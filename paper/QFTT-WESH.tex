\documentclass[12pt,a4paper]{article}
\usepackage{amsthm}
\newtheorem{example}{Example}

\usepackage[utf8]{inputenc}
\usepackage[T1]{fontenc}
\usepackage{amsmath,amssymb,amsthm,mathtools}
\usepackage{physics}
\usepackage{times}
\usepackage{graphicx}
\usepackage[english]{babel}
\usepackage{setspace}
\usepackage{booktabs}
\usepackage{chngcntr}
\usepackage[hidelinks]{hyperref}
\usepackage{url}
\usepackage{empheq}
\usepackage{hyperref}
\hypersetup{
    breaklinks=true,
    colorlinks=true,
    linkcolor=black,
    citecolor=black,
    urlcolor=black
}
\counterwithin{figure}{section}

\DeclareMathOperator{\sdet}{sdet}
\newcommand{\Thm}{\ref}

% Define 'note' environment
\newenvironment{note}{\begin{quote}\itshape}{\end{quote}}
\newcommand{\Var}{\operatorname{Var}}
\usepackage{newunicodechar}
\newunicodechar{̂}{\^{} }
\usepackage{float}
\usepackage[most]{tcolorbox}
\usepackage{tabularx,booktabs,array}
\usepackage{afterpage}
\usepackage{enumitem}

% --- Riferimenti compatti e consistenti
\newcommand{\Eq}[1]{Eq.~\eqref{#1}}
\newcommand{\Eqs}[1]{Eqs.~\eqref{#1}}
\newcommand{\Fig}[1]{Fig.~\ref{#1}}
\newcommand{\Sec}[1]{Sec.~\ref{#1}}
\newcommand{\App}[1]{App.~\ref{#1}}

\newcommand{\Gtime}{\Gamma[\Psi]}
\newcommand{\Gdec}{\Gamma_{\mathrm{dec}}}
\newcommand{\Gspec}{\Gamma_{\mathrm{WESH}}}

% Theorem environments
\newtheorem{lemma}{Lemma}[section]
\newtheorem{proposition}{Proposition}[section]
\newtheorem{definition}{Definition}[section]
\newtheorem{corollary}{Corollary}[section]
\newtheorem{remark}{Remark}[section]
\newtheorem{theorem}{Theorem}[section]
\newtheorem{axiom}{Axiom}
\newtheorem{principle}{Principle}[section]


% Commands for convenience
\renewcommand{\Tr}{\operatorname{Tr}}
\newcommand{\That}{\hat{T}}
\newcommand{\eff}{\mathrm{eff}}
\newcommand{\tot}{\mathrm{tot}}
\newcommand{\loc}{\mathrm{loc}}
\newcommand{\corr}{\mathrm{corr}}
\newcommand{\coh}{\mathrm{coh}}
\newcommand{\Eig}{\mathrm{Eig}}
\newcommand{\thr}{\mathrm{thr}}
\newcommand{\refs}{\mathrm{ref}}

\newcommand{\Ttil}{\tilde T}      % dimensionless time field: \Ttil = \hat T / \taus
\newcommand{\Lxy}{L_{xy}}         % bilocal Lindblad operator (dimensionless)
% Correlation 4-volume (coarse-graining scale, O(\xi^4)):
\newcommand{\Vxi}{V_\xi}          % with \Vxi \sim \xi^4
% Normalized 4D measures:
\newcommand{\intx}{\int \frac{d^4x}{\Vxi}}
\newcommand{\intxy}{\iint \frac{d^4x\,d^4y}{\Vxi^2}}

\newcounter{boxinternal}
\setcounter{boxinternal}{0}

\numberwithin{equation}{section}

% Eigentime notation (consistency)
\newcommand{\NEig}{\mathcal{N}_{\mathrm{Eig}}}
\newcommand{\lambdaEig}{\lambda_{\mathrm{Eig}}}

\usepackage{array,tabularx,booktabs,caption,ragged2e}
\newcolumntype{Y}{>{\RaggedRight\arraybackslash}X}
\captionsetup[table]{font=small,labelfont=bf,justification=raggedright,singlelinecheck=false}

\usepackage{enumitem}  
\makeatletter
\def\thm@space@setup{
  \thm@preskip=0.9cm \thm@postskip=0.9cm}
\makeatother

\renewenvironment{abstract}{
  \vspace*{-3cm}  
  \begin{center}
  \Large\textbf{Abstract} 
  \end{center}
  \vspace{0.5cm} 
  \noindent\ignorespaces  
}{\par\vspace{1cm}}  

\newcommand{\Iloc}{I_{\text{loc}}}
\newcommand{\Ibi}{I_{\text{bi}}}
\begin{document}

\begin{titlepage}
{\centering 
\vspace*{2cm}
{\LARGE\bfseries
QUANTUM FIELD THEORY OF TIME\\[0.5cm]
}
\vspace{1cm}
{\Large
The Weak Entanglement Symmetry Hypothesis\\
and Emergent Spacetime\\[1cm]
}
\vspace{1cm}

{\large
Luca Casagrande\\[0.5cm]
}
\vfill
{\normalsize
First Edition\\
January 24, 2026\\[0.5cm]
}
}
\end{titlepage}

\vspace{0.3cm}
\begin{abstract}
In canonical quantum gravity the Wheeler--DeWitt equation freezes evolution, 
while standard quantum theory treats time as an external parameter. The Weak 
Entanglement Symmetry Hypothesis (WESH) promotes entanglement as the carrier 
of statistical Lorentz-type symmetry and as an engine of endogenous dissipative 
dynamics. Lorentz symmetry is recovered in expectation over the Poissonian 
ensemble of \emph{Eigentime} events (discrete stochastic localizations of the 
time-field operator $\hat{T}(x)$ within its continuous spectrum, induced by 
the quadratic WESH dissipators $\mathcal{D}[\hat{T}^2(x)]$ and $\mathcal{D}[L_{xy}]$ --- mean spacing $\tau_{\rm Eig}$), whereas global conserved charges obey a stronger 
operator-level constraint (WESH--Noether, $\mathcal L^\dagger[\hat Q_a]=0$). On this foundation, physical time becomes a local quantum field operator $\hat T(x)$, subject to superposition and objective collapse, with dynamics constructed from first principles. An open-system master equation in an auxiliary label $s$ generates physical time through $dt/ds=\Gamma[\rho]\ge 0$. Complete positivity, no-signaling, and a pre-geometric WESH--Noether conservation principle single out a CPT-even Lindblad structure with quadratic local dissipators $\mathcal D[\hat T^2(x)]$ and a Rényi-2 weighted bilocal channel. In the infrared, spontaneous gradient alignment $\partial_\mu \langle \hat T \rangle = k\,\partial_\mu \Phi$ defines an emergent metric and reproduces the Einstein equations with Newton constant fixed by the same matching; no background geometry is assumed. The same eigentime statistics yields the cosmological constant as irreducible shot noise, with scaling $\delta\Lambda \sim H^2$. The framework predicts collective coherence scaling $\tau_{\mathrm{coh}}\propto N^2$ and a robust $\cos^2\theta$ angular law for parity decay. Collision-model simulations ($N=2$--$16$) and experiments on IBM Eagle and Rigetti Ankaa-3 show trends consistent with both signatures. A near-horizon analysis of the $\hat T^2$ channel and the emergent KMS structure yields
\[
S_{\rm BH}=\frac{A}{4L_P^2}+\gamma_{\rm BH}\ln\!\frac{A}{L_P^2}+\cdots
\]
without thermal ansatz. The construction has no free dimensionless parameters; deviations from the $1/4$ prefactor, the $N^2$ scaling, or the angular law would falsify the framework.
\vspace{0.8cm}

\begin{center}
\textit{
All resources are openly available.\\
Complete experimental data and analysis code are provided at \url{https://github.com/Luca-Casagrande/QFTT-WESH}\\
Sections 1, 5, 6 and Appendices A--D and G--J have been formally verified in Lean 4/Mathlib.
}
\end{center}
\end{abstract}

\clearpage 
\thispagestyle{empty}  
\enlargethispage{2.5cm}
\vspace*{-2cm} 

\begin{tcolorbox}[
  enhanced, breakable,
  title={\centering\textbf{Box 1 — Logical Map}},
  colback=gray!4, colframe=black, boxrule=0.6pt, arc=2pt,
  left=8pt, right=8pt, top=8pt, bottom=8pt]

\small
\textbf{Motivation.} Neither imposing an external time nor accepting a strictly timeless universe is satisfactory. 
QFTT--WESH derives physical time from quantum events weighted by entanglement, yielding a consistent route from the Wheeler--DeWitt constraint to general relativity and black–hole thermodynamics.

\vspace{0.4em}
\textbf{Primitives.}
\begin{enumerate}[leftmargin=1.2em,itemsep=0.25em]
\item \emph{Timeless constraint.} $\hat H_{\rm tot}\ket{\Psi}=0$ (WDW: no $t$, no metric).
\item \emph{Quantum time field.} Local observable $\hat T(x)$ on an extended kinematical space, with mean eigentime profile $\tau(x):=\langle\hat T(x)\rangle_\rho$ in the effective (mean-field) description.
\item \emph{Eigentimes \& bootstrap.} $\,\tfrac{dt}{ds}=\Gamma[\rho]\ge 0$, $\,t(s)=\!\int_0^s\Gamma[\rho(u)]du$:
      eigentime events create the very flow that regulates their statistics.
\end{enumerate}

\vspace{0.4em}
\textbf{Operational structure.}
\begin{itemize}[leftmargin=1.2em,itemsep=0.25em]
\item \emph{GKSL structure.} Locally GKSL (frozen-gate micro-steps), globally 
      a CP/TP concatenation; causal generator $\mathcal L$ with Hermitian 
      channels $\hat T^2(x)$ and $L_{xy}=\hat T^2(x)-\hat T^2(y)$.
\item \emph{Covariant construction.} Once the emergent metric 
      $g^{(T)}_{\mu\nu}=\zeta^{-1}\langle \partial_\mu \hat T\,\partial_\nu \hat T\rangle$ is defined, 
      the master equation is made covariant by replacing $\partial\!\to\!\nabla$ and using a scalar 
      causal kernel $K_\xi(\sigma(x,y))$ in the bilocal weight; WESH--Noether enforces 
      conservation of global charges.
\item \emph{Finite-range Markov window.} Correlation time $\tau_{\rm corr}\!\sim\!\xi/c$ with 
      $\xi\simeq L_P$ and exponential-causal support.
\item \emph{Two-derivative IR truncation.} In $D{=}4$, diffeomorphism invariance together with a 
      $\leq 2$-derivative truncation fixes the IR geometric sector to Einstein--Hilbert plus a cosmological constant; 
      higher-derivative operators are treated as controlled EFT corrections.
\item \emph{No free parameters.} The normalization \(k\) is fixed by internal consistency via 
      \(k^2/(4\pi G)=\lambda_1+3\lambda_2\); no tunable dimensionless couplings remain.
\end{itemize}

\vspace{0.4em}
\textbf{Structure \& Derived Laws.}
\begin{enumerate}[leftmargin=1.2em,itemsep=0.25em]
\item \emph{WESH--Noether conservation.} $\mathcal L^\dagger[\hat Q_a]=0$: generator-level conservation 
      on CPTP maps, path-independent global charges.
\item \emph{Spacetime bootstrap dynamics.} WDW $\Rightarrow$ continuous WESH evolution with 
      instantaneous $C(\rho)$ feedback; eigentime events occur with propensity $\Gamma[\rho]\!\ge\!0$ ($>0$ under ND) and 
      generate $t$ via $dt/ds=\Gamma[\rho]$; the state is driven toward the swap-even ($\mathbb Z_2$) projected physical subset and maintains self-consistent equilibrium without introducing any discrete external time step. The bootstrap architecture $\rho\mapsto C[\rho]\mapsto\mathcal L_{C[\rho]}\mapsto\rho$ is the unique structure capable of representing chronogenesis; nonlinearity is its natural consequence.
\item \emph{Fixed point and alignment.} The bootstrap map 
      $\rho\mapsto\exp(\delta s\,\mathcal L_{C[\rho]})\rho$ is nonlinear but continuous; 
      by Schauder--Tychonoff a fixed point exists, and Dobrushin contraction ensures uniqueness. 
      At equilibrium, $\partial_\mu\tau=k\,\partial_\mu\Phi$.
\item \emph{Hidden-sector cancellation.}
      At the fixed point the hidden-sector stresses cancel in the continuum limit,
$T^{(T)}_{\mu\nu}+T^{(\mathrm{nl})}_{\mu\nu}=0$ as $N\to\infty$, with controlled
finite-$N$ residual ${\cal O}(1/N)$; hence
$G_{\mu\nu}+\Lambda g_{\mu\nu}=8\pi G\,T^{(m)}_{\mu\nu}+{\cal O}(1/N)$.
\end{enumerate}

\vspace{0.4em}
\textbf{Falsifiable Signatures.}
\begin{itemize}[leftmargin=1.2em,itemsep=0.25em]
\item \emph{Collective stability.} $\tau_{\mathrm{coh}}\!\propto\!N^2$ — distinctive of $N^{-2}$ bilocal weighting.
\item \emph{Angular law.} Decoherence $\!\propto\!(1+\varepsilon\cos^2\theta)$, with a state– and geometry–dependent 
      modulation parameter $\varepsilon$; $W$-state anti-modulation ($\varepsilon\!<\!0$) confirms state dependence.
\item \emph{BH sector.} A near-horizon KMS fixed point yields
      $S_{\rm BH}=A/(4L_P^2)+\gamma_{\rm BH}\ln(A/L_P^2)+\cdots$  
      without imposing a thermal ansatz.
\end{itemize}

\vspace{0.4em}
\textbf{Scope \& Controlled Approximations.}
\begin{itemize}[leftmargin=1.2em,itemsep=0.25em]
\item \emph{Domain.} IR continuum regime ($L\!\gg\!\xi$), $D{=}4$, swap-even projected physical subset.
\item \emph{Kernel \& memory.} Exponential-causal kernel of range $\xi$, finite-memory Markov limit 
      $\tau_{\rm corr}\!\sim\!\xi/c$; curvature-induced deformations treated as EFT corrections.
\item \emph{Finite-$N$ control.} $1/N$ residuals bounded via graph-to-manifold convergence.
\end{itemize}

\vspace{0.5em}
\textbf{Summary.} A single CP/Markov pre-geometric dynamics (WESH), fixed by Noether consistency and without tuning, bootstraps spacetime from quantum events and yields both Einstein–Hilbert gravity and black-hole thermodynamics as its infrared manifestations in four dimensions.

\end{tcolorbox}

\vspace*{0.5cm}
\newpage  

\section*{\centering QFTT--WESH --- Graphical Synopsis}
{
\setcounter{figure}{0}
\renewcommand{\thefigure}{\arabic{figure}}

\begin{figure}[H]  
\centering
\includegraphics[width=1\textwidth]{Picture1.png}  
\vspace{0.1cm}
\caption{\textbf{Conceptual overview of the QFTT--WESH framework.}
From the Wheeler--DeWitt constraint, the theory introduces a quantum time field
evolving via the WESH master equation under pre-geometric charge conservation.
Eigentime collapse events generate a spacetime bootstrap converging to emergent
Einstein--Hilbert gravity and horizon thermodynamics.}
\label{fig:architecture}
\end{figure}

\vspace{1cm}

\begin{figure}[H]
\centering
\includegraphics[width=1\textwidth]{Picture2.png}
\vspace{0.1cm}
\caption{\textbf{WESH as the unique dissipative extension of Wheeler--DeWitt.}
The frozen time paradox ($\Delta H_{\rm tot}=0 \Rightarrow \Delta T \to \infty$) 
requires a GKSL dissipative structure. Three constraints---CPT symmetry with 
no-signaling causality, WESH--Noether charge conservation, and collective 
$N^{2}$ coherence scaling---uniquely determine the generator: a local channel 
$\mathcal{D}[\hat T^{2}(x)]$ plus a bilocal channel $\mathcal{D}[L_{xy}]$ 
weighted by Rényi--2 correlators. 
The dashed arrow indicates foundational closure: in the limit $G\to 0$, 
dissipation vanishes and WESH reduces to unitary Wheeler--DeWitt.}
\label{fig:wdw-wesh}
\end{figure}

\vspace{1cm}

\begin{figure}[H]
\centering
\includegraphics[width=1\textwidth]{Picture3.png}
\vspace{0.1cm}
\caption{\textbf{Spacetime bootstrap from pre-geometric dynamics.}
Starting from pre-geometric WESH constraints, the Lyapunov functional
$\mathcal{M}_\epsilon[\rho]$ determines a unique fixed point on the emergent
Einstein--Hilbert manifold. The global state $\rho(s)$ reaches this fixed point
through continuous GKSL evolution punctuated by stochastic Poisson eigentime events ($\Gamma[\rho]\ge 0$ - red crosses), with the hybrid flow driving
spacetime emergence from the pre-geometric sector (green annotations mark the
self-consistent stationary state).}
\label{fig:bootstrap}
\end{figure}

\thispagestyle{empty}  
\vspace*{-2.9cm} 

\begin{figure}[H]
\centering
\includegraphics[width=1\textwidth]{Picture4.png}
\vspace{0.1cm}
\caption{\textbf{Emergence of general relativity from WESH.}
WESH--Noether symmetry and the stationary bootstrap fix the metric decomposition
$g_{\mu\nu}=g^{(T)}_{\mu\nu}+g^{(C)}_{\mu\nu}+g^{(E)}_{\mu\nu}$ and determine the
coupling matching. The Lyapunov flow converges to a unique stationary point
characterized by gradient alignment $\partial_\mu \tau = k\,\partial_\mu \Phi$.
At this point, hidden--sector cancellation reduces the infrared dynamics to the
Einstein equations with cosmological constant and conserved matter stress--energy,
up to $1/N$ corrections.}
\label{fig:metric}
\end{figure}

\vspace{1cm}

\begin{figure}[H]
\centering
\includegraphics[width=1\textwidth]{Picture4b.png}
\vspace{0.1cm}
\caption{\textbf{Black--hole thermodynamics from WESH horizon dynamics.}
WESH--Noether consistency and the unique KMS fixed point on the near--horizon
wedge provide the entropy framework. Bipartite halving with swap-even ($\mathbb{Z}_2$) projection yields the $1/4$ prefactor; the WESH effective action Hessian with
replica regularization produces logarithmic corrections. The condition $\Gamma[\rho]\ge 0$ implies the holographic bound
$S[\mathcal{H}]\le A/(4L_P^2)+\cdots$.}
\label{fig:blackhole}
\end{figure}

} 

\clearpage 
\thispagestyle{empty}  
\enlargethispage{2.5cm}
\vspace*{-3cm} 


\begin{tcolorbox}[
  enhanced, breakable,
  title=\centering\textbf{Box 2: Foundational clarifications and core principles}]
\noindent
For later reference we collect a few structural points that are easy to misread
if one assumes standard Hamiltonian or semiclassical intuition. They follow from
the constraints introduced in Sec.~1, rather than being added as independent
postulates.
\setlength{\itemsep}{8pt}

~\\
\textbf{Auxiliary label $s$ and emergence of physical time.}
The WESH master equation is written in an auxiliary ordering parameter $s$ with
dimensions of time, but $s$ itself is never an observable. Physical time is
reconstructed afterwards from eigentime events via $dt/ds=\Gamma[\rho(s)]\ge 0$ ($\Gamma>0$ under nondegeneracy). Canonical commutators are imposed on an arbitrary
kinematical Cauchy slice, so that the algebra does not depend on any prior choice
of time coordinate; the distinction between $s$ (ordering) and $t(s)$ (emergent
physical time) removes the usual circularity of “time–evolution in time”.

~\\
\textbf{Local time field and the scope of Pauli’s theorem.}
QFTT--WESH employs a local quantum time field $\hat T(x)$ on an extended
kinematical space subject to the Wheeler--DeWitt constraint and open–system
dynamics. Pauli’s no–go theorem concerns a single global self–adjoint time
operator canonically conjugate to the Hamiltonian in a closed, unitary theory,
which is not the setting considered here. Operational meaning is instead attached
to $\hat T(x)$ through eigentime collapse events and the statistics of their rate.

~\\
\textbf{Dissipator structure, entanglement gate, nonlinearity, and Markov window.}
CPT symmetry, WESH--Noether charge conservation, and the requirement of collective
stability select Hermitian jumps $\hat T^2(x)$ and
$L_{xy}=\hat T^2(x)-\hat T^2(y)$
(equivalently $\tilde T^2(x)-\tilde T^2(y)$ in dimensionless units with $\tilde T=\hat T/\tau_s$) and fix the
$N^{-2}$ bilocal scaling, rather than leaving a large family of GKSL terms.
The Rényi‑2 gate $C(\Psi;x,y)$ is a bounded scalar functional of the reduced state
$\rho_{xy}$ that weights the bilocal channel and couples only genuinely entangled
subsystems. The resulting state-dependence $\rho\mapsto C[\rho]\mapsto\mathcal L_{C[\rho]}\mapsto\rho$ renders the evolution nonlinear; this is the necessary architecture of any chronogenetic theory where time emerges from the dynamics it regulates. For each frozen $C$ the generator is of standard GKSL form, and the full evolution is a concatenation of positivity- and trace-preserving maps; the Schauder--Tychonoff theorem guarantees existence of a fixed point despite the nonlinearity. The interaction
kernel has light–cone–restricted, exponential–causal support of range $\xi$,
defining a finite–memory Markov window with $\tau_{\rm corr}\sim\xi/c$ where
memory effects are parametrically suppressed. In this regime the local rate
$\nu$ is fixed by self–consistency with the bilocal sector on causal diamonds,
so that $\Gamma_{\rm loc}$ and $\Gamma_{\rm bi}$ are not independently tunable
and $\nu$ is \emph{not} a free normalization parameter.

~\\
\textbf{Hidden–sector cancellation as a dynamical outcome.}
The cancellation between time–sector and nonlocal stresses,
$T^{(T)}_{\mu\nu}+T^{(\mathrm{nl})}_{\mu\nu}=0$ (in the $N\to\infty$ continuum limit), is not imposed by hand.
It arises at the unique globally attractive fixed point of the WESH flow, where
the Lyapunov functional $\mathcal M_\epsilon$ enforces gradient alignment of the mean field
$\partial_\mu\tau = k\,\partial_\mu\Phi$ (with $\tau:=\langle\hat T\rangle_\rho$) and the matching relation between $k$
and the WESH couplings fixes the relative normalization of the two sectors.
In this regime the Einstein tensor is sourced only by ordinary matter, with
residual $\mathcal O(1/N)$ corrections controlled by the continuum limit.

~\\
\textbf{Black–hole entropy and the role of $(\kappa\xi)^2$.}
Near a stationary horizon, the same WESH dynamics leads to a Hartle--Hawking KMS
state on the wedge and to a mean pair entropy
$\bar s = 1 + \mathcal O\big((\kappa\xi)^2\big)$.
Bipartite halving and swap-even projection then produce the universal area term
$S_{\rm BH}=A/(4L_P^2)$, while the Hessian of the WESH effective action yields
the logarithmic correction. The combination $(\kappa\xi)^2$ is the natural small
parameter of the calculation; for spherical horizons it reduces to the familiar
$\mathcal O(L_P^2/A)$ scaling, but the more general form continues to apply to
rotating or distorted horizons.
\end{tcolorbox}


\vspace{1cm}

\section{QFTT Framework \& WESH}
\label{sec:qftt-framework}

\vspace{0.5cm}

\textit{In this section we promote physical time to a local quantum field operator
$\hat{T}(x)$ on an extended kinematical algebra and justify its operational
meaning in constrained (Wheeler--DeWitt) cosmologies. From CPT symmetry,
WESH--Noether conservation, and collective stability (bootstrap closure),
within a completely positive and causal GKSL evolution, we derive a unique
quadratic \textsc{wesh} Lindblad generator $\mathcal{D}[\hat{T}^{2}(x)]$,
and couple it to entanglement through a normalized R\'enyi-2 correlator
$C(\Psi; x, y)$. A self-consistent bootstrap driven by eigentime collapse
events generates physical time and an entanglement-weighted dissipative flow,
with a bilocal kernel $\gamma(x, y)$ enforcing causal support and collective
$N^{-2}$ scaling. The section concludes by fixing the fundamental scales in
terms of Planck units and summarizing the resulting parameters and kernels
in Table~\ref{tab:parameters}.
}

\vspace{0.5cm}

\subsection{Problem of time and quantum gravity strategy}
\vspace{0.3cm}

In canonical quantum cosmology, the Wheeler--DeWitt constraint
$\hat H_{\rm tot}\ket{\Psi}=0$ freezes global evolution, while standard
quantum theory still treats time as an external parameter. Most approaches
to quantum gravity, loop quantum gravity (Ashtekar et al., 1995), string
theory, and causal dynamical triangulations, either quantize an already
geometric spacetime or work semiclassically on a fixed background, leaving
the operational status of time itself unresolved. Alternative proposals such
as the Page--Wootters mechanism (Page \& Wootters, 1983) tie the emergence
of time to entanglement between system and clock but do not derive
dissipative dynamics or metric structure. 

\noindent Here we address both issues at once: we promote physical time to a local
quantum field operator $\hat T(x)$ on an extended kinematical algebra, evolve
the state by a completely positive WESH master equation in an auxiliary
ordering parameter $s$, and let physical time and metric relations emerge
from eigentime events and entanglement--weighted dissipative flow. In the
remainder of this section we derive falsifiable pre-geometric signatures and
show how general relativity and black-hole thermodynamics emerge in the
infrared limit of the theory.


\subsection{Time field and canonical structure}

As mentioned, the resolution proposed here elevates physical time from an external
parameter to a local quantum field operator $\hat{T}(x)$ defined on an
extended kinematical Hilbert space. This extension follows from the
canonical structure of constrained systems.

\noindent In the minisuperspace approximation, the Wheeler--DeWitt constraint
decomposes as
\begin{equation}
\hat{H}_{\text{tot}}=\hat{H}_{\text{universe}}+\hat{P}_T \approx 0, \qquad
[\hat{T},\hat{P}_T]=i\hbar,
\label{eq:minisspace-split}
\end{equation}
where $\hat{P}_T$ is the momentum conjugate to a global time variable. The
constraint does not forbid dynamics; it enforces energy balance between the
time sector and the remaining degrees of freedom. Passing to the local
field-theoretic extension, we postulate canonical commutation relations on an arbitrary
kinematical slice $\Sigma$ (since the algebra is independent of the choice of coordinates prior to metric emergence):
\begin{equation}
[\hat{T}(\mathbf{x}),\hat{\Pi}_T(\mathbf{y})]
= i\hbar\,\delta^{(3)}_{\Sigma}(\mathbf{x}-\mathbf{y}),\quad
[\hat{T},\hat{T}]=[\hat{\Pi}_T,\hat{\Pi}_T]=0.
\label{eq:canonical-commutators}
\end{equation}
On the constraint surface, the conjugate momentum is identified with (minus) the local energy density,
\begin{equation}
\hat{\Pi}_T(\mathbf{x})=-\hat{\mathcal{H}}(\mathbf{x}),\qquad
[\hat{\mathcal{H}}(\mathbf{x}),\hat{T}(\mathbf{y})]
= i\hbar\,\delta^{(3)}_{\Sigma}(\mathbf{x}-\mathbf{y}),
\label{eq:conjugacy}
\end{equation}
so that the local Hamiltonian constraint
$\hat{\mathcal{H}}(x)+\hat{\Pi}_T(x)\approx 0$ becomes the generator of
local time translations. This identification transforms the Wheeler--DeWitt
``frozen time'' into a dynamical equilibrium condition: the constraint
balances the quantum time frame against the local energy content.

\noindent The spectral decomposition
\begin{equation}
\hat{T}(x)\,|t\rangle_x = t\,|t\rangle_x, \quad
\int_{-\infty}^{+\infty}\!dt\; |t\rangle_x\,{}_x\langle t| = \mathbb{1}_x
\label{eq:decomposition}
\end{equation}
allows one to expand the global state as a superposition over temporal
configurations:
\begin{equation}
|\Psi\rangle = \!\int\!\mathcal{D}t(x)\,\mathcal{D}\phi\;
\Psi[t(x),\phi]\;\bigotimes_x |t(x)\rangle_x\!\otimes|\phi\rangle.
\label{eq:expansion}
\end{equation}
In the WDW sector, temporal indeterminacy is therefore physical, not
artifactual: the universe exists in a quantum superposition of chronologies.
Resolving this superposition requires a dynamical mechanism that cannot be
unitary in the conventional sense, there is no external $t$ to generate a
Schr\"odinger flow, but must be dissipative. The structure of this
dissipation is fixed by symmetry, as we now explain.

\vspace{0.5cm}
\subsection{WESH--Noether conservation}
\label{sec:wesh-noether}

\vspace{0.3cm}

The dissipative extension of Wheeler--DeWitt operates in a regime where 
no background metric exists: the constraint 
$\hat{H}_{\mathrm{tot}}|\Psi\rangle = 0$ removes the generator of time 
translations, leaving only algebraic structure. In a closed universe 
there is no external bath to absorb conserved charges, so the generator 
must preserve them exactly:
\begin{equation}
\mathcal{L}^\dagger[\hat{Q}_a] = 0 \quad \forall\, a.
\label{eq:wesh-noether-constraint}
\end{equation}
This \emph{WESH--Noether conservation} is pre-geometric: it is formulated
in the auxiliary parameter $s$, before physical time or spacetime have
emerged, and guarantees path independence in state space, so that
$\langle\hat{Q}_a\rangle$ does not depend on the particular trajectory
$\rho(s)$.

\paragraph{Algebraic mechanism.}
The conservation law~\eqref{eq:wesh-noether-constraint} translates into 
commutant conditions on the generator. For Hermitian Lindblad operators 
$L$, the adjoint dissipator acts on observables as
\begin{equation}
\mathcal{D}^\dagger_L[\hat{Q}] = -\tfrac{1}{2}[L,[L,\hat{Q}]].
\label{eq:double-commutator}
\end{equation}
Under the WESH structure ($\nu>0$, $\gamma(x,y)\,C(\Psi;x,y)\ge 0$, Hermitian 
jumps) and mild spectral regularity (Appendix~G, Prop.~G.2), 
$\mathcal{L}^\dagger[\hat{Q}_a] = 0$ is equivalent to
\begin{equation}
\begin{split}
&[H_{\mathrm{eff}}, \hat{Q}_a] = 0, \qquad 
[\hat{T}^2(x), \hat{Q}_a] = 0, \\
&[L_{xy}, \hat{Q}_a] = 0 \quad \text{for all $(x,y)$ with nonzero rate}.
\end{split}
\label{eq:commutant-conditions}
\end{equation}
The double commutator~\eqref{eq:double-commutator} is the algebraic engine 
of conservation: it vanishes precisely when each Lindblad channel commutes 
with the charge.

\paragraph{Difference structure.}
For the bilocal channel, the commutant condition together with the requirement 
of pairwise-balanced redistribution (zero net injection) selects the difference form,
\begin{equation}
L_{xy} = \hat{T}^2(x) - \hat{T}^2(y).
\label{eq:difference_Lxy}
\end{equation}
This structure redistributes coherence between causally connected points 
without injecting or removing total charge: any increase at $x$ is 
compensated by a decrease at $y$. The local channel 
$\mathcal{D}[\hat{T}^2(x)]$ satisfies the same commutant provided the 
time field is \emph{T-neutral},
\begin{equation}
[\hat{T}(x), \hat{Q}_{\mathrm{tot}}] = 0 \quad \forall\, x.
\label{eq:T-neutrality}
\end{equation}
T-neutrality is not imposed externally; it is the unique condition 
compatible with path independence, universal time emergence, and 
no-signaling (Appendix~G, Remark~G.4).

\begin{theorem}[WESH--Noether]
\label{thm:wesh-noether}
Let $\mathcal{L}$ be a GKSL generator with Hermitian jump operators 
$L_\alpha \in \{\hat{T}^2(x),\, L_{xy} = \hat{T}^2(x) - \hat{T}^2(y)\}$ 
and let $\hat{Q}_a$ be any global charge. Then the following are equivalent:
\begin{enumerate}[label=(\roman*)]
\item $\mathcal{L}^\dagger[\hat{Q}_a] = 0$ \quad (generator-level conservation);
\item $[H_{\mathrm{eff}}, \hat{Q}_a] = 0$ and $[L_\alpha, \hat{Q}_a] = 0$ 
for all $\alpha$ \quad (commutant conditions);
\item for every trajectory $\rho(s)$ one has
\begin{equation}
\frac{d}{ds}\langle\hat{Q}_a\rangle_{\rho(s)} 
= \Tr\bigl(\mathcal{L}^\dagger[\hat{Q}_a]\,\rho(s)\bigr) = 0,
\label{eq:charge-conservation}
\end{equation}
\quad (path independence in $s$).
\end{enumerate}
Under T-neutrality~\eqref{eq:T-neutrality}, all three conditions hold for 
every conserved charge.
\end{theorem}

\noindent This theorem extends Noether's result to dissipative quantum dynamics:
constants of motion are identified with fixed points of the Heisenberg
semigroup generated by $\mathcal L^\dagger$. The proof uses the 
double-commutator identity~\eqref{eq:double-commutator} together with
mild spectral regularity assumptions.

\paragraph{Functional role in the present framework.}
In the remainder of this section, WESH--Noether is used as a selection 
principle: together with CPT symmetry and collective stability, it fixes 
the admissible form of the dissipative generator and singles out the 
quadratic WESH structure employed in Eq.~\eqref{eq:wesh-master}. The same 
constraint plays a second role in the infrared analysis. There, its 
pre-geometric character becomes essential: the WESH flow admits a unique 
fixed point that minimizes a Lyapunov functional subject to the Noether 
constraints, and at this fixed point the gradients of the time field and of the entanglement potential align, $\partial_\mu\tau = k\,\partial_\mu\Phi$ (with $\tau:=\langle\hat T\rangle_\rho$).
The emergent Einstein--Hilbert dynamics and hidden-sector cancellation 
follow from this aligned fixed point. Both applications derive from the same algebraic condition; only the 
physical interpretation shifts from constraint selection to geometric 
emergence.

\vspace{0.3cm}

\noindent The complete operator-level derivation, including spectral regularity 
conditions and the bridge to alignment, is developed in Appendix~G.

\vspace{0.8cm}

\subsection{Focus on the Weak Entanglement Symmetry Hypothesis.}

\vspace{0.5cm}

In this framework ``weak'' does not refer to a small coupling, but to the 
level at which Lorentz-type symmetry is enforced. Lorentz invariance is not 
required pointwise on each individual collapse trajectory, but is recovered 
in expectation over the Poissonian ensemble of eigentime events. Global 
conserved charges obey a stronger, operator-level constraint 
(Sec.~\ref{sec:wesh-noether}). Entanglement correlations $C(\Psi;x,y)$ distribute the effect of a local collapse across all correlated regions, so that apparent local violations are globally compensated when averaged over the process. Such symmetry is therefore a property of the process measure rather than of individual realizations.

\vspace{0.5cm}

\subsection{Constraint-driven selection of the WESH generator}
\label{sec:quadratic-selection}

\vspace{0.3cm}

We now address the mechanism by which a directed temporal order can emerge from within the Wheeler–DeWitt sector itself. Instead of coupling to an external bath, we model dissipation as \emph{endogenous}: an effective redistribution of information among relational degrees of freedom in a closed universe, where the environment is internal to the WDW kinematics and no net charge or energy can be injected. This zero-sum condition, together with complete positivity, light-cone causality, WESH–Noether charge conservation, and collective stability at large $N$, severely constrains the admissible GKSL generators. Under these requirements the space of allowed structures collapses to a single option: quadratic local channels $\mathcal{D}[\hat T^{2}(x)]$ and difference-form bilocal operators $L_{xy} = \hat T^{2}(x) - \hat T^{2}(y)$, modulated by the entanglement gate. The remainder of this subsection derives this uniqueness from first principles.

\paragraph{CPT symmetry.}
Under time reversal one has $\hat T \to -\hat T$, so any admissible local
Lindblad operator must be an even function of the field. Writing
$L_x = F(\hat T(x))$ with a power series
$F(z)=\sum_{m\ge1} a_{2m} z^{2m}$ excludes odd powers and confines the dynamics
to CPT-even channels. The lowest non-trivial choice compatible with
spontaneous collapse is the quadratic form $\mathcal{D}[\hat T^2(x)]$.

\paragraph{WESH--Noether conservation.}
Pre-geometric WESH--Noether conservation requires the generator to preserve
all global charges exactly, $\mathcal{L}^\dagger[\hat Q_a]=0$ for all $a$
(Eq.~\eqref{eq:wesh-noether-constraint}). For the WESH channels introduced in
Sec.~\ref{sec:wesh-noether}, this is implemented by the commutant conditions
\eqref{eq:commutant-conditions}, which in particular fix the bilocal difference
structure $L_{xy} = \hat{T}^2(x) - \hat{T}^2(y)$ (Eq.~\eqref{eq:difference_Lxy})
and ensure $[L_{xy},\hat Q_{\rm tot}]=0$. Here we use these algebraic
constraints as selection principles for the admissible GKSL generators.

\paragraph{Collective coherence scaling.}
The bilocal kernel must suppress pairwise coupling as the system size grows,
keeping total collapse power bounded. Imposing $\gamma_{ij} \propto N^{-2}$
leads to a coherence time
\begin{equation}
\tau_{\mathrm{coh}} \propto N^{2},
\label{eq:N2-scaling}
\end{equation}
a falsifiable prediction that distinguishes WESH from standard decoherence
models, where $\tau_{\mathrm{coh}}$ typically scales as $N^{-1}$ or remains of
order unity.

\begin{proposition}[Quadratic dissipator from symmetry and stability]
\label{prop:quadratic-dissipator}
Let the local Lindblad operators $L_x$ be CPT-even, satisfy the WESH--Noether
condition \eqref{eq:wesh-noether-constraint}, and be coupled through a bilocal
kernel $\gamma(x,y)\propto N^{-2}$ in such a way that the infrared coherence
time obeys $\tau_{\mathrm{coh}}(N)\propto N^{2}$ (Eq.~\ref{eq:N2-scaling} and
Eq.~\ref{eq:fixed-regime}). Then the only admissible local dissipator is
$\mathcal{D}[\hat T^2(x)]$.
\end{proposition}

\begin{proof}
CPT symmetry implies that $L_x$ is an even function of $\hat T(x)$,
$L_x \sim \hat T^{2n}(x)$ with $n\ge1$. For a general even-power dissipator
$\mathcal{D}[\hat T^{2n}]$ the combination of $\gamma\propto N^{-2}$ with
dimensional analysis yields
\begin{equation}
\text{If } L_x \sim \hat{T}^{2n} \quad\Rightarrow\quad
\tau_{\mathrm{coh}}(N)\propto N^{2n}.
\label{eq:scaling-condition}
\end{equation}
The collective stability requirement $\tau_{\mathrm{coh}}(N)\propto N^{2}$
then fixes $n=1$ uniquely, hence the local channel is $\mathcal{D}[\hat T^2(x)]$.
\end{proof}

\begin{remark}[Convergent IR rationale (ancillary)]
Independently of the collective-stability selection, a Wilsonian IR argument
also singles out the quadratic operator: among CPT-even local monomials
$F(\hat T)\sim \hat T^{2n}$, the lowest-dimension term dominates the long-wavelength
fixed-regime, while higher even powers are irrelevant deformations. This IR
minimality rationale is consistent with (and secondary to) the structural
selection $n=1$ enforced by $\tau_{\mathrm{coh}}(N)\propto N^2$.
\end{remark}

\begin{remark}
The role of $\alpha=2$ is dynamical rather than kinematical: it emerges from
bootstrap self-consistency (Eq.~\ref{eq:fixed-regime}), not from an external constraint.
\end{remark}

\vspace{0.3cm}
\subsection{The WESH master equation}

\vspace{0.5cm}

\noindent The constraints derived above—CPT symmetry, WESH–Noether conservation, 
and collective $N^2$ stability—uniquely determine the structure of the 
generator. Before writing it explicitly, we fix the integration conventions.

\paragraph{Measure conventions.}
We use coarse-grained integration measures normalized to the correlation four-volume:
\begin{equation}
\int_x := \frac{1}{V_\xi}\int d^4x,\qquad
\int_{xy}:=\frac{1}{V_\xi^2}\int d^4x\,d^4y,
\qquad V_\xi = \mathcal{O}(\xi^4).
\label{eq:measure-conventions}
\end{equation}
With these conventions, $\nu$ carries dimensions of inverse time; Prop.~\ref{prop:Gto0} uses unnormalized integrals, so $\nu$ and $\int d^4y\,\gamma(x,y)$ scale as $\mathcal{O}(\gamma_0 V_\xi)$ and $\mathcal{O}(\gamma_0 V_\xi/N^2)$, respectively.

\paragraph{Generator.}

\vspace{0.3cm}
\noindent The unique GKSL generator compatible with the constraints takes the form
\begin{empheq}[box={\setlength\fboxsep{14pt}\fbox}]{equation}
\begin{split}
\frac{d\rho}{ds} \;=\; &{-i[\hat{H}_{\mathrm{eff}}, \rho]}
\;+\; \nu \intx \mathcal{D}[\tilde{T}^2(x)]\,\rho \\[4pt]
&+\; \intxy \gamma(x,y)\,C(\Psi;x,y)\,\mathcal{D}[L_{xy}]\,\rho
\end{split}
\label{eq:wesh-master}
\end{empheq}
where $\tilde{T} = \hat{T}/\tau_s$ is the dimensionless field and
$\Vxi \sim \xi^4$ sets the coarse-graining four-volume. The effective
Hamiltonian splits as
\begin{equation}
\hat{H}_{\mathrm{eff}} = \hat{H}_{\mathrm{local}} + \hat{H}_{\mathrm{ent}},
\label{eq:split}
\end{equation}
with $\hat{H}_{\mathrm{local}}$ collecting matter and time-field
contributions and $\hat{H}_{\mathrm{ent}}$ the entanglement-mediated
correction.

\noindent The dissipative part uses the standard Lindblad superoperator
\begin{equation}
\mathcal{D}[O]\,\rho \equiv O\rho O^\dagger - \tfrac12\{O^\dagger O,\rho\},
\label{eq:superoperator}
\end{equation}
so that the local channel is generated by the Hermitian jump operator
$\tilde{T}^2(x)$,
\begin{equation}
\mathcal{D}[\tilde{T}^2(x)]\,\rho \;=\; \tilde{T}^2(x)\,\rho\,\tilde{T}^2(x)
\;-\; \tfrac12\{\tilde{T}^4(x),\rho\},
\label{eq:wesh-local-diss}
\end{equation}
and can be interpreted as the continuous counterpart of the conditional
update
\begin{equation}
\rho \ \longrightarrow \
\frac{\hat{T}^2(x)\,\rho\,\hat{T}^2(x)}{\Tr[\hat{T}^2(x)\rho\hat{T}^2(x)]}
\label{eq:cupdate}
\end{equation}
at the point $x$.

\paragraph{Causal kernel.}
The bilocal weight combines exponential falloff, causal support, and
collective suppression:
\begin{equation}
\gamma(x,y) = \frac{\gamma_0}{N^2}\, K_\xi(x-y)\, \Theta[\mathrm{causal}(x,y)],
\label{eq:gamma-kernel}
\end{equation}
with $K_\xi(z)$ a short-range profile of range $\xi \simeq L_P$
(e.g.\ $K_\xi(z) = e^{-d(z)/\xi}$), $\gamma_0 > 0$ a rate scale, and
$\Theta$ the causal step function vanishing for spacelike separation. The
$N^{-2}$ prefactor ensures that the sum over all pairs yields a total rate
of order $\gamma_0$, independent of system size.

\paragraph{Terminological note.}
Two kernels appear with distinct roles. The function $\gamma(x,y)$ in
Eq.~\eqref{eq:gamma-kernel} is the exponential--causal dissipative weight
entering the generator~\eqref{eq:wesh-master}. A Yukawa-type kernel
$K(x-y)$, introduced in Appendix~A, defines an entanglement potential
$\Phi(x)=\int d^4y\ K(x-y)\,C(\Psi;x,y)$ and enters the metric-emergence
analysis of Sec.~4. Both kernels have the same short range
$\xi\simeq L_P$, but $\gamma$ weights collapse rates whereas $K$ averages
correlations for the emergent geometry.

\paragraph{Entanglement gate.}
The bilocal channel is modulated by a normalized R\'{e}nyi-2 correlator,
\begin{equation}
C(\Psi;x,y) = \frac{\big[\Tr\rho_{xy}^{2} - \Tr\rho_{x}^{2}\,\Tr\rho_{y}^{2}\big]_+}
{1 - \Tr\rho_{x}^{2}\,\Tr\rho_{y}^{2}} \;\in\; [0,1],
\qquad [u]_+:=\max(u,0).
\label{eq:renyi-gate}
\end{equation}

\vspace{0.3cm}
\noindent When $\Tr\rho_x^2\,\Tr\rho_y^2=1$ (pure product marginals), we set $C=0$ by convention 
(no synchronization channel for uncorrelated pure states).

\vspace{0.3cm}
\noindent For comparison with standard connected correlators, one may also define
\begin{equation}
C_{\text{loc}}(\Psi;x,y)
= \langle \hat{f}(x)\hat{f}(y)\rangle_\Psi
 - \langle \hat{f}(x)\rangle_\Psi\,\langle \hat{f}(y)\rangle_\Psi,
\label{eq:local-correlator}
\end{equation}
for a local observable $\hat f(x)$. This quantity is used only as a
diagnostic; it never appears in the weights of
Eq.~\eqref{eq:wesh-master}. When $\Tr\rho_{xy}^{2} \ge \Tr\rho_{x}^{2}\,\Tr\rho_{y}^{2}$, the complement takes
the normalized deficit form
\begin{equation}
1 - C(\Psi;x,y)\;=\;\frac{1 - \Tr\rho_{xy}^{2}}
{1 - \Tr\rho_{x}^{2}\,\Tr\rho_{y}^{2}}.
\label{eq:normalized_dform}
\end{equation}
The $[\cdot]_+$ construction ensures $0\leq C\leq 1$ in all cases and yields $C=1$ whenever $\rho_{xy}$
is pure (non-factorized). When $x$ and $y$ are maximally entangled,
$C \to 1$ and the synchronization channel operates at full strength; for
product states, $C = 0$ and the points evolve independently. This
state-dependent weighting is the core of the Weak Entanglement Symmetry
Hypothesis: geometry emerges from the entanglement structure, not the
reverse.

\paragraph{Difference form.}
The bilocal Lindblad operator
\begin{equation}
L_{xy} = \tilde{T}^2(x) - \tilde{T}^2(y)
\label{eq:Lxy-def}
\end{equation}
is simply the rescaled version of the difference form
$L_{xy} = \hat{T}^2(x) - \hat{T}^2(y)$ of Eq.~\eqref{eq:difference_Lxy},
using the dimensionless field $\tilde{T}(x)=\hat{T}(x)/\tau_s$.
It penalizes temporal desynchronization between causally connected points. Its difference structure guarantees that no net charge is introduced or removed by the bilocal channel, 
a property that will later underpin the pre-geometric conservation principle.

\paragraph{No-signaling.}
Because $\gamma(x,y)$ vanishes for spacelike pairs, the bilocal generator has
causal support only. For any region $A\subset\Sigma$ with complement $A^c$,
the reduced state obeys
\begin{equation}
\frac{d}{ds}\rho_A(s)\;=\;\Tr_{A^c}\big(\mathcal L_{J(A)}[\rho(s)]\big),
\label{eq:no-signaling}
\end{equation}
where $\mathcal L_{J(A)}$ denotes the restriction of the generator to operators
supported in the causal domain $J(A)$. Hence spacelike operations on $A^c$
cannot instantaneously affect $\rho_A$ (Appendix~H).

 \noindent Causality is thus built into the generator itself. Here `causal' denotes 
pre-geometric admissibility of pairs enforced by the WESH kernel; at the 
alignment fixed point it coincides with metric causal separation.

\begin{remark}[Nonlinearity, CP/TP, and no-signaling]
\label{rem:nonlinear-cp}
Under the structural assumptions in Box~2, freezing the state-dependent
gate $C(\Psi;x,y)$ on each micro-interval yields a standard GKSL generator
with Hermitian jumps and nonnegative rates, so that each step of the WESH
evolution is completely positive, trace preserving, and causal (McCauley
et al., 2020). The state dependence of $C$ can then be viewed as inducing a
time-dependent generator; complete positivity and trace preservation for the
resulting non-autonomous dynamics follow from product-integral constructions
for time-dependent GKSL evolutions (Rivas \& Huelga, 2012). Appendix~H
provides the full operator-level proof of CP/TP and no-signaling for
Eq.~\eqref{eq:wesh-master}.
\end{remark}

\subsection{Bootstrap, Eigentimes, and the Emergence of Spacetime}
\label{sec:eigentime-bootstrap}

\vspace{0.3cm}
In this subsection we analyse how a directed temporal ordering, and subsequently spacetime geometry, may emerge once time is promoted to a local quantum field operator $\hat T(x)$ in the Wheeler--DeWitt sector. Spatial geometry, though not quantized directly, emerges from the alignment dynamics governed by entanglement correlations and the causal kernel, as developed in the Einstein–Hilbert derivation of Sec.~\ref{sec:gr-dynamic}. The construction must generate temporal order and geometric structure without reintroducing an external time parameter or any pre-assigned causal layering.

\vspace{0.3cm}

\noindent Throughout this subsection $\langle\cdot\rangle_{\rho(s)}$ denotes expectation in the state $\rho(s)$, $\Delta\hat T(x):=\hat T(x)-\langle\hat T(x)\rangle_{\rho(s)}$, and we set $\hbar=c=1$.

\paragraph{Architectural necessity.}
Once time is encoded by the operator $\hat T(x)$, it cannot simultaneously act as an external evolution parameter; introducing a separate clock would restart the hyper-time regress. Any emergent temporal order must therefore arise \emph{endogenously}, from the statistics and structure of collapse events of $\hat T(x)$ itself. 

A theory that aims to generate spacetime from a setting that, by necessity, lies \emph{prior to} geometric structure---and simultaneously excludes any external evolution parameter---must satisfy strict architectural constraints tied to the endogenous handling of information. We may therefore identify the following prohibitions:
\begin{itemize}[leftmargin=1.5em,itemsep=0.1em]
\item \emph{Sequential} constructions presuppose a pre-existing temporal ordering;
\item \emph{Hierarchical} constructions presuppose layered or stratified causality;
\item \emph{Iterative} algorithms presuppose discrete temporal steps.
\end{itemize}
The bootstrap is not a modeling choice but the \emph{only self-sufficient mathematical architecture} compatible with these constraints. This places our construction in continuity with earlier self-consistency programmes---Chew’s $S$-matrix bootstrap (Chew, 1961) and the conformal bootstrap of Rattazzi, Rychkov, Tonni, and Vichi (2008), where physical quantities are fixed by global consistency rather than external inputs. Here, the same philosophy governs \emph{chronogenesis} and the induced emergence of spacetime structure. In short, the bootstrap provides a self-consistent, atemporal configuration in which the Markovian dynamics and Poissonian stochasticity conspire to generate temporal order and geometric structure without any external temporal input or historical memory.

\paragraph{Bootstrap as an atemporal configuration.}
In WESH, the bootstrap is \emph{not} a temporal sequence. It is an atemporal, non-hierarchical configuration in which multiple elements co-determine one another, closer to a Nash-type equilibrium than to a causal chain. To illustrate (as \emph{mutual constraints}, not chronological stages): the quantum state fixes entanglement correlations $C(\Psi;x,y)$; these correlations weight dissipative channels; stochastic eigentime events realize localized outcomes of the time field; the concentration of such events constitutes the emergent spacetime arena in which the state is defined. No element is ``first''; each requires the others for consistency.

This dissolves apparent circularities. As in the Page--Wootters mechanism (Page \& Wootters, 1983), where a timeless global state encodes relational temporal correlations, and as in Wheeler--DeWitt, which admits no ``before'' once time emerges, here there is no ``before the first eigentime'' because eigentimes \emph{constitute} the temporal structure. Questions that presuppose a pre-existing sequence are simply ill-posed. Combined with truly stochastic localization (which carries no temporal memory) and with the causal kernel $\gamma(x,y)$ enforcing a finite memory window $\tau_{\mathrm{corr}}\ll\tau_{\mathrm{Eig}}$, this yields the \emph{minimal sufficient} pre-geometric framework: the bootstrap eliminates external temporal scaffolding; Poissonian statistics eliminate dynamical memory; the short-range causal kernel coordinates the process without long-term history dependence.

\paragraph{Eigentimes as time quanta and geometric seeds.}
The basic quanta of this construction are \emph{Eigentimes}: stochastic localizations of the time field at definite outcomes of the spectral measure of $\hat T(x)$. Let $E_x(dt)$ be the spectral measure of $\hat T(x)$, so that
\begin{equation}
\hat T(x) = \int_{\mathbb R} t\,E_x(dt),
\qquad
\mathbb 1_x = \int_{\mathbb R} E_x(dt).
\end{equation}
For a local time-sector state $|\psi_T(x)\rangle$, the spectral expansion reads
\begin{equation}
|\psi_T(x)\rangle = \int_{\mathbb R} E_x(dt)\,|\psi_T(x)\rangle,
\end{equation}
and an eigentime event in a Borel window $\Delta\subset\mathbb R$ is modeled operationally by the localization
\begin{equation}
|\psi_T(x)\rangle \;\longrightarrow\;
\frac{E_x(\Delta)\,|\psi_T(x)\rangle}
{\sqrt{\langle\psi_T(x)|E_x(\Delta)|\psi_T(x)\rangle}},
\quad
p_x(\Delta)=\langle\psi_T(x)|E_x(\Delta)|\psi_T(x)\rangle.
\label{eq:eigentime-collapse}
\end{equation}
Eigentimes are therefore \emph{time quanta}: discrete stochastic realizations of the continuous spectrum of $\hat T(x)$, not quanta of space. Spatial geometry is induced indirectly: the pattern of eigentime events, weighted by entanglement via $C(\Psi;x,y)$ and correlated by the causal kernel $\gamma(x,y)$, seeds the smooth spacetime geometry reconstructed in Sec.~\ref{sec:gr-dynamic}. Regions where eigentimes concentrate densely support a well-defined geometric structure; sparsity leaves geometry weakly defined.

\noindent The mean production rate of eigentimes along the auxiliary flow is
\begin{equation}
\begin{split}
\Big\langle\frac{dN_{\mathrm{Eig}}}{ds}\Big\rangle
&= \frac{\Gamma[\Psi(s)]}{\tau_{\rm Eig}} \\[6pt]
&= \nu \intx \Tr\!\big[\tilde{T}^2(x)\,\rho(s)\,\tilde{T}^2(x)\big] \\
&\quad + \intxy \gamma(x,y)\,C(\Psi;x,y)\,\Tr\!\big[L_{xy}^\dagger L_{xy}\,\rho(s)\big],
\end{split}
\label{eq:eigrate}
\end{equation}
with $\nu>0$ a state-independent intensity scale. The local hazard---the instantaneous propensity at point $x$---satisfies
\begin{equation}
\lambda(x)=\nu\,\Tr\!\big[\tilde{T}^2(x)\,\rho\,\tilde{T}^2(x)\big]
\ \ge\ \nu\,\big(\Var_\rho[\tilde{T}(x)]\big)^2,
\label{eq:local-hazard}
\end{equation}
(Cauchy--Schwarz). Since $\tilde{T}(x)=\hat{T}(x)/\tau_s$ shares the same
spectral projectors as $\hat{T}(x)$, the activation condition can equivalently be
expressed in terms of the physical variance of $\hat{T}(x)$:
\begin{equation}
\big\langle(\Delta\hat T(x_0))^2\big\rangle_{\rho(s)}>\tau_{\mathrm{thr}}^{\,2}
\ \Longrightarrow\ \text{eigentime at }x_0.
\label{eq:locthreshold}
\end{equation}
Two scales control the statistics:
\begin{equation}
\tau_{\rm Eig} \equiv \nu^{-1},\qquad
\tau_{\mathrm{ref}}(x)\;\sim\;\frac{1}{\nu}\,
\frac{1}{\big(\Var_{\rho}[\tilde{T}(x)]\big)^{2}}.
\label{eq:eigscale}
\end{equation}
Here $\tau_{\rm Eig}$ is the mean spacing in the homogeneous limit, whereas $\tau_{\mathrm{ref}}(x)$ acts as a local refractory period after strong localization.

\paragraph{Time-production functional.}
The relation between the auxiliary coordinate $s$ (a reparametrization-invariant bookkeeping parameter; see Appendix~B) and emergent time $t$ is fixed by
\begin{equation}
\frac{dt}{ds} = \Gamma[\Psi], \qquad
t(s) = \int_0^s \Gamma[\Psi(s')]\, ds',
\label{eq:s-to-t-bootstrap}
\end{equation}
\noindent The lower integration bound is a convention fixing the origin of the auxiliary
parameter $s$; physical statements depend only on differences
$t(s_2)-t(s_1)=\int_{s_1}^{s_2}\Gamma[\Psi(s')]\,ds'$ and are invariant under $s$-translations.

\noindent The functional $\Gamma[\Psi]$ is given by
\begin{equation}
\begin{split}
\Gamma[\Psi] = \tau_{\mathrm{Eig}} \Bigl[
&\nu \intx \Tr\bigl[\tilde{T}^2(x)\,\rho\,\tilde{T}^2(x)\bigr] \\[4pt]
&+ \intxy \gamma(x,y)\,C(\Psi;x,y)\,\Tr\bigl[L_{xy}^\dagger L_{xy}\,\rho\bigr]
\Bigr].
\end{split}
\label{eq:gamma-decomposition}
\end{equation}

The prefactor $\tau_{\mathrm{Eig}}=\nu^{-1}$ renders $\Gamma$ dimensionless; by the law of large numbers, $t(s)\approx \tau_{\mathrm{Eig}}N_{\mathrm{Eig}}(s)$. For later use we isolate the bilocal term:
\begin{equation}
\Gamma_{\mathrm{bi}}[\Psi(s)] \;=\; 
\intxy \gamma(x,y)\,C(\Psi;x,y)\,
\big\langle\,(\tilde{T}^2(x)-\tilde{T}^2(y))^2\,\big\rangle_{\rho(s)}.
\label{eq:Gamma-bilocal}
\end{equation}

\begin{lemma}[Monotonicity of $t$]
\label{lem:Gamma-positive}
Let $\rho(s)\!\ge\!0$ with $\Tr\,\rho(s)=1$. Assume time-sector
nondegeneracy (ND): either $\exists\,x_0$ with
$\Tr[\tilde{T}^2(x_0)\,\rho(s)\,\tilde{T}^2(x_0)]>0$, or $\exists\,(x,y)$ with
$C[\rho(s);x,y]>0$ and
$\Tr[\rho(s)\,( \tilde{T}^2(x)-\tilde{T}^2(y) )^2]>0$. With $\nu>0$,
$\gamma(x,y)\ge 0$, and $0\le C[\rho(s);x,y]\le 1$, one has
$\Gamma[\rho(s)]\ge 0$, with $\Gamma>0$ under (ND); hence $dt/ds=\Gamma[\rho(s)]\ge 0$ and $t(s)$ is strictly
increasing under (ND).
\end{lemma}

\begin{lemma}[Eigentime activation on chronogenetic segments]
\label{lem:eigentime-activation}
Under the standing integrability assumptions on $\hat T(x)$ (finite fourth
moments), assume nondegeneracy (ND) holds along a segment of the $s$-flow, so that
$\Gamma[\rho(s)]>0$ on that segment (Lemma~\ref{lem:Gamma-positive}).
Define the survival probability of having no eigentime event on a segment of length
$\delta>0$ by
\[
S_{s_0}(\delta):=\exp\!\left(-\int_{s_0}^{s_0+\delta}\frac{\Gamma[\rho(s')]}{\tau_{\rm Eig}}\,ds'\right).
\]
Then $S_{s_0}(\delta)<1$ on any chronogenetic segment (equivalently: eigentime events have
non-zero intensity on every such segment). No segment is privileged: the statement is
$s$-translation invariant.
\end{lemma}

\begin{lemma}[Contraction to equilibrium and $\alpha=2$]
\label{lem:contraction}
Let $\Phi_{s,\delta s}$ be the update--mix map over $[s,s+\delta s]$. In
the Markov regime $\mu=\tau_{\mathrm{corr}}/\tau_{\rm Eig}\ll 1$ there exist
$L\gg\xi$ and $\varepsilon=\Theta(\mu)>0$ such that
\[
\|\Phi_{s,\delta s}(\rho)-\Phi_{s,\delta s}(\sigma)\|_1
\le (1-\varepsilon)\,\|\rho-\sigma\|_1,\quad\forall\,\rho,\sigma.
\]
\noindent
\textbf{(Uniqueness and mixing)}
Since the state space is complete in trace norm, the strict contraction above implies
(by the Banach fixed-point theorem) a \emph{unique} fixed point $\rho_\infty$ of $\Phi_{s,\delta s}$,
and exponential convergence of iterates:
\[
\|\Phi_{s,\delta s}^{\,n}(\rho)-\rho_\infty\|_1 \le (1-\varepsilon)^n\|\rho-\rho_\infty\|_1\,,\qquad
\|\Phi_{s,\delta s}^{\,n}(\rho)-\Phi_{s,\delta s}^{\,n}(\sigma)\|_1 \le (1-\varepsilon)^n\|\rho-\sigma\|_1.
\]
This uniqueness/mixing mechanism is \emph{purely Markovian} (Dobrushin-type) and does \emph{not}
rely on any KMS/detailed-balance/thermal structure. Appendix~D provides an existence proof
via Schauder--Tychonoff under constraints even without a contraction estimate; Sec.~\ref{sec:KMS-Rindler}
gives an \emph{independent} (specialized) uniqueness route via primitivity and KMS spectral gap.
With $\varepsilon=\Theta(\mu)$ and the collective-stability law
$\tau_{\mathrm{coh}}(N)\propto N^\alpha$, the fixed-point balance
$\mu\,\sigma_\alpha=\mathcal{O}(1)$ uniquely selects $\alpha=2$.
\end{lemma}

\begin{proof}[Sketch]
Finite interaction range $\xi$ and the $N^{-2}$ normalization imply bounded per-site influence
$\mathcal{O}(\mu)$ on blocks $L\gg\xi$. Standard Dobrushin-type arguments for finite-range
Markov mixing then yield trace-norm contraction with $\varepsilon=\Theta(\mu)$, and the
Banach fixed-point theorem gives uniqueness and exponential mixing.
An independent uniqueness proof in the KMS/detailed-balance specialization is given in
Sec.~\ref{sec:KMS-Rindler}.
\end{proof}

\paragraph{Self-consistent closure.}
Lemma~\ref{lem:Gamma-positive} ensures $\Gamma[\rho]\ge 0$, with $\Gamma>0$ under nondegeneracy (ND), so $t(s)$ is non-decreasing in general and strictly increasing under (ND), defining an intrinsic arrow of time. No external clock is introduced at any stage: the statistics of eigentimes, constrained by WESH--Noether conservation and collective stability in the Markov window, fully determine the emergent temporal order. Markovianity, Poissonian statistics, and the atemporal bootstrap architecture thus form a single logical package: the evolution depends only on the present state, carries no temporal memory, and is closed under self-consistency. The foundational circularity (WESH derived from Wheeler--DeWitt by constraints,
and the WESH dynamics contracting back to the frozen Wheeler--DeWitt constraint
in the $G\!\to\!0$ limit) is not a defect but a signature of self-consistency
in a pre-geometric theory: time, geometry, and dynamics co-emerge from the same
WESH completion of the timeless constraint.

\vspace{0.5cm}

\subsection{Foundational consistency and parameter fixing}

\vspace{0.3cm}
A necessary check on any dissipative extension of Wheeler--DeWitt is the behavior of the theory in two limiting regimes: the decoupling of gravitational effects and the thermodynamic scaling at horizons.

\paragraph{Finite memory (Born--Markov).}
\begin{equation}
\int_0^\infty dt\,\|\langle \mathcal{C}(t)\mathcal{C}(0)\rangle\| < \infty,
\qquad 
\tau_{\mathrm{corr}}=\xi/c \ll \tau_{\rm Eig}\equiv 1/\nu. 
\label{eq:finite-memory}
\end{equation}

\noindent\emph{Note.} We use $\mathcal{C}(t)$ for the memory kernel to avoid
clashes with the entanglement gate $C(\Psi;x,y)$.

\paragraph{Compatibility with $N^2$ stability (ratio scaling).}
\begin{equation}
\frac{\tau_{\mathrm{corr}}}{\tau_{\rm Eig}^{(\mathrm{eff})}} \sim N^{-2}
\quad\text{(bilocal }\gamma\propto N^{-2},\ \
\tau_{\rm Eig}^{(\mathrm{eff})}\propto N^2\text{).} 
\label{eq:ratio-scaling}
\end{equation}

\paragraph{Fundamental scales.}
The fundamental scales of the theory are anchored to Planck units:
\begin{equation}
\xi = \frac{\hbar}{m_T c} \simeq L_P, \qquad
\gamma_0 \simeq t_P^{-1}.
\label{eq:fundamental-scales}
\end{equation}

\noindent\textit{Terminology.} $m_T$ fixes the range $\xi$ of the kernel used
in $\gamma(x,y)$; it is not an on-shell degree of freedom.

\begin{proposition}[Decoupling in the $G\!\to\!0$ limit]
\label{prop:Gto0}
Let $\xi\simeq L_P\propto\sqrt{G}$ and
$\gamma_0\simeq t_P^{-1}\propto G^{-1/2}$, and write
\[
\gamma(x,y)=\frac{\gamma_0}{N^2}\,K_\xi(x-y)\,
\Theta[\mathrm{causal}(x,y)],\qquad
K_\xi\ge0,\quad \|K_\xi\|_{L^1(\mathbb{R}^{1,3})}\sim V_\xi=\mathcal O(\xi^4) .
\]
Define the shorthand
\[
\int\!\gamma \;\equiv\; \int d^4y\,\gamma(x,y).
\]
Then
\[
\nu\propto \gamma_0\,V_\xi \to 0,\qquad
\int\!\gamma=\frac{\gamma_0}{N^2}\,\|K_\xi\|_{L^1}
\sim \frac{\gamma_0}{N^2}\,V_\xi \to 0,
\]
so for any bounded operator $A$,
\[
\Big\|\tfrac{d}{ds}\Tr(A\rho)\Big\|_{\mathrm{diss}}
\;\le\; C_A\big(\nu+\!\int\!\gamma\big)\;\to\;0.
\]
\end{proposition}

\noindent\emph{Proof sketch.} $\|K_\xi\|_{L^1}\sim c\,\xi^4$ yields
$\nu,\int\gamma=\mathcal{O}(\gamma_0\xi^4)\to 0$; standard Lindblad bounds
give $\|\mathcal{D}\|\le \mathrm{const}\cdot(\nu+\int\gamma)\to 0$, hence
strong convergence to the unitary group.

\paragraph{Constraint continuity (The Frozen Limit).}
Since the fundamental scales satisfy $\xi \sim L_P \propto \sqrt{G}$ and 
$\gamma_0 \sim t_P^{-1}$, taking the formal limit $G\to 0$ implies 
$\nu\to 0$ and $\int\gamma\to 0$. In this limit, the dynamics undergoes 
a structural contraction:
\begin{enumerate}[label=(\roman*),leftmargin=1.5em]
\item \textbf{Vanishing dissipation.} The non-unitary terms in the master 
equation scale to zero, leaving a unitary flow generated solely by 
$\hat{H}_{\mathrm{eff}}$.
\item \textbf{Halting of temporal flow.} Crucially, the eigentime event
rate $\big\langle dN_{\rm Eig}/ds\big\rangle$ (Eq.~\ref{eq:eigrate})
vanishes as $\nu\to 0$ and $\int\gamma\to 0$ (Prop.~\ref{prop:Gto0}):
no eigentimes occur, and the arrow of time ceases to emerge.
\end{enumerate}

\paragraph{Physical interpretation.}
The limit $G\to 0$ enables us to recognize WESH and Wheeler--DeWitt as 
two regimes of the same fundamental constraint. When gravitational coupling 
is active, the constraint unfolds into a dissipative, dynamical structure 
that generates temporal flow; when gravity is turned off, both the production 
of time and the emergence of spacetime geometry halt simultaneously, and the 
theory continuously contracts back to the frozen, timeless sector 
$\hat{H}_{\mathrm{tot}}|\Psi\rangle=0$. 

Physical time and Einstein gravity emerge as complementary aspects of a 
single dissipative resolution of the Wheeler--DeWitt constraint: time arises 
through the eigentime bootstrap ($dt/ds=\Gamma>0$), while spacetime geometry 
crystallizes through gradient alignment and hidden-sector cancellation, which 
together constitute the unique infrared attractor of the WESH flow 
(Sec.~\ref{sec:gr-dynamic}). Neither structure is logically prior; both 
derive from the same pre-geometric algebraic constraints---CPT invariance, 
WESH--Noether charge conservation, and collective stability---without 
postulating a background arena.

\paragraph{Fixed-regime summary.}
Combining the $N^{-2}$ kernel scaling with the Markov condition
$\tau_{\mathrm{corr}} \ll \tau_{\mathrm{Eig}}$ yields a self-consistent
fixed point:
\begin{equation}
\boxed{\;
\big(\tau_{\mathrm{coh}}(N)\propto N^2\big)\ \&\
\big(\mu:=\tfrac{\tau_{\mathrm{corr}}}{\tau_{\rm Eig}}\ll 1\big)
\ \Longleftrightarrow\ \alpha=2
\;}
\label{eq:fixed-regime}
\end{equation}
The bootstrap mechanism should be understood as a self-consistency requirement 
that constrains the dynamics of the theory, analogous to closure conditions 
successfully employed in conformal field theory to fix operator dimensions 
(Rattazzi et al., 2008). It is not a circular definition, but rather a 
dynamical equilibrium condition: the parameters that govern dissipation 
($\gamma_0$, $\xi$) and the resulting statistical properties of eigentime 
events ($\nu$, $\tau_{\mathrm{coh}}$) must mutually adjust to maintain 
Markovian memory within a collective $N^2$ stability window.

\begin{remark}
The scaling statement in Eq.~\eqref{eq:fixed-regime} is encoded by the
bilocal part of the WESH generator \eqref{eq:wesh-master} with causal kernel
\eqref{eq:gamma-kernel}: the $N^{-2}$ prefactor provides collective
protection (per-site rate $\Gamma_i\!\sim\!\gamma_0/N$), while the finite
correlation length $\xi$ fixes a state-independent correlation time
$\tau_{\mathrm{corr}}\!\sim\!\xi/c$. Together these imply the Markov window
$\mu:=\tau_{\mathrm{corr}}/\tau_{\rm Eig}\ll 1$ in the large-$N$ regime
(see the heuristics in Appendix~E).
\end{remark}

\paragraph{Preview: thermodynamic validation.}
The same dissipative structure, applied to the near-horizon region of a
black hole, reproduces the Bekenstein--Hawking entropy
$S = A/4L_P^2$ with calculable logarithmic corrections (Sec.~6). This
cross-scale consistency---from Planck-scale dynamics to macroscopic
thermodynamics---provides a non-trivial validation of the framework.

\vspace{0.5cm}

\subsection{Foundational Considerations}
\label{sec:structural-necessity}

\vspace{0.3cm}
\noindent We conclude this foundational exposition by highlighting the strict parsimony of the construction. The transition from the static Wheeler--DeWitt constraint to the dynamic WESH evolution is not an ad hoc extension, but establishes a rigorous lineage from the first principles of the quantum universe: the WESH dynamics is derived directly from the Wheeler--DeWitt constraint as its necessary dissipative realization. This unfolding into a causal history is achieved without expanding the fundamental physics: the theory posits no extra spatial dimensions, no supersymmetric partners, no discrete spacetime atoms, nor ad hoc "chronon" particles. Instead, the framework relies exclusively on established theoretical components: the kinematics of quantum theory, the canonical constraints of General Relativity, and the structure of quantum entanglement. The novelty resides entirely in the architectural synthesis, which rests on two non-trivial structural keystones:\begin{itemize}[leftmargin=1.5em,itemsep=0.1em]\item The pre-geometric algebraic foundation of conservation, where WESH--Noether symmetry is enforced at the operator level prior to the emergence of any background metric (essential for deriving the Einstein tensor in Sec.~\ref{sec:gr-dynamic} and the Bekenstein--Hawking law in Sec.~\ref{sec:KMS-Rindler});\item The bootstrap closure, which replaces linear evolution with a self-consistent cycle where the geometry necessary to define the field $\hat{T}$ is generated by the field's own dissipative activity.\end{itemize}In this architecture, entanglement plays a dual, constitutive role: it acts simultaneously as the carrier of statistical symmetry and as the engine of Poissonian stochastic dissipation. The parameters appearing in the final structure ($\nu$, $k$, $\alpha$) are therefore not free couplings, but structural constants locked by the stability of this interdependence. In this sense, QFTT--WESH represents a \emph{minimal sufficiency} approach to quantum gravity: it demonstrates that the ingredients necessary to resolve the frozen-time paradox are already present in quantum mechanics and relativity, but their consistent integration demands a methodological shift, from unitary linearity to a bootstrap-closed, algebraic dissipative dynamics, capable of accessing solutions (chronogenesis, hidden-sector cancellation) that remain invisible to standard background-dependent approaches.

\vspace{1cm}

\begin{tcolorbox}[title={\centering Box 3: Beyond Detailed Balance -- Pre-geometric Consistency}]
\noindent
In standard open quantum systems, convergence to a stationary state is often
tied to \emph{Detailed Balance} (DB) with respect to a fixed thermal state, a
condition that presupposes a background time and an externally given
equilibrium ensemble. In QFTT--WESH we are in a pre-temporal, pre-geometric
regime: such structures are not available, and DB cannot be imposed as an
input.

\vspace{0.3em}
\noindent\textbf{Pre-temporal constraints (in $s$).} The fundamental
constraints are algebraic, not thermal.
(i) \emph{WESH--Noether:} the generator exactly preserves all global charges
at the operator level (path independence; App.~G). In particular,
\begin{equation}
\mathcal L^\dagger[\hat Q_a]=0 \quad \forall a .
\label{eq:H-noether}
\end{equation}
(ii) \emph{Causal support:} the bilocal kernel has light-cone support and
forbids superluminal signaling (Eq.~\eqref{eq:no-signaling}). 
(iii) \emph{Positivity/trace preservation \& finite memory:} the flow preserves 
positivity and unit trace (stepwise CPTP) within a finite Markov window (Eq.~\eqref{eq:finite-memory}).

\vspace{0.3em}
\noindent\textbf{Emergence.} Under (i)--(iii) and mixing (contraction of the
WESH update--mix map on coarse-grained windows; App.~D), the pre-temporal GKSL
flow admits a unique stationary state. Detailed Balance is not
assumed at this level: it reappears later as a KMS/detailed-balance
property of the emergent near-horizon equilibrium (Sec.~\ref{sec:KMS-Rindler}),
once physical time and geometry have formed.

\vspace{0.3em}
\noindent\textbf{Implication.} Placing algebraic conservation first, 
and thermodynamic notions only at the emergent level,
removes circularity and keeps the pre-geometric
generator \eqref{eq:wesh-master} free of thermal priors.
\end{tcolorbox}

\vspace{1cm}

\begin{table}[H]
\centering
\caption{\textbf{Summary of fundamental parameters and scales in the QFTT--WESH formalism.} All quantities use natural units with $\hbar=c=1$ where applicable.}
\label{tab:parameters}
\footnotesize
\setlength{\tabcolsep}{4pt}
\renewcommand{\arraystretch}{1.10}
\begin{tabularx}{\linewidth}{l Y Y Y l}
\toprule
\textbf{Symbol} & \textbf{Meaning} & \textbf{Dimensions} & \textbf{Scale / Definition} & \textbf{Eq.} \\
\midrule
$\xi$ &  Corr.\ length & [length] & Planck ($\simeq L_P$) & \eqref{eq:fundamental-scales}, \eqref{eq:gamma-kernel} \\
$\gamma_0$  & WESH rate scale & [time]$^{-1}$ &
$\simeq t_P^{-1}\!\times\!\Theta_0$ &
\eqref{eq:fundamental-scales}, (A.6) \\
$N_{\mathrm{Eig}}$ & eigentime count & --- &
via Eq.~\eqref{eq:eigrate} &
\eqref{eq:eigrate}, \eqref{eq:s-to-t-bootstrap} \\
$\tau_{\rm Eig}$  & Mean eigentime spacing & [time] & $1/\nu$ &
\eqref{eq:eigscale} \\
$\nu$ & Local eigentime rate & [time]$^{-1}$ &
matched (no explicit $N$) &
\eqref{eq:eigrate}, \eqref{eq:nu-matching} \\
$\tau_{\mathrm{corr}}$ & Correlation time & [time] & $\xi/c$ &
\eqref{eq:finite-memory} \\
$\tau_{\mathrm{coh}}$ & Coherence time & [time] & $\propto N^{2}$ &
\eqref{eq:fixed-regime} \\
$\alpha$ & Stability exponent & --- & $2$ &
\eqref{eq:fixed-regime} \\
$\Gamma[\Psi]$ & Time-production functional & --- &
$\ge 0$ & \eqref{eq:s-to-t-bootstrap}, \eqref{eq:gamma-decomposition} \\
$C(\Psi;x,y)$ & Entanglement corr.\ & --- & $[0,1]$ &
\eqref{eq:renyi-gate} \\
$L_{xy}$ & Bilocal jump operator & --- & $\tilde{T}^2(x)-\tilde{T}^2(y)$ &
\eqref{eq:difference_Lxy}, \eqref{eq:Lxy-def} \\
$m_T$ & Kernel mass scale & [mass] & $\propto 1/\xi$ &
\eqref{eq:fundamental-scales} \\
$k$ & Alignment factor & [time] & via $\lambda_{1,2}$ matching &
\eqref{eq:stationary-alignment}, \eqref{eq:parameter-relations} \\
\bottomrule
\end{tabularx}
\vspace{0.3cm}
\parbox{\linewidth}{\footnotesize 
\noindent\emph{Units Note.} With $\Ttil=\hat T/\tau_s$ and normalized measures
$\intx$ and $\intxy$, each additive term in the generator carries units
of inverse time. With $\|K_\xi\|_{L^1(\mathbb{R}^{1,3})}\sim V_\xi=\mathcal O(\xi^4)$,
the integrated bilocal weight scales as 
\[
\int\!\gamma=\frac{\gamma_0}{N^2}\,\|K_\xi\|_{L^1}\sim\frac{\gamma_0}{N^2}\,V_\xi.
\] Using the
normalized measures ensures consistent dimensions in Eq.~\eqref{eq:wesh-master}. 
The time-production functional $\Gamma[\Psi]$ is rendered dimensionless by the prefactor
$\tau_{\rm Eig}=1/\nu$ in Eq.~\eqref{eq:gamma-decomposition}, ensuring that
the map $dt/ds=\Gamma[\Psi]$ is dimensionally consistent.
}
\end{table}

\vspace{1cm}

\section{Key Theoretical Predictions and Conservation Laws}
\label{sec:theoretical-predictions}

\vspace{0.5cm}

\textit{Overview. Building on the WESH--Noether structure developed in Sec.~1, we now isolate two falsifiable signatures of the framework: collective stability in the $N$--scaling of coherence, and a $\cos^2\theta$ angular law in parity decay. Both follow directly from the WESH master equation and its causal kernel, and they provide the quantitative targets for the simulations and hardware tests of Sec.~\ref{sec:experimental}.}

\vspace{1cm}


\subsection{Collective stability and $N$-scaling}
\vspace{0.3cm}

The fundamental signature of WESH dynamics is the suppression of pairwise dissipative couplings as the system size grows. Unlike standard decoherence, where independent local errors accumulate linearly, the WESH interaction strength scales inversely with the square of the number of constituents:
\begin{equation}
\gamma_{ij}(N) \;=\; \frac{\gamma_0}{N^2}\,,\qquad i\neq j. 
\label{eq:cs-scaling}
\end{equation}
This $N^{-2}$ scaling is not arbitrary but required to maintain the total collapse power finite in the thermodynamic limit ($\sum \gamma_{ij} \sim \mathcal{O}(\gamma_0)$), ensuring that the effective rate per subsystem decreases as $\Gamma_i \sim 1/N$.

Consequently, we predict a \textbf{collective enhancement of coherence time} that defies standard local noise models:
\begin{equation}
\tau_{\mathrm{coh}}(N)\;\propto\;N^2. 
\label{eq:coherence-enhancement}
\end{equation}
This quadratic protection is valid in the Markovian regime where the correlation time $\tau_{\mathrm{corr}}$ is small compared to the eigentime scale $\tau_{\rm Eig}$ (see Appendix~E for the heuristic derivation).
\vspace{0.5cm}

\subsection{Angular dependence: projection $\to$ geometry $\to$ state}
\label{subsec:ang-bridge}
\vspace{0.3cm}

The second falsifiable signature connects the geometry of the detector to the quantum state. The decay rate of parity oscillations is predicted to follow a robust angular modulation:
\begin{equation}
\Gamma_{\mathrm{dec}}(\theta)=\overline\Gamma\big(1+\varepsilon\cos^2\theta\big),
\qquad
\varepsilon=\frac{a_2}{a_0}\,\frac{G_2}{G_0}.
\label{eq:angular-law}
\end{equation}
Here, $\varepsilon$ factorizes into an intrinsic state anisotropy ($a_2/a_0$) and a device-specific geometric factor ($G_2/G_0$). Intuitively, the rotated‑parity measurement projects two‑body contributions onto the $\cos^2\theta$ harmonic; the bilocal WESH channel then aggregates these pairwise projections over the chip geometry, while the prepared state supplies the intrinsic quadrupolar bias. Importantly, this allows us to distinguish WESH effects from isotropic background noise:
\begin{itemize}
    \item \textbf{GHZ states:} exhibit positive modulation ($\varepsilon > 0$).
    \item \textbf{W-states:} are expected to show anti-modulation ($\varepsilon < 0$), providing a crucial control for state dependence.
\end{itemize}

\vspace{0.8cm}

\section{Experimental Validation}  
\label{sec:experimental}
\setcounter{figure}{0}
\renewcommand{\thefigure}{3.\arabic{figure}}

\vspace{0.5cm}
We test these predictions through a dual approach combining controlled classical simulations and quantum hardware experiments.

\vspace{0.3cm}
\noindent First, we employ a pre-asymptotic collision model on CPUs to isolate the collective stability mechanism. This model deliberately implements inverse couplings in a tractable regime ($N \leq 16$) to verify the scaling laws without the overhead of full quantum simulation.

\vspace{0.3cm}
\noindent Second, we execute production runs on superconducting quantum processors (IBM Eagle and Rigetti Ankaa-3). These experiments probe the angular dependence and protection gaps in a physical setting, leveraging large-scale statistics ($\sim 3 \times 10^6$ shots) to extract signals from the NISQ noise floor.
\vspace{0.3cm}

\noindent All datasets and analysis scripts are available in the companion repository:\\
\href{https://github.com/Luca-Casagrande/QFTT--WESH}{\texttt{https://github.com/Luca-Casagrande/QFTT--WESH}}.
\vspace{0.5cm}

\subsection{Numerical Evidence: Collective Stability}
\vspace{0.3cm}

The collision model simulations are consistent with the non-standard scaling predicted by Eq.~\eqref{eq:cs-scaling}. As shown in Figure~\ref{fig:3-1}, the coherence time for the WESH model grows with system size ($\alpha_{\rm coh} \approx 2.5$), in stark contrast to the standard local decoherence model where coherence remains flat or degrades ($\alpha_{\rm coh} \approx 0$).

\begin{figure}[H]
\centering
\includegraphics[width=\textwidth]{Picture5.png}  
\caption{\textbf{Purity and coherence scaling ($N=3\text{--}9$).}
Comparison of WESH (collective jump $L \propto T/N$), Standard (local jumps $\sigma_x$), and Pure WDW baselines. 
\textbf{Left:} Time traces at $N=9$. WESH maintains significantly higher purity and coherence.
\textbf{Right:} Log-log scaling. WESH coherence grows super-linearly ($\alpha \approx 2.5$), supporting the collective stability hypothesis. Standard decoherence shows no such enhancement.}
\label{fig:3-1}
\end{figure}

\noindent The robustness of this scaling is further tested under different coupling profiles (Figure~\ref{fig:3-2}). Only inverse-scaling profiles ($1/N, 1/N^2$) yield a negative loss exponent $b$, confirming that collective suppression is necessary for stability.

\begin{figure}[H]
\centering
\includegraphics[width=\textwidth]{Picture6.png}
\caption{\textbf{Robustness of protection.} Final coherence $C(N)$ for various noise channels and coupling profiles. Only inverse scaling ($1/N^k$) enables coherence preservation (rising trends), whereas constant coupling leads to rapid decay (flat/dropping trends).}
\label{fig:3-2}
\end{figure}

\noindent Extending the analysis to $N=16$ (Figure~\ref{fig:3-3}), we extract a rate scaling exponent $\alpha \approx -1.80$. This value approaches the theoretical prediction of $-2$ and is clearly distinguishable from the $\alpha = -1$ scaling of standard local control models (Figure~\ref{fig:3-4}).

\begin{figure}[H]
\centering
\includegraphics[width=\textwidth]{Picture7.png}
\caption{\textbf{Pre-asymptotic rate scaling.} Log-log regression yields $\gamma(N) \propto N^{-1.80}$, consistent with an approach to the asymptotic $N^{-2}$ WESH prediction.}
\label{fig:3-3}
\end{figure}

\begin{figure}[H]
\centering
\includegraphics[width=\textwidth]{Picture7a.png}
\caption{\textbf{Local baseline check.} The same analysis on a local first-order model yields $\alpha = -1.00$ exactly, validating the sensitivity of the metric.}
\label{fig:3-4}
\end{figure}

\vspace{0.8cm}
\subsection{Hardware Evidence: Angular Law and Protection}

On quantum hardware, we focus on the angular signature (Eq.~\eqref{eq:angular-law}). Experiments on IBM Eagle ($N=3,4$) reveal a clear $\cos^2\theta$ modulation of the parity decay rate. Figure~\ref{fig:3-5} shows a high-quality fit ($R^2=0.947$) with modulation amplitude $\varepsilon \approx 0.31$. Crucially, preparing a W-state reverses the sign of the effect (anti-modulation), confirming the state-dependence predicted by the theory.

\begin{figure}[H]
\centering
\includegraphics[width=\textwidth]{Picture8.png}
\caption{\textbf{Angular dependence on IBM Eagle.} 
\textbf{Top Left:} Decay rate modulation follows the predicted $\cos^2\theta$ law ($R^2=0.947$).
\textbf{Top Right:} W-state control shows the predicted sign reversal (anti-modulation).}
\label{fig:3-5}
\end{figure}

\noindent To rule out gate-overhead artifacts, we performed a controlled comparison (Figure~\ref{fig:3-6}) between GHZ states and "fake-GHZ" states (entanglement created and immediately undone). The GHZ state retains significantly higher parity, with a decay rate ratio $\Gamma_{\rm PROD}/\Gamma_{\rm GHZ} \approx 2.6$, far exceeding the factor attributable to gate depth alone ($\sim 1.2$).

\begin{figure}[H]
\centering
\includegraphics[width=\textwidth]{Picture9.png}
\caption{\textbf{Gate-matched control.} The "Fake-GHZ" (disentangled) behaves like the product state, while the true GHZ state maintains high parity. This isolates entanglement as the protective resource.}
\label{fig:3-6}
\end{figure}

\noindent Comprehensive data from 1.9M shots (Figure~\ref{fig:3-7}) confirms this separation across the frequency spectrum. The distribution of parity outcomes for GHZ states is bimodally separated from product states (Figure~\ref{fig:3-8}), providing model-independent evidence of entanglement-enhanced stability.

\begin{figure}[H]
\centering
\includegraphics[width=\textwidth]{Picture10.png}
\caption{\textbf{Spectral consistency.} (a) Raw decay rates show GHZ (blue) consistently lower than product states (orange) across the 0.1--2.0 MHz band. (d) Protection persists across varying CPMG depths.}
\label{fig:3-7}
\end{figure}

\begin{figure}[H]
\centering
\includegraphics[width=\textwidth]{Picture11.png}
\caption{\textbf{Distributional separation.} Violin plots of parity outcomes ($N=576$) show a statistically significant gap ($21\sigma$) between GHZ (blue, protected) and product states (red, decaying).}
\label{fig:3-8}
\end{figure}

\noindent\textit{Note.} In this dataset, multipartite entanglement is observed to suppress decoherence, as quantified by the $\sim 2.6\times$ slower decay of the GHZ state. This protective effect, measured on a 127-qubit superconducting processor, highlights entanglement not only as a computational resource but also as a potential stabilizing mechanism for quantum information in near-term hardware, and merits further investigation independently of any specific theoretical framework.

\vspace{1cm}

\begin{figure}[H]
\centering
\includegraphics[width=\textwidth]{Picture12.png}
\caption{\textbf{Cross-platform check (Rigetti Ankaa-3).} The angular law is reproduced for $N=3,4$. Signal-to-noise ratio degrades for $N \geq 5$, defining the current hardware sensitivity limit.}
\label{fig:3-9}
\end{figure}

\vspace{1cm}

\begin{tcolorbox}[title=\centering Box 4: Experimental Summary]
The combined numerical and experimental evidence supports the two pillars of WESH phenomenology:
\begin{enumerate}
    \item \textbf{Collective stability:} simulations are consistent with $\tau_{\mathrm{coh}} \propto N^{\alpha}$ with $\alpha \approx 2$, driven by inverse-scaling couplings.
    \item \textbf{Geometric structure:} hardware data are consistent with the specific $\cos^2\theta$ modulation and its dependence on the quantum state (GHZ vs W), distinguishing it from isotropic noise.
\end{enumerate}
\end{tcolorbox}

\vspace{0.8cm}
\begin{table}[h]
\centering
\footnotesize
\setlength{\tabcolsep}{6pt}
\renewcommand{\arraystretch}{1.2}
\begin{tabular}{@{}lccc@{}}
\toprule
Metric & WESH (Collision Model) & Standard Noise & Pure WDW \\
\midrule
Coherence Scaling ($\alpha_C$) & $\approx +2.1$ \textbf{(protected)} & $\approx 0.0$ (flat) & — \\
Rate Scaling ($\alpha_\gamma$) & $-1.80$ \textbf{($N^{-2}$ trend)} & $-1.00$ (local) & — \\
Purity Scaling ($\alpha_{\mathrm{pur}}$) & $-1.91$ & $-3.51$ & $\sim 0$ \\
Thermodynamic Arrow & \textbf{Yes} & Yes & No \\
\bottomrule
\end{tabular}
\caption{\textbf{Scaling metric comparison.} The WESH model uniquely combines a thermodynamic arrow of time with collective protection of coherence, distinct from both unitary evolution (Pure WDW) and standard local decoherence.}
\label{tab:metrics}
\end{table}

\vspace{0.8cm}

\section{General Relativity, part I: Emergence from QFTT--WESH}
\label{sec:gr-emergence}

\vspace{0.5cm}

\textbf{Overview:} This section and Section~\ref{sec:gr-dynamic} offer two 
complementary perspectives on the emergence of gravity within QFTT--WESH. 
Here we adopt a consistency-check approach: we introduce the Einstein--Hilbert 
term in $S_{\text{eff}}$ (Eq.~\eqref{eq:effective-action}) and verify its 
compatibility with WESH through hidden-sector cancellation, recovering 
$G_{\mu\nu} + \Lambda g_{\mu\nu} = 8\pi G\, T^{(m)}_{\mu\nu} + O(1/N)$ 
(Eqs.~\eqref{eq:hidden-cancel}--\eqref{eq:emergent-einstein}). 
Here and below, $\Lambda$ denotes the effective IR cosmological constant
(i.e.\ $\Lambda\equiv\Lambda_{\rm eff}$); its IR interpretation and stochastic width are discussed
in the subsection ``$\Lambda$ as entanglement IR imprint'' and Appendix~J.
Section~\ref{sec:gr-dynamic} complements this by deriving the EH term 
dynamically, without postulating it.
Together, these routes can be viewed as a microscopic realization of Sakharov's induced-gravity scenario (Sakharov, 1967), where spacetime curvature emerges from quantum vacuum fluctuations rather than being assumed from the outset.

\vspace{0.8cm}

\subsection{Stationary alignment: the bridge from time to geometry}

\vspace{0.5cm}

\noindent We now discuss the emergence of spacetime. The WESH dynamics induce a stationary alignment between the time field and the entanglement potential; this alignment underlies the subsequent metric reconstruction.

\begin{lemma}[Bootstrap alignment — stationary condition]\label{lem:bootstrap-alignment}
\hfill\break  
\vspace{0.3cm}

\noindent\textbf{Entanglement potential and its functional variation.}  
\vspace{0.3cm}
In the present effective description we treat the local eigentime profile
$\tau(x):=\langle\hat T(x)\rangle_\rho$ as the relevant variational degree
of freedom. Accordingly, the functional derivative $\delta C/\delta\hat T(z)$
in Eq.~\eqref{eq:entanglement-potential} is to be understood as taken with
respect to local shifts of the mean field $\tau(z)$ (mean-field limit of the
operator variation).
\begin{equation}
\begin{aligned}
\Phi(x) &= \int d^4y\ K(x-y)\,C(\Psi;x,y),\\[0.25cm]
\delta\Phi(x) &= \int d^4y\ K(x-y)\!\int d^4z\;
\frac{\delta C(\Psi;x,y)}{\delta \hat T(z)}\,\delta\hat T(z).
\label{eq:entanglement-potential}
\end{aligned}
\end{equation}

\noindent\textbf{Stationary alignment.}  
\vspace{0.3cm}
\begin{equation}
\text{Assume (stationary, mean-field):}\qquad \partial_\mu \tau(x) \;=\; k\,\partial_\mu \Phi(x),
\qquad \tau(x):=\langle \hat T(x)\rangle_\rho.
\label{eq:stationary-alignment}
\end{equation}
\end{lemma}

\begin{proof}[Proof (sketch)]
In the Markovian regime ($\tau_{\mathrm{corr}}\!\ll\!\tau_{\rm Eig}$), stationarity of the CP–TP flow
is equivalent to vanishing first variation of a WESH monotone
($\delta\mathcal M_\epsilon\!=\!0$). The variation produces a mixed term
$\mathcal{J}^\mu(x)\,\partial_\mu \tau(x)$ with $\mathcal{J}^\mu\!\propto\!\partial^\mu\Phi(x)$
(cf. Part~II, Thm.~\ref{thm:alignment-PartII}).
Here \(\mathcal J^\mu\) denotes this variational mixed-term current and should not be confused with the
entanglement current \(J^\mu\) introduced in Lemma~\ref{lem:wesh-noether-field}.
The stationarity condition thus requires
$\partial_\mu \tau \parallel \partial_\mu\Phi$; the proportionality constant
$k$ is fixed by the linear matching (Eq.~\eqref{eq:parameter-relations}). 
\end{proof}

\vspace{0.8cm}

\subsection{Emergent metric decomposition (uniqueness)}

\vspace{0.5cm}

\noindent\textbf{Complete metric: time + entanglement + eigentime substrate.}
\vspace{0.3cm}

\noindent The emergence of Einstein's equations requires the physical metric $g_{\mu\nu}$ to be a composite field, reconstructed from three distinct sectors: the time-field contribution $g^{(T)}_{\mu\nu}$ (set by gradients of $\hat T$), the entanglement contribution $g^{(C)}_{\mu\nu}$ (encoded by the nonlocal potential $\Phi$), and the discrete eigentime substrate $g^{(E)}_{\mu\nu}$ (the inertial seed). This decomposition is minimal and dictated by the WESH architecture:
\begin{equation}
g_{\mu\nu}(x)=g^{(T)}_{\mu\nu}(x)+g^{(C)}_{\mu\nu}(x)+g^{(E)}_{\mu\nu}(x). 
\label{eq:complete-metric}
\end{equation}

\vspace{0.8cm}
\noindent\textbf{Definitions of $g^{(T)}$ and $g^{(C)}$.} 
\vspace{0.3cm}
\begin{equation}
\begin{split}
g^{(T)}_{\mu\nu} &= \zeta^{-1}\,\big\langle \partial_{\mu}\hat T\,\partial_{\nu}\hat T\big\rangle,\\
g^{(C)}_{\mu\nu}(x) &= \beta_{\mathrm{G}}\,V_\xi\!\int\! d^4y\ K(x-y)\,\partial_\mu^{(x)}\partial_\nu^{(x)}\ln C(\Psi;x,y),
\end{split}
\label{eq:decomposition-definitions}
\end{equation}
\vspace{0.3cm}
\noindent where $\beta_{\mathrm{G}}=\frac{1}{4\pi G}$ (see Eq.~\eqref{eq:beta-normalization}).
In the geometric regime where the metric is well-defined, $C(\Psi;x,y)\geq C_{\min}>0$ 
throughout the domain of interest; the logarithm is thus well-defined.

\vspace{0.8cm}
\noindent\textbf{Eigentime contribution (mollified; smooth continuum limit).} 
\vspace{0.3cm}
\begin{equation}
g^{(E)}_{\mu\nu,\varepsilon}(x)=\frac{V_\xi}{N}\sum_{i=1}^N \eta_{\mu\nu}\,w_i\,K_\varepsilon(x-x_i),\quad \lim_{\substack{N\to\infty\\ \varepsilon\to0}}g^{(E)}_{\mu\nu,\varepsilon}\!=g^{(E)}_{\mu\nu}\ \text{(smooth)}.
\label{eq:eigentime-contribution}
\end{equation}

\noindent\emph{Weights.}
We assume bounded weights with $\sup_i |w_i|=\mathcal{O}(1)$ and normalization $\sum_i w_i = \mathcal{O}(N)$, consistent with the coarse-grained estimator.


\noindent\emph{Seed inertial structure.}
The tensor $\eta_{\mu\nu}$ in $g^{(E)}$ is the local inertial seed (Lorentz signature) on the pre-geometric label space;
the event mesh then dresses it into a smooth, macroscopic background.

\noindent\emph{Derivatives.}
All $\partial_\mu$ are taken with respect to the Minkowski label coordinates.
In the emergent geometric regime they may be covariantized $\partial_\mu\to\nabla_\mu$ with respect to $g_{\mu\nu}$,
with differences suppressed on coarse-graining windows $L\gg \xi$.

\vspace{0.8cm}
\noindent\textbf{Emergent metric (concise statement).} 
\vspace{0.3cm}
\begin{equation}
\boxed{\,g_{\mu\nu}(x)=g^{(T)}_{\mu\nu}(x)+g^{(C)}_{\mu\nu}(x)+g^{(E)}_{\mu\nu}(x)\,}
\label{eq:unique-decomp}
\end{equation}

\vspace{0.3cm}
\begin{theorem}[Uniqueness]
\label{thm:uniqueness-metric}
Under the constraints \emph{Emergence}, \emph{Physicality} and \emph{Dynamic Consistency},
the decomposition \eqref{eq:unique-decomp} is unique among functionals built from the
QFTT--WESH primitives $\hat T$, $C(\Psi;x,y)$ and the eigentime substrate $\{x_i\}$.
\end{theorem}

\noindent\textit{Sketch (metric decomposition uniqueness).}
With gradient alignment $\partial_\mu\tau=k\,\partial_\mu\Phi$ (Eq.~\eqref{eq:stationary-alignment}) and the cancellation matching
\eqref{eq:cancellation-matching}, the $(\partial\Phi)^2$ terms in
$T^{(T)}_{\mu\nu}$ cancel those in the nonlocal sector at continuum, leaving Einstein
with matter up to $\mathcal{O}(1/N)$ (Eqs.~\eqref{eq:stress-energy}--\eqref{eq:emergent-einstein}). A formal proof of this uniqueness theorem is given in Sec.~\ref{subsec:IR-uniqueness-PartII}.
\begin{example}[CPT as an atemporal symmetry]
Let $\mathcal{G}_{\rm CPT}$ act on observables as the antiunitary CPT transformation.
Since $\mathcal{D}[\hat T^2(x)]$ and $\mathcal{D}[L_{xy}]$ are CPT-even (cf. Eqs.~\eqref{eq:wesh-local-diss}, \eqref{eq:Lxy-def}),
one has $\mathcal{G}_{\rm CPT}\!\circ\!\Phi_{s\to s'}=\Phi_{s\to s'}\!\circ\!\mathcal{G}_{\rm CPT}$ 
for the CP-TP maps generated by Eq.~\eqref{eq:wesh-master} with frozen coefficients on each micro-interval.
Therefore, CPT parity is exactly preserved along $s$: 
$\frac{d}{ds}\langle \hat \Pi_{\rm CPT}\rangle=0$.
Via $dt/ds=\Gamma[\Psi]$ (Eq.~\eqref{eq:s-to-t-bootstrap}), this implies preservation in $t$ as well, 
up to $O(1/N)$ corrections when feedback is restored.
\end{example}
\vspace{0.8cm}

\subsection{Connection induced by \texorpdfstring{$\nabla\Phi$}{nabla Phi} and linear curvature}

\vspace{0.8cm}

\noindent To determine the dynamics, we examine how the emergent geometry responds to the entanglement potential $\Phi$. Since $\Phi$ is a nonlocal entanglement potential, the curvature tensor arises as a linear response to its gradients.

\vspace{0.5cm}

\noindent In the IR regime ($L\gg\xi$) and for $C\ge C_{\min}>0$, the functional $\ln C$ admits a controlled derivative expansion; the potential $\Phi$ then provides the closed nonlocal variable controlling the leading curvature response.

\vspace{0.5cm}

\noindent\textbf{Linearized Ricci response to the entanglement potential.}
\vspace{0.3cm}

\noindent Let $\lambda_1$ and $\lambda_2$ denote the effective IR response coefficients characterizing the nonlocal
entanglement backreaction (see $g^{(C)}$ in Eq.~\eqref{eq:decomposition-definitions}). At linear order in the
entanglement potential $\Phi$ the curvature response can be written as
\begin{equation}
\delta R_{\mu\nu}
= (\lambda_1+3\lambda_2)\,\big(\nabla_\mu\nabla_\nu\Phi-g_{\mu\nu}\Box\Phi\big)
+ \mathcal O(\Phi^2).
\label{eq:linearized-ricci}
\end{equation}

\vspace{0.8cm}

\noindent\textbf{Coefficient matching, yielding GR normalization.}  
\vspace{0.3cm}
\begin{equation}
\frac{k^2}{4\pi G}=\lambda_1+3\lambda_2. 
\label{eq:parameter-relations}
\end{equation}

\vspace{0.8cm}

\subsection{Effective action and variation}

\vspace{0.5cm}

\noindent The field equations are derived from a variational principle. In the stationary regime, the alignment conditions can be expressed as Euler--Lagrange equations of an effective action $S_{\mathrm{eff}}[g,\Psi]$. By IR closure at $\le 2$ derivatives, $S_{\mathrm{eff}}$ reduces to the Einstein--Hilbert form plus matter and WESH-sector terms.

\vspace{0.5cm}

\noindent\textbf{Effective action: EH + matter + WESH sector terms.} 
\vspace{0.3cm}
\begin{equation}
S_{\mathrm{eff}}[g,\Psi]=\frac{1}{16\pi G}\!\int\! d^4x\,\sqrt{-g}\,(R-2\Lambda) + S_{\mathrm{mat}}[\Psi,g] + S_{T}[g^{(T)}]+S_{C}[g^{(C)}]+S_{E}[g^{(E)}]. 
\label{eq:effective-action}
\end{equation}

\vspace{0.8cm}

\noindent\textbf{Field equations from variation of $S_{\text{eff}}$.} 
\vspace{0.3cm}
\begin{equation}
G_{\mu\nu}+\Lambda g_{\mu\nu}
= 8\pi G\Big(T^{(m)}_{\mu\nu}
             +T^{(T)}_{\mu\nu}
             +T^{(nl)}_{\mu\nu}\Big).
\label{eq:field-equations}
\end{equation}

\vspace{0.8cm}

\noindent\textbf{Stress energy of the time sector.} 
\vspace{0.3cm}
\begin{equation}
T^{(T)}_{\mu\nu}=\zeta^{-1}\!\left(\big\langle \partial_{\mu}\hat T\,\partial_{\nu}\hat T\big\rangle
-\tfrac12 g_{\mu\nu}\big\langle \partial_{\lambda}\hat T\,\partial^{\lambda}\hat T\big\rangle\right). 
\label{eq:stress-energy}
\end{equation}

\vspace{0.8cm}

\noindent\textbf{Nonlocal stress from entanglement and eigentime sectors.} 
\vspace{0.3cm}
\begin{equation}
T^{(nl)}_{\mu\nu}
= -\,\frac{2}{\sqrt{-g}}\frac{\delta}{\delta g^{\mu\nu}}
\Big[S_{C}[g^{(C)}]+S_{E}[g^{(E)}]\Big] .
\label{eq:nonlocal-stress}
\end{equation}

\begin{lemma}[WESH--Noether (field form), continuity and closure]
\label{lem:wesh-noether-field}
With the finite-range Yukawa mediator $K$ and the normalized Rényi-2 gate $C\in[0,1]$,
the entanglement current reads
\begin{equation}
\label{eq:J-current}
J^\mu(x)\;=\;\int d^4y\ K(x-y)\,C(\Psi;x,y)\,\hat I^\mu(x,y),
\end{equation}
where the bilocal information-flow operator $\hat I^\mu(x,y)$ satisfies
\[
\hat I^\mu(x,y)=-\hat I^\mu(y,x),\qquad 
\partial_\mu^{(x)}\hat I^\mu(x,y)=\delta^{(4)}(x-y)\,\hat S(x),
\]
with $\hat S(x)$ a local source operator (neutral in the mean).
Taking expectations one obtains the continuity law with neutral source:
\begin{equation}
\label{eq:cont-neutral}
\partial_\mu\langle J^\mu(x)\rangle=\langle \hat S(x)\rangle,\qquad 
\int d^4x\,\langle \hat S(x)\rangle=0.
\end{equation}
Since $K(x-y)$ (and $\gamma(x,y)$) has finite range $\sim\xi$, coarse graining over any window
$L\gg\xi$ yields 
\begin{equation}
\label{eq:cg-conservation}
\nabla_\mu\langle J^\mu\rangle\;\approx\;0\qquad (L\gg\xi),
\end{equation}
consistent with the $\mathcal O(1/N)$ control of global sources used in Sec.~\ref{subsec:TSL-PartII}.
\end{lemma}

\vspace{0.8cm}

\subsection{Hidden-sector cancellation and the emergent Einstein equations}
\label{subsec:hidden-cancel-einstein}
\vspace{0.5cm}

With the effective action framework and the metric decomposition $g_{\mu\nu}=g^{(T)}_{\mu\nu}+g^{(C)}_{\mu\nu}+g^{(E)}_{\mu\nu}$ in place, we turn to the dynamical consequence of WESH at the level of the field equations: the cancellation of the hidden-sector stresses $T^{(T)}_{\mu\nu}$ and $T^{(nl)}_{\mu\nu}$. This cancellation is not assumed but follows from three ingredients:
\vspace{0.3cm}
\begin{enumerate}[label=(\roman*),leftmargin=2.0cm,itemsep=0.3cm]
    \item \textbf{Stationary gradient alignment} $\partial_\mu \tau=k\,\partial_\mu \Phi$,
    \item \textbf{WESH--Noether continuity (neutral source)} (Lemma~\ref{lem:wesh-noether-field}, Eq.~\eqref{eq:cont-neutral}),
    \item \textbf{Cancellation matching} $\frac{k^2}{\zeta} = \lambda_1 + 3\lambda_2$ (Eq.~\eqref{eq:cancellation-matching}).
\end{enumerate}
\vspace{0.3cm}
This leads to Einstein's equations with matter only, up to controlled $\mathcal{O}(1/N)$ corrections.


\paragraph{Linear response of $T^{(nl)}$ and dominant structure.}
For IR kernels and the normalizations introduced in Sec.~\ref{sec:qftt-framework}, the variation of $S_C+S_E$ gives
\begin{equation}
T^{(nl)}_{\mu\nu}(x)\;=\;-\!\int d^4y\ K(x-y)\,\nabla_\mu\nabla_\nu \ln C(\Psi;x,y)
\;+\;\text{(lower–order covariant terms)}.
\label{eq:Tnl-linear}
\end{equation}

\noindent\emph{IR tensorial closure.}
Finite-range Markov locality (kernel support $\sim\xi$) ensures that on coarse-grained 
windows $L\gg\xi$ the entanglement functional admits a local derivative expansion in 
$\Phi$. Under stationary alignment, diffeomorphism invariance and the $\le2$-derivative 
truncation then force the dominant contribution of $T^{(\mathrm{nl})}_{\mu\nu}$ into 
the universal scalar-kinetic form quadratic in first derivatives of $\Phi$
(Part~II, Lemma~\ref{lem:tensorial-closure}, Eq.~\eqref{eq:tensorial-closure-form});
subleading terms are suppressed by $\xi/L$.

\paragraph{Key cancellation mechanism.}
Under the stationary gradient alignment
\begin{equation}
\partial_\alpha \hat T = k\,\partial_\alpha \Phi, \qquad \Phi(x)=\int d^4y\ K(x-y)\,C(\Psi;x,y),
\end{equation}
the leading quadratic contributions are
\[
T^{(T)}_{\mu\nu}\ \sim\ \frac{k^2}{\zeta}\,\partial_\mu\Phi\,\partial_\nu\Phi\;+\;\cdots,
\qquad
T^{(nl)}_{\mu\nu}\ \sim\ -\big(\lambda_1+3\lambda_2\big)\,\partial_\mu\Phi\,\partial_\nu\Phi\;+\;\cdots.
\]
The unique WESH \emph{cancellation matching} condition
\begin{equation}
\frac{k^2}{\zeta}=\lambda_1+3\lambda_2
\label{eq:cancellation-matching}
\end{equation}
ensures exact cancellation of the $(\partial\Phi)^2$ terms at continuum.

\begin{theorem}[Hidden-sector cancellation yields Einstein equations]
\label{thm:hidden-cancel-EH}
Assume stationarity with gradient alignment \eqref{eq:stationary-alignment}, the cancellation matching \eqref{eq:cancellation-matching}, and the self-consistency matching for the local rate (Eq.~\eqref{eq:nu-matching}). Then:
\begin{enumerate}[label=(\roman*)]
    \item In the continuum limit ($N\to\infty$),
    \begin{equation}
    T^{(T)}_{\mu\nu}+T^{(nl)}_{\mu\nu}=0.
    \label{eq:hidden-cancel}
    \end{equation}
    \item At finite $N$, the residual satisfies
    \begin{equation}
    \big\|\,T^{(T)}_{\mu\nu}+T^{(nl)}_{\mu\nu}\,\big\|_{L^\infty(B_L)}=\mathcal O(1/N)\qquad (L\gg\xi),
    \label{eq:hidden-cancel-bound}
    \end{equation}
    consistently with the graph-to-manifold convergence bounds (Eqs.~\eqref{eq:discrete-to-continuum}--\eqref{eq:curvature-and-hausdorff}).
    \item Consequently, the emergent field equations are
    \begin{equation}
    \boxed{\,G_{\mu\nu}+\Lambda g_{\mu\nu}=8\pi G\,T^{(m)}_{\mu\nu}+\mathcal O(1/N)\,}.
    \label{eq:emergent-einstein}
    \end{equation}
\end{enumerate}
\end{theorem}

\noindent\textit{Physical interpretation.}
The time-field stress $T^{(T)}_{\mu\nu}$ sources curvature, while the entanglement backreaction $T^{(nl)}_{\mu\nu}$ provides the precise counter-stress required by Wheeler--DeWitt consistency. This cancellation is \emph{not} a fine-tuning: it follows uniquely from the WESH bootstrap, gradient alignment, and the cancellation matching \eqref{eq:cancellation-matching}.

\noindent\textit{Link to matter conservation.}
The cancellation \eqref{eq:hidden-cancel} ensures that the full field equations reduce to Einstein's equations with matter stress only. Matter conservation follows from the contracted Bianchi identity (see below for the detailed derivation).
\vspace{0.3cm}

\noindent\textit{Dimensional consistency.}
We work in the dimensionless normalization used throughout QFTT--WESH (natural units
$\hbar=c=1$, with lengths expressed in units of the microscopic scale $\xi\simeq L_P$).
With this convention the matchings \eqref{eq:cancellation-matching} and
\eqref{eq:parameter-relations} are dimensionally consistent, and at the GR fixed point
they identify the time-sector normalization as $\zeta=4\pi G$.
A complete unit-restoration (including the role of $V_\xi$ and $\xi$-scaling) is given
in Appendix~J.

\vspace{0.8cm}

\subsection{Bianchi Identity and Matter Conservation}

\vspace{0.5cm}

The geometric consistency of the emergent Einstein equations requires the total stress-energy tensor to be covariantly conserved. In the stationary regime where the metric is well-defined, the contracted Bianchi identity enforces this constraint.

\vspace{0.8cm}

\noindent\textbf{Total conservation from Bianchi.}
\vspace{0.3cm}

\noindent With $G_{\mu\nu} + \Lambda g_{\mu\nu} = 8\pi G\,T^{\mu\nu}_{\rm tot}$ and 
\[
T^{\mu\nu}_{\rm tot} = T^{(m)\,\mu\nu}+T^{(T)\,\mu\nu}+T^{(nl)\,\mu\nu},
\]
the identity $\nabla_\mu G^{\mu\nu}\equiv 0$ implies
\begin{equation}
\nabla_\mu T^{\mu\nu}_{\rm tot} = \nabla_\mu\Big(T^{(m)\,\mu\nu}+T^{(T)\,\mu\nu}+T^{(nl)\,\mu\nu}\Big) = 0.
\label{eq:total-conservation}
\end{equation}

\vspace{0.8cm}

\noindent\textbf{Emergence of matter conservation.}
\vspace{0.3cm}

\noindent At the stationary point, the hidden-sector stresses cancel to leading order (Eqs.~\eqref{eq:hidden-cancel}--\eqref{eq:hidden-cancel-bound}):
\begin{equation}
\nabla_\mu\Big(T^{(T)\,\mu\nu}+T^{(nl)\,\mu\nu}\Big) = \mathcal O(1/N).
\end{equation}
Substituting into \eqref{eq:total-conservation} isolates the visible sector:
\begin{equation}
\nabla_\mu T^{(m)\,\mu\nu} = \mathcal O(1/N),
\label{eq:matter-conservation}
\end{equation}
recovering standard matter conservation and geodesic motion in the continuum limit.

\vspace{0.8cm}

\subsection{\texorpdfstring{$\Lambda$}{Lambda} as entanglement IR imprint}

\vspace{0.8cm}

\noindent\textbf{IR status: the unique zero-derivative coupling.}
\vspace{0.3cm}

\noindent By IR uniqueness at $\le 2$ derivatives, the coarse-grained gravitational sector closes on the Einstein--Hilbert class. The only diffeomorphism-invariant contribution at \emph{zero derivatives} is the volume operator, whose coefficient defines the effective cosmological constant:
\begin{equation}
S_{\rm eff}^{\rm IR}[g]\ \supset\ \frac{1}{16\pi G}\int d^4x\sqrt{-g}\,\big(-2\Lambda_{\rm eff}\big).
\label{eq:Lambda-IR-def}
\end{equation}

\vspace{0.5cm}

\noindent\textbf{Dynamical origin: intrinsic shot noise.}
\vspace{0.3cm}

\noindent While the hidden-sector cancellation constrains the gradient terms (the tensor sector), it does not eliminate the global scalar residue of the GKSL process. Since eigentime production is a stochastic counting process, the vacuum term acquires an \emph{irreducible shot-noise width} on any finite four-volume $V_4$.
Consistent with the Central Limit (martingale) scaling derived in Appendix~J, the typical amplitude is:
\begin{equation}
\delta\Lambda_{\rm typ}(V_4)\ \sim\ \frac{\alpha_\Lambda}{\sqrt{V_4}},\qquad \alpha_\Lambda=\mathcal O(1),
\label{eq:Lambda-shotnoise-main}
\end{equation}
yielding the cosmological scaling $\delta\Lambda_{\rm typ}\sim H^2$ for $V_4\sim H^{-4}$. This estimate concerns the typical fluctuation amplitude; the instantaneous sign remains stochastic.

\vspace{0.8cm}

\subsection{Linearized limit and Newtonian recovery}

\vspace{0.5cm}

\noindent Following the hidden-sector cancellation (Theorem~\ref{thm:hidden-cancel-EH}), the visible matter sector remains as the sole macroscopic source at $\mathcal{O}(1/N)$. We now verify that this residual dynamics correctly recovers standard General Relativity in the weak-field regime (here we temporarily restore factors of $c$ for clarity).

\vspace{0.3cm}

\noindent\textbf{Linearized Einstein in Lorentz gauge.}  
\vspace{0.3cm}
\begin{equation}
g_{\mu\nu}=\eta_{\mu\nu}+h_{\mu\nu},\ \bar h_{\mu\nu}=h_{\mu\nu}-\tfrac12\eta_{\mu\nu}h,\ \partial^\mu \bar h_{\mu\nu}=0 \ \Rightarrow\ \Box\,\bar h_{\mu\nu}=-16\pi G\,T^{(m)}_{\mu\nu}. 
\label{eq:linearized-lorentz}
\end{equation}

\vspace{0.8cm}

\noindent\textbf{Newtonian limit and Poisson equation.}  
\vspace{0.3cm}

\noindent Here $\rho_m$ denotes the nonrelativistic mass density.

\vspace{0.3cm}
\begin{equation}
\text{Newtonian: }\ \nabla^2\phi_N=4\pi G\,\rho_m,\quad \bar h_{00}=-\frac{4\phi_N}{c^2}. 
\label{eq:newtonian-poisson}
\end{equation}

\vspace{0.8cm}

\subsection{Example (vacuum, spherically symmetric): Schwarzschild}

\vspace{0.8cm}

\noindent Since the emergent equations reduce to the standard GR vacuum form (and to $R_{\mu\nu} = 0$ when $\Lambda_{\rm eff}$ is neglected on the scales considered), the theory admits the usual vacuum solutions of General Relativity.

\vspace{0.3cm}

\noindent\textbf{Schwarzschild line element (vacuum).} 
\vspace{0.3cm}
\begin{equation}
ds^2=-\Big(1-\frac{2GM}{rc^2}\Big)c^2 dt^2 + \Big(1-\frac{2GM}{rc^2}\Big)^{-1}dr^2+r^2 d\Omega^2,\qquad R_{\mu\nu}=0. 
\label{eq:schwarzschild-element}
\end{equation}

\vspace{0.8cm}

\noindent\textbf{Kretschmann scalar.} 
\vspace{0.3cm}
\begin{equation}
\text{Kretschmann: }\ R_{\mu\nu\rho\sigma}R^{\mu\nu\rho\sigma}=\frac{48\,G^2 M^2}{c^4\,r^6}. 
\label{eq:kretschmann-scalar}
\end{equation}

\vspace{0.8cm}

\subsection{Eigentime substrate: convergence to continuum}

\vspace{0.5cm}

\noindent\textbf{Discrete-to-continuum bounds for $g^{(E)}$.} 
\vspace{0.3cm}

\noindent Assumptions (A1)--(A3) for the convergence bound $\mathcal{O}(N^{-1/4})$. (A1) Local correlation with length $\xi$; (A2) coarse-grained sampling of eigentimes within causal diamonds; (A3) bounded curvature and mollifier with compact support. Under (A1)--(A3) the kernel estimator achieves $\|g^{(E)}_{\mu\nu,\varepsilon}-g^{(E)}_{\mu\nu}\|_{\infty}=\mathcal{O}(N^{-1/4})$.

\noindent Let $n_{\mathrm{Eig}}:=N/V_4$ denote the eigentime event density, where $V_4$ is the four-volume of the coarse-graining window $L\gg\xi$; in the quasi-uniform regime $n_{\mathrm{Eig}}\sim\bar\lambda_{\mathrm{Eig}}$ (see Proposition below).

\begin{equation}
\ell_{\mathrm{Eig}}\sim n_{\mathrm{Eig}}^{-1/4},\qquad \big\|g^{(E)}_{\mu\nu,\varepsilon}-g^{(E)}_{\mu\nu}\big\|_\infty=\mathcal O\!\big(N^{-1/4}\big). 
\label{eq:discrete-to-continuum}
\end{equation}

\begin{proposition}[smoothing and optimal convergence]

\vspace{0.3cm}
~\\

\noindent Let
\[
\lambda_{\mathrm{Eig}}(x)\;=\;\nu\,\Tr\!\big[\rho\,\hat T^4(x)\big] + \iint d^4y\,\gamma(x,y)\,C(\Psi;x,y)\,\Tr\!\big[\rho\,(\hat T^2(x)-\hat T^2(y))^2\big],
\]
be the hazard rate governing eigentime events (so that it contributes to $dt/ds=\Gamma[\Psi(s)]$).
For any ball $B_L$ with $L\gg \xi$ there exists a constant 
$\kappa=\mathcal{O}(1)$, depending only on $\|\gamma\|_{L^1(B_L)}$, the variance $\mathrm{Var}_\rho(\hat T^2)$, and the geometry of $B_L$, such that

\vspace{0.3cm}

\[
\frac{\sup_{B_L}\lambda_{\mathrm{Eig}}}{\inf_{B_L}\lambda_{\mathrm{Eig}}} \;\le\; 1+\kappa\,e^{-L/\xi}.
\]

\vspace{0.3cm}

\noindent Hence eigentime events form a \emph{quasi-uniform} inhomogeneous-Poisson mesh on scales $\gg\xi$, 
with characteristic spacing $\ell_{\mathrm{Eig}}\sim \lambda_{\mathrm{Eig}}^{-1/4}$ in four dimensions and fill distance 
$h_N=\mathcal{O}(N^{-1/4})$ for $N$ events in $B_L$.
\vspace{0.3cm}

\noindent With the kernel regularization of Eq.~\eqref{eq:eigentime-contribution}, the relevant fields entering the continuum limit 
are Lipschitz on $B_L$; therefore the sampling error obeys

\vspace{0.3cm}

\[
\big\|\mathcal{F}_{\mathrm{disc}}-\mathcal{F}_{\mathrm{cont}}\big\|_{L^\infty(B_L)} \;\le\; \mathrm{Lip}(\mathcal{F})\,h_N \;=\; \mathcal{O}(N^{-1/4}),
\]

\vspace{0.3cm}

\noindent which yields the convergence rate $\mathcal{O}(N^{-1/4})$ in Eq.~\eqref{eq:discrete-to-continuum}. 
(Without  smoothing one only obtains $\mathcal{O}(N^{-1/4+\varepsilon})$ for any $\varepsilon>0$.)
\end{proposition}

\vspace{0.5cm}

\begin{tcolorbox}[title=\centering Box 5: Assumptions leading to $\mathcal{O}(N^{-1/4})$ convergence]

(A1) Short-range clustering: the  kernel with correlation length $\xi$ enforces the rapid decay of connected correlators beyond $\mathcal{O}(\xi)$.

\smallskip
(A2) Quasi-uniform sampling: eigentime events provide a mesh with bounded hole size and aspect ratio in the emergent 3+1 geometry.

\smallskip
(A3) Regular mollifier: the kernel estimator uses a $C^2$ mollifier with bounded moments and curvature-controlled bandwidth.

\smallskip
Under (A1)--(A3) the graph-to-manifold estimator achieves $\|g^{(E)}_{\mu\nu,\varepsilon}-g^{(E)}_{\mu\nu}\|_{\infty}= \mathcal{O}(N^{-1/4})$, faster than i.i.d. kernel rates due to correlated sampling induced by WESH dynamics.
\vspace{0.3cm}
\end{tcolorbox}

\vspace{0.8cm}

\noindent\textbf{Curvature convergence and Hausdorff dimension.} 
\vspace{0.3cm}
\begin{equation}
\big\|R^{(\mathrm{disc})}_{\mu\nu}-R_{\mu\nu}\big\|=\mathcal O\!\big(\ell_{\mathrm{Eig}}^{\,2}\big),\qquad \lim_{N\to\infty} d_H(\mathcal G_N)=4. 
\label{eq:curvature-and-hausdorff}
\end{equation}

\vspace{0.5cm}

\begin{remark}[Scaling of the finite-$N$ hidden-sector residual]
\label{rem:geometric-to-stress}
The hidden-sector cancellation $T^{(T)}_{\mu\nu}+T^{(\mathrm{nl})}_{\mu\nu}=0$ 
holds exactly in the continuum limit $N\to\infty$. At finite $N$, potential 
sources of residual error are:
\begin{enumerate}[label=(\roman*),leftmargin=1.2cm]
\item geometric discretization errors, bounded by $O(N^{-1/2})$ at the level 
of curvature (Eq.~\eqref{eq:curvature-and-hausdorff});
\item finite-$N$ corrections to the variational alignment and matching, 
controlled by the $N^{-2}$ prefactor of the bilocal WESH channel.
\end{enumerate}
Crucially, the errors in (i) enter $T^{(T)}_{\mu\nu}$ and $T^{(\mathrm{nl})}_{\mu\nu}$ 
through the same functional dependence on $\Phi$ and its derivatives; since the cancellation matching \eqref{eq:cancellation-matching} (and, at the GR fixed point, \eqref{eq:parameter-relations}) is algebraic and $N$-independent,
these geometric errors cancel at leading order in the sum $T^{(T)}+T^{(\mathrm{nl})}$, leaving only alignment deviations as the dominant finite-$N$ residual. On coarse-grained windows $B_L$ with $L\gg\xi$, spatial averaging 
further suppresses local fluctuations. The bound
\begin{equation}
\big\|\,T^{(T)}_{\mu\nu}+T^{(\mathrm{nl})}_{\mu\nu}\,\big\|_{B_L}
= O(1/N)
\end{equation}
is therefore controlled by the $N^{-2}$ dissipative scaling and is consistent 
with the collective-stability law $\tau_{\mathrm{coh}}(N)\propto N^2$.
\end{remark}

\subsection{Markovianity, causality, no-signaling}

\vspace{0.5cm}

\noindent The finite range $\xi$ of the WESH kernel ensures that the dynamics is effectively Markovian on coarse-grained timescales and respects relativistic causality in the emergent spacetime.

\vspace{0.5cm}

\noindent\textbf{Finite memory time from $\xi$; Markovian regime.}
\vspace{0.3cm}
\begin{equation}
\tau_{\mathrm{corr}}\sim \xi/c \Rightarrow \text{finite memory; Markov for } \Delta t\gg \tau_{\mathrm{corr}}.
\label{eq:finite-memory-s5}
\end{equation}

\vspace{0.5cm}

\noindent\textit{No-signaling.}
The causal structure of the WESH kernel ensures
\begin{equation}
(x-y)^2>0 \ \Rightarrow\ \gamma(x,y)\,K(x-y)= 0 \qquad \text{(exponential-causal; no-signaling)}.
\label{eq:no-signaling}
\end{equation}

\vspace{0.8cm}

\subsection{Collective stability: \texorpdfstring{$N^2$}{N^2} scaling}

\vspace{0.5cm}

\noindent A key prediction of the WESH channel is the suppression of decoherence rates for large entangled systems. The quadratic scaling $\tau_{\rm coh} \sim N^2$ reflects the collective protection mechanism underlying the semiclassical limit.

\vspace{0.5cm}

\noindent\textbf{Quadratic stability law for coherence lifetime.} 
\vspace{0.3cm}
\begin{equation}
\gamma(N)\sim \frac{1}{N^2},\qquad \tau_{\mathrm{coh}}(N)\sim N^2. \label{eq:quadratic-stability}
\end{equation}

\vspace{0.5cm}

\noindent\textbf{collision model: $\gamma \propto N^{-2}$.} 
\vspace{0.3cm}
\begin{equation}
\text{Collision model:}\quad \theta_i=\sqrt{\kappa\,dt}\,N^{-1} w_i,\ \ \sum_i w_i^2=1 \ \Rightarrow\ \gamma\propto N^{-2}. 
\label{eq:collision-model}
\end{equation}

\vspace{0.8cm}

\subsection{Normalizations and matching to GR}

\vspace{0.5cm}

\noindent The free coefficients in the metric decomposition are fixed by consistency with the GR normalization of Newton's constant.

\vspace{0.5cm}
\noindent\textbf{$\beta$ normalization and explicit $g^{(C)}$.} 
\vspace{0.3cm}
\begin{equation}
\beta_{\mathrm{G}}=\frac{1}{4\pi G},\qquad g^{(C)}_{\mu\nu}(x)=\beta_{\mathrm{G}}\,V_\xi\!\int\! d^4y\ K(x-y)\,\partial_\mu^{(x)}\partial_\nu^{(x)}\ln C(\Psi;x,y). 
\label{eq:beta-normalization}
\end{equation}

\vspace{0.5cm}

\noindent\textbf{$\zeta$ normalization from GR matching.} 

\vspace{0.3cm}
\noindent The time-sector normalization is fixed at the GR fixed point by the cancellation matching \eqref{eq:cancellation-matching} together with the GR response normalization \eqref{eq:parameter-relations}, yielding $\zeta = 4\pi G$. The Minkowski vacuum corresponds to the state where the composite metric reduces to $\eta_{\mu\nu}$ in the absence of sources:
\begin{equation}
\zeta = 4\pi G, \qquad g^{(T)}_{\mu\nu}=\zeta^{-1}\,\big\langle \partial_\mu \hat T\,\partial_\nu \hat T\big\rangle. 
\label{eq:alpha-normalization}
\end{equation}

\vspace{0.3cm}
\noindent\textbf{Parameter relations ($k$, $\lambda$) and alignment.} 
\vspace{0.3cm}
\begin{equation}
\boxed{\,\frac{k^2}{4\pi G}=\lambda_1+3\lambda_2\,,\quad \partial_\mu \tau=k\,\partial_\mu\Phi\,} 
\label{eq:k-lambda-box}
\end{equation}
\vspace{0.3cm}

\begin{tcolorbox}[title=\centering\textbf{Box 6: Two roles of \(G\) — microscopic scale vs.\ macroscopic coupling}]
\noindent The symbol \(G\) enters the theory in two \emph{a priori} distinct ways:

\begin{itemize}[leftmargin=1.2cm, itemsep=0.25cm]
\item \textbf{Microscopic scale anchor.} In Eqs.~\eqref{eq:fundamental-scales}, \eqref{eq:scales-mediatormass}, \(G\) appears only through Planck units,
\(L_P=\sqrt{\hbar G/c^3}\) and \(t_P=\sqrt{\hbar G/c^5}\), which fix the correlation length \(\xi\) and the kernel scale \(\gamma_0\).
Here, \(G\) acts as a \emph{dimensional} parameter setting the discreteness scale, not a classical coupling.

\item \textbf{Macroscopic gravitational coupling.} In Eqs.~\eqref{eq:parameter-relations}, \eqref{eq:emergent-einstein}, \eqref{eq:beta-normalization}, \(G\) is the Newton constant in Einstein's equations,
with \(\beta_G = 1/(4\pi G)\) fixing the normalization of the entanglement-metric sector.
\end{itemize}

\noindent The theory does \emph{not} assume these two appearances coincide. Rather, consistency of the hidden-sector cancellation matching (Eq.~\eqref{eq:cancellation-matching})
with the GR normalization (Eq.~\eqref{eq:parameter-relations})
\emph{identifies} them as a consequence of internal consistency between microphysics (eigentimes, entanglement) and macrophysics (metric, curvature).
Consistency via the cancellation matching \eqref{eq:cancellation-matching} 
\emph{enforces} a relation between the microscopic correlation length (set by $L_P$ through $G$) and the macroscopic coupling measured in Einstein's equations. The identification emerges from hidden–sector cancellation, not as an external input.
\end{tcolorbox}

\vspace{0.8cm}

\subsection{Summary of the emergence scheme}

\vspace{0.5cm}

\noindent\textbf{Metric decomposition (T + C + E).} 
\vspace{0.3cm}
\begin{equation}
\boxed{\ \ g_{\mu\nu}=g^{(T)}_{\mu\nu}+g^{(C)}_{\mu\nu}+g^{(E)}_{\mu\nu}\ \ } \label{eq:metric-summary}
\end{equation}

\vspace{0.5cm}

\noindent\textbf{Einstein equations from hidden-sector cancellation.} 
\vspace{0.3cm}

\begin{equation}
\boxed{\ \ T^{(T)}_{\mu\nu}+T^{(nl)}_{\mu\nu}=0\ \Longrightarrow\  G_{\mu\nu}+\Lambda g_{\mu\nu}=8\pi G\,T^{(m)}_{\mu\nu}+\mathcal O(1/N)\ \ } 
\label{eq:emergence-hs}
\end{equation}

\vspace{0.5cm}

\noindent\textbf{Conservation, $\Lambda$, and Planck-scale parameters.} 
\vspace{0.3cm}
\begin{equation}
\boxed{\ \ \nabla_\mu T^{(m)\,\mu\nu}=\mathcal O(1/N),\quad
\Lambda_{\rm eff}\ \text{(IR parameter; see App.~J)},\quad
\xi\simeq L_P,\ \gamma_0\simeq t_P^{-1}\ \ }
\label{eq:conservation-planck}
\end{equation}

\vspace{1cm}

\section{General Relativity, part II: Dynamic derivation}
\label{sec:gr-dynamic}

\vspace{0.5cm}

\subsection*{Overview.}

\vspace{0.5cm}
This section derives Einstein–Hilbert gravity \emph{dynamically} from the WESH evolution in a fully covariant Tomonaga–Schwinger–Lindblad (TSL) representation that preserves locality, CP/TP, and no‑signaling. A pre‑geometric condition (operator level path independence of global charges; App.~H) constrains the generator. On this basis we establish three results.

\medskip
\noindent\textbf{(1) Variational fixed point (alignment).}
There exists a Lyapunov monotone for the WESH flow with a unique globally attractive fixed point characterized by the gradient‑alignment relation
\[
\partial_\mu \tau \;=\; k\,\partial_\mu \Phi,
\]
with \(k\) fixed by internal consistency (Eq.~\eqref{eq:parameter-relations}). Existence, uniqueness and global attractivity follow from mixing/primitivity (reversible spectral gap) in the KMS geometry (Thm.~\ref{thm:alignment-PartII}, App.~D).

\medskip
\noindent\textbf{(2) Hidden‑sector cancellation \(\Rightarrow\) Einstein equations.}
At the fixed point the time‑sector and nonlocal stresses cancel as a tensor identity,
\[
T^{(T)}_{\mu\nu}+T^{(\mathrm{nl})}_{\mu\nu}=0 \quad (\text{Eq.~\eqref{eq:hidden-cancel}}),
\]
yielding the matter‑only field equations up to controlled corrections,
\[
G_{\mu\nu}+\Lambda g_{\mu\nu}=8\pi G\,T^{(m)}_{\mu\nu}+\mathcal O(1/N) \quad (\text{Eq.~\eqref{eq:emergent-einstein}}).
\]

\medskip
\noindent\textbf{(3) IR uniqueness of the gravitational sector.}
Among diffeomorphism‑invariant local completions compatible with CP/Markov and causal support and truncated at second derivative order, the gravitational action is uniquely Einstein–Hilbert with \(\Lambda\). This identification is consistent with the normalization fixed by Eq.~\eqref{eq:parameter-relations}.

\medskip

\noindent The section first introduces the TSL packaging, then establishes the alignment fixed point, and uses it to derive the hidden‑sector cancellation and IR uniqueness, closing the WDW\(\,\to\,\)WESH\(\,\to\,\)(\(s\!\to\!t\)) bootstrap in a covariant route to GR. In Sec.~\ref{subsec:VIGNE} we also reformulate the cancellation as a \emph{variational generalized Nash equilibrium (v–GNE)}. In that formulation, the standard equivalence between generalized Nash problems and variational inequalities (Facchinei \& Pang, 2003) appears as a strongly monotone VI/KKT stationarity condition on the feasible set $\mathcal Z$, whose unique solution reproduces Einstein’s equations.

\medskip
\noindent\emph{Notation consistency.} We use the GKSL convention $\mathcal D[O]\rho = O\rho O^\dagger - \frac12\{O^\dagger O,\rho\}$ as in \Sec{sec:qftt-framework}.

\vspace{0.8cm}

\subsection{Atemporal symmetry on CP--TP evolution and emergent conservation}
\label{subsec:Noether-atemporal-PartII}

\vspace{0.5cm}

\begin{definition}[Atemporal symmetry of the evolution]
Let $\{\Phi_{s\to s'}\}_{s\le s'}$ be the positivity- and trace-preserving (stepwise CPTP) family solving the (possibly nonlinear) WESH master equation.  
A $*$-automorphism $\mathcal G:\mathfrak A\!\to\!\mathfrak A$ is an \emph{atemporal symmetry} if 
$\mathcal G\!\circ\!\Phi_{s\to s'}=\Phi_{s\to s'}\!\circ\!\mathcal G$ for all $s\le s'$.
\end{definition}

\begin{theorem}[Extended WESH--Noether atemporal (map-based)]
If $\mathcal G$ is an atemporal symmetry, the associated charge $\hat Q_{\mathcal G}$ (in the GNS representation) obeys
\vspace{0.3cm}
\begin{equation}
\label{eq:Noether-atemporal-PartII}
\frac{d}{ds}\,\langle \hat Q_{\mathcal G}\rangle=0,\qquad 
\frac{d}{dt}\,\langle \hat Q_{\mathcal G}\rangle=0,\quad \frac{dt}{ds}=\Gtime>0\ .
\end{equation}

\vspace{0.3cm}

\noindent Discrete symmetries (e.g.\ CPT-evenness of the dissipator) produce exact $s$-conservations (hence exact $t$-conservations when $dt/ds=\Gamma>0$); 
deformed continuous symmetries (scale/rotations tied to $N^{-2}$ weights and support) are only \emph{effective} at finite $N$ and yield drifts suppressed as $\mathcal O(1/N)$ in the emergent description.
\end{theorem}

\begin{example}[CPT as atemporal symmetry]
\label{ex:CPT}
Since $\mathcal D[\hat T^2(x)]$ and $L_{xy}=\hat T^2(x)-\hat T^2(y)$ are CPT-even, the CPT transform $\mathcal G_{\rm CPT}$ satisfies 
$\mathcal G_{\rm CPT}\!\circ\!\Phi_{s\to s'}=\Phi_{s\to s'}\!\circ\!\mathcal G_{\rm CPT}$, hence 
$d\langle \hat Q_{\rm CPT}\rangle/ds=0$ and $d\langle \hat Q_{\rm CPT}\rangle/dt=0$.
\end{example}

\vspace{0.8cm}

\subsection{Covariant packaging: Tomonaga--Schwinger--Lindblad and IR Ward--Takahashi/Bianchi}
\label{subsec:TSL-PartII}

\vspace{0.5cm}

\begin{lemma}[Tomonaga--Schwinger--Lindblad evolution and WESH current]
\label{lem:TSL-PartII}

\vspace{0.3cm}
~\\

\noindent Define a Tomonaga--Schwinger--Lindblad (TSL) evolution for a foliation $\{\Sigma_\Lambda\}$ (label $\Lambda$ is a foliation parameter, not the cosmological constant) with unit normal $n^\mu$ and induced determinant $h$ by
\[
\frac{d\rho}{d\Lambda}
= \mathcal L[\rho]
= \mathcal L_{\rm unitary}[\rho]
+ \int_{\Sigma_\Lambda} d^3x\,\sqrt{h}\; n^\mu J_\mu[\rho] ,
\]

\vspace{0.3cm}

\noindent with the WESH‑compatible current (with $\nu>0$)
\[
\begin{aligned}
J_\mu[\rho](x)
&= -\,\nu\,\mathcal D[\hat T^2(x)]\,\rho\; n_\mu(x)\\
&\quad -\!\int d^4y\,\gamma(x,y)\,C(\Psi;x,y)\,\mathcal D[L_{xy}]\,\rho\;n_\mu(x).
\end{aligned}
\]

\vspace{0.3cm}

\noindent Here $\nu$ and $\gamma(x,y)$ are understood as rate densities per unit proper
time and per unit three–volume on the leaf $\Sigma_\Lambda$, so that the
contraction $n^\mu J_\mu$ defines a proper current density and the integral
over $\Sigma_\Lambda$ in the TSL generator is dimensionally consistent.

\vspace{0.3cm}

\noindent Let $S(x)\!:=\!\nabla_\mu J^\mu(x)$. On coarse‑grained windows $L\!\gg\!\xi$ one has
$\nabla_\mu \langle J^\mu\rangle \approx 0$. This follows from the WESH–Noether continuity law \eqref{eq:H-noether}, which guarantees global neutrality of the source term $\int d^4x\,\langle S(x)\rangle=0$, combined with the finite‑range (exponential‑causal) support of $\gamma(x,y)$. This TSL evolution reproduces the WESH dissipator in the IR, is foliation‑independent, and preserves no‑signaling.
\end{lemma}

\vspace{0.3cm}
\noindent\emph{Bilocal covariance.} The bilocal term is weighted by the normal at the insertion point \(x\),
so \(J_\mu(x)\) is manifestly leaf-local. Since \(\gamma(x,y)\) has finite range \(\xi\), one has
\(n_\mu(y)=n_\mu(x)+\mathcal O(\xi/L)\) on coarse-graining windows \(L\gg\xi\); thus the difference between
a symmetric prescription and the leaf-local choice is IR-suppressed, while \(\nabla_\mu\langle J^\mu\rangle\approx 0\)
remains the correct large-scale statement.

\begin{equation}
\label{eq:finite-memory-TSL}
\mathcal K(t-t')=0\quad \text{for } |t-t'|>\tau_{\rm corr}\sim \xi/c,
\qquad 
\|K_\xi\|_{L^1({\rm causal\ LC})}=O(V_\xi)=O(\xi^4).
\end{equation}

\vspace{0.3cm}
\noindent\emph{(The $L^1$ norm refers to the unnormalized kernel $K_\xi^{\rm unnorm}=V_\xi K_\xi$; see App.~A. Used in Lemma~\ref{lem:IR-filter} as the finite-memory hypothesis.)}
\vspace{0.3cm}

\begin{proposition}[IR Ward--Takahashi $\Rightarrow$ Bianchi]
\label{prop:Ward-Bianchi-PartII}
If the coarse-grained TSL is diffeomorphism-invariant on scales $L\gg \xi$ (i.e.\ invariant under normal deformations of $\Sigma_\Lambda$ up to boundary terms), then the associated generating functional is diffeomorphism-invariant and the Ward--Takahashi identities imply 
\[
\nabla_\mu\,\langle T^{\mu\nu}_{\rm tot}\rangle=0
\]
in the IR (cf.\ Eqs.~\eqref{eq:bianchi-identity}--\eqref{eq:matter-conservation}).
\end{proposition}

\begin{definition}[Total stress tensor from the TSL-effective action]
\label{def:Ttot}
Let $S_{\rm eff}[\rho;g]$ be the coarse-grained effective action that reproduces the
TSL generator on a foliation $\{\Sigma_\Lambda\}$, in the sense that its first
functional variation in the metric yields the local/bilocal GKSL currents when
pulled back to each leaf. We define the \emph{total} stress tensor by
\begin{equation}
\label{eq:Ttot-def}
T^{\mu\nu}_{\rm tot}[\rho]
\;:=\;
\frac{2}{\sqrt{-g}}\,
\frac{\delta S_{\rm eff}[\rho;g]}{\delta g_{\mu\nu}}\,.
\end{equation}
\end{definition}

\begin{remark}[Foliation covariance and Noether content]
Under WESH--Noether and TSL covariance, the definition \eqref{eq:Ttot-def} is
foliation independent up to boundary terms: normal deformations of
$\Sigma_\Lambda$ generate the same local currents that appear in the TSL evolution,
so that Ward--Takahashi identities for diffeomorphisms imply the Bianchi--type
closure
$\nabla_\mu\langle T^{\mu\nu}_{\rm tot}\rangle=0$ on scales $L\gg\xi$ (Prop.~\ref{prop:Ward-Bianchi-PartII}).
Thus $T^{\mu\nu}_{\rm tot}$ is the Noether current of normal deformations encoded
variationally in $S_{\rm eff}$, not an \emph{ad hoc} object.
\end{remark}
\vspace{0.8cm}

\subsection{Alignment as a unique stationary fixed point of the WESH flow}
\label{subsec:alignment-variational-PartII}

\vspace{0.5cm}

\begin{definition}[Regularized WESH monotone (dimensionless form)]
\label{def:monotone-dimless}
Define
\[
\mathcal M[\rho]=\int d^4x\Big(\langle \Ttil^4(x)\rangle-\langle \Ttil^2(x)\rangle^2\Big)
\;+\!\!\iint d^4x\,d^4y\,\gamma(x,y)\,C(\Psi;x,y)\,\langle \Lxy^2\rangle,
\]
and the technical regularization
\[
\mathcal M_\epsilon[\rho]=\mathcal M[\rho]+\epsilon\,S_2[\rho],
\qquad
S_2[\rho]:=\Tr\rho^2,\quad \epsilon>0.
\]
Here $\Ttil=\hat T/\tau_s$ and $\Lxy=\Ttil^2(x)-\Ttil^2(y)$.
\end{definition}

\begin{remark}
All terms are dimensionless by construction (via $\Ttil=\hat T/\tau_s$).
This dimensionless normalization matches Appendix~D (see Def.~\ref{def:monotone-dimless}).
In dissipative many-body dynamics, Lyapunov functionals arise naturally as generalized 
free energies driving relaxation toward equilibrium (Spohn, 1991). Here $\mathcal M_\epsilon$ can be interpreted as playing that role in the WESH setting,
driving the flow toward the GR fixed point.
\end{remark}

\begin{remark}[Gradient flow structure]
The WESH dynamics admits a gradient-flow–like interpretation in quantum state space, 
reminiscent of the structures developed for quantum Markov semigroups with detailed balance 
(Carlen \& Maas, 2017), where the relative entropy to the fixed point plays the role of 
Lyapunov functional.
\end{remark}

\begin{lemma}[First variation of the WESH monotone]
Let $\mathcal{M}_\epsilon$ be as in Def.~\ref{def:monotone-dimless}. Then
\[
\frac{\delta \mathcal{M}_\epsilon}{\delta \Ttil(x)} 
= 4\Big(\langle\Ttil^3(x)\rangle - \langle\Ttil(x)\rangle\,\langle\Ttil^2(x)\rangle\Big)
+ {\cal J}^\mu(x)\,\partial_\mu \Ttil(x) + \ldots,
\]
with ${\cal J}^\mu\propto\partial^\mu\Phi$ and $\Phi(x)=\!\int\! d^4y\ K(x-y)\,C(\Psi;x,y)$ (Eq.~\eqref{eq:entanglement-potential}).
Stationarity entails $\partial_\mu \tilde\tau=(k/\tau_s)\,\partial_\mu\Phi$, hence $\partial_\mu \tau=k\,\partial_\mu\Phi$ (Eqs.~\eqref{eq:stationary-alignment}, \eqref{eq:parameter-relations}).
\end{lemma}

\begin{theorem}[Variational alignment via a WESH monotone]\label{thm:alignment-PartII}
\emph{Statement.} In the Markov window $\Delta t\gg \tau_{\rm corr}$, the WESH flow with the regularized Lyapunov functional $\mathcal M_\epsilon$ admits a \emph{unique} stationary point and it is characterized by the alignment
\medskip
\[
\partial_\mu \tau = k\,\partial_\mu \Phi,
\]
\medskip
with $k$ fixed by Eq.~\eqref{eq:parameter-relations}. This stationary point is globally attractive.
\emph{Full proof and strengthened form (monotonicity, uniqueness, global attractivity and error control) are given in Appendix~D.}
\end{theorem}

\vspace{0.3cm}
\noindent\emph{Uniqueness mechanism.} Uniqueness and global attractivity follow from mixing/primitivity (reversible spectral gap) of the WESH semigroup in the KMS geometry on the Markov window, as established in Appendix~D.

\vspace{0.2cm}
\noindent\textit{Orthogonal suppression.}
Decompose the gradient of the time field into components parallel and orthogonal to the
entanglement potential,
\[
\partial_\mu \Ttil \;=\; \alpha\,\partial_\mu \Phi \;+\; \partial_\mu \Ttil^{\perp},
\qquad 
\partial_\mu \Ttil^{\perp}\,\partial^\mu\Phi = 0.
\]
Mixing/primitivity (reversible spectral gap) implies exponential suppression of the
orthogonal component \(\partial_\mu \Ttil^{\perp}\) by the contractive WESH semigroup (Appendix~D). The only stationary direction is therefore
\(\partial_\mu \Ttil \parallel \partial_\mu \Phi\), with the proportionality constant
\(k\) fixed by the matching condition \eqref{eq:parameter-relations}.

\vspace{0.8cm}

\subsection{IR uniqueness at \texorpdfstring{$\leq 2$}{≤} derivatives (EH selection)}
\label{subsec:IR-uniqueness-PartII}

\vspace{0.3cm}
\paragraph{Noether input.}

\noindent We use WESH--Noether (Appendix~H; pre-geometric path independence) together with the TSL covariant packaging (Lemma~\ref{lem:TSL-PartII}) as symmetry constraints for the IR selection. Combined with causality/no-signaling and the Markov window (see Lemma~\ref{lem:TSL-PartII} and Eq.~\eqref{eq:finite-memory-TSL}), these constraints restrict the admissible local completion at $\leq 2$ derivatives to the Einstein–Hilbert sector.

\begin{lemma}[CP/Markov IR filter: only EH+$\Lambda$ survives in $D=4$]
\label{lem:IR-filter}
Consider a TSL-covariant Markovian evolution whose generator is GKSL with exponential-causal kernels of range $\xi$ (Sec.~\ref{subsec:TSL-PartII}). Let $\tau_{\rm corr}\sim \xi/c$ be the memory time and suppose the finite-memory condition \eqref{eq:finite-memory-TSL} holds on the wedge for the coarse-grained kernel $\mathcal K$. Then:
\begin{enumerate}[leftmargin=1.45cm,itemsep=0.25cm]
\item[(i)] Any effective gravitational action whose linearized metric equations, about an IR background, require time derivatives of order $>2$ \emph{for the metric alone} either (a) can be rewritten as a second-order system at the price of introducing additional propagating fields, or (b) propagates ghost-like degrees of freedom with indefinite kinetic energy (Ostrogradsky instabilities). Case~(b) is incompatible with complete positivity and with the contractive decay of the WESH Lyapunov functional $\mathcal M_\epsilon$ (App.~F); in case~(a) the extra modes are not part of the WESH gravitational sector (exhausted by $\hat T$, $C$ and the eigentime substrate) and must instead be treated as matter. In both situations such terms cannot define the IR gravitational dynamics generated by the WESH GKSL semigroup on the metric sector.
\item[(ii)] In $D=4$, among local diffeomorphism-invariant actions whose metric equations are second order and propagate only the two massless spin-2 polarizations, Lovelock’s theorem reduces the admissible class to Einstein--Hilbert plus cosmological constant (EH+$\Lambda$). Curvature-squared pieces ($R^2$, $R_{\mu\nu}R^{\mu\nu}$, ${\rm Weyl}^2$) enter only as radiative counterterms renormalizing couplings and/or introducing extra propagating modes as in (i), and therefore do not define the deterministic IR equations of motion under the CP/Markov WESH constraints.
\end{enumerate}
\end{lemma}

\begin{proof}[Proof sketch]
Generic curvature-squared corrections such as $R_{\mu\nu}R^{\mu\nu}$ or ${\rm Weyl}^2$ yield fourth-order-in-time metric equations whose linearized spectrum contains, besides the massless graviton, a massive spin-2 mode with negative norm (the standard higher-derivative spin-2 ghost). Recasting such dynamics as a first-order system exhibits an unbounded-from-below Hamiltonian and exponentially growing solutions, incompatible with the contractivity of completely positive quantum Markov semigroups and with the monotone decay of $\mathcal M_\epsilon$.

\noindent When higher derivatives are cured by introducing additional healthy auxiliary fields (as in $f(R)$ models with $f''(R)>0$, dynamically equivalent to GR plus a scalaron), the extra propagating modes lie outside the WESH gravitational sector and are absorbed into matter. In either case, the metric equations relevant for the GKSL/TSL WESH generator must be second order and propagate only the two GR polarizations. In $D{=}4$ Lovelock’s theorem then selects EH+$\Lambda$ as the unique local, diffeomorphism-invariant, second-order metric action. Curvature-squared pieces reduce to perturbative counterterms that renormalize couplings without altering the IR classical equations subject to the CP/Markov constraints.
Note: pure conformal gravity (Weyl$^2$ in $D=4$) is a paradigmatic example of case~(ii); Its fourth-order Bach equations propagate a massless spin--2 ghost, so that no
finite-memory GKSL representation with positive energy exists.

\end{proof}

\subsubsection*{Example: $R_{\mu\nu}R^{\mu\nu}$ and a massive spin-2 ghost}

Consider
\[
S_g = \frac{1}{16\pi G}\int d^4x\,\sqrt{-g}\,\big(R + \beta R_{\mu\nu}R^{\mu\nu}\big),
\qquad \beta\neq 0.
\]
Linearizing around Minkowski and decomposing into spin components shows that,
besides the massless graviton, the spectrum contains a massive spin-2 excitation
with negative residue in the propagator (a spin-2 ghost). The associated mode
admits exponentially growing solutions for suitable initial data and an
unbounded-from-below energy density. Such behaviour cannot be generated by a
completely positive, trace-preserving semigroup with finite memory, whose norm
is contractive in any KMS geometry; hence this type of higher-derivative
correction is excluded by Lemma~\ref{lem:IR-filter}.

\begin{lemma}[Ghost instabilities vs GKSL semigroups]
\label{lem:ghost-vs-GKSL}
Let $x(t)$ denote a linearized gravitational mode evolving under a GKSL semigroup
$e^{t\mathcal L}$: in any KMS norm one has $\|e^{t\mathcal L}\|\le 1$ for all $t\ge 0$.
A mode generated by a higher-derivative ghost, with solutions containing factors
$e^{+\gamma t}$ for some $\gamma>0$, cannot be represented in this way. Therefore
any metric theory whose linearized spectrum contains ghost modes (as in the
$R_{\mu\nu}R^{\mu\nu}$ example above) is incompatible with the CP/Markov WESH
dynamics on the gravitational sector.
\end{lemma}

\begin{remark}[Higher dimensions]
In $D>4$, the Lovelock tower may contribute additional second-order terms; the same CP/Markov filter selects the Lovelock class and excludes genuine higher-time-derivative dynamics. Our derivation in $D=4$ thus matches the standard uniqueness of EH+$\Lambda$.
\end{remark}

\begin{theorem}[EH as the unique IR covariant option under WESH constraints]
\label{thm:IR-uniqueness-PartII}
Assuming locality for $L\gg\xi$, CP--TP and complete positivity, Markovianity (Eq.~\eqref{eq:finite-memory-TSL}), no-signaling (Eq.~\eqref{eq:no-signaling}), and TSL covariance (Lemma~\ref{lem:TSL-PartII}), 
the only IR covariant metric functional with $\leq 2$ derivatives and positive energy is the Einstein--Hilbert form (up to boundary/$\Lambda$), with normalization
\[
\frac{k^2}{4\pi G}=\lambda_1+3\lambda_2\quad(\text{Eq.~\eqref{eq:parameter-relations}}).
\]
\end{theorem}

\begin{corollary}[IR selection at $\leq 2$ derivatives]
Within the Markov window ($\tau_{\rm corr}\!\sim\!\xi/c$) and under CP, the causal kernel and no‑signaling, together with Markovianity (Eq.~\eqref{eq:finite-memory-TSL}), exclude any diffeomorphism‑invariant local completion beyond the Einstein–Hilbert sector at $\leq 2$ derivatives at leading order. In particular, $R^2$, $R_{\mu\nu}R^{\mu\nu}$, and ${\rm Weyl}^2$ terms either violate the Markov locality (invoke higher-derivative memory), or require fine-tuned ghost-avoidance inconsistent with the cutoff $\xi$.
\end{corollary}

\begin{corollary}[``Quantum Lovelock'' under CP/Markov: exclusions and reasons]
\label{cor:Lovelock-quantum}
In $D=4$ the following are excluded under WESH constraints:
\begin{itemize}
\item $R^2$: for $f(R)=R+\alpha R^2$ with $f''(R)>0$ the theory is dynamically
equivalent to GR plus a healthy massive scalaron $\phi$ with
$(\Box-m_\phi^2)\phi=0$ and positive kinetic energy. In the WESH setting such
an extra degree of freedom belongs to the \emph{matter} sector and must be
encoded in $T^{(m)}_{\mu\nu}$, not in the purely geometric IR functional singled
out by the CP/Markov filter. Hence $R^2$ does not define an independent
gravitational completion at $\leq 2$ derivatives, but a scalar-matter
contribution that is already accounted for in $T^{(m)}_{\mu\nu}$.
\item $R_{\mu\nu}R^{\mu\nu}$: in the generic quadratic action
$R+\alpha R^2+\beta R_{\mu\nu}R^{\mu\nu}$ the $R_{\mu\nu}R^{\mu\nu}$ term
produces a massive spin--2 mode with \emph{negative} norm (Stelle-type ghost)
and fourth-order equations of motion. This is an Ostrogradsky-unstable sector
with an energy functional unbounded from below, incompatible with complete
positivity and with the contractivity of the GKSL/WESH semigroup on the
Markov window $\Delta t\gg\tau_{\rm corr}$.
\item $\mathrm{Weyl}^2$: in pure conformal gravity (Weyl$^2$ in $D=4$) the Bach
equations are fourth order and propagate additional spin--2 ghost degrees of
freedom. This again realizes an Ostrogradsky-unstable spin--2 sector (a
massless ghost in the conformal limit), which falls under case~(ii) of
Lemma~\ref{lem:IR-filter} and is excluded by the CP/Markov IR filter.
\end{itemize}
Thus, the admissible class reduces to $R-2\Lambda$ with normalization fixed by
Eq.~\eqref{eq:parameter-relations}.
\end{corollary}

\vspace{0.3cm}

\subsection{Quantum features and falsifiability: connection to gravitational wave squeezing}

\vspace{0.5cm}
Recent analyses in the effective field theory (EFT) of gravity have highlighted that nonlinear self-interactions of the Einstein–Hilbert action can act as sources of genuinely nonclassical gravitational-wave states. In particular, Guerreiro (2025) shows that high-frequency gravitational waves propagating on weakly curved backgrounds undergo a three-wave mixing process where cubic GR self‑interactions generate squeezed and entangled states, with squeezed variances below the vacuum level ($V_{\mathrm{sq}}<1$) and logarithmic-negativity oscillations between coupled modes. These results clarify that nonlinear general relativity can both \emph{produce} and \emph{detect} quantum correlations among gravitational degrees of freedom, even in the geometric-optics limit.

\vspace{0.3cm}
\noindent The QFTT–WESH framework arrives at closely related structures from the opposite direction. Rather than quantizing perturbations on a pre-given metric, WESH begins with pre-geometric quantum correlations and reconstructs spacetime through eigentime production and hidden-sector cancellation. Yet, the quantum signatures that emerge in the two approaches display notable parallels. The $\cos^2\theta$ angular modulation that appears in WESH collective decoherence protection (Sec.~\ref{sec:experimental}) plays a role similar to the angular dependence of Guerreiro's three-mode couplings, where polarization tensors mediate the redistribution of quantum fluctuations across interacting frequency bands. In both cases, the angular structure encodes how gravitational dynamics select phase-sensitive quantum correlations.

\vspace{0.3cm}
\noindent Moreover, the quadratic Hermitian dissipator $\mathcal{D}[\hat T^2]$ that drives eigentime localization in WESH naturally builds non-Gaussian correlations between modes. After the continuum limit, where the same composite operator contributes to the cubic Einstein–Hilbert vertex, this structure inherits exactly the ingredients required to generate squeezed and entangled states of the type identified in the EFT treatment. In other words, the microscopic WESH mechanism and the macroscopic EFT analysis isolate the same operational cause: nonlinear graviton couplings acting as parametric amplifiers.

\vspace{0.3cm}
\noindent This comparison yields a sharp falsifiability requirement. Any emergent-gravity framework claiming to reproduce general relativity in the infrared must also accommodate the quantum features generated by nonlinear gravitational dynamics, squeezing, multipartite entanglement, and state-dependent mode-coupling,
as demonstrated in the EFT regime. Structurally, QFTT–WESH satisfies this requirement: its pre-geometric correlations, dissipative generator, and continuum limit contain the requisite non-Gaussianity to support such states. A full quantitative comparison, matching WESH spectral weights to Guerreiro's squeezed-state parameters, is a natural next step.

\vspace{0.8cm}

\subsection{Hidden-sector cancellation as an operator identity}
\label{subsec:hidden-operator-PartII}

\vspace{0.5cm}

\begin{theorem}[Operator-level hidden-sector cancellation at the fixed point]
\label{thm:operator-cancellation-PartII}
Let $\mathfrak A_{\rm wedge}$ be the von Neumann algebra of the near-horizon wedge. Assume:
\begin{enumerate}[leftmargin=1.45cm,itemsep=0.25cm]
\item[(i)] The stationary state is the unique wedge KMS state $\sigma_{\rm KMS}\!\propto\! e^{-\beta_H H_R}$ (Sec.~\ref{sec:KMS-Rindler}, Thm.~\ref{thm:KMS-unique}), and $\mathfrak A_{\rm wedge}$ is a Type~III$_1$ factor.
\item[(ii)] The WESH wedge semigroup is primitive (irreducible) and detailed-balanced with respect to $\sigma_{\rm KMS}$ (Sec.~\ref{sec:KMS-Rindler}), so that the GNS vector $\Omega_{\rm KMS}$ is \emph{cyclic and separating} for $\mathfrak A_{\rm wedge}$, as required by Tomita–Takesaki theory. Modular theory (Takesaki, 1970) provides the operator–algebraic foundation for this structure and its KMS implementation.
\item[(iii)] Variational alignment holds at the global attractor (Thm.~\ref{thm:alignment} / Thm.~\ref{thm:alignment-PartII}), and the cancellation matching 
\eqref{eq:cancellation-matching} is satisfied. In the geometric IR, together with
\eqref{eq:parameter-relations} this identifies $\zeta=4\pi G$ (i.e.\ $\beta_G=\zeta^{-1}$).
\end{enumerate}
Then, in the sense of quadratic forms on a common core,
\begin{equation}
\label{eq:op-cancel}
T^{(T)}_{\mu\nu} \;+\; T^{(\mathrm{nl})}_{\mu\nu} \;=\; 0 \qquad \text{on } \mathfrak A_{\rm wedge}.
\end{equation}
At finite $N$, one has $\|\,T^{(T)}+T^{(\mathrm{nl})}\,\|=\mathcal O(1/N)$ on coarse-grained windows $L\gg\xi$, hence
\[
G_{\mu\nu}+\Lambda g_{\mu\nu}
\;=\;
8\pi G\,T^{(m)}_{\mu\nu}\;+\;\mathcal O(1/N)\,.
\]
\end{theorem}

\begin{proof}[Proof sketch]
By Sec.~\ref{sec:KMS-Rindler} the wedge KMS state is unique and the WESH semigroup is primitive/detailed-balanced; thus the GNS representation is standard and $\Omega_{\rm KMS}$ is cyclic and separating for $\mathfrak A_{\rm wedge}$. Using Thm.~\ref{thm:alignment} (variational alignment) and the cancellation matching \eqref{eq:cancellation-matching}, one shows that for all local polynomials \(\mathcal O\) with support in the wedge 
(which generate a $\sigma$-weakly dense $*$-subalgebra of \(\mathfrak A_{\rm wedge}\)),
\[
\langle\Omega_{\rm KMS},\,(T^{(T)}_{\mu\nu}+T^{(\mathrm{nl})}_{\mu\nu})\;\mathcal O\;\Omega_{\rm KMS}\rangle=0.
\]
Cyclicity and separability then imply $(T^{(T)}_{\mu\nu}+T^{(\mathrm{nl})}_{\mu\nu})=0$ as a quadratic form on a common core (Tomita--Takesaki), i.e.\ an operator identity on the wedge.\qedhere
\end{proof}

\begin{example}[Cancellation of $(\partial\Phi)^2$ terms (schematic)]
\label{ex:cancel}
In a local frame, the quadratic time-sector contribution reads 
$T^{(T)}_{00} \sim \zeta^{-1}\,(\partial_0 \hat T)^2 
= \frac{k^2}{\zeta}\,(\partial_0\Phi)^2$ under alignment,
while the nonlocal backreaction yields 
$T^{(\mathrm{nl})}_{00} \sim -\,\beta_G\,\delta_{g^{00}}[S_{\rm ent}] 
\sim -\,\frac{k^2}{\zeta}\,(\partial_0\Phi)^2$.
At the fixed point, combining the cancellation matching \eqref{eq:cancellation-matching} with the GR normalization \eqref{eq:parameter-relations} identifies $\beta_G=\zeta^{-1}$ (equivalently, $\zeta=4\pi G$).
Hence $T^{(T)}_{00}+T^{(\mathrm{nl})}_{00}=0$ at continuum, and similarly for spatial components.
\end{example}

\noindent\emph{Symmetry protection.} The cancellation is protected by the atemporal symmetry (WESH--Noether, Eq.~\eqref{eq:H-noether}): any deformation that breaks the commutant constraints or the alignment $\partial_\mu\tau\propto\partial_\mu\Phi$ violates conservation and spoils the emergent Einstein equation.

\vspace{0.5cm}
\noindent\textit{Tensorial closure (IR uniqueness under alignment).}
The cancellation established above is not restricted to a single component (e.g.\ $00$):
at two-derivative order, the WESH primitives admit only one admissible
\emph{gradient-sector} stress-tensor structure.

\begin{lemma}[Tensorial closure of the gradient sector]
\label{lem:tensorial-closure}
Work on a coarse-grained window $L\gg\xi$ in the Markov/IR regime, and adopt the
$\le 2$-derivative truncation used in the IR completion. Assume stationary
gradient alignment $\nabla_\mu \tau = k\,\nabla_\mu \Phi$
(Eq.~\eqref{eq:stationary-alignment}). Then the derivative-dependent parts of the
hidden-sector stresses are forced to share the universal scalar-kinetic tensor:
\begin{equation}
\label{eq:tensorial-closure-form}
\begin{split}
T^{(T)}_{\mu\nu}\big|_{\nabla\Phi}
&=\frac{k^2}{\zeta}\Big(\nabla_\mu\Phi\,\nabla_\nu\Phi-\tfrac12 g_{\mu\nu}(\nabla\Phi)^2\Big),\\[4pt]
T^{(\mathrm{nl})}_{\mu\nu}\big|_{\nabla\Phi}
&=-(\lambda_1+3\lambda_2)\Big(\nabla_\mu\Phi\,\nabla_\nu\Phi-\tfrac12 g_{\mu\nu}(\nabla\Phi)^2\Big),
\end{split}
\end{equation}
up to higher-derivative corrections suppressed by $\xi/L$ and finite-$N$ corrections
(collectively denoted $O(\xi/L)+O(1/N)$).
Any remaining zero-derivative contribution is proportional to $g_{\mu\nu}$ and is
therefore absorbed into $\Lambda_{\rm eff}$ (Appendix~J).
\end{lemma}

\begin{proof}[Sketch]
Alignment makes the time sector depend on the system only through the scalar field
$\Phi$ at first-derivative order, since $\nabla \tau = k\,\nabla\Phi$; inserting
this into the definition of $T^{(T)}_{\mu\nu}$ yields the first identity in
Eq.~\eqref{eq:tensorial-closure-form}.
For the nonlocal sector, finite-range Markov locality (kernel support $\sim\xi$) 
implies that on $L\gg\xi$ the IR contribution of $S_C+S_E$ admits a \emph{local} 
derivative expansion in the single scalar $\Phi(x)=\int d^4y\,K(x-y)\,C(\Psi;x,y)$.
Diffeomorphism invariance together with the $\le2$-derivative truncation leaves a
unique nontrivial kinetic term proportional to $(\nabla\Phi)^2$; its metric variation
is fixed and produces exactly the canonical tensor
$\nabla_\mu\Phi\nabla_\nu\Phi-\tfrac12 g_{\mu\nu}(\nabla\Phi)^2$.
The coefficient of this term is the IR response combination $\lambda_1+3\lambda_2$
(cf.\ Eq.~\eqref{eq:parameter-relations}), giving the second identity in
Eq.~\eqref{eq:tensorial-closure-form}.
\end{proof}

\noindent Consequently, once the cancellation matching
$\frac{k^2}{\zeta}=\lambda_1+3\lambda_2$ is imposed
(Eq.~\eqref{eq:cancellation-matching}), the entire gradient-sector tensor cancels,
not just a particular component; residuals are controlled by $O(\xi/L)+O(1/N)$ as
claimed.

\vspace{0.3cm}
\noindent\emph{Discrete-to-continuum control.}
On scales $L\gg\xi$, eigentime events form a quasi-uniform inhomogeneous Poisson mesh with spacing $\ell_{\rm Eig}\!\sim\!\lambda_{\rm Eig}^{-1/4}$ and fill distance $h_N=\mathcal O(N^{-1/4})$. With kernel regularization the fields entering the limit are Lipschitz, so that discrete-to-continuum errors satisfy $\|\mathcal F_{\rm disc}-\mathcal F_{\rm cont}\|_{L^\infty}=\mathcal O(N^{-1/4})$, and curvature errors scale as $\mathcal O(\ell_{\rm Eig}^2)$. These bounds yield the stated $\mathcal O(1/N)$ precision for the hidden-sector cancellation on coarse-grained windows.
\vspace{0.8cm}

\subsection{Variational GNE formulation of the hidden-sector cancellation}
\label{subsec:VIGNE}

\paragraph*{Motivation.}
Up to now, hidden–sector cancellation has been shown as a dynamical consequence of alignment and matching. Here we repackage it as a \emph{variational fixed point}: the WESH flow minimizes a regularized Lyapunov functional $\mathcal M_\epsilon$ under WESH–Noether/TSL constraints, so that cancellation is the \emph{unique} stationary solution of a strongly monotone VI—equivalently, a variational generalized Nash equilibrium (v–GNE). Strong monotonicity is guaranteed by mixing/primitivity (reversible spectral gap) in the KMS geometry (Appendix~D).

\begin{definition}[Feasible set, operator, and VI/KKT]\label{def:VIGNE}
Let the feasible set be
\begin{align}
\mathcal Z := \Big\{\, z=(\rho,\hat T)\ :\ 
&\rho\ge0,\ \Tr\rho=1,\notag\\
&\text{WESH--Noether (App.~H, Eq.~\eqref{eq:H-noether})},\notag\\ 
&\text{TSL covariance (Sec.~\ref{subsec:TSL-PartII})}\Big\},
\end{align}
a closed subset of the natural state/field space. For any $z=(\rho,\hat T)\in\mathcal Z$ we define the entanglement potential
$\Phi[\rho,\hat T]$ by the finite-range Yukawa mediator $K$ and the normalized Rényi-2 gate $C\in[0,1]$
(Sec.~4.1, Eq.~\eqref{eq:entanglement-potential}, App.~F). Define the operator
\begin{equation}
F(z) := \nabla\mathcal M_\epsilon\big(\rho,\hat T,\Phi[\rho,\hat T]\big)\big|_{\mathcal Z},
\quad z:=(\rho,\hat T)\in\mathcal Z,
\end{equation}
where $\mathcal M_\epsilon$ is the regularized WESH Lyapunov functional of Appendix~D.
Endow the wedge algebra with the reversible KMS inner product
\begin{equation}
\langle A,B\rangle_{*} := \Tr\!\big(\sigma_{\rm KMS}^{1/2}A^\dagger\sigma_{\rm KMS}^{1/2}B\big)\,,
\label{eq:KMS-inner}
\end{equation}
the GNS/KMS inner product on the wedge algebra (Sec.~\ref{sec:KMS-Rindler}).
The \emph{VI problem} $\mathrm{VI}(F,\mathcal Z)$ is: find $z^\star\in\mathcal Z$ such that
\begin{equation}
\big\langle F(z^\star),\, z-z^\star\big\rangle_{*} \ge 0
\quad \forall\,z\in\mathcal Z,
\label{eq:VI}
\end{equation}
equivalently the KKT stationarity for the constrained minimization of $\mathcal M_\epsilon$ on $\mathcal Z$.
\end{definition}

\begin{lemma}[Strong monotonicity and contraction]\label{lem:monotone}
Under CP--TP and complete positivity, finite--memory/Markov locality
(Eq.~\eqref{eq:finite-memory-TSL}) and mixing/primitivity (reversible spectral gap) in the KMS geometry,
the operator $F=\nabla\mathcal M_\epsilon$ is (strongly) monotone on $\mathcal Z$, i.e.
\begin{equation}
\big\langle F(z)-F(w),\, z-w\big\rangle_{*} \ge \mu\,\|z-w\|_{*}^2
\quad(\mu>0).
\label{eq:strongmono}
\end{equation}
Moreover, the wedge--restricted GKSL semigroup is contractive in the KMS geometry, and detailed balance
(Sec.~\ref{subsec:KMS-fixed-unique}) implies a positive spectral gap on the orthogonal complement of the fixed point.
\end{lemma}

\begin{theorem}[Unique v-GNE and hidden-sector cancellation]\label{thm:VIGNE-cancel}
Let $\mathcal Z$ and $F=\nabla\mathcal M_\epsilon$ be as above. Then:
\begin{enumerate}[leftmargin=1.6cm,itemsep=0.25cm]
\item The VI~\eqref{eq:VI} admits a \emph{unique} solution $z^\star\in\mathcal Z$. 
Existence follows from the compactness of the sublevel sets of $\mathcal M_\epsilon$ in the KMS topology
and the continuity of $F$. Uniqueness follows from the strong monotonicity \eqref{eq:strongmono} of $F$ on $\mathcal Z$, induced by mixing/primitivity (reversible spectral gap) in the KMS geometry.
\item The KKT stationarity at $z^\star$ enforces the \emph{gradient alignment}
$\partial_\mu\tau = k\,\partial_\mu\Phi$ (Thm.~\ref{thm:alignment}, Appendix~D), and, substituting this relation in the
stress tensors (Eqs.~\eqref{eq:stress-energy}--\eqref{eq:nonlocal-stress}), one obtains, at continuum and with the cancellation matching
\eqref{eq:cancellation-matching}, the operator identity
\[
T^{(T)}_{\mu\nu}(x)+T^{(\mathrm{nl})}_{\mu\nu}(x)=0
\]
in the sense of local quadratic forms on the wedge algebra (Sec.~\ref{subsec:hidden-operator-PartII}).
\item The semigroup $e^{s\mathcal L}$ converges exponentially to $z^\star$ in the KMS geometry,
with mixing rate set by the reversible spectral gap (Sec.~\ref{subsec:KMS-fixed-unique}).
\end{enumerate}
\end{theorem}

\begin{corollary}[Robustness and coarse-grained stability]\label{cor:robust}
For any fixed $\epsilon>0$, the solution $z^\star(\epsilon)$ of $\mathrm{VI}(F,\mathcal Z)$ is
Lyapunov--stable; as $\epsilon\downarrow 0$, $z^\star(\epsilon)\to z^\star$ by $\Gamma$--convergence
(Appendix~D). Hence the cancellation persists under coarse--graining/noise regularization, and the
Einstein--Hilbert phase is the unique, robust IR fixed point under WESH constraints.
\end{corollary}

\begin{remark}[Quantum-coordinated equilibrium and Araki/KMS]\label{rem:Araki-bridge}
On the wedge, Araki's relative entropy w.r.t.\ the KMS state is a Lyapunov; its first--law variation
yields the same VI stationarity near equilibrium. Null--plane Markov saturation (Sec. 6.2)
means that conditional relative entropy vanishes only on scalars, as characterized by the Petz recovery map
(Petz, 1986), enforcing the uniqueness of the fixed point.
Entanglement thus acts as the \emph{coordination resource} that allows the local (time) and nonlocal (backreaction)
sectors to reach the unique v-GNE: the operator cancellation is the \emph{quantum--coordinated} equilibrium on the wedge.
\end{remark}

\vspace{1cm}

\begin{tcolorbox}[
  enhanced, breakable,
  title={\centering\textbf{Box 7: The WESH Bridge – From Timeless Constraint to Spacetime and Holography}},
  colback=gray!4, colframe=black, boxrule=0.6pt, arc=2pt,
  left=8pt, right=8pt, top=8pt, bottom=8pt,
  before skip=10pt, after skip=10pt
]
\textbf{Start (timeless):} Wheeler--DeWitt constraint
\[
\hat H_{\rm tot}\ket{\Psi}=0 \qquad \text{(no time, no evolution)}
\]
\textbf{Unique extension (pre-geometric dynamics):} WESH master equation (Eq.~\eqref{eq:wesh-master})
\begin{align}
\partial_s\rho &= \mathcal L[\rho], \notag\\
\mathcal L &= \int\! d^4x\, \mathcal D_{\hat T^2(x)}[\rho] + \iint\! d^4x d^4y\, \gamma(x,y)C(\Psi;x,y)\,\mathcal D_{L_{xy}}[\rho], \notag
\end{align}
with Hermitian jumps $\hat T^2(x)$, $L_{xy}=\hat T^2(x)-\hat T^2(y)$, and satisfying the fundamental constraints (Sec.~1): 
(i)~global charge neutrality, (ii)~pre-geometric Noether conservation, and (iii)~CPT-causality with collective coherence (CS).
This uniquely fixes the WESH dissipative completion and yields emergent physical time.

\vspace{3mm}
\textbf{Emergent chain:}
\vspace{2mm}

\begin{tabular}{@{}l@{\hspace{8mm}}l@{}}
$dt/ds = \Gamma[\Psi] > 0$ & (physical time emerges)\\[4pt]
$\partial_\mu \tau = k\,\partial_\mu\Phi$ & (unique attractor: gradient alignment)\\[4pt]
$T^{(T)}_{\mu\nu} + T^{(\mathrm{nl})}_{\mu\nu} = 0$ & (hidden-sector cancellation, operator level)\\[6pt]
\multicolumn{2}{@{}l}{$\Downarrow$ \hspace{10mm} (IR CP/Markov filter in $D{=}4$)}\\[6pt]
$G_{\mu\nu}+\Lambda g_{\mu\nu} = 8\pi G\,T^{(m)}_{\mu\nu}$ & (Einstein eq.; matching $k$ via Eq.~\eqref{eq:parameter-relations})\\[6pt]
\multicolumn{2}{@{}l}{$\Downarrow$ \hspace{10mm} (Rindler wedge + KMS + RAQ)}\\[6pt]
$S_{\rm BH} = \dfrac{A}{4L_P^2}+c_{\rm log}\,\ln\!\dfrac{A}{L_P^2}+\cdots$ & (Bekenstein--Hawking law)
\end{tabular}

\vspace{5mm}
\textbf{Interpretation.} \emph{General Relativity and black hole thermodynamics are not independent inputs. They arise as dual macroscopic consequences of resolving the timeless Wheeler--DeWitt constraint through WESH: the same pre-geometric, CP/Markov, entanglement-driven dynamics yields both the Einstein--Hilbert action and the Bekenstein--Hawking area law.}
\end{tcolorbox}
\vspace{1cm}

\section{Black Hole Thermodynamics: a Cross-Scale Validation of WESH}
\label{sec:BH-validation}
\vspace{0.8cm}

\subsection*{Overview}

\vspace{0.5cm}
If WESH is a fundamental description of quantum spacetime, its microscopic 
structure—Planck-scale correlation length $\xi \simeq L_P$, bilocal dissipator 
$L_{xy} = \hat{T}^2(x) - \hat{T}^2(y)$, and pre-geometric charge conservation via 
WESH–Noether—must remain consistent across all gravitational scales. Black-hole 
horizons provide the decisive test: quantum information, geometry, and thermodynamics 
converge in regimes separated by 45–50 orders of magnitude from the Planck length.

This section proves that black-hole thermodynamics is not an external input but a 
necessary consequence of the same dynamical principles that govern Planckian eigentimes. 
The Bekenstein–Hawking entropy
\begin{equation}
S_{\rm BH} = \frac{A}{4L_P^2} + \gamma \ln \frac{A}{L_P^2} + \cdots
\end{equation}
emerges \textbf{uniquely}, without additional free parameters and without ad hoc 
projections. The derivation requires no thermal ansatz and no fine-tuning. Unlike 
Euclidean path-integral approaches, which target equilibrium, the present Lorentzian 
treatment describes dynamical relaxation toward equilibrium, including slowly evolving 
horizons. The argument proceeds as follows.

\begin{enumerate}[leftmargin=1.6cm,itemsep=0.3cm]

\item \textbf{Pre-geometric consistency.} WESH--Noether enforces path independence of global charges in the auxiliary $s$-flow. This pre-geometric constraint, formulated at Planck scale, must hold universally if WESH is fundamental.

\item \textbf{Thermal emergence.} In a Rindler wedge, the same GKSL dynamics admits a unique KMS fixed point at $\beta_H=2\pi/\kappa$. This is precisely the Kubo--Martin--Schwinger characterization of equilibrium for quantum statistical mechanics on operator algebras (Haag, Hugenholtz, \& Winnink, 1967). No thermal prior is imposed; KMS emerges from modular covariance of the $\hat{T}^2$-channel.

\item \textbf{Universal prefactor.} The $1/4$ coefficient arises from bipartite halving across the horizon ($1/2$) combined with RAQ projection, continuous + discrete $\mathbb{Z}_2$, onto the horizon-even physical subspace (another $1/2$), both inherited from constraint structure.

\item \textbf{Robustness and EFT control.} Spectral concentration bounds yield $\bar{s}=1+\mathcal{O}(L_P^2/A)$. Curvature-induced kernel deformations are suppressed by $(L_P/M)^4$ for macroscopic horizons, confirming $\xi\simeq L_P$ as a controlled approximation. Here $\xi$ is the correlation length of Table~1 (Sec.~1), understood as its infrared fixed-point value; possible ultraviolet running can be absorbed into the renormalization of $G$, leaving the Bekenstein--Hawking combination $A/(4G)$ invariant.

\item \textbf{Subleading structure.} Logarithmic corrections $\gamma\ln(A/L_P^2)$ are fixed by the heat-kernel expansion of the gauge-fixed quadratic form $\Delta_{\rm phys}$, tied to the variational alignment (Thm.~\ref{thm:alignment-PartII}, Appendix~D). For regular horizons, $\gamma=\sigma_{\rm phys}\chi(\mathcal{H})$ with $\chi(S^2)=2$.

\item \textbf{Extension to rotating horizons.} The framework extends to Kerr via co-rotating KMS structure ($\chi=\partial_t+\Omega_H\partial_\phi$) and hidden-symmetry covariance. The leading $A/(4L_P^2)$ is universal; spin dependence enters only through extrinsic-curvature terms in $\gamma(J)$.

\item \textbf{Holography as chronogenetic stability.} The bound $S[\mathcal{H}]\le A/(4L_P^2)+\cdots$ is \emph{derived} as a stability condition: the chronogenetic requirement $\Gamma[\Psi]>0$ (forward time production) implies the holographic entropy bound. This aligns with the covariant formulation of the entropy bound (Bousso, 1999), where entropy on light-sheets cannot exceed the area of the boundary surface.

\end{enumerate}

\vspace{0.3cm}

\noindent\textbf{Logical status.} This is not a ``WESH model of black holes.'' It is a \textbf{cross-scale consistency theorem}: \emph{if} WESH is fundamental at $L_P$, \emph{then} black hole thermodynamics must hold as observed, with the numerical coefficients derived above. Any deviation, in the leading $1/4$, in the KMS temperature $\beta_H=2\pi/\kappa$, in the logarithmic coefficient $\gamma$, or in the holographic bound, would falsify the framework. Conversely, its reproduction on a macro scale constitutes a non-trivial validation of WESH as a candidate theory of quantum spacetime. 

\vspace{0.3cm} 
\noindent\textbf{Assumptions and falsifiability.} The derivation requires: (i)~Hartle--Hawking KMS regularity in the near-horizon window; (ii)~exponential-causal support of the WESH kernel with $\xi\simeq L_P$ undeformed to leading order; (iii)~gauge-invariant bipartite splitting implemented via RAQ projection. These are not fine-tunings but \emph{consistency requirements} for the scaling hypothesis. Curvature-induced deformations are treated within effective field theory and shown to be negligible ($\mathcal{O}((L_P/M)^4)$) for macroscopic horizons, while providing falsifiable signatures, deviations in $\gamma$, shifts in spectral weights, breakdown of KMS balance, for near-Planckian or rapidly rotating black holes. 

\vspace{0.3cm} 
\noindent\textbf{Roadmap.} The section proceeds from pre-geometric conservation (Sec.~\ref{sec:wesh-noether}) through wedge KMS structure (Sec.~\ref{sec:KMS-Rindler}) to RAQ projection (Sec.~\ref{sec:RAQ-quarter}) and spectral analysis (Sec.~\ref{sec:phase4-leading-robust}), yielding the entropy law and its holographic interpretation.

\vspace{0.3cm}
\begin{table}[H]
\centering
\label{tab:BH-chain}
\begin{tabular}{p{0.07\textwidth}p{0.43\textwidth}p{0.43\textwidth}}
\hline
Step & Fundamental input / tool & Output / consequence \\
\hline
1 & Pre-geometric coherence: WESH–Noether ($\mathcal L^\dagger[\hat Q_a]=0$) & Athermal GKSL class commuting with charges; path-independence in $s$.\\
2 & Modular covariance in a Rindler wedge; null-plane Markov property & Unique KMS fixed point at $\beta_H=2\pi/\kappa$.\\
3 & Bipartite pairing $+$ RAQ even projection ($\mathbb{Z}_2$ swap) & Universal prefactor $1/4$ in $S_{\rm BH}=A/(4L_P^2)+\cdots$.\\
4 & One-loop Hessian $\Delta_{\rm phys}$ $+$ replica/heat kernel & Logarithmic correction $\gamma\ln(A/L_P^2)$.\\
5 & Chronogenetic split $\Gamma=\Gamma_{\rm loc}+\Gamma_{\rm bi}$ $+$ KMS calibration & Holographic bound as stability condition; saturation at HH.\\
\hline
\end{tabular}
\end{table}

\vspace{0.8cm}

\subsection{Pre‑geometric foundation: the WESH–Noether principle (path independence)}

\vspace{0.5cm}

\paragraph*{Overview.}
This subsection fixes the pre‑geometric conservation law that selects the physically admissible $s$–dynamics in QFTT–WESH before the emergence of physical time $t$. The law is imposed directly on the GKSL generator and expresses \emph{path independence} of global currents (no vorticity) in the auxiliary flow. It is formulated as an algebraic necessity—on par with the closure of a constraint algebra—rather than as a thermodynamic prior. Near‑horizon thermality (KMS) and detailed balance are \emph{not} assumed here; they will follow as stationary properties of the wedge dynamics once the pre‑geometric generator is constrained by this principle and locality/causality.

\paragraph{GKSL generator and Heisenberg dual.}
The $s$–evolution of the state $\rho(s)$ is generated by

\begin{equation}
\partial_s \rho \;=\; \mathcal L[\rho]
\;=\; -\,i\,[H_{\mathrm{eff}},\rho]
\;+\; \sum_{\alpha}\Big(L_\alpha\,\rho\,L_\alpha^\dagger - \tfrac{1}{2}\bigl\{L_\alpha^\dagger L_\alpha,\rho\bigr\}\Big).
\label{eq:gksl}
\end{equation}
\vspace{0.3cm}

\noindent where local and bilocal channels are admitted (e.g.\ $L_x\!\sim\!\hat T^2(x)$ and $L_{xy}\!\sim\!\hat T^2(x)-\hat T^2(y)$) with exponential‑causal kernel $K_\xi$ (Sec.~1), and $H_{\mathrm{eff}}$ collects coherent terms and gauge contributions (Sec.~4). The Heisenberg dual acts as

\vspace{0.3cm}
\begin{equation}
\mathcal L^\dagger[\hat O] = i\,[H_{\rm eff},\hat O] \;+\; \sum_\alpha\Big(L_\alpha^\dagger\,\hat O\,L_\alpha - \tfrac12\{L_\alpha^\dagger L_\alpha,\hat O\}\Big).
\label{eq:heis-dual}
\end{equation}
\vspace{0.3cm}

\begin{axiom}[WESH–Noether: pre‑geometric conservation/path independence]\label{ax:WESH-Noether}
Let $\{\hat Q_a\}$ be the set of global first‑class charges (total energy/momentum, total $T$‑charge, Hamiltonian/diffeomorphism constraints, etc.). The physical pre‑geometric dynamics is the maximal subclass of GKSL generators that \emph{annihilates} every charge in the adjoint:

\begin{equation}
\label{eq:H-noether}
\mathcal L^\dagger[\hat Q_a]=0 \qquad \text{for all $a$}.
\end{equation}
\end{axiom}

\begin{lemma}[Equivalent formulations]\label{lem:equivalences}
The following are equivalent to \eqref{eq:H-noether}:
\begin{enumerate}[leftmargin=1.6cm]
\item \textbf{Conservation in mean:} $\displaystyle \frac{d}{ds}\Tr(\rho\,\hat Q_a)=\Tr(\rho\,\mathcal L^\dagger[\hat Q_a])=0$ for every state $\rho$.
\item \textbf{Path independence (no vorticity):}
the functional one-form $\omega_a[\rho]:=\Tr(\hat Q_a\,\mathcal L[\rho])\,ds$ vanishes identically,
hence $\oint \omega_a=0$ on any closed loop in state space.
\item \textbf{Structural constraint:} in canonical (minimal) GKSL form, a sufficient condition is
\begin{equation}
\label{eq:comm-suff}
[H_{\rm eff},\hat Q_a]=0,\qquad [L_\alpha,\hat Q_a]=0\ \ \forall\alpha.
\end{equation}
Under standard linear‑independence/minimality assumptions for the channel set, \eqref{eq:comm-suff} is also necessary up to unitary mixing within the Lindblad span.
\end{enumerate}
\end{lemma}

\begin{remark}[Geometric content]
Equation~\eqref{eq:H-noether} enforces the absence of circulation of global charges along the $s$‑flow: the auxiliary evolution is \emph{path independent} with respect to $\{\hat Q_a\}$. This elevates conservation to an internal consistency requirement of the pre‑geometric generator (cf.\ the role of first‑class constraints).
\end{remark}

\begin{proposition}[Closedness and monotonicity]\label{prop:selector}
Let $\mathfrak G_{\rm phys}:=\{\mathcal L:\ \mathcal L^\dagger[\hat Q_a]=0\ \forall a\}$. Then:

\begin{enumerate}[leftmargin=1.6cm]
\item $\mathfrak G_{\rm phys}$ is convex and stable under unitary conjugation; it preserves the level sets of $\{\Tr(\rho\,\hat Q_a)\}$.
\item For any $\mathcal L\in\mathfrak G_{\rm phys}$, the CPTP semigroup $e^{s\mathcal L}$ admits a Lyapunov functional $\mathcal M[\rho]$ (e.g.\ a relative entropy with respect to the fixed‑point manifold) such that $\partial_s\mathcal M\le 0$, with equality iff $\rho$ is stationary (LaSalle invariance in quantum state space).
\end{enumerate}
\end{proposition}

\begin{proof}[Sketch]
(1) follows from linearity of \eqref{eq:H-noether}. For (2), contractivity of CPTP maps ensures monotonicity of suitable $f$‑divergences; invariance of $\{\hat Q_a\}$ forbids coherent drift along constrained directions, so fixed points are precisely the zero‑derivative locus.
\end{proof}

\begin{lemma}[Charge-preserving descent of relative entropy]\label{lem:charges-relent}
Let $\sigma$ be any stationary state of the pre-geometric flow on $\mathfrak A$, with the same global 
charges as $\rho_0$. If $\mathcal L^\dagger[\hat Q_a]=0$ for all $a$ (WESH–Noether), then along 
$\rho(s)=e^{s\mathcal L}{}^\ast[\rho_0]$ the relative entropy $S_{\mathfrak A}(\rho(s)\Vert\sigma)$ 
is non-increasing, with trajectories confined to the charge level sets. 
Stationarity conditions $\delta S_{\mathfrak A}(\rho\Vert\sigma)=0$ under local variations 
reduce to the gradient-alignment equations of Theorem~\ref{thm:alignment}.
\end{lemma}

\begin{proof}[Sketch]
Noether conservation fixes the admissible directions in state space; contractivity of $S(\cdot\Vert\sigma)$ under the CPTP semigroup yields monotonic descent. First-order optimality on the constrained manifold enforces collinearity of functional gradients, reproducing \eqref{eq:grad-align}.
\end{proof}

\begin{theorem}[Gradient alignment as the attracting fixed manifold]\label{thm:alignment}
Assume locality/causality of the mediator ($K_\xi$ exponential‑causal, Sec.~1) and impose \eqref{eq:H-noether} for energetic and gauge charges. Then the attracting stationary manifold of the $s$‑flow is characterized by the \emph{gradient alignment}
\begin{equation}
\label{eq:grad-align}
\partial_\mu \tau \;=\; k\,\partial_\mu \Phi,
\end{equation}
with $k$ fixed by the normalizations used in Sec.~4. On this manifold, hidden‑sector cancellation yields the effective action reproducing Einstein dynamics.
\end{theorem}

\begin{proof}[Sketch]
This theorem is a direct corollary of the global alignment result proved in 
Appendix~D (Thm.~\ref{thm:alignment-PartII}). 
There, a WESH monotone $\mathcal M_\epsilon$ is constructed and one shows that the 
stationarity condition $\delta\mathcal M_\epsilon/\delta\tilde T=0$ forces 
collinearity of the functional gradients of $\hat T$ and the geometric potential 
$\Phi$. Imposing the WESH–Noether constraint \eqref{eq:H-noether} eliminates 
forbidden drift directions and pins the stationary manifold to the solutions of 
\eqref{eq:grad-align}. The proportionality constant $k$ is uniquely fixed by the 
normalizations in Sec.~4, where the emergent Einstein sector is matched.
\end{proof}

\begin{remark}[Role of RAQ vs swap projection]
When some $\hat Q_a$ are first‑class (Hamiltonian/diffeomorphisms), 
\eqref{eq:H-noether} ensures compatibility of the generator with the constrained 
subspace, and the RAQ/group‑averaging map
$\eta=\int_{\mathcal G_{\rm cont}}d\mu(g)\,U(g)$ implements the continuous 
constraint surface in the usual way. This continuous RAQ projector 
$\hat{\mathcal P}_{\rm RAQ}$ does \emph{not} halve the mode count at leading order: 
it projects out gauge orbits of measure zero and ensures that the determinants 
are well defined, while preserving the $\mathcal{O}(A/\xi^{2})$ scaling.

\noindent The additional factor $1/2$ relevant for black–hole entropy arises from selecting the
\emph{swap-even superselection sector} (state-level projection $P_+=\tfrac12(\mathbf 1+G_{xy})$).
At the observable level this is conveniently expressed via the conditional expectation
$\mathbb E_{\rm even}(X)=\tfrac12(X+G_{xy}XG_{xy})$, which block-diagonalizes the swap decomposition. 
This is a boundary superselection associated with the modular/CPT structure of the 
wedge, not a first–class constraint. Combined with the bipartite halving ($1/2$ 
from pairing out/in modes), the swap–even projection yields the universal $1/4$ 
prefactor of the Bekenstein–Hawking law.
\end{remark}

\paragraph{EFT normalization of the kernel.}
The mediator scale $\xi$ is identified with the fundamental short distance of the WESH sector (naturally $\xi\sim L_P$). In curved backgrounds, $m_T=\xi^{-1}$ receives curvature corrections of effective‑field‑theory type (e.g.\ from the Kretschmann scalar), but for macroscopic black holes the relative shift is suppressed by powers of $(L_P/M)$; thus $\xi=L_P$ is a controlled leading‑order approximation.

\paragraph{Prelude to horizon thermodynamics.}
Equation~\eqref{eq:H-noether} is athermal; it imposes algebraic coherence of the pre‑geometric flow. In a near‑horizon (Rindler) wedge, the same coherence singles out, as the normal stationary state on the wedge algebra, a KMS state with respect to boosts. Uniqueness follows from the structure of GKSL semigroups with quantum detailed balance on the wedge algebra. This provides the thermal fixed point needed for the entropy analysis without assuming thermality \emph{a priori}.

\begin{definition}[Araki relative entropy and basic properties]\label{def:Araki}
Let $\rho,\sigma$ be normal states on a von Neumann algebra $\mathfrak A$. 
The Araki relative entropy (Araki, 1976) $S_{\mathfrak A}(\rho\Vert\sigma)$ is the algebraic extension of 
$\mathrm{Tr}\,\rho(\log\rho-\log\sigma)$ that is well-defined also for type–III local algebras.
It obeys:
\begin{itemize}
\item \textbf{Positivity and faithfulness:} $S_{\mathfrak A}(\rho\Vert\sigma)\ge 0$, with equality iff $\rho=\sigma$.
\item \textbf{Monotonicity (data processing):} for any inclusion of von Neumann algebras 
$\mathfrak A\subset\mathfrak B$ and normal states $\rho,\sigma$ on $\mathfrak B$,
\[
S_{\mathfrak A}(\rho|_{\mathfrak A}\Vert\sigma|_{\mathfrak A})\ \le\ S_{\mathfrak B}(\rho\Vert\sigma)
\]
(restriction to subsystems does not increase relative entropy).
\end{itemize}
For bipartite algebras, the mutual information is a relative entropy,
\[
I(A\!:\!B)_\rho \;=\; S(\rho_{AB}\Vert\rho_A\!\otimes\!\rho_B).
\]
\emph{Remark.} In algebraic quantum field theory (AQFT; Haag, 1996), local algebras are typically type–III, 
so von Neumann entropies of subregions may diverge; Araki's $S(\rho\Vert\sigma)$ is the correct, 
regulator-independent quantity for all local statements used below.
\end{definition}

\begin{definition} (Modular–KMS spectral regularity of $\hat T^2$ on the wedge)
\label{def:modular-KMS-regularity}
\vspace{0.5cm}

\noindent Let $\mathfrak A(\mathcal W_R)$ be the right–Rindler wedge algebra and $H_R$ its boost generator. 
We say that the composite channel density $\hat T^2$ enjoys \emph{modular–KMS spectral regularity} on $\mathcal W_R$ if:

\begin{enumerate}[leftmargin=1.45cm,itemsep=0.25cm]

\item[(i)] \textbf{Tempered KMS two–point structure.} The KMS two–point Wightman distributions of $O:=\hat T^2$ are tempered and satisfy the standard strip analyticity with respect to the modular flow generated by $H_R$; equivalently, their boost–frequency spectra obey the KMS reflection relations on the strip $\Im t\in(0,\beta_H)$.

\item[(ii)] \textbf{Finite null–second moment for the even four–point.}
On any null generator (affine parameter $u$) of the horizon cut, consider the
\emph{connected} four–point function of $O=\hat T^2$,
\[
G_c(u_1,u_2,u_3,u_4)
:=\langle O(u_1)O(u_2)O(u_3)O(u_4)\rangle_c.
\]
We define its even part under the swap $(u_1,u_3)$ as
\[
G_c^{\rm even}(u_1,u_2,u_3,u_4)
:=\tfrac12\Big(G_c(u_1,u_2,u_3,u_4)
+G_c(u_3,u_2,u_1,u_4)\Big),
\]
and require that, for coincident pairs $(u_1,u_3)=(u, u)$ and $(u_2,u_4)=(0,0)$,
the null–second moment
\[
\int_{\mathbb R} du\,u^2\,
\big\|G_c^{\rm even}(u,0,u,0)\big\|
\]
is finite. This guarantees that the even modular spectral form factor of $O$ 
admits a $\mathcal C^1$ Taylor expansion at $\omega=0$.

\item[(iii)] \textbf{Absolutely continuous modular spectrum near $\omega=0$.} The modular spectral measure of $O$ (with respect to $H_R$) is absolutely continuous in a neighborhood of $\omega=0$ with locally bounded density.
\end{enumerate}
\end{definition}

\begin{lemma}[Spectral regularity implies charge--commutant equivalence]
\label{lem:KMS-regularity}
Let $\mathcal L$ be the WESH generator with Hermitian channels 
$\{\hat T^2(x), L_{xy}=\hat T^2(x)-\hat T^2(y)\}$ on the near-horizon 
Rindler wedge $\mathcal W_R$. Under the above spectral regularity conditions, 
the charge--commutant equivalence of Appendix~G (Prop.~G.2) applies: for any 
global charge $\hat Q$,
\begin{equation}
\mathcal L^\dagger[\hat Q]=0 
\quad\Longleftrightarrow\quad
[H_{\rm eff},\hat Q]=0 \text{ and } [L_\alpha,\hat Q]=0 
\text{ for all channels } L_\alpha.
\label{eq:KMS-wesh-equivalence}
\end{equation}
\end{lemma}

\begin{proof}[Proof]
The Modular--KMS spectral regularity conditions (Definition~\ref{def:modular-KMS-regularity}) — specifically: (i) tempered two-point KMS correlator, (ii) finite null second moment for the even connected four-point function, and (iii) absolutely continuous modular spectrum near $\omega=0$ — together provide the mild spectral regularity hypothesis required by Proposition~G.2 (Appendix~G).

\noindent In detail: conditions (ii) and (iii) ensure that the operator $\hat T^2$ possesses a non-degenerate spectral resolution in the measure-theoretic sense on the wedge. This nondegeneracy allows the transition from the double-commutator identity $\mathcal D^\dagger_L[\hat Q]=-\tfrac{1}{2}[L,[L,\hat Q]]=0$ (implied by $\mathcal L^\dagger[\hat Q]=0$) to the single-commutator condition $[L,\hat Q]=0$ for each Hermitian Lindblad operator $L\in\{\hat T^2(x),L_{xy}\}$.

\noindent Therefore, Proposition~G.2 applies in the wedge geometry, establishing the stated equivalence.
\end{proof}

\vspace{0.3cm}

\noindent\textit{Remark.} The near-horizon KMS regime is a concrete physical realization where the general algebraic framework applies. The equivalence is not re-derived here but follows from verifying that the wedge satisfies the regularity conditions of Proposition~G.2.
\vspace{0.3cm}

\vspace{0.8cm}

\subsection{Emergence of thermality: the KMS state as the unique stationary solution in a Rindler frame}
\label{sec:KMS-Rindler}
\vspace{0.5cm}

\paragraph*{Overview.}
This subsection derives, rather than assumes, the near-horizon KMS structure by solving the WESH master equation in a uniformly accelerated (Rindler) frame. In the Rindler wedge, the WESH generator is KMS-symmetric with respect to the boost modular flow; consequently, the unique normal stationary state on the wedge algebra is a KMS state at inverse temperature $\beta_H=2\pi/\kappa$, where $\kappa$ is the surface gravity. Detailed balance for the $\hat T^2$-channel and the horizon thermodynamic relations follow as corollaries. No thermodynamic prior is used; the result is anchored in pre‑geometric conservation (Sec.~\ref{sec:wesh-noether}), locality/causality of the mediator, and modular covariance on the wedge.

\subsubsection*{Rindler wedge, boost modular flow, and $\beta_H$}
\label{subsec:Rindler-data}
Fix a regular, non-extremal Killing horizon with surface gravity $\kappa>0$ and introduce Rindler coordinates $(\eta,\rho,\mathbf{x}_\perp)$ so that, in leading order,
\begin{equation}
\label{eq:Rindler-metric}
ds^2 \;=\; -\,\kappa^2 \rho^2\, d\eta^2 \;+\; d\rho^2 \;+\; d\mathbf{x}_\perp^2,
\qquad \rho>0.
\end{equation}
The right wedge $\mathcal W_R=\{\rho>0\}$ carries the boost Killing field $\partial_\eta$ with wedge Hamiltonian $H_R$ generating $\eta$-translations on the wedge algebra $\mathfrak A(\mathcal W_R)$. The Unruh/Hawking inverse temperature is
\begin{equation}
\label{eq:betaH}
\beta_H \;=\; \frac{2\pi}{\kappa}.
\end{equation}
On $\mathfrak A(\mathcal W_R)$, the restriction of the Minkowski (or Hartle–Hawking) vacuum is a KMS state with respect to the modular flow generated by $H_R$; on null cuts of the wedge boundary, the modular Hamiltonian admits a local expression in terms of null stress components, supplying the near-horizon template for what follows.
\vspace{0.8cm}

\subsubsection*{WESH generator in a uniformly accelerated frame}
\label{subsec:WESH-Rindler}

The pre‑geometric $s$‑evolution is governed by the GKSL generator in Eq.~\eqref{eq:gksl}. Its wedge-restricted form reads

\vspace{0.3cm}
\begin{align}
\label{eq:Dloc-wedge}
\mathcal D_{\rm loc}[\rho] 
&= \int_{\mathcal W_R}\! d\Sigma_x \;\gamma(x)\,\Big( \hat T^2(x)\,\rho\,\hat T^2(x) - \tfrac{1}{2}\bigl\{\hat T^2(x)^2,\rho\bigr\}\Big),\\[0.3cm]
\label{eq:Dbi-wedge}
\mathcal D_{\rm bi}[\rho] 
&= \iint_{\mathcal W_R\times\mathcal W_R}\! d\Sigma_x d\Sigma_y \; \Gamma(x,y)\,\Big( L_{xy}\,\rho\,L_{xy}^\dagger - \tfrac{1}{2}\bigl\{L_{xy}^\dagger L_{xy},\rho\bigr\}\Big),
\end{align}
\vspace{0.3cm}

\noindent where $\gamma,\Gamma$ are smooth, exponentially causal kernels with correlation length $\xi$, and $H_{\rm eff}$ commutes with the global charges by WESH–Noether (Sec.~\ref{sec:wesh-noether}). The algebra of accessible observables is $\mathfrak A(\mathcal W_R)$, and all integrals are understood in the sense of operator-valued distributions on wedge Cauchy sections.

\paragraph{Boost spectral decomposition.}
Let $\mathrm{ad}_{H_R}(A):=[H_R,A]$. Any $A\in\mathfrak A(\mathcal W_R)$ admits the Bohr (boost-frequency) decomposition
\vspace{0.3cm}
\begin{equation}
\label{eq:boost-Bohr}
A \;=\; \int_{\mathbb R} d\omega\; A(\omega),
\qquad \mathrm{ad}_{H_R}\!\big(A(\omega)\big) \;=\; \omega\,A(\omega),
\end{equation}
which diagonalizes the action of the modular flow and isolates frequency-resolved rate densities in \eqref{eq:Dloc-wedge}–\eqref{eq:Dbi-wedge}.

\subsubsection*{KMS symmetry (quantum detailed balance) on the wedge}
\label{subsec:KMS-DBC}
For a faithful, normal state $\sigma$ on $\mathfrak A(\mathcal W_R)$, define the GNS inner product
\vspace{0.3cm}
\begin{equation}
\label{eq:sigma-inner}
\langle A,B\rangle_\sigma \;:=\; \langle \Omega_\sigma,\,A^\dagger B\,\Omega_\sigma\rangle
\quad
(\text{in a $\xi$-regulated/type-I representation: }
\mathrm{Tr}(\sigma^{1/2}A^\dagger\sigma^{1/2}B)).
\end{equation}
A GKSL generator $\mathcal L$ satisfies $\sigma$-detailed balance (KMS symmetry) iff $\langle A,\mathcal L[B]\rangle_\sigma = \langle \mathcal L[A],B\rangle_\sigma$ for all $A,B$, in which case $\sigma$ is stationary, $\mathcal L^\ast[\sigma]=0$.

\begin{lemma}[KMS symmetry of the wedge-restricted WESH generator]\label{lem:KMS-symmetry}
Let $\sigma_{\rm KMS}\propto e^{-\beta_H H_R}$ be the KMS state on $\mathfrak A(\mathcal W_R)$ at inverse temperature $\beta_H$. Suppose the $\hat T^2$–correlators in $\sigma_{\rm KMS}$ enjoy standard KMS analyticity and that $\Gamma(x,y)$ is boost-covariant and decays exponentially on scales $\gtrsim\xi$. Then the frequency-resolved rates from \eqref{eq:Dloc-wedge}–\eqref{eq:Dbi-wedge} obey
\begin{equation}
\label{eq:rates-DBC}
\Gamma(+\omega)\;=\;e^{-\beta_H\omega}\,\Gamma(-\omega),
\end{equation}
and $\mathcal L_{\rm WESH}$ is $\sigma_{\rm KMS}$-detailed balanced, hence $\mathcal L_{\rm WESH}^\ast[\sigma_{\rm KMS}]=0$.
\end{lemma}

\begin{proof}[Sketch]
Insert \eqref{eq:boost-Bohr} into \eqref{eq:Dloc-wedge}–\eqref{eq:Dbi-wedge} and express the Golden rule rate densities as Fourier transforms of connected four-point functions of $\hat T$ in $\sigma_{\rm KMS}$. KMS analyticity yields \eqref{eq:rates-DBC}. With jump operators $A(\omega)$ and $A^\dagger(\omega)$ organized by $\omega$, one verifies $\sigma$-self-adjointness of $\mathcal L_{\rm WESH}$ relative to \eqref{eq:sigma-inner}, implying stationarity. Boost covariance of $\Gamma$ ensures closure on $\mathfrak A(\mathcal W_R)$.
\end{proof}
\vspace{0.3cm}

\begin{proposition}[Relative-entropy Lyapunov functional]\label{prop:relent-lyap}
Let $\mathcal L_{\rm WESH}$ be $\sigma_{\rm KMS}$-detailed balanced on $\mathfrak A(\mathcal W_R)$, 
and let $\rho(s)=e^{s\mathcal L_{\rm WESH}}{}^\ast[\rho_0]$. 
Then the Araki relative entropy to the KMS state,
\[
\mathcal S(s)\;:=\;S_{\mathfrak A(\mathcal W_R)}\!\big(\rho(s)\,\Vert\,\sigma_{\rm KMS}\big),
\]
is non-increasing in $s$ and strictly decreasing unless $\rho(s)=\sigma_{\rm KMS}$. 
Hence $\mathcal S$ is a Lyapunov functional for the wedge semigroup; combined with primitivity, 
it implies uniqueness of the stationary state.
\end{proposition}

\begin{proof}[Sketch]
Detailed balance makes $\mathcal L_{\rm WESH}$ self-adjoint in the $\sigma_{\rm KMS}$ inner product, so the generator is a gradient flow for a convex functional; contractivity of CPTP maps entails the data‑processing inequality for $S(\cdot\Vert\sigma_{\rm KMS})$, ensuring $\partial_s\mathcal S\le 0$ with equality only at the fixed point.
\end{proof}

\subsubsection*{Fixed point and uniqueness}
\label{subsec:KMS-fixed-unique}
\vspace{0.5cm}

\begin{proposition} (Irreducibility from null–plane Markovity).
Let $\mathfrak A(\mathcal W_R)$ be the wedge algebra and let the WESH generator on the wedge be
\[
\begin{aligned}
\mathcal L_{\rm WESH}[\rho]
&= -\,i[H_{\rm eff},\rho]
  + \int_{\mathcal W_R}\! d\Sigma_x\,\gamma(x)\,\mathcal D_{\hat T^2(x)}[\rho] \\
&\quad
  + \iint_{\mathcal W_R\times \mathcal W_R}\! d\Sigma_x\, d\Sigma_y\,
    \Gamma(x,y)\,\mathcal D_{L_{xy}}[\rho],
\end{aligned}
\qquad
L_{xy}:=\hat T^2(x)-\hat T^2(y).
\]
with Hermitian jumps and kernels as in Sec.~\ref{subsec:WESH-Rindler}
(Eqs.~\eqref{eq:Dloc-wedge}--\eqref{eq:Dbi-wedge}). Assume:

\begin{enumerate}[label=(K\arabic*),leftmargin=1.6cm,itemsep=0.25cm]
\item \emph{Spectral regularity (Definition~\ref{def:modular-KMS-regularity}).} The quadratic channel $\hat T^2$ enjoys modular–KMS spectral regularity on the wedge (tempered KMS two–point, finite second moment of the even connected four–point on null cuts, absolutely continuous modular spectral measure near $\omega{=}0$).
\item \emph{Cross–horizon support.} $\Gamma(x,y)$ has exponentially causal support of range $\xi$ and is non–vanishing for a set of $(x,y)$ of positive measure with $x$ outside and $y$ inside the horizon neighborhood (Planck–thickened cross–sections), i.e.\ the bilocal channel truly \emph{couples} the two sides.
\item \emph{Null–plane Markov property.} The vacuum restricted to $\mathfrak A(\mathcal W_R)$ is Markov on nested null intervals (saturation of strong subadditivity / conditional independence along null generators). This ensures that the modular flow is local and generated by a horizon–local 
integral of the stress tensor (Casini, Testé, \& Torroba, 2017).
\end{enumerate}

\noindent Then the quantum Markov semigroup $\{e^{s\mathcal L_{\rm WESH}}\}_{s\ge 0}$ is \emph{primitive} (irreducible) on $\mathfrak A(\mathcal W_R)$: it admits a unique faithful normal stationary state. Since $\mathcal L_{\rm WESH}$ satisfies $\sigma_{\rm KMS}$–detailed balance at $\beta_H=2\pi/\kappa$, that unique stationary state is the $\chi$–KMS state at Hawking temperature,
\[
\rho_{\rm st}=\sigma_{\rm KMS}\propto e^{-\beta_H H_R}\,.
\]
\end{proposition}

\begin{remark}[On the uniqueness hypotheses]
Assumptions (K1)–(K3) play disjoint roles. (K1) guarantees that Hermitian jumps generated by $\hat T^2$ admit a non-degenerate modular spectral resolution; this collapses invariant corners to scalars via $\mathcal D^\dagger_L[Q]=-\tfrac{1}{2}[L,[L,Q]]$. (K2) ensures cross-horizon \emph{connectivity} (no decoupled components) at range $\xi$. (K3) (null-plane Markov property) rules out “shielded’’ subalgebras on nested null intervals by saturating strong subadditivity along the generators. Together with detailed balance, these imply primitivity and thus uniqueness of the KMS fixed point.
\end{remark}

\paragraph*{Proof sketch.}

\vspace{0.3cm}
~\\

\noindent\emph{Step 1 (Trivial commutant under spectral regularity).}
By Hermiticity of the jumps and Definition~\ref{def:modular-KMS-regularity}, the double–commutator identity
$\mathcal D^\dagger_L[Q]=-\tfrac12[L,[L,Q]]$ holds on the relevant domain. If a von Neumann subalgebra $\mathcal N\subset\mathfrak A(\mathcal W_R)$ were invariant under the semigroup, then for every $Q\in\mathcal N$ one would have $\mathcal D^\dagger_L[Q]\in\mathcal N$ for all $L\in\{\hat T^2(x),L_{xy}\}$. Spectral regularity (non–degenerate modular spectral measure for the local polynomials of $\hat T$) implies that $[Q,\hat T^2(x)]=[Q,L_{xy}]=0$ almost everywhere forces $Q$ to lie in the commutant of the local quadratic field algebra, hence $Q$ is scalar. Thus any invariant von Neumann subalgebra is trivial. (Appendix~G, Prop.\,G.2 gives the same conclusion for global charges from $\mathcal L^\dagger[Q]=0$.)

\vspace{0.3cm}
\noindent\emph{Step 2 (Connectivity from the bilocal kernel).}
Coarse–grain the horizon into Planck–thickened null cells $\{C_i\}$ of diameter $\sim\xi$ along each generator. By (K2), the weighted adjacency matrix $A_{ij}:=\int_{C_i\times C_j}\Gamma(x,y)\,d\Sigma_x d\Sigma_y$ is irreducible (one giant connected component): every cell on the ``out'' side is coupled to the ``in'' side within $\mathcal O(\xi)$ along the null direction. Therefore, the $*$–algebra generated by the set $\{\hat T^2(C_i),\,\hat T^2(C_i)-\hat T^2(C_j)\}$ acts cyclically on the wedge algebra: there is no nontrivial decomposition preserved by all jumps. 

\vspace{0.3cm}
\noindent\emph{Step 3 (Null–plane Markovity kills hidden corners).}
Assumption (K3) states that along any nested null intervals $I_1\subset I_2\subset I_3$
on a generator the vacuum (and its $\chi$–KMS deformations) is Markov,
$I(I_1:I_3\mid I_2)=0$. Equivalently, there exists a Petz recovery map
$\mathbb E_{I_2}:\mathfrak A(I_1\cup I_3)\to\mathfrak A(I_2)$ saturating strong
subadditivity.

\noindent Suppose now that $\mathcal N\subset\mathfrak A(\mathcal W_R)$ were a nontrivial
von Neumann subalgebra invariant under the WESH semigroup. By Step~2, $\mathcal N$
must contain operators with support on both sides of the horizon. Pick
$N\in\mathcal N\cap\mathfrak A(I_1)$ with nonzero projection onto $\mathfrak A(I_3)$.
Invariance of $\mathcal N$ under the dynamics and locality of the bilocal kernel
then imply that $\mathbb E_{I_2}(N)$ carries nontrivial correlations between $I_1$
and $I_3$.

\noindent But Markov saturation forces any such operator to have vanishing conditional mutual
information unless it is proportional to the identity: $I(I_1:I_3\mid I_2)=0$
together with data processing implies $N\propto\mathbf 1$ in the GNS space of
$\sigma_{\rm KMS}$. By translating this argument along the generators and using
cross–horizon connectivity (Step~2), one concludes that
$\mathcal N=\mathbb C\,\mathbf 1$. Hence the semigroup is primitive.

\vspace{0.3cm}
\noindent\emph{Step 4 (Detailed balance $\Rightarrow$ uniqueness).}
Under Sec.\,6.2, $\mathcal L_{\rm WESH}$ satisfies $\sigma_{\rm KMS}$ detailed balance; hence it is self–adjoint in the 
$\langle A,B\rangle_{\sigma_{\rm KMS}}=\Tr(\sigma_{\rm KMS}^{1/2}A^\dagger\sigma_{\rm KMS}^{1/2}B)$ inner product. By Steps 1–3 the kernel of $\mathcal L_{\rm WESH}$ is one–dimensional (scalars), and the orthogonal complement has strictly positive spectrum. Standard results for detailed–balance quantum Markov semigroups therefore give a unique faithful stationary state and mixing $e^{s\mathcal L_{\rm WESH}}{}^\ast[\rho]\to\sigma_{\rm KMS}$ for all normal $\rho$. \hfill$\square$

\paragraph*{Remarks.}
\begin{enumerate}[leftmargin=1.3cm,itemsep=0.25cm]
\item \emph{What could fail?} If the cross–horizon gate vanished on a set of nonzero measure (no Rényi–2 correlations across some pairs), one could engineer a nontrivial invariant corner. Assumption (K3) excludes this near the horizon: the vacuum is Markov on null cuts and the modular Hamiltonian is local, so two–point (and controlled four–point) data extend across arbitrarily small null separations.
\item \emph{Alignment with the pre–geometric structure.} Appendix~D–H provide the Lyapunov functional $\mathcal M$ and the Noether–level commutant arguments used implicitly in Step 1; they ensure that no global charge generates hidden invariant directions for the wedge semigroup. 
\item \emph{Spectral gap.} In the detailed–balance geometry the primitive property implies a strictly positive spectral gap on the orthogonal complement of the scalars in the $\sigma_{\rm KMS}$–GNS space. This gap sets the wedge mixing time and coincides with the width of the low–frequency peak entering the spectral weight estimates in Sec.\,6.4.
\end{enumerate}

\begin{remark}[Null-plane Markovity and relative entropy]
On nested null intervals $I_1\subset I_2\subset I_3$, the vacuum (and its $\chi$–KMS deformations) 
satisfies $I(I_1:I_3\!\mid\! I_2)=0$, i.e.\ saturation of strong subadditivity. 
Equivalently, the conditional mutual information equals a conditional relative entropy that 
vanishes, and a Petz recovery map exists localized on $I_2$. 
This formulation makes precise why no “screened’’ invariant corner survives on the wedge: 
data processing along the null inclusion chain collapses relative entropy to zero only on scalars, 
which enforces primitivity once cross‑horizon connectivity is present.
\end{remark}

\begin{remark}[Ergodicity from cross-horizon coupling]
\label{rem:ergodicity}
The uniqueness of the KMS fixed point relies on primitivity, not 
merely on energy conservation. While $[L_{xy}, H_R]=0$ ensures 
stationarity, the cross-horizon support of the bilocal kernel 
(Assumption K2) prevents the formation of decoupled sectors that 
would otherwise relax to non-thermal Generalized Gibbs Ensembles. 
Combined with the null-plane Markov property (K3), this guarantees 
that the only invariant subalgebra is trivial, forcing unique 
relaxation to the $\chi$-KMS state.
\end{remark}

\begin{theorem}[Uniqueness of the KMS stationary state]\label{thm:KMS-unique}
Assume the hypotheses of Lemma~\ref{lem:KMS-symmetry}, exponential clustering of connected cumulants of $\hat T$ with range $\xi$, and primitivity (irreducibility) of the dissipative representation generated by $\{\hat T^2(x),L_{xy}\}$ on $\mathfrak A(\mathcal W_R)$. Then the semigroup $e^{s\mathcal L_{\rm WESH}}$ admits a unique normal stationary state on $\mathfrak A(\mathcal W_R)$,
\begin{equation}
\label{eq:rho-staz}
\begin{split}
\rho_{\rm staz} &\;=\; \sigma_{\rm KMS}\quad(\chi\text{-KMS state at }\beta_H=\tfrac{2\pi}{\kappa}),\\
&\quad (\text{in a $\xi$-regulated/type-I model: }\sigma_{\rm KMS}\propto e^{-\beta_H H_R})
\end{split}
\end{equation}
and $e^{s\mathcal L_{\rm WESH}}{}^\ast[\rho_0]\to\sigma_{\rm KMS}$ for any normal $\rho_0$ (weak$^\ast$), with mixing rate governed by the spectral gap of $\mathcal L_{\rm WESH}$ in the $\langle\cdot,\cdot\rangle_{\sigma_{\rm KMS}}$ geometry.
\end{theorem}

\begin{proof}[Sketch]
By Lemma~\ref{lem:KMS-symmetry}, $\sigma_{\rm KMS}$ is stationary. Exponential clustering and finite-range kernels imply (quasi-)compactness of the dissipator on frequency sectors; primitivity excludes invariant subalgebras other than scalars. Standard results for detailed-balance quantum Markov semigroups then give uniqueness and convergence with a strictly positive gap.
\end{proof}

\begin{corollary}[Hawking/Unruh detailed balance]\label{cor:UD-balance}
For the $\hat T^2$–channel, the upward/downward transition rates obey
\begin{equation}
\label{eq:UD-balance}
\frac{\Gamma_\uparrow(\omega)}{\Gamma_\downarrow(\omega)} \;=\; e^{-\beta_H\omega},
\qquad \beta_H = \frac{2\pi}{\kappa},
\end{equation}
providing the near-horizon detailed balance used in the thermodynamic analysis.
\end{corollary}

\subsubsection*{Spectral weights and the horizon pair entropy}
\label{subsec:W-HH-weight}
Combining Corollary~\ref{cor:UD-balance} with the Golden-rule rate densities yields the WESH–HH spectral weight
\begin{equation}
\label{eq:W-HH-weight}
W_{\rm HH}(\omega) \;=\; \Gspec(\omega)\,\rho(\omega)\,\frac{e^{-\beta_H\omega}}{1-e^{-\beta_H\omega}},
\end{equation}
where $\Gspec(\omega)$ is the $\hat T^2$ spectral response and $\rho(\omega)$ the smooth near-horizon density of states on $\mathfrak A(\mathcal W_R)$. Equation~\eqref{eq:W-HH-weight} governs the KMS-weighted pair-entropy average that fixes the leading area coefficient in the entropy law (Sec.~\ref{sec:RAQ-quarter}), and it appears here as a consequence of the KMS fixed point rather than a modeling choice.

\subsubsection*{Emergent thermality: $s$ versus $t$}
\label{subsec:roles}
KMS symmetry and detailed balance are emergent properties of the wedge-restricted dynamics \emph{after} bootstrapping to physical time via $dt/ds=\Gamma[\Psi]$. The WESH–Noether constraint (Sec.~\ref{sec:wesh-noether}) remains the athermal, pre‑geometric input selecting admissible generators; thermality is a stationary property induced by boost modular flow in the near-horizon regime.

\subsubsection*{Domain of validity and controlled deformations}
\label{subsec:validity-KMS}
The derivation holds in the Rindler window where (i) boost covariance and wedge modular structure are accurate; (ii) kernels are undeformed up to curvature-suppressed corrections $\mathcal O((\kappa\xi)^n)$; and (iii) clustering at range $\xi$ persists. Curvature-induced renormalizations of kernel parameters are treated within EFT and are parametrically suppressed for macroscopic horizons; they do not alter the existence or temperature of the KMS fixed point, but only dress approach-to-equilibrium rates and higher cumulants that feed subleading (logarithmic/power-law) corrections.

\subsubsection*{Outlook: rotating horizons}
\label{subsec:Kerr-outlook}
For stationary rotating black holes, replace $H_R$ with the co-rotating generator $H_R-\Omega_H J$ on the wedge algebra adapted to Kerr (or NHEK). The fixed point becomes $\sigma_{\rm KMS}\propto e^{-\beta_H (H_R-\Omega_H J)}$, i.e.\ a KMS state with an angular momentum chemical potential; the modular picture on null generators remains the organizing principle. Hidden symmetries (Killing–Yano tower) ensure separability and allow a covariant spectral decomposition of the $\hat T^2$ channel, paving the way to the analysis of superradiant channels and logarithmic corrections in the rotating case.

\vspace{0.8cm}

\subsection{Physical projection via RAQ and the universal $\tfrac{1}{4}$ prefactor}
\label{sec:RAQ-quarter}
\vspace{0.5cm}

\paragraph*{Overview.}
This subsection replaces the heuristic 'physical projector' with a mathematically well–defined construction based on canonical first-class constraints and group averaging (Refined Algebraic 
Quantization, RAQ; Ashtekar et al., 1995). The near–
horizon bipartite Hilbert space factors as $\mathcal H_{\rm kin}\!\simeq\!\mathcal H_{\rm out}\!\otimes\!\mathcal H_{\rm in}$ with entanglement–effective dimension $D$ on a regulator window set by the correlation length $\xi$. WESH–Noether guarantees that the constraints commute with the adjoint pre–geometric generator, so the RAQ projection is dynamically consistent. We show that the gauge average including the discrete horizon swap $\mathbb Z_2$ yields an even–sector projection whose normalized trace is asymptotically $1/2$; combined with bipartite halving, this produces the universal $\tfrac{1}{4}$ prefactor of the Bekenstein–Hawking law. From the wedge point of view, the $\mathbb Z_2^{\rm (swap)}$ factor implements the discrete CPT reflection associated with the modular structure of the Rindler algebra; imposing invariance under the swap thus enforces that physical states respect the fundamental CPT symmetry of the vacuum, rather than introducing an \emph{ad hoc} gauge choice. Edge/center modes and RAQ measure issues affect only $\mathcal O(1)$ terms.

\paragraph{Constraints and gauge group.}
\vspace{0.3cm}

Collect the constraints
\vspace{0.3cm}
\begin{equation}
\label{eq:constraints}
\begin{aligned}
\widehat{\mathcal C}\;=\;\Big\{\ 
&\hat H_{\rm tot}(x)\approx 0\ \ \text{(Wheeler--DeWitt/Hamiltonian)},\\
&\hat{\mathcal H}_a(x)\approx 0\ \ \text{(spatial diffeomorphisms)},\\
&\hat Q_T\approx 0\ \ \text{(global $T$-charge neutrality)}\ \Big\},
\end{aligned}
\end{equation}
\vspace{0.3cm}

\noindent with $\hat Q_T$ fixed by WESH-Noether. In the near-horizon (Rindler) window the wedge modular flow (boosts) commutes with the constraint set on the regulated algebra (domain subtleties understood), so that the physical gauge group can be taken as the (semi--)direct product
\begin{equation}
\label{eq:gaugegroup}
\mathcal G_{\rm cont}\;=\;\exp\!\big(i\alpha^A \hat{\mathcal C}_A\big),
\qquad \hat{\mathcal C}_A\in \widehat{\mathcal C}.
\end{equation}
The trans--horizon swap $G_{xy}$ is implemented \emph{separately} as the horizon-even
conditional expectation $\mathbb E_{\rm even}$ (Eq.~\eqref{eq:even-expectation}),
i.e.\ a boundary superselection compatible with the wedge modular/KMS structure.

\paragraph{RAQ (group averaging) and the physical projector.}
In RAQ the rigging map $\eta:\mathcal D\!\subset\!\mathcal H_{\rm kin}\!\to\!\mathcal D^{*}$ averages kinematical vectors over $\mathcal G$,

\begin{equation}
\label{eq:rigging-map}
\eta\big(|\psi\rangle\big)[|\phi\rangle]
:= \int_{\mathcal G_{\rm cont}}\! d\mu_{\rm cont}(\alpha)\;
\langle \phi|\,e^{\,i\alpha^A\widehat{\mathcal C}_A}\,|\psi\rangle,
\qquad |\phi\rangle,|\psi\rangle\in\mathcal D.
\end{equation}
\vspace{0.3cm}
\noindent Here $d\mu_{\rm cont}(\alpha)$ is the (generalized) Haar measure on $\mathcal G_{\rm cont}$. 
The induced (formal) projector is

\begin{equation}
\label{eq:phys-projector}
\begin{split}
\hat{\mathcal P}_{\rm RAQ} &\;\propto\;\int_{\mathcal G_{\rm cont}}\! d\mu_{\rm cont}(\alpha)\;
e^{\,i\alpha^A\widehat{\mathcal C}_A},\\
\hat{\mathcal P}_{\rm phys} &:= \mathbb E_{\rm even}\circ \hat{\mathcal P}_{\rm RAQ},
\quad
\mathbb E_{\rm even}(X)=\tfrac12(X+G_{xy}XG_{xy}).
\end{split}
\end{equation}
\vspace{0.3cm}
Standard RAQ technology (choice of dense domain $\mathcal D$, distributional completion, gauge–volume normalization for non–compact groups) ensures well–posedness; for linear constraints on free fields the continuous average implements $\prod_A\delta(\hat{\mathcal C}_A)$ as a sesquilinear form, while interactions are handled by adiabatic/regularized averages before taking the physical limit.

\paragraph{Horizon bipartition, modular flow, and swap symmetry.}
On the Rindler wedge algebra the modular Hamiltonian equals the boost generator; locality on null cuts renders it quasi–local, so the swap action $G_{xy}$ and the modular flow commute on the algebra generated by trans–horizon pairs (up to edge–mode centers). This validates the evaluation of \eqref{eq:phys-projector} directly on $\mathcal H_{\rm out}\!\otimes\!\mathcal H_{\rm in}$ using the symmetric/antisymmetric decomposition.

\begin{theorem}[Asymptotic halving under RAQ gauge projection]
\label{thm:halving}

Let $\mathcal H_{\rm kin}=\mathcal H_{\rm out}\otimes\mathcal H_{\rm in}$ with $\dim\mathcal H_{\rm out}=\dim\mathcal H_{\rm in}=D<\infty$. Assume: \textup{(i)} the continuous constraints in \eqref{eq:constraints} act reducibly with equal multiplicities across out/in factors on the near–horizon regulated algebra; \textup{(ii)} the swap $G_{xy}$ exchanges the factors and commutes with the continuous average; \textup{(iii)} the Hartle–Hawking/KMS sector is invariant under the diagonal action of $G_{xy}$. Then the projector \eqref{eq:phys-projector} induces an \emph{even} projection on $\mathcal H_{\rm kin}$ with normalized trace
\begin{equation}
\label{eq:trace-ratio}
\frac{\Tr_{\mathcal H_{\rm kin}}\!\big[\hat{\mathcal P}_{\rm phys}\big]}{\Tr_{\mathcal H_{\rm kin}}\!\big[\mathbf 1\big]}
\;=\;\frac{D(D\pm 1)/2}{D^2}\;=\;\frac{1}{2}\ \pm\ \frac{1}{2D}\ \xrightarrow[D\to\infty]{}\ \frac{1}{2},
\end{equation}
\vspace{0.3cm}
where $\pm$ select the even/odd sector; the physical sector is even. Hence the gauge projection contributes a universal factor $1/2$ to the leading area term.
\end{theorem}

\begin{proof}[Sketch]
The continuous average restricts to the joint kernel of $\widehat{\mathcal C}$ with equal multiplicities on out/in (assumption (i)), so the residual multiplicity factorizes. The discrete average splits $\mathcal H_{\rm out}\!\otimes\!\mathcal H_{\rm in}$ into the symmetric/antisymmetric $\mathbb Z_2$ representations of dimensions $D(D\pm1)/2$, giving \eqref{eq:trace-ratio}. Invariance of the HH/KMS sector under $G_{xy}$ and its commutation with modular flow single out the even (orientation–preserving) sector as physical. The $1/(2D)$ correction is subleading and vanishes in the thermodynamic ($D\!\to\!\infty$) limit.
\end{proof}

\begin{proposition}[Edge/center modes and robustness]
\label{prop:edge}
Gauge centers localized at the entangling surface induce superselection sectors (edge modes). The RAQ projection implements the gauge–invariant algebra; center degrees contribute only additive $\mathcal O(1)$ terms to the entropy and do not modify the $A$–proportional leading coefficient. Measure–normalization ambiguities in RAQ likewise shift only the constant part of $S$.
\end{proposition}

\begin{corollary}[The universal $\tfrac{1}{4}$ prefactor]
\label{cor:quarter}
Let $N_{\rm naive}=A/\xi^2$ be the UV mode count at correlation length $\xi$ (Sec.~5). Bipartite pairing across the horizon yields $N_{\rm bi}=\tfrac12\,A/\xi^2$. The RAQ gauge projection (Theorem~\ref{thm:halving}) contributes another factor $\tfrac12$, so the number of physical horizon modes is
\[
N_{\rm phys}\;=\;\frac{1}{4}\,\frac{A}{\xi^2}.
\]
With the KMS–weighted mean pair entropy $\bar s=\mathcal O(1)$ (cf.\ Prop.~\ref{prop:concentration}), the entropy is
\begin{equation}
\label{eq:final-quarter}
S_{\rm BH}\;=\;N_{\rm phys}\,\bar s\;=\;\frac{A}{4\,\xi^2}\,\Big(1+\mathcal O(\xi^2/A)\Big)
\;\xrightarrow{\ \xi=L_P\ }\;\frac{A}{4L_P^2}\,+\,\mathcal O(1).
\end{equation}
The $\mathcal O(1)$ and further subleading terms collect edge/center contributions, RAQ measure choices, and curvature–sensitive one–loop (heat–kernel) corrections, to be quantified in subsequent sections.
\end{corollary}

\paragraph{Consistency with WESH–Noether and modular structure.}
WESH–Noether enforces $\mathcal L^\dagger[\hat{\mathcal C}_A]=0$, ensuring that the constraint surface is invariant under the pre‑geometric flow (path independence). Thus the RAQ projector is not an external device but the representation–theoretic implementation of the very conservation principle encoded in the generator. On the wedge algebra, the modular Hamiltonian (boost) commutes with $G_{xy}$, so thermality (Sec.~\ref{sec:KMS-Rindler}) and RAQ projection are compatible. The resulting chain,
\[
\text{WESH–Noether}\ \Rightarrow\ \text{RAQ–consistent constraints}\ \Rightarrow\ \text{even projection}\ \Rightarrow\ \text{universal $1/2$},
\]
combined with bipartite halving and $\bar s\simeq 1$, yields the $\tfrac{1}{4}$ law from first principles.

\vspace{0.8cm}

\subsection*{RAQ constraints vs.\ horizon superselection: a clean factorization}
\label{subsec:raq-vs-superselection}

\vspace{0.5cm}
\begin{definition}[Two--step physical projection]
\label{def:two-step-projection}
Let $\widehat{\mathcal C}_A$ denote the first--class constraints 
(WDW Hamiltonian and spatial diffeomorphisms, including possible global $T$--charge)
acting on the regulated bipartite Hilbert space 
$\mathcal H_{\rm kin}\!\simeq\!\mathcal H_{\rm out}\!\otimes\!\mathcal H_{\rm in}$,
and let $G_{xy}$ be the unitary implementing the trans-horizon swap. Define:

\vspace{0.5cm}
\noindent
\textbf{(i) RAQ / constraint average.}
\begin{equation}
\hat{\mathcal P}_{\rm RAQ}
   := \int_{G_{\rm cont}}\! d\mu(\alpha)\;
      e^{\,i\alpha^A\widehat{\mathcal C}_A},
\label{eq:RAQ-proj}
\end{equation}

\noindent When acting on operators $X\in\mathcal B(\mathcal H_{\rm kin})$, the average is understood via the adjoint action, $X\mapsto \int d\mu(\alpha)\,e^{i\alpha^A\widehat{\mathcal C}_A}\,X\,e^{-i\alpha^A\widehat{\mathcal C}_A}$.

\vspace{0.6cm}
\noindent
\textbf{(ii) Horizon even--sector expectation.}
\begin{subequations}
\begin{align}
\mathbb E_{\rm even}(X)
   &= \tfrac12\,\big(X + G_{xy}XG_{xy}\big),
   \\[0.3cm]
X &\in \mathcal B(\mathcal H_{\rm kin}) .
\end{align}
\label{eq:even-expectation}
\end{subequations}

\noindent\emph{Swap projectors.} On the regulated bipartite space, the swap satisfies
$G_{xy}^\dagger=G_{xy}$ and $G_{xy}^2=\mathbf 1$. Let $P_\pm:=\tfrac12(\mathbf 1\pm G_{xy})$. Then
$\mathbb E_{\rm even}(X)=\tfrac12(X+G_{xy}XG_{xy})=P_+XP_+ + P_-XP_-$.

\vspace{0.3cm}
\noindent The \emph{physical superselection sector} is the $+$ block (swap-even states),
so the halving in the mode count uses $P_+$:
$\Tr(P_+)/\Tr(\mathbf 1)=D(D+1)/(2D^2)=\tfrac12+\tfrac{1}{2D}$.

\vspace{0.5cm}
\noindent
The \emph{physical} map factorizes as
\begin{equation}
\boxed{
\hat{\mathcal P}_{\rm phys}
  = \mathbb E_{\rm even}\circ\hat{\mathcal P}_{\rm RAQ}\,.
}
\label{eq:phys-factorization}
\end{equation}
\end{definition}

\begin{remark}[Swap–even is \emph{not} a gauge average]
The second factor $\mathbb E_{\rm even}$ is a normal, completely positive idempotent (a conditional expectation) onto the even subalgebra fixed by $G_{xy}$; it implements a \emph{boundary superselection} tied to edge–mode coherence at the entangling surface (horizon), rather than a quotient by a first–class gauge constraint. In gravitational and gauge theories, the correct localization of degrees of freedom on boundaries requires an extended phase space with edge modes; horizon “soft” charges provide precisely such boundary data and organize physical sectors. The even sector singled out by modular covariance/KMS on the wedge coincides with the sector compatible with these boundary symmetries (soft hair / edge–mode literature).
\end{remark}

\begin{proposition}[Commutation, trace factorization, and halving]
\label{prop:commute-halving}
Suppose (a) the constraint set $\{\widehat{\mathcal C}_A\}$ and the swap $G_{xy}$ commute on the regulated wedge algebra (modulo standard domain issues), which is ensured when the adjoint WESH generator conserves the global charges (WESH–Noether) and the wedge modular flow respects the bipartition; (b) $\dim\mathcal H_{\rm out}=\dim\mathcal H_{\rm in}=D<\infty$ at fixed regulator. Then:
\begin{enumerate}\itemsep=0.25cm
\item $\mathbb E_{\rm even}$ and $\hat{\mathcal P}_{\rm RAQ}$ commute on $\mathcal B(\mathcal H_{\rm kin})$, hence \eqref{eq:phys-factorization} is well–defined and independent of order.
\item The swap–even subspace $P_+(\mathcal H_{\rm out}\otimes\mathcal H_{\rm in})$ has dimension $\Tr(P_+)=D(D+1)/2$. Consequently,
\begin{equation}
\frac{\Tr_{\mathcal H_{\rm kin}}\!\big[P_+\big]}
     {\Tr_{\mathcal H_{\rm kin}}\!\big[\mathbf 1\big]}
\;=\;\frac{D(D+1)/2}{D^2}
\;=\;\frac12\,+\,\frac{1}{2D}
\ \xrightarrow[D\to\infty]{}\ \frac12\,.
\label{eq:even-halving}
\end{equation}
\end{enumerate}
\emph{Proof sketch.} (1) WESH–Noether at generator level implies constraint charges commute with the Lindblad set and with the effective Hamiltonian; modular covariance on the wedge makes $G_{xy}$ a symmetry of the near–horizon algebra. These properties imply $[\widehat{\mathcal C}_A,G_{xy}]=0$ on the regulated domain, hence $\hat{\mathcal P}_{\rm RAQ}\mathbb E_{\rm even}=\mathbb E_{\rm even}\hat{\mathcal P}_{\rm RAQ}$. (2) The swap decomposition into symmetric/antisymmetric tensors has dimensions $D(D\pm1)/2$. Traces factorize at fixed regulator because $\hat{\mathcal P}_{\rm RAQ}$ acts diagonally on the joint kernel of the constraints, while $\mathbb E_{\rm even}$ is block–diagonal in the swap basis; \eqref{eq:even-halving} follows. \hfill$\square$ \\

\vspace{0.3cm}
\noindent\emph{Note.} \textit{Generator-level conservation is established in Appendix~G, Prop.~G.2. 
Modular compatibility follows from the pre-geometric path independence detailed in 
Appendix~C.2. Edge-mode/soft-symmetry structures justify the boundary superselection 
through the horizon-even conditional expectation.}
\end{proposition}

\begin{corollary}[From two–step projection to the $\tfrac14$ law]
\label{cor:quarter-projection}
At leading order, the physical mode count is
\(
N_{\rm phys}
= \big(A/\xi^2\big)\times\underbrace{\tfrac12}_{\text{bipartite halving}}
\times\underbrace{\tfrac12}_{\text{swap–even}}
= A/(4\xi^2)\!,
\)
and with $\xi\simeq L_P$ and $\bar s=\mathcal O(1)$ (cf.\ Prop.~\ref{prop:concentration} and Remark~\ref{rem:sbar-role}) one obtains
\(
S_{\rm BH}=A/(4L_P^2)+\mathcal O(\ln A,1).
\)
The swap–even factor originates from boundary superselection (not gauge fixing) and is therefore robust under the inclusion of gravitational edge modes/soft charges. 
\end{corollary}

\begin{remark}[Independence of the two halving factors]
\label{rem:double-halving}
The prefactor $1/4$ in the Bekenstein--Hawking law arises from two 
independent reductions acting at distinct levels:

\begin{enumerate}[leftmargin=1.5cm,itemsep=0.2cm]

\item \textbf{Bipartite pairing (geometric/combinatorial).}
The bilocal mediator $L_{xy}=\hat T^2(x)-\hat T^2(y)$ couples 
trans-horizon modes within the correlation length $\xi$. 
As a consequence, the $A/\xi^2$ horizon cells organize into 
approximately $A/(2\xi^2)$ entangled pairs 
$(\mathrm{out}_i,\mathrm{in}_i)$.
The entanglement entropy is, by definition, the von Neumann 
entropy of the reduced state on $\mathcal{H}_{\rm out}$ (partial trace 
over $\mathcal{H}_{\rm in}$), but the leading extensive factor is set 
by \emph{how many} independent pairs exist. In the regime where each 
pair contributes on average $\bar s = \mathcal{O}(1)$ (cf.\ Prop.~\ref{prop:concentration}) to the 
entropy (Sec.~\ref{sec:phase4-leading-robust}), this geometric/combinatorial 
pairing yields the first factor $1/2$.

\item \textbf{Modular superselection (algebraic/boundary).}
On top of this pairing structure, the conditional expectation 
$\mathbb{E}_{\rm even}$ onto the swap-symmetric sector implements 
the modular/CPT structure of the KMS state on the horizon edge algebra. 
This is a boundary superselection rule---not a gauge constraint---which 
removes antisymmetric configurations incompatible with the modular 
horizon algebra. As shown in Proposition~\ref{prop:commute-halving}, 
$\mathbb{E}_{\rm even}$ reduces the normalized trace ratio by 
an additional factor $1/2$ in the large-area limit.
\end{enumerate}

\noindent These reductions are logically independent: the first concerns 
how many entangled units exist (geometric/combinatorial pairing), the 
second concerns which states are physically admissible on the horizon 
edge (algebraic/modular superselection). Their product, combined with 
the spectral concentration $\bar s = \mathcal{O}(1)$ (Prop.~\ref{prop:concentration}), yields
\[
S_{\rm BH} = \frac{A}{4L_P^2} + \gamma \ln\frac{A}{L_P^2}
+ k_0 + \mathcal{O}\!\left(\frac{L_P^2}{A}\right)
\]
without any double counting of trans-horizon degrees of freedom.
\end{remark}

\paragraph{Context note (internal consistency).}
The separation \eqref{eq:phys-factorization} is compatible with the pre–geometric WESH–Noether constraint algebra at generator level 
(no–vorticity/
path–independence) and with wedge modular structure on null surfaces, ensuring that RAQ and horizon superselection can be implemented without logical overlap.
\vspace{0.8cm}

\subsection{Leading order and robustness: spectral control and EFT stability of the kernel}
\label{sec:phase4-leading-robust}

\vspace{0.5cm}
\paragraph*{Overview.}
We place the leading Bekenstein–Hawking law on firm ground by controlling the near‑horizon spectral weight that enters the mean pair‑entropy and by promoting the undeformed‑kernel choice $\xi\simeq L_P$ to a controlled effective‑field‑theory (EFT) approximation in curved backgrounds. The first part derives the low‑frequency structure of the WESH–HH weight and a quantitative concentration bound for the spectral average $\bar s$; the second part quantifies curvature-induced deformations of the mediator scale $\xi$ and shows that they are parametrically suppressed on macroscopic horizons.

\paragraph{Spectral structure at low frequency and concentration of the entropy average.}
With the KMS fixed point on the wedge (Sec.~\ref{sec:KMS-Rindler}) and the physical projection (Sec.~\ref{sec:RAQ-quarter}), the mean pair‑entropy is
\begin{equation}
\label{eq:phase4-spectral-average}
\bar s \;=\; 
\frac{\displaystyle \int_{0}^{\infty} \frac{d\omega}{2\pi}\,W_{\rm HH}(\omega)\; s(\beta_H\omega)}
     {\displaystyle \int_{0}^{\infty} \frac{d\omega}{2\pi}\,W_{\rm HH}(\omega)}\,,
\qquad
s(x)=\frac{x}{e^{x}-1}-\ln\!\big(1-e^{-x}\big),\quad \beta_H=\frac{2\pi}{\kappa}.
\end{equation}
Here
\begin{equation}
W_{\rm HH}(\omega)
\;=\;
\Gspec(\omega)\,\varrho(\omega)\,
\frac{e^{-\beta_H\omega}}{1-e^{-\beta_H\omega}}
\;=\;
\omega^{2+p}\,F(\omega\xi)\,
\frac{e^{-\beta_H\omega}}{1-e^{-\beta_H\omega}},
\end{equation}
where $p\ge 0$ parametrizes the near-horizon density of states and $F$ is smooth,
positive and single-peaked with $F(0)=1$ and UV suppression for $\omega\gtrsim\xi^{-1}$.
\noindent For a minimally coupled scalar field on a Schwarzschild background one has
\[
p=0,
\]
corresponding to an essentially flat density in tortoise coordinates. For higher
spins or in Kerr backgrounds, $p$ receives $\mathcal O(1)$ corrections (e.g. 
of order $(a/M)^2$ in the slow–rotation regime), but remains bounded and does not 
alter the conclusion that $\bar s=\mathcal O(1)$ independently of $A$.
\vspace{0.3cm}
\noindent Let $x=\beta_H\omega$ and write $\tilde W(x):=x^{n}\,\tilde F(x)\,e^{-x}(1-e^{-x})^{-1}$ with $n=2+p$ and $\tilde F(x)=F(x\,\xi/\beta_H)$. The peak $x_\star$ satisfies
\begin{equation}
\label{eq:phase4-stationary}
\frac{d}{dx}\ln\tilde W(x)\Big|_{x_\star}=0
\;\Longleftrightarrow\;
\frac{n}{x_\star}-1-\frac{1}{e^{x_\star}-1}\;=\;-\frac{d}{dx}\ln\tilde F(x)\Big|_{x_\star}
\;=\;\mathcal O(\kappa\xi),
\end{equation}
so $x_\star=\mathcal O(1)$ and, for slowly varying $\tilde F$, $x_\star=x_n+\mathcal O(\kappa\xi)$,
where $x_n$ is the $\mathcal O(1)$ solution of $\frac{n}{x}-1-\frac{1}{e^x-1}=0$.
\vspace{0.3cm}
\begin{proposition}[Concentration of the spectral average]
\label{prop:concentration}
If $\tilde W$ is positive, smooth, single‑peaked with thermal width $\Delta x=\mathcal O(1)$, then
\[
\bar s\;=\;s(x_\star)\;+\;\delta s,\qquad 
|\delta s|\ \le\ \sup_{x\in I}|s'(x)|\,\sqrt{\mathrm{Var}_{\tilde W}(x)}
\;+\;\tfrac12\,\sup_{x\in I}|s''(x)|\;\mathrm{Var}_{\tilde W}(x),
\]
where $\mathrm{Var}_{\tilde W}(x):=\mathbb E_{\tilde W}\!\big[(x-x_\star)^2\big]$ denotes the second moment around the mode,
for any compact interval $I=[\varepsilon,\Lambda]$ with $0<\varepsilon<x_\star<\Lambda$ covering the bulk of $\tilde W$. The tail $x>\Lambda$ is exponentially suppressed by the thermal factor (and further suppressed by $F$); the tail $x<\varepsilon$ is power-law suppressed ($\tilde W\sim x^{n-1}$) and has total weight $\mathcal O(\varepsilon^{n})$ (hence a negligible contribution to $\bar s$, at worst logarithmically enhanced). Since $\mathrm{Var}_{\tilde W}(x)=\mathcal O(1)$ (thermal) and $s'$, $s''$ are bounded on $I$,
\begin{equation}
\label{eq:phase4-sbar-concentration}
\bar s\;=\; s(x_\star)\;+\;\delta s.
\end{equation}
In particular, $\bar s$ remains an $\mathcal O(1)$ quantity fixed by the KMS weight (no hidden $A$--dependence).
\end{proposition}
\begin{proof}[Sketch]
Normalize $\tilde W$ to a probability density on $\mathbb R_+$ and expand $s$ to second order around $x_\star$ and use $\mathbb E|x-x_\star|\le\sqrt{\mathrm{Var}_{\tilde W}(x)}$. The thermal width $\Delta x=\mathcal O(1)$ gives $\mathrm{Var}_{\tilde W}(x)=\mathcal O(1)$; with $s'$, $s''$ bounded on the bulk interval $I=[\varepsilon,\Lambda]$, the correction $\delta s$ is $\mathcal O(1)$ but $A$--independent. The tail $x<\varepsilon$ has weight $\mathcal O(\varepsilon^{n})$ by the $x^{n-1}$ behavior near zero, hence its contribution to $\bar s$ is negligible (at worst logarithmically enhanced); the tail $x>\Lambda$ is exponentially suppressed by the thermal factor (and further suppressed by $F$).
\end{proof}
\begin{remark}[Origin of $\bar{s}=\mathcal{O}(1)$]
\label{rem:sbar-origin}
The result $\bar{s} = s(x_\star) + \delta s$ with 
$x_\star = \mathcal{O}(1)$ is a structural consequence of the KMS 
spectral weight, not a numerical coincidence. The bosonic pair-entropy 
function $s(x) = x/(e^x-1) - \ln(1-e^{-x})$ is of order unity 
for $x = \mathcal{O}(1)$; since the KMS weight peaks at 
$x_\star = \mathcal{O}(1)$, the spectral average $\bar{s}$ is 
automatically of order unity. The value of $\bar{s}$ 
is fixed by the KMS spectral structure (not by $A$); it is not a tunable 
parameter but a consequence of the microscopic physics encoded in $p$ 
and the WESH kernel.
\end{remark}

\begin{remark}[Role of the single-pair entropy in the $\tfrac14$ law]
\label{rem:sbar-role}
The $1/4$ prefactor is fixed by the two structural $1/2$ reductions (bipartite pairing and swap–even boundary superselection, Sec.~\ref{sec:RAQ-quarter}). The spectral average $\bar s=\mathcal O(1)$ (Prop.~\ref{prop:concentration}) multiplies $N_{\rm phys}$ as an overall $A$–independent constant and only affects the identification of the microscopic cutoff $\xi$ with the phenomenological Planck scale in the matching to the emergent gravitational coupling.
\end{remark}

\paragraph{EFT stability of the kernel: curvature–induced deformations and domain of validity.}
Interpreting the mediator scale as $m_T=\xi^{-1}$, curvature renormalizes the quadratic kernel via diffeomorphism‑invariant local terms:

\begin{equation}
\label{eq:phase4-meff}
m_{\rm eff}^2 \;=\; m_T^2 \;+\; \alpha_1 R \;+\; \alpha_2 R_{\mu\nu}R^{\mu\nu} \;+\; \alpha_3 R_{\mu\nu\rho\sigma}R^{\mu\nu\rho\sigma} \;+\;\cdots,
\end{equation}
\vspace{0.3cm}
with Wilson coefficients $\alpha_i$ set at the Planck scale. For Schwarzschild, $R=0$ while $K:=R_{\mu\nu\rho\sigma}R^{\mu\nu\rho\sigma}\big|_{r\sim r_H}\sim r_H^{-4}$. Naturalness suggests $\alpha_3=\mathcal{O}(L_P^2)$, giving

\vspace{0.3cm}
\begin{align}
\label{eq:phase4-mshift}
\delta m_T^2 &\;\sim\; \alpha_3 K \;\sim\; L_P^2 K, \notag\\[0.3cm]
\frac{\delta m_T^2}{m_T^2} &\;\sim\; K L_P^4 \;=\; \mathcal{O}\!\big((L_P/r_H)^4\big), \\[0.3cm]
\frac{\delta\xi}{\xi} &\;\simeq\; -\tfrac{1}{2}\,\frac{\delta m_T^2}{m_T^2}. \notag
\end{align}
\vspace{0.3cm}

\noindent Extrinsic-curvature deformations at the horizon affect subleading coefficients (e.g. the logarithmic term) at order $\kappa L_P$ but do not alter the leading area law.

\begin{proposition}[Controlled kernel approximation and leading law]
\label{prop:kernel-controlled}
For macroscopic Schwarzschild horizons ($r_H\gg L_P$) one has

\[
\kappa\xi \;=\; \mathcal O(L_P/r_H)\ll 1,\qquad \frac{\delta\xi}{\xi} \;=\; \mathcal O\big((L_P/r_H)^4\big).
\]
\vspace{0.3cm}

\noindent Consequently: 

\vspace{0.3cm}
\textup{(i)} $x_\star=x_n+\mathcal O(\kappa\xi)$, where $x_n=\mathcal O(1)$ solves $\frac{n}{x}-1-\frac{1}{e^x-1}=0$ (cf.\ Prop.~\ref{prop:concentration}); 

\vspace{0.3cm}
\textup{(ii)} $\mathrm{Var}_{\tilde W}(x)=\mathcal O(1)$ (thermal width, cf.\ Prop.~\ref{prop:concentration}); 

\vspace{0.3cm}
\textup{(iii)} $\bar s=\mathcal O(1)$ (no hidden $A$-dependence). Therefore the leading Bekenstein--Hawking term derived in Secs.~\ref{sec:KMS-Rindler}--\ref{sec:RAQ-quarter} is stable under the EFT kernel deformations controlled here, and one retains

\begin{equation}
\label{eq:phase4-leading-robust-final}
S_{\rm BH} \;=\; \frac{A}{4L_P^2} \;+\; \gamma\ln\!\frac{A}{L_P^2} \;+\; k_0 \;+\; \mathcal O\!\Big(\frac{L_P^2}{A}\Big),
\end{equation}

\noindent where the subleading terms are controlled within the EFT expansion.
\end{proposition}

\begin{proof}[Sketch]
Use $\kappa\sim r_H^{-1}$ and $\xi\sim L_P$ to obtain $\kappa\xi\sim L_P/r_H$. Item (i) follows from \eqref{eq:phase4-stationary}; item (ii) from Prop.~\ref{prop:concentration}. For (iii), $\bar s=\mathcal O(1)$ (Remark~\ref{rem:sbar-origin}) guarantees that the spectral integral cannot generate any hidden $A$-dependent prefactor; hence it cannot renormalize the leading coefficient fixed by the two $1/2$ factors (pairing and swap–even boundary superselection) in Secs.~\ref{sec:KMS-Rindler}--\ref{sec:RAQ-quarter}. Kernel deformations are suppressed by \eqref{eq:phase4-mshift}, so only subleading terms are affected at this order.
\end{proof}

\begin{remark}[Thermal width vs.\ coarse-graining error]
The estimate $\mathrm{Var}_{\tilde W}(x)=\mathcal O(1)$ refers to the thermal width of the KMS weight in the dimensionless variable $x=\beta_H\omega$. The $\mathcal O((\kappa\xi)^2)$ terms appearing in Lemma~\ref{lem:pair-Markov} arise instead from $\xi$-scale null-buffer coarse-graining on the modular length $\kappa^{-1}$, and are logically independent of the thermal variance.
\end{remark}

\begin{remark}[Domain of validity]
For astrophysical black holes $r_H/L_P\gg 1$, the deformation $\delta\xi/\xi$ is negligible at leading order. For primordial or near‑Planckian horizons the expansion in $L_P/r_H$ must be resummed; shifts of $x_\star$ and dressing of subleading coefficients then provide a concrete arena for beyond‑leading‑order predictions within the same formalism.
\end{remark}
\vspace{0.3cm}

\subsection{Subleading corrections and unification with GR: \texorpdfstring{$\Delta_{\rm phys}$}{Delta_phys} as Hessian and the logarithmic term via replicas}
\label{sec:subleading-GR}
\vspace{0.5cm}

\paragraph*{Overview.}
Quantum corrections to black–hole entropy in QFTT–WESH are controlled by the same effective action whose first variation yields the emergent Einstein sector. At one loop, the physical fluctuation operator is the Hessian of the gauge–fixed WESH effective action evaluated on a black–hole background. The logarithmic coefficient is then obtained from the replica (or conical) variation of the horizon–local heat–kernel coefficient. This subsection formalizes these statements and makes the connection to the pre‑geometric and modular structures developed in Secs.~\ref{sec:wesh-noether}, \ref{sec:KMS-Rindler} and \ref{sec:RAQ-quarter} explicit.

\paragraph{Physical fluctuation operator.}
\vspace{0.3cm}

Let $S_{\rm eff}[\Phi]$ denote the (gauge-fixed) WESH effective action whose stationary points reproduce the GR limit (Sec.~4). For a stationary black-hole background $\bar\Phi$ (metric, time-field, mediator, gauge-fixing and ghost data), write $\varphi=\Phi-\bar\Phi$. The quadratic expansion is

\vspace{0.3cm}
\begin{equation}
\label{eq:Hessian}
S_{\rm eff}[\bar\Phi+\varphi]
= S_{\rm eff}[\bar\Phi] \;+\; \frac{1}{2}\,\langle \varphi,\,\Delta_{\rm phys}\,\varphi\rangle \;+\; \mathcal{O}(\varphi^3),
\end{equation}
\vspace{0.2cm}
\noindent where
\begin{equation}
\label{eq:Hessian-def}
\Delta_{\rm phys} \;\equiv\; \left.\frac{\delta^2 S_{\rm eff}}{\delta\Phi\,\delta\Phi}\right|_{\bar\Phi}.
\end{equation}
\vspace{0.3cm}

\noindent Blockwise,

\vspace{0.3cm}
\begin{equation}
\label{eq:blockwise}
\Delta_{\rm phys}\;=\;
\begin{bmatrix}
\Delta_{g} & \Delta_{gT} & \cdots\\[0.3cm]
\Delta_{Tg} & \Delta_{T} & \cdots\\[0.3cm]
\vdots & \vdots & \ddots
\end{bmatrix}
\ \oplus\ \Delta_{\rm gh},
\end{equation}
\vspace{0.3cm}

\noindent where $\Delta_{\rm gh}$ arises from Faddeev--Popov ghosts and gauge fixing. The refined constraint projection (RAQ/group averaging, Sec.~6.3) implements the physical subspace at quadratic order: null directions associated with first-class constraints are quotiented before taking determinants, so that only gauge-inequivalent modes contribute.

\begin{theorem}[One–loop generator of subleading entropy]
\label{thm:Hessian-entropy}
Let $S_{\rm eff}$ be the WESH effective action whose classical limit yields GR. Then the one–loop correction to black–hole entropy is governed by the quadratic form defined by $\Delta_{\rm phys}=\delta^2 S_{\rm eff}/\delta\Phi^2|_{\bar\Phi}$ on the RAQ‑projected fluctuation space. In particular, the logarithmic coefficient $\gamma$ is determined by the horizon–local Seeley–DeWitt density of $\Delta_{\rm phys}$ on the replica/conical geometry.
\end{theorem}

\begin{tcolorbox}[
  enhanced, breakable,
  title={\centering\textbf{Box 8: Logarithmic Corrections via the WESH Hessian}},
  colback=gray!10, colframe=black, boxrule=0.4pt, arc=2pt,
  left=6pt, right=6pt, top=8pt, bottom=8pt,
  before skip=10pt, after skip=14pt
]

\noindent\textbf{Setup.}
Let $\mathcal M^{(\alpha)}$ be the $\alpha$-replicated Euclidean manifold with angular
periodicity $2\pi\alpha$ around the smooth horizon cross-section $\mathcal H$.
Define the gauge-fixed WESH effective action pulled back to $\mathcal M^{(\alpha)}$
as $S_{\rm eff}^{(\alpha)}[\Phi]$, with background $\bar\Phi$ solving the classical
equations away from the conical locus (and regular at $\alpha=1$),
with the standard conical/brane prescription at the tip. Writing $\Phi=\bar\Phi+\varphi$ and expanding to quadratic order,

\vspace{0.3cm}
\[
S_{\rm eff}^{(\alpha)}[\bar\Phi+\varphi]
= S_{\rm eff}^{(\alpha)}[\bar\Phi] + \tfrac12\,\langle\varphi,\Delta_{\rm phys}^{(\alpha)}\varphi\rangle
+ \mathcal O(\varphi^3),
\]
\vspace{0.3cm}
\[
\Delta_{\rm phys}^{(\alpha)}:=\left.\frac{\delta^2 S_{\rm eff}^{(\alpha)}}{\delta\Phi\,\delta\Phi}\right|_{\bar\Phi}.
\]

\vspace{0.3cm}
\noindent Here $\Delta_{\rm phys}^{(\alpha)}$ is the \emph{projected} quadratic operator on the physical
fluctuation space: it includes Faddeev-Popov ghosts, gauge-fixing, and the RAQ/constraint projection (Sec.~6.3),
so that pure-gauge/constraint zero-modes are removed at the outset (cf.\ App.~F-H).

\vspace{0.3cm}
\noindent\textbf{Gaussian path integral and zeta regularization.}
The one-loop partition function on $\mathcal M^{(\alpha)}$ is Gaussian,

\vspace{0.3cm}
\[
Z_{1\text{-loop}}(\alpha)\propto\big(\text{sdet}\,\Delta_{\rm phys}^{(\alpha)}\big)^{-1/2},
\quad
W_1(\alpha):=-\ln Z_{1\text{-loop}}(\alpha)=\tfrac12\,\ln\text{sdet}\,\Delta_{\rm phys}^{(\alpha)}.
\]

\vspace{0.3cm}
\noindent Using zeta-function regularization,

\vspace{0.3cm}
\begin{align*}
\ln\det\nolimits_\zeta \Delta
&= -\left.\frac{d}{dz}\right|_{z=0}\frac{1}{\Gamma(z)}\!\int_0^\infty\! d\tau\,\tau^{\,z-1}\,\text{Tr}(e^{-\tau\Delta})\\
&= -\!\int_{0^+}\!\frac{d\tau}{\tau}\,\text{Tr}(e^{-\tau\Delta})+\cdots,
\end{align*}

\vspace{0.3cm}
\noindent and in $d=4$ the heat-kernel expansion 
$\text{Tr}(e^{-\tau\Delta})\sim (4\pi \tau)^{-2}\sum_{k\ge0} a_k(\Delta)\,\tau^k$
shows that the \emph{logarithmic} sensitivity comes entirely from $a_2$.

\vspace{0.3cm}
\noindent\textbf{Replica extraction of the logarithmic coefficient.}
The von Neumann entropy follows from
$S_{1\text{-loop}}=(\alpha\partial_\alpha-1)W_1(\alpha)\big|_{\alpha=1}$.
Isolating the log term in $d=4$:

\vspace{0.3cm}
\begin{equation}
\boxed{\begin{aligned}
\gamma&=\tfrac12\,\partial_\alpha\!\left[a_2\!\left(\Delta_{\rm phys}^{(\alpha)}\right)-\alpha\,a_2\!\left(\Delta_{\rm phys}^{(1)}\right)\right]_{\alpha=1},\\[4pt]
S_{\rm BH}&=\frac{A}{4L_P^2}+\gamma\ln\!\frac{A}{L_P^2}+\cdots\,.
\end{aligned}}
\label{eq:gamma-from-a2}
\end{equation}

\vspace{0.3cm}
\noindent Equation \eqref{eq:gamma-from-a2} is the precise Hessian$\to$replica link used in Sec.~6.5.

\vspace{0.3cm}
\noindent\textbf{Horizon-local form.}
On smooth sections $\mathcal H$, the conical variation localizes on $\mathcal H$:

\vspace{0.3cm}
\begin{multline*}
a_2\!\big(\Delta_{\rm phys}^{(\alpha)}\big)-\alpha\,a_2\!\big(\Delta_{\rm phys}^{(1)}\big)
=\!\int_{\mathcal H}\! dA\,\Big[b_1(\alpha)R_{\mathcal H}+b_2(\alpha)K_{ab}K^{ab}\\
+b_3(\alpha)\text{Tr}\,E+b_4(\alpha)\text{Tr}(\Omega_{ab}\Omega^{ab})\Big], \end{multline*} where the coefficient functions $b_i(\alpha)$ encode the purely conical contribution and satisfy $b_i(1)=0$.

\vspace{0.3cm}
\noindent so that for bifurcation surfaces of stationary Killing horizons (in particular Schwarzschild, where $K_{ab}=0$) one gets
$\gamma=\sigma_{\rm phys}\,\chi(\mathcal H)$ with
$\chi(\mathcal H)=\tfrac{1}{4\pi}\!\int_{\mathcal H}\!R_{\mathcal H}\,dA$;
for non-minimal/deformed horizon sections the extrinsic piece
$\propto \partial_\alpha b_2(\alpha)|_{\alpha=1}\int_{\mathcal H}\!K_{ab}K^{ab}\,dA$
yields a calculable shift. For stationary Kerr, rotation dependence is encoded in the full
horizon-local invariants of $\Delta_{\rm phys}$ (in particular through the $E$ and $\Omega$ sectors),
with extrinsic terms contributing only when the chosen section is not minimal.

\vspace{0.3cm}
\noindent\textbf{Remarks.}
\begin{enumerate}[leftmargin=1.35cm,itemsep=0.3cm]
\item \emph{Same $S_{\rm eff}$ on and off the cone.} $\Delta_{\rm phys}^{(\alpha)}$ is the Hessian of the same
$S_{\rm eff}$ that yields GR at tree level (Sec.~4); RAQ/constraint projection ensures gauge-invariant determinants.

\item \emph{Consistency with gravitational replica / cosmic branes.}
For $S_{\rm eff}\!\to\!S_{\rm EH}$, \eqref{eq:gamma-from-a2} reproduces Lewkowycz-Maldacena/Dong; extra WESH fields
modify $a_2$ in the standard way.

\item \emph{Boundary terms and edge modes.} GHY/Hayward and edge modes enter via gauge-fixing and physical projection:
they shift $a_2$ with local functions on the horizon without affecting the leading $A/(4L_P^2)$.

\item \emph{Scheme independence.} Power divergences renormalize the couplings (e.g., $1/G_N$); the log coefficient $\gamma$ is universal.
\end{enumerate}

\end{tcolorbox}

\vspace{0.5cm}

\paragraph{Replica/cone evaluation and the log coefficient.}
On the Euclidean $\alpha$–replica ($2\pi\alpha$ periodicity around the horizon), the one–loop generating functional is
\begin{equation}
W_{1}(\alpha)\;=\;-\ln Z_{1\text{-loop}}(\alpha)\;=\;\frac{1}{2}\,\ln\det\!\big(\Delta_{\rm phys}^{(\alpha)}\big),
\end{equation}
with $\Delta_{\rm phys}^{(\alpha)}$ the lifted operator (including ghosts and the physical projection). The von Neumann entropy follows from
\begin{equation}
\label{eq:replica-entropy}
S_{\rm 1\text{-loop}}
\;=\; \left.(\alpha\,\partial_\alpha-1)\,W_{1}(\alpha)\right|_{\alpha=1}
\end{equation}

\noindent Using the heat-kernel representation and the small-$s$ expansion in $d=4$,

\vspace{0.3cm}
\begin{align}
\ln\det\Delta &\;=\; -\int_{0}^{\infty}\!\frac{d\tau}{\tau}\,\mathrm{Tr}\,e^{-\tau\Delta}, \notag\\[0.3cm]
\mathrm{Tr}\,e^{-\tau\Delta_{\rm phys}^{(\alpha)}} &\sim \frac{1}{(4\pi \tau)^{2}}\sum_{k\ge0} a_k\!\left[\Delta_{\rm phys}^{(\alpha)}\right]\,\tau^{k}, \notag
\end{align}
\vspace{0.3cm}

\noindent the $\ln$-sensitive piece is controlled by $a_2$. Varying with respect to $\alpha$ yields

\vspace{0.3cm}
\begin{equation}
\boxed{\ 
\gamma \;=\; \frac{1}{2}\,\partial_\alpha\bigg|_{\alpha=1}\Big(a_2\!\left[\Delta_{\rm phys}^{(\alpha)}\right]-\alpha\,a_2\!\left[\Delta_{\rm phys}^{(1)}\right]\Big),
\qquad 
S_{\rm BH}\;=\;\frac{A}{4L_P^2}\;+\;\gamma\,\ln\!\frac{A}{L_P^2}\;+\;\cdots\ .
\ }
\label{eq:gamma-master}
\end{equation}

\paragraph{Horizon-local structure.}
\vspace{0.3cm}

The coefficient $a_2$ on a smooth horizon cross-section $\mathcal{H}$ decomposes into intrinsic/extrinsic invariants:

\vspace{0.3cm}
\begin{equation}
\label{eq:a2-horizon}
\begin{aligned}
a_2\!\left[\Delta_{\rm phys}^{(\alpha)}\right]-\alpha\,a_2\!\left[\Delta_{\rm phys}^{(1)}\right]
= \int_{\mathcal{H}}\! dA\ \Big[\ 
&b_1(\alpha)\,R_{\mathcal{H}}
\;+\; b_2(\alpha)\,K_{ab}K^{ab}\\
&\;+\; b_3(\alpha)\,\mathrm{Tr}\,E
\;+\; b_4(\alpha)\,\mathrm{Tr}\,\Omega_{ab}\Omega^{ab}\,\Big],
\end{aligned}
\end{equation}
where the coefficient functions $b_i(\alpha)$ encode the purely conical contribution and satisfy $b_i(1)=0$.
\vspace{0.3cm}

\noindent where $R_{\mathcal{H}}$ is the intrinsic scalar curvature of $\mathcal{H}$, $K_{ab}$ its extrinsic curvature, $E$ the endomorphism term in the second-order operator, and $\Omega$ the field-strength of the relevant bundle connection after gauge fixing. The sector decomposition inherited from $\Delta_{\rm phys}$ naturally splits a time-field contribution and a constraint/ghost (projection) contribution. For regular spherical horizons ($K_{ab}=0$),

\vspace{0.3cm}
\begin{equation}
\boxed{\begin{aligned}
\gamma \;=\; \sigma_{\rm phys}\,\chi(\mathcal{H}), \qquad 
\sigma_{\rm phys}\;=\;\sigma_T+\sigma_{\rm proj},\\[0.3cm]
\chi(\mathcal{H}) \;=\;\frac{1}{4\pi}\int_{\mathcal{H}}\! R_{\mathcal{H}}\,dA\ ,
\end{aligned}}
\label{eq:gamma-topological}
\end{equation}
\vspace{0.3cm}

\noindent while for deformed horizons (e.g. rotating cases) the extrinsic term produces a calculable shift

\vspace{0.3cm}
\begin{equation}
\label{eq:Delta-gamma}
\Delta\gamma \;=\; \frac{1}{2}\,\partial_\alpha b_2(\alpha)\big|_{\alpha=1}\ \int_{\mathcal{H}} K_{ab}K^{ab}\,dA,
\end{equation}
\vspace{0.3cm}

\noindent encoding spin-dependent corrections.

\begin{remark}[Numerical value of \texorpdfstring{$\sigma_{\rm phys}$}{sigma\_phys}]
The coefficient $\sigma_{\rm phys}$ is fixed by the horizon–local Seeley–DeWitt density
of the physical Hessian $\Delta_{\rm phys}$ on the replica/conical geometry. Its explicit
evaluation requires the full mode decomposition of $\Delta_{\rm phys}$, including time‑field,
metric, ghost and constraint sectors, and will be carried out in a dedicated follow‑up.
Here we only use the structural decomposition
\[
\gamma=\sigma_{\rm phys}\,\chi(\mathcal H)+\Delta\gamma(J),
\]
which is the falsifiable prediction of the WESH framework: different microscopic
realizations give different $\sigma_{\rm phys}$ and $\Delta\gamma(J)$, but must respect
this horizon–local form.
\end{remark}

\paragraph{Renormalization and universality.}
The one–loop functional $\tfrac12\ln\det(\Delta_{\rm phys}/\mu^2)$ contains a renormalization scale $\mu$. Area‑divergent pieces renormalize $1/G_N$ and are absorbed by the classical $A/(4L_P^2)$ term; the remaining finite $\ln$–piece is universal and fixed by $a_2$. Matching to the microscopic WESH scale $\xi^{-1}\sim L_P^{-1}$ ensures $\gamma=\mathcal O(1)$ with no dependence on long‑distance details. Because $\Delta_{\rm phys}$ is the Hessian of the same $S_{\rm eff}$ that yields GR at leading order, the field content, normalizations and gauge choices entering $\gamma$ are not tunable inputs but are already fixed by the emergent sector and by the RAQ projection (Sec.~\ref{sec:RAQ-quarter}).

\paragraph{Example: smooth $S^2$ and deformations.}
For a regular $S^2$ cross–section, $\chi(S^2)=2$ and $K_{ab}=0$, hence
\[
\gamma_{S^2}=2\,\sigma_{\rm phys} \qquad (\sigma_{\rm phys}=\sigma_T+\sigma_{\rm proj}).
\]
For Kerr‑type horizons, $K_{ab}K^{ab}\neq 0$ on generic sections and the $b_2$–term in \eqref{eq:a2-horizon} generates a spin‑dependent $\Delta\gamma[J]$. The same framework applies in NHEK limits and admits checks via null‑surface modular Hamiltonians and Rényi continuations.

\paragraph{Synthesis.}
Equations \eqref{eq:Hessian}–\eqref{eq:gamma-master} close the logical chain: the pre‑geometric WESH–Noether constraint (Sec.~\ref{sec:wesh-noether}) selects admissible generators; modular/KMS analysis on the wedge (Sec.~\ref{sec:KMS-Rindler}) fixes the thermal structure; RAQ implements the physical projection (Sec.~\ref{sec:RAQ-quarter}); and the same $S_{\rm eff}$ that yields the Einstein sector at leading order provides, through its Hessian $\Delta_{\rm phys}$, the universal logarithmic correction via the replica/cone method. The subleading structure is thus a horizon‑local prediction tied to the very mechanism that produces the classical gravitational dynamics.
\vspace{0.8cm}

\subsection{Kerr horizons: rotating KMS, hidden symmetries, and WESH thermodynamics}
\label{sec:Kerr-WESH}

\vspace{0.5cm}
\paragraph*{Overview.}
The WESH near-horizon thermodynamics extends from static to stationary rotating (Kerr) black holes once (i) the thermal flow is generated by the horizon Killing field
$\chi^\mu=\partial_t^\mu+\Omega_H\,\partial_\phi^\mu$, so that KMS relations involve the corotating frequency $\tilde\omega\equiv \omega-m\Omega_H$, and (ii) the $\hat T^2$ channel is organized covariantly under the hidden symmetry tower generated by the principal conformal Killing–Yano tensor. With these replacements, the fixed-point/KMS derivation, the bipartite halving plus swap–even projection, and the Hessian/replica evaluation of logarithmic corrections carry over verbatim: the leading $A/4L_P^2$ law is universal, while subleading terms acquire Kerr-specific (spin-dependent) structure through the horizon-local invariants of $\Delta_{\rm phys}$ (in particular the $E$ and $\Omega$ sectors) and superradiant shifts in the spectral weights, with extrinsic terms contributing only for non-minimal/deformed sections.

\paragraph{Assumptions and notation.}
Work on a four-dimensional stationary axisymmetric Kerr spacetime with mass $M$ and angular momentum $J=aM$.
All KMS statements are understood on the near-horizon corotating wedge algebra
(a local statement). A global $\chi$–KMS/Hartle--Hawking completion in asymptotically
flat Kerr requires suitable boundary conditions to control superradiance, but is not
needed for the wedge analysis used here.
The outer horizon has surface gravity $\kappa$ and angular velocity $\Omega_H$. Frequencies decompose in azimuthal modes $(\omega,m)$; the corotating combination is $\tilde\omega=\omega-m\Omega_H$. Units $\hbar=c=k_B=G=1$ are used. The WESH short‑range scale remains $\xi\simeq L_P$ to leading order; curvature‑induced renormalizations are treated in EFT and are suppressed by powers of $L_P/M$ for macroscopic horizons.

\paragraph{Rotating KMS and detailed balance.}
Let $\mathcal A_{\rm wedge}$ be the algebra of observables in the exterior corotating wedge adapted to $\chi^\mu$. A state $\sigma_{\chi{\rm KMS}}$ is \emph{$\chi$–KMS at} $\beta_H=2\pi/\kappa$ if for all $A,B\in\mathcal A_{\rm wedge}$, $F_{A,B}(t)=\langle A\,\alpha_t^\chi(B)\rangle$ analytically extends to $0<\Im t<\beta_H$ and obeys $F_{A,B}(t+i\beta_H)=\langle \alpha_t^\chi(B)\,A\rangle$, where $\alpha_t^\chi$ is the Heisenberg evolution generated by $K_\chi=i\mathcal L_\chi$. In Fourier modes,
\begin{equation}
\widetilde C_\chi(-\omega,-m)=e^{-\beta_H(\omega-m\Omega_H)}\,\widetilde C_\chi(\omega,m),
\qquad \beta_H=\frac{2\pi}{\kappa}.
\label{eq:kerr-kms}
\end{equation}

\begin{theorem}[Rotating KMS fixed point]
\label{thm:Kerr-KMS}
Under the pre‑geometric WESH–Noether constraint (Sec.~\ref{sec:wesh-noether}), locality/causality of the kernels, and wedge modular covariance, and assuming the Kerr analogue of the primitivity hypotheses used in the static wedge case (spectral regularity, cross-horizon support, null-plane Markovity), the WESH generator reduced to $\mathcal A_{\rm wedge}$ is $\sigma_{\chi{\rm KMS}}$‑detailed balanced, and $\sigma_{\chi{\rm KMS}}$ is the unique faithful normal stationary state on $\mathcal A_{\rm wedge}$. Consequently, the frequency‑resolved up/down rates for the $\hat T^2$ channel satisfy
\begin{equation}
\frac{\Gamma_\uparrow(\omega,m)}{\Gamma_\downarrow(\omega,m)}
=\exp\!\Big[-\beta_H\big(\omega-m\Omega_H\big)\Big],
\label{eq:kerr-ud}
\end{equation}
and the Kerr horizon weight entering spectral averages is
\begin{equation}
W_{\rm Kerr}(\omega,m)
=\Gspec(\omega,m)\,\rho_{\rm Kerr}(\omega,m)\,
\frac{e^{-\beta_H(\omega-m\Omega_H)}}{1-e^{-\beta_H(\omega-m\Omega_H)}}.
\label{eq:kerr-weight}
\end{equation}
\end{theorem}

\paragraph{Superradiance and positivity.}
Spectral averages (e.g.\ \eqref{eq:kerr-sbar}) are understood over the corotating positive-frequency sector $\tilde\omega=\omega-m\Omega_H>0$. The $\tilde\omega<0$ (superradiant) contributions are accounted for by the KMS/detailed-balance pairing of modes, i.e.\ by mapping $(\omega,m)\leftrightarrow(-\omega,-m)$ using \eqref{eq:kerr-kms}--\eqref{eq:kerr-ud}, so that the effective weight entering the averages is positive. Greybody factors/boundary conditions enter only through the positive response $\Gspec\,\rho_{\rm Kerr}$. The bilocal WESH channel and the RAQ physical projection maintain complete positivity of the reduced dynamics as in the static case, with modular flow generated by $K_\chi$.

\paragraph{Hidden‑symmetry covariance of the dissipator.}
Kerr admits a principal CKY 2‑form $h_{\mu\nu}$; its square yields the Killing tensor $K_{\mu\nu}$ that organizes separability. To align dissipation with this structure, define quadratic densities
\begin{equation}
\mathcal O_q(x)=\hat T_{\mu\nu}(x)\,\big(\mathcal Q^{\mu\nu}{}_{\alpha\beta}\big)_q\,\hat T^{\alpha\beta}(x),
\qquad \big(\mathcal Q\big)_q\in\mathrm{Alg}\!\big[g_{\mu\nu},K_{\mu\nu},h_{\mu\nu},\nabla\big],
\label{eq:kerr-hidden}
\end{equation}
and select the scalar in the hidden‑symmetry decomposition as the dominant dissipative component. Near the horizon, Cauchy sections degenerate to null sheets generated by $\chi^\mu$; the modular Hamiltonian is local along such null generators for weakly‑coupled sectors, matching the corotating KMS structure \eqref{eq:kerr-kms}.
\vspace{0.5cm}

\begin{tcolorbox}[
  enhanced, 
  breakable,
  break at=-\baselineskip,
  title={\centering Box 9: Kerr Dissipator Selection Principle},
  colback=gray!10, 
  colframe=black, 
  boxrule=0.4pt, 
  arc=2pt,
  left=6pt, 
  right=6pt, 
  top=8pt, 
  bottom=8pt,
  before skip=10pt, 
  after skip=14pt
]

\paragraph{Set--up.}
Let $(\mathcal{M},g)$ be a Kerr (or NHEK) background with horizon Killing field
$\chi^\mu=\partial_t^\mu+\Omega_H\,\partial_\phi^\mu$, principal conformal Killing--Yano form
$h_{\mu\nu}$ and associated Killing tensor $K_{\mu\nu}$. Denote by $K_\chi\equiv i\mathcal{L}_\chi$ the
corotating modular generator and by $\sigma_{\chi{\rm KMS}} \propto e^{-\beta_H K_\chi}$ the
$\chi$--KMS state at $\beta_H=2\pi/\kappa$. On the wedge algebra $\mathfrak A(\mathcal W_\chi)$,
equip observables with the reversible inner product
$\langle A,B\rangle_\chi := \Tr(\sigma_{\chi{\rm KMS}}^{1/2}A^\dagger\sigma_{\chi{\rm KMS}}^{1/2}B)$.
Consider admissible quadratic densities for the WESH dissipator of the form
\[
\mathcal O_q(x)\;=\;T_{\mu\nu}(x)\,\big(\mathcal Q^{\mu\nu}{}_{\alpha\beta}\big)_q\,T^{\alpha\beta}(x),
\qquad
\big(\mathcal Q\big)_q \in \mathrm{Alg}\!\big[g_{\cdot\cdot},K_{\cdot\cdot},h_{\cdot\cdot},\nabla_{\!\cdot}\big],
\]
with exponentially causal kernel of range $\xi$ as in Sec.~6.2, and subject to the WESH--Noether commutant
constraints $[H_{\rm eff},\hat Q_a]=[\,\mathcal O_q(x),\hat Q_a]=0$ for all global charges $\hat Q_a$ (Appendix~H).
\vspace{0.3cm}

\paragraph{Dirichlet form and spectral gap.}
For a GKSL generator $\mathcal L_q$ built from $\mathcal O_q$ and its bilocal difference $L_{xy}^{(q)}$,
reversibility with respect to $\sigma_{\chi{\rm KMS}}$ (detailed balance) implies that
$\mathcal L_q$ is self--adjoint in $\langle\cdot,\cdot\rangle_\chi$ and defines the non--negative Dirichlet form
\[
\mathcal E_\chi^{(q)}[A]\;:=\;-\langle A,\mathcal L_q[A]\rangle_\chi, 
\qquad
\lambda_{\rm gap}(\mathcal L_q)\;:=\;\inf_{\substack{A=A^\dagger\\ \langle A,\mathbf 1\rangle_\chi=0}}
\frac{\mathcal E_\chi^{(q)}[A]}{\langle A,A\rangle_\chi},
\]
which governs the mixing rate $\,\|e^{s\mathcal L_q}[A]-\langle A\rangle_{\sigma}\mathbf 1\|_\chi
\le e^{-\lambda_{\rm gap}s}\|A-\langle A\rangle_{\sigma}\mathbf 1\|_\chi$.
\vspace{0.3cm}

\begin{principle}[Hidden--symmetry alignment]\label{prin:hidden-alignment}
Among all admissible quadratic densities $\{\mathcal O_q\}$ built from the hidden--symmetry tower,
\emph{choose} $\mathcal O_{q^\star}$ to \textbf{maximize} the reversible spectral gap:
\[
q^\star\;\in\;\arg\max_{q}\ \lambda_{\rm gap}(\mathcal L_q),
\]
\emph{subject to} (i) WESH--Noether commutant constraints, (ii) $\chi$--KMS detailed balance, and
(iii) exponential--causal kernel of range $\xi$. The maximizer is the \emph{$K_\chi$--invariant scalar}
in the hidden--symmetry decomposition, i.e.\ the scalar constructed from the principal tensor tower
that commutes with $K_\chi$ and the axial $U(1)$, thereby yielding the fastest mixing to the
$\chi$--KMS fixed point.
\end{principle}

\begin{proof}[Proof sketch]
(\emph{1}) By axial and corotating invariance, $\mathfrak A(\mathcal W_\chi)$ decomposes into
Fourier modules labelled by $(\omega,m)$ for $K_\chi$ and $\partial_\phi$. Reversibility block--diagonalizes
$\mathcal L_q$ across these modules. \\
(\emph{2}) Within each block, the WESH--Noether commutant condition suppresses directions that transform
nontrivially under the hidden--symmetry algebra generated by $\{K_{\mu\nu},h_{\mu\nu}\}$. By Schur's
lemma, the \emph{only} choice whose jump set has trivial commutant on every $(\omega,m)$ block is the scalar
component of the hidden--symmetry tower. Any vector/tensor choice leaves a nontrivial commutant, creating
$\mathcal L_q$--invariant subalgebras and strictly reducing the Rayleigh quotient that defines
$\lambda_{\rm gap}$. \\
(\emph{3}) Because the bilocal channel furnishes cross-horizon connectivity and the vacuum is Markov on null cuts,
each $(\omega,m)$ block is ergodic once the dissipator is $K_\chi$--invariant and scalar; detailed balance then
identifies the unique fixed point with the $\chi$--KMS state and the corresponding gap is maximal among admissible
choices. Finally, the scalar constructed from the principal tensor tower is the unique such $K_\chi$--invariant. 
\qedhere
\end{proof}

\paragraph{Consequences.}
(i) \emph{Uniqueness and fastest mixing.} The selected dissipator ensures primitivity and maximizes the reversible
spectral gap, giving the fastest approach to $\chi$--KMS at $\beta_H=2\pi/\kappa$ (Sec.~6.2). (ii) \emph{Continuity to Schwarzschild.}
In the limit $\Omega_H\!\to 0$ the principle reproduces the static scalar channel used in Sec.~6.2. (iii) \emph{Kerr/CFT
compatibility.} $K_\chi$--invariance aligns the dissipative weights with the corotating Virasoro zero--mode used in the
Kerr/CFT literature, ensuring that subleading data (e.g.\ $\gamma(J)$) can be cross--checked against horizon symmetry
charges. This principle leverages that structure for the variational selection of the dissipator. 

\paragraph{Link to the WESH monotone.}
In the reversible geometry $\langle\cdot,\cdot\rangle_\chi$, the Dirichlet form $\mathcal E_\chi^{(q)}$ is the quadratic
dissipation rate of the WESH Lyapunov functional (Appendix~D); hence maximizing $\lambda_{\rm gap}$ is equivalent to
\emph{minimizing} the relaxation time of the WESH monotone under the constraints above, i.e.\ an optimal--control
selection consistent with the Phase--F variational framework.

\end{tcolorbox}

\paragraph*{Status.}
The Hidden–Symmetry Alignment principle is \emph{not} derived from the core WESH
axioms. It is introduced as an additional selection criterion for the Kerr
extension, motivated by: (i) uniqueness and fastest mixing to the $\chi$–KMS state;
(ii) smooth reduction to the static case as $\Omega_H\to 0$; and (iii) compatibility
with Kerr/CFT structures. The argument in Box~9 should therefore be read as:
\emph{given} this principle, the admissible dissipator is uniquely fixed to the
scalar hidden–symmetry channel aligned with $K_\chi$; it does not constitute a
derivation of the principle itself from WESH–Noether.

\begin{remark}[Physical rationale]
The reversible spectral gap $\lambda_{\rm gap}$ controls the mixing time $s_{\rm mix}\!\sim\!\lambda_{\rm gap}^{-1}$ and quantifies the positivity-improving power of the semigroup on each Fourier block $(\omega,m)$.  
Any jump set leaving a non-trivial commutant—i.e.\ hidden-symmetry components not invariant under $K_\chi$—necessarily yields a smaller Rayleigh quotient and hence slower relaxation.  
The scalar component aligned with the principal Killing–Yano tower is the unique choice whose commutant is trivial, making the semigroup primitive and "fastest'' under all symmetry constraints.  
This variational criterion is the rotating analogue of the static local quadratic channel selected in Sec.~6.2.
\end{remark}

\paragraph{Leading area law and pair entropy in Kerr.}
\vspace{0.3cm}

Let $x=\beta_H(\omega-m\Omega_H)$. With the pair-entropy functional $s(x)=x/(e^x-1)-\ln(1-e^{-x})$, the Kerr average reads

\vspace{0.3cm}
\begin{equation}
\bar{s}_{\rm Kerr}
=\frac{\displaystyle \sum_m\!\int_0^\infty\!\frac{d\omega}{2\pi}\,\Theta(\omega-m\Omega_H)\,
W_{\rm Kerr}(\omega,m)\,s\!\big(\beta_H(\omega-m\Omega_H)\big)}
{\displaystyle \sum_m\!\int_0^\infty\!\frac{d\omega}{2\pi}\,\Theta(\omega-m\Omega_H)\,
W_{\rm Kerr}(\omega,m)}.
\label{eq:kerr-sbar}
\end{equation}
\vspace{0.3cm}

\noindent Assuming the low-frequency scalings justified in Sec.~\ref{sec:phase4-leading-robust} (Kerr wedge density $\rho_{\rm Kerr}\sim \tilde\omega^{\,p}$ and response $\Gspec\sim \tilde\omega^{\,k}\,\tilde F(\tilde\omega\,\xi)$ with smooth UV cutoff at $\xi^{-1}$), the weight $\tilde W(x)\propto x^n\,e^{-x}(1-e^{-x})^{-1}\tilde F(x)$ is single-peaked with $x_\star=x_n+\mathcal{O}(\kappa\xi)$ (cf.\ Prop.~\ref{prop:concentration}), hence

\vspace{0.3cm}
\begin{equation}
\bar{s}_{\rm Kerr}=\mathcal{O}(1)\quad\text{(no hidden $A$-dependence, cf.\ Remark~\ref{rem:sbar-origin})}.
\label{eq:kerr-sbar-one}
\end{equation}
\vspace{0.3cm}

\noindent Short-range mode counting gives $N\sim A/\xi^2$; bipartite halving yields $A/(2\xi^2)$; swap–even boundary superselection (Sec.~\ref{sec:RAQ-quarter}) contributes an additional factor $1/2$ in the $\chi$-adapted even sector. Therefore,

\vspace{0.3cm}
\begin{equation}
S_{\rm BH}^{\rm Kerr}
=\frac{A}{4L_P^2} \;+\; \gamma_{\rm Kerr}\,\ln\!\frac{A}{L_P^2} \;+\;c_{\rm Kerr}\;+\;\mathcal{O}\!\Big(\frac{L_P^2}{A}\Big),
\label{eq:kerr-leading}
\end{equation}
\vspace{0.3cm}

\noindent with the same universal leading coefficient.

\vspace{0.5cm}

\paragraph{Logarithmic corrections from the Hessian.}
\vspace{0.3cm}

Let $\Delta_{\rm phys}$ be the Hessian of the WESH effective action evaluated on the Kerr background (Sec.~\ref{sec:subleading-GR}). On the Euclidean replica/conical geometry, the log coefficient is

\vspace{0.3cm}
\begin{equation}
\label{eq:kerr-gamma}
\begin{aligned}
\gamma_{\rm Kerr}
&=\frac{1}{2}\,\partial_\alpha\Big|_{\alpha=1}\,\Big(a_2\!\left[\Delta_{\rm phys}^{(\alpha)}\right] 
-\alpha\,a_2\!\left[\Delta_{\rm phys}^{(1)}\right]\Big)\\[0.3cm]
&=\sigma_{\rm phys}\,\chi(\mathcal{H})\;+\;\frac{1}{2}\,\partial_\alpha b_2(\alpha)\Big|_{\alpha=1}\!
\int_{\mathcal{H}}\!K_{ab}K^{ab}\,dA\;+\;\cdots,
\end{aligned}
\end{equation}
\vspace{0.3cm}

\noindent where $\mathcal H$ is a smooth horizon section (topologically $S^2$), $\chi(\mathcal H)=2$, and the ellipsis denotes additional gauge/ghost terms fixed by the RAQ projection. Rotation dependence is encoded in the Kerr-sensitive horizon-local invariants of $\Delta_{\rm phys}$ (notably the $E$ and $\Omega$ sectors); extrinsic terms contribute only when the chosen horizon section is non-minimal/deformed. The resulting $\Delta\gamma(J)$ vanishes smoothly as $J\to0$.

\paragraph{Extremal (NHEK) limit and Kerr/CFT interface.}
In the NHEK limit ($\kappa\to 0$) the corotating KMS structure reduces to a zero‑temperature (ground‑state) KMS with respect to $K_\chi$. Hidden‑symmetry covariance dovetails with the integrable mode structure and offers a natural interface to Kerr/CFT tools (e.g., emergent Virasoro) for organizing subleading corrections and boundary data in the throat region.

\paragraph{EFT control of kernel deformations.}
Treating WESH as an EFT, the mediator mass $m_T=\xi^{-1}$ receives curvature‑dependent renormalizations $m_{\rm eff}^2=m_T^2+c_1 R + c_2 R_{\mu\nu\rho\sigma}R^{\mu\nu\rho\sigma}+\cdots$. For Kerr, $R=0$ while the Kretschmann scalar is nonzero; thus $\delta\xi/\xi=\mathcal O((L_P/M)^4)$ on macroscopic horizons. Leading area scaling is unaffected; Kerr‑specific shifts enter only subleading coefficients ($\gamma_{\rm Kerr}$, $c_{\rm Kerr}$).

\begin{corollary}[Leading law and Kerr log shift]
\label{cor:Kerr-leading-log}

\vspace{0.3cm}

With $\xi\simeq L_P$ and hidden-symmetry-covariant dissipation, 
\[
\bar{s}_{\rm Kerr}=\mathcal{O}(1)
\quad\text{and}\quad
S_{\rm BH}^{\rm Kerr}=\frac{A}{4L_P^2}+\cdots
\]
as in Eq.~\eqref{eq:kerr-leading}. The logarithmic coefficient splits as

\vspace{0.3cm}
\begin{equation}
\gamma_{\rm Kerr}=2\,\sigma_{\rm phys}+\Delta\gamma(J),
\end{equation}
\vspace{0.2cm}

\noindent with $\Delta\gamma(J)$ determined by the Kerr-sensitive horizon-local invariants of $\Delta_{\rm phys}$ (in particular the $E$ and $\Omega$ sectors, and extrinsic terms only for non-minimal sections),
\[
\Delta\gamma\!\to\!0\ \ \text{for}\ \ J\!\to\!0.
\]

\vspace{0.3cm}

\end{corollary}

\paragraph{Falsifiable signatures.}
(i) A spin‑dependent shift $\Delta\gamma(J)$ in logarithmic corrections (entering Rényi slopes) controlled by horizon-local invariants; (ii) superradiant deformations of near‑horizon spectral densities consistent with the $\tilde\omega$ KMS shift; (iii) hidden‑symmetry selection rules in WESH response correlators aligned with Kerr separability.

\paragraph{Synthesis.}
The rotating extension shows that once the correct modular flow ($K_\chi$) is used, WESH thermodynamics reproduces the universal leading area law and encodes rotation through controlled subleading structures governed by horizon-local invariants and hidden symmetries. Together with Secs.~\ref{sec:wesh-noether}, \ref{sec:KMS-Rindler}, \ref{sec:RAQ-quarter} and \ref{sec:subleading-GR}, this yields a unified chain: pre‑geometric conservation $\Rightarrow$ rotating KMS fixed point $\Rightarrow$ physical projection $\Rightarrow$ $A/4L_P^2$ universality $+$ Kerr‑specific logarithmic corrections.

\vspace{0.8cm}

\subsection{The holographic bound from WESH stationarity}
\label{sec:holographic-bound}

\vspace{0.5cm}
\paragraph*{Guiding claim.}
In QFTT--WESH the very existence of a well–defined emergent time map $t(s)$ with non-negative production rate,
\begin{equation}
\frac{dt}{ds} \;=\; \Gamma[\Psi]\;\ge\;0\,,
\label{eq:chrono-dt}
\end{equation}
with $\Gamma>0$ for forward-time production and $\Gamma=0$ at the stationary HH/BH fixed point,
\vspace{0.3cm}
imposes an information–geometric \emph{stability} constraint that \emph{upper–bounds} the admissible cross–sectional entanglement on codimension–two boundaries. In a near–horizon window this becomes a holographic inequality
\begin{equation}
S[\mathcal H]\;\le\;\frac{A(\mathcal H)}{4L_P^2}\;+\;\text{subleading},
\label{eq:chrono-holo-intro}
\end{equation}

\vspace{0.3cm}
\noindent with saturation at the Hartle–Hawking/KMS fixed point. The bound is \emph{derived} from $\Gamma[\Psi]\ge0$ together with the WESH–Noether conservation principle (Sec.~\ref{sec:wesh-noether}) and the wedge–modular structure underpinning the KMS fixed point (Sec.~\ref{sec:KMS-Rindler}); the normalization of the bound is fixed by the bipartite halving$\times$swap–even superselection (Sec.~\ref{sec:RAQ-quarter}) and the KMS spectral calibration (Sec.~\ref{sec:phase4-leading-robust}).

\paragraph{Local/bilocal split of the chronogenesis rate.}
The emergent–time production functional admits a canonical decomposition
\begin{equation}
\Gamma[\Psi]\;=\;\Gamma_{\rm loc}[\Psi]\;-\;\Gamma_{\rm cost}[\Psi],
\qquad 
\Gamma_{\rm loc}\ge 0,\quad \Gamma_{\rm cost}\ge 0,
\label{eq:Gamma-split}
\end{equation}

\vspace{0.3cm}
\noindent where $\Gamma_{\rm loc}$ collects strictly local GKSL contributions, fixed in sign by complete positivity and constrained by WESH--Noether, while $\Gamma_{\rm cost}$ encodes bilocal, entanglement–driven channels weighted by the mediator kernel $K_\xi$ (exponential–causal support with range $\xi\simeq L_P$). The sign convention emphasizes that cross–boundary correlations \emph{consume} chronogenetic capacity. 

\vspace{0.3cm}
\noindent Here $\Gamma_{\rm cost}$ denotes the information-theoretic cost of maintaining trans-horizon 
correlations; it is controlled by the bilocal channel and enters with a minus sign because 
cross-boundary entanglement \emph{consumes} chronogenetic capacity.

\begin{remark}[Physical interpretation of the split]
Local channels act independently on each side of the horizon and \emph{generate temporal order}
by aligning gradients (cf.\ Theorem~\ref{thm:alignment}); their contribution 
$\Gamma_{\rm loc}\!\ge\!0$ measures the intrinsic rate of time production inside each wedge. 
The bilocal channel, instead, couples trans-horizon pairs within a Planck-thick layer and must 
\emph{expend} part of this order to maintain cross-cut correlations. 
The associated information cost—proportional to the average mutual information of the pairs—appears with a minus sign as $\Gamma_{\rm cost}$. 
At the Hartle–Hawking/KMS fixed point the two effects balance ($\Gamma=0$): the horizon saturates the maximal sustainable cross-horizon correlation compatible with stationarity. 
For $\Gamma>0$, forward chronogenesis requires a sub-holographic correlation content, leading to the entropy bound derived below.
\end{remark}

\paragraph{Near–horizon modular control.}
Restricting to a Rindler wedge, the modular Hamiltonian generates boosts and admits a local, null–surface expression for suitable QFT sectors. This supplies sharp null–plane inequalities (relative entropy, Markov property) that relate modular energy to mutual information and provide area–type control of cross–horizon entanglement on $\mathcal H$.

\paragraph{Kernel geometry and area scaling.}
Boundary–crossing pairs are localized within a Planck–thick layer ($\sim\xi$) around $\mathcal H$. The same geometry that controls the mode count ($\sim A/\xi^2$, reduced by halving+projection) also bounds the bilocal subtraction in \eqref{eq:Gamma-split} through an $L^1$ kernel norm times an information density per pair:
\begin{equation}
\Gamma_{\rm cost}[\Psi]\;\lesssim\;\tilde c\,\frac{A}{\xi^2}\,\overline{\mathcal I}_\Psi,
\qquad 
\tilde c:=\|K_\xi\|_{L^1}\,\xi^{-4}=\mathcal O(1),
\label{eq:Gamma-threshold}
\end{equation}
with $\overline{\mathcal I}_\Psi$ a KMS–weighted pair–information functional (defined below).
\vspace{0.5cm}

\begin{lemma}[Pair–information control on null wedges]\label{lem:pair-Markov}
Let $\mathfrak A(\mathcal W_R)$ be the right–Rindler wedge algebra and consider three contiguous intervals $(A,B,C)$ along a fixed null generator with $B$ a buffer of thickness $\gtrsim\xi$ around the bifurcation surface. For the vacuum (or any $\chi$–KMS deformation with $\beta_H=2\pi/\kappa$), the null–plane Markov property implies $I(A:C\mid B)=0$ (saturation of strong subadditivity). For trans–horizon pairs $(x\in A,y\in C)$ drawn with kernel density $K_\xi$ and weighted by the horizon modular spectrum, the KMS–weighted pair information
\[
\overline{\mathcal I}_\Psi\;:=\;\frac{\displaystyle \int_0^\infty\!\frac{d\omega}{2\pi}\,W_{\rm HH}(\omega)\ \mathbb E_{(x,y)\sim K_\xi}\!\big[\,\mathcal I_\Psi(x:y)\,\big]}
{\displaystyle \int_0^\infty\!\frac{d\omega}{2\pi}\,W_{\rm HH}(\omega)}
\qquad\Big(\ \mathcal I_\Psi(x:y)=\tfrac12\,S_{\mathfrak A_{xy}}\!\big(\rho_{xy}\Vert\rho_x\!\otimes\!\rho_y\big)\ \Big)
\]
obeys the bound
\begin{equation}
\overline{\mathcal I}_\Psi\ \le\ \bar s\;+\;\mathcal O\!\big((\kappa\xi)^2\big),
\label{eq:pair-info-bound}
\end{equation}
where $\bar s$ is the KMS–weighted pair–entropy average (Sec.\,6.4) and the error term is controlled by $\xi$-scale null-buffer coarse-graining on the modular length $\kappa^{-1}$ (hence by $\kappa\xi$), and is $A$-independent. 
\end{lemma}

\begin{proof}[Sketch]
Markov saturation $I(A:C\mid B)=0$ on null triads gives a chain–rule identity $I(A:C)=I(A:BC)-I(A:B)$; with a buffer $B$ of size $\gtrsim\xi$, the long–range part of $I(A:C)$ is exhausted by local null contributions near the cut. Express $2\,\mathcal I_\Psi(x:y)$ as a relative entropy, $2\,\mathcal I_\Psi(x:y)=D\!\big(\rho_{xy}\,\|\,\rho_x\otimes\rho_y\big)$, and decompose fields in boost modes with weight $W_{\rm HH}(\omega)$ (Sec.\,6.2). By the entanglement first law around the $\chi$–KMS state, the leading variation of $D$ is set by the modular–energy variation and coincides with the bosonic pair–entropy functional $s(x)$ at $x=\beta_H\omega$, up to $\mathcal O\big((\kappa\xi)^2\big)$ corrections controlled by $\xi$-scale null–buffer coarse–graining on the modular length $\kappa^{-1}$ (and independent of $A$). Averaging over $\omega$ and the kernel $K_\xi$ gives \eqref{eq:pair-info-bound}, with $\bar s$ defined in Eq.~\eqref{eq:phase4-spectral-average}. The entanglement first-law step and modular-energy variation follow Casini, Testé, \& Torroba (2017), while the null-plane Markov property and relative-entropy control build on Blanco, Casini, Hung, \& Myers (2013); see Appendix~G for a detailed discussion.  \, \mbox{} \\
\end{proof}

\begin{corollary}[Calibrated chronogenetic threshold]\label{cor:chrono-threshold}
With the chronogenetic split $\Gamma=\Gamma_{\rm loc}-\Gamma_{\rm cost}$, $\Gamma_{\rm loc}\ge0$ and
\vspace{0.3cm}
\begin{equation}
\Gamma_{\rm cost}\ \lesssim\ \tilde c\,\frac{A}{\xi^2}\,\overline{\mathcal I}_\Psi,
\qquad 
\tilde c=\mathcal O(1)\,,
\label{cor:Gamma-threshold}
\end{equation}
\vspace{0.3cm}
At the Hartle–Hawking state, detailed balance enforces $\Gamma_{\rm loc}=\Gamma_{\rm cost}$, and the bound saturates at the Bekenstein–Hawking value derived independently in Sec.~6.4; in particular the leading coefficient is the same universal $1/4$ fixed by pairing$\times$swap–even projection.
Consequently,
\vspace{0.3cm}
\begin{equation}
\boxed{\ \Gamma>0\ \Longrightarrow\ S[\mathcal H]\ \le\ \frac{A(\mathcal H)}{4\,L_P^2}\;+\;\gamma\,\ln\!\frac{A}{L_P^2}\;+\;k_0\;+\;\mathcal O\!\Big(\frac{L_P^2}{A}\Big)\ ,}
\label{eq:chrono-bound}
\end{equation}
\vspace{0.3cm}
with $\gamma$ and $k_0$ the one–loop/log and constant terms computed from the physical Hessian on the cone (Sec.\,\ref{sec:subleading-GR}).
\end{corollary}

\begin{remark}[Non-circularity of the calibration]
The value $A/(4L_P^2)$ is \emph{not} assumed here as an ansatz for the chronogenetic bound.
It has already been obtained independently in Secs.~6.2–6.4 from WESH–Noether, uniqueness of
the KMS fixed point, spectral concentration $\bar s=\mathcal O(1)$ and the bipartite+swap projection
yielding the universal $1/4$ prefactor. The chronogenetic argument of this subsection provides
a second, dynamical route to the \emph{same} threshold by demanding $\Gamma=0$ at the
Hartle–Hawking fixed point. The calibration at $A/(4L_P^2)$ is therefore a consistency
requirement between two logically independent derivations, not a circular assumption.
\end{remark}

\begin{proof}[Sketch]
Use \eqref{eq:Gamma-threshold} with Lemma~\ref{lem:pair-Markov}: $\Gamma_{\rm cost}\lesssim \tilde c\,\frac{A}{\xi^2}\,[\,\bar s+\mathcal O((\kappa\xi)^2)\,]$. From Sec.\,6.4, $\bar s=\mathcal O(1)$ (no hidden $A$-dependence, cf.\ Remark~\ref{rem:sbar-origin}), and from Sec.\,6.3 the physical mode count carries the universal factor $1/4$ (bipartite halving $\times$ even–sector projection). At the HH point, detailed balance yields $\Gamma_{\rm loc}=\Gamma_{\rm cost}$; this calibrates the threshold at $S_{\rm BH}=A/(4L_P^2)+\gamma\ln(A/L_P^2)+\cdots$. Therefore $\Gamma>0$ forces $S$ below that calibrated value, establishing \eqref{eq:chrono-bound}. \textit{Refs:} modular Markov and KMS weights; Appendix~D–H for $\Gamma$, the RAQ/constraint average, and the swap–even projector. \, \mbox{} \
\end{proof}

\begin{proposition}[Relative-entropy formulation of the chronogenetic bound]\label{prop:chrono-relent}
For trans-horizon pairs $(x,y)$ distributed by the mediator kernel $K_\xi$, 
the pair information used in Eq.~\eqref{eq:Gamma-threshold} satisfies
\(
2\,\mathcal I_\Psi(x:y)=S\!\big(\rho_{xy}\Vert\rho_x\otimes\rho_y\big).
\)
By monotonicity under inclusions adapted to the null wedge and by the KMS calibration at the HH point, 
\[
\Gamma_{\rm cost}\ \lesssim\ \tilde c\,\frac{A}{\xi^2}\ 
\overline{\,S\!\big(\rho_{xy}\Vert\rho_x\otimes\rho_y\big)\,}
\qquad (\tilde c=\mathcal O(1)).
\]
Hence the forward-time condition $\Gamma>0$ enforces
\(
S[\mathcal H]\le \frac{A}{4L_P^2}+\gamma\ln(A/L_P^2)+k_0+\mathcal O(L_P^2/A),
\)
with equality at the HH fixed point where the relative‑entropy "budget'' is exhausted.
\end{proposition}

\begin{theorem}[Holographic bound from chronogenetic stability]
\label{thm:chrono}
Assume:
\begin{enumerate}[label=(\roman*),leftmargin=1.5cm,itemsep=0.2cm]
\item \textbf{WESH--Noether conservation:} Pre-geometric path independence 
      (Sec.~\ref{sec:wesh-noether}, Eq.~\eqref{eq:H-noether});
      
\item \textbf{GKSL structure:} Quadratic channels $\{\hat{T}^2(x), L_{xy}\}$ 
      with exponential--causal mediator $K_\xi$ of range $\xi \simeq L_P$ 
      (Sec.~\ref{sec:qftt-framework}, Eqs.~\eqref{eq:wesh-master}--\eqref{eq:gamma-kernel});
      
\item \textbf{Near-horizon KMS:} Wedge-modular control with thermal fixed point 
      at $\beta_H = 2\pi/\kappa$ (Sec.~\ref{sec:KMS-Rindler}, Thm.~\ref{thm:Kerr-KMS});
      
\item \textbf{Physical projection:} Bipartite halving $\times$ swap–even boundary superselection 
      yields factor $1/4$ (Sec.~\ref{sec:RAQ-quarter}, Thm.~\ref{thm:halving}).
\end{enumerate}

\vspace{0.3cm}
\noindent Then the chronogenetic condition $\Gamma[\Psi] \ge 0$ in a near-horizon window implies
\begin{equation}
S[\mathcal{H}] \;\le\; \frac{A(\mathcal{H})}{4L_P^2} \;+\; \gamma\,\ln\!\frac{A}{L_P^2} 
\;+\; k_0 \;+\; \mathcal{O}\!\Big(\frac{L_P^2}{A}\Big),
\label{eq:chrono-holo}
\end{equation}
with \textbf{sharp saturation} at the Hartle--Hawking/KMS fixed point, 
i.e., at the Bekenstein--Hawking equilibrium. 
The coefficient of the leading term is universally $1/4$.
\end{theorem}

\paragraph{Sketch of proof.}

\begin{enumerate}[label=(\arabic*),leftmargin=2cm,itemsep=0.3cm]

\item \textbf{Local positivity.} 
By complete positivity and WESH--Noether, $\Gamma_{\rm loc} \ge 0$ with strict positivity 
on any wedge of finite modular energy: local channels \emph{produce} $t$ 
(no conservative drift along forbidden directions).

\item \textbf{Bilocal subtraction and area control.} 
Using \eqref{eq:Gamma-threshold} and null--plane modular inequalities, 
the pair--information density obeys 
$\overline{\mathcal{I}}_\Psi \le \bar{s} + \mathcal{O}((\kappa\xi)^2)$, 
where $\bar{s}$ is the KMS--weighted pair entropy. 
Secs.~\ref{sec:KMS-Rindler} and \ref{sec:phase4-leading-robust} give 
$\bar{s} = \mathcal{O}(1)$ (no hidden $A$-dependence) with $\xi \simeq L_P$. 
Hence $\Gamma_{\rm cost} \lesssim \tilde{c}\,\frac{A}{\xi^2}\,[1 + \mathcal{O}((\kappa\xi)^2)]$, 
with $\tilde{c}$ fixed by WESH normalization and $\|K_\xi\|_{L^1}$.

\item \textbf{Calibration at the fixed point.} 
At the HH fixed point detailed balance enforces pairwise cancellation in frequency space, 
so $\Gamma_{\rm loc} = \Gamma_{\rm cost}$ and $\Gamma = 0$. 
The threshold value of the cross-horizon entropy at which equality holds is precisely 
the BH value: the factor $1/4$ is fixed by bipartite halving$\times$swap–even superselection 
      (Sec.~\ref{sec:RAQ-quarter}) together with $\bar{s} = \mathcal{O}(1)$ 
      (Sec.~\ref{sec:phase4-leading-robust}). 
\noindent Therefore $\Gamma > 0$ implies $S[\mathcal{H}] \le A/(4L_P^2) + \cdots$, 
while $S$ above that value would force $\Gamma \le 0$.

\item \textbf{Subleading structure.} 
The $\ln$ and $A^{-1}$ terms come from the one--loop determinant of the physical 
quadratic form (Sec.~\ref{sec:subleading-GR}) and from the finite width of the 
KMS weight, respectively; they do not shift the leading coefficient.

\end{enumerate}

\paragraph{Operational domain and ``WESH stability cone''.}
Equation~\eqref{eq:chrono-holo} identifies the \emph{WESH stability cone}
\begin{equation}
\mathcal C_{\rm WESH}\;=\;\Big\{\,(\mathcal H,\Psi)\ :\ \Gamma[\Psi]\ge0,\quad S[\mathcal H]\le \tfrac{A(\mathcal H)}{4L_P^2}+\gamma\ln(A/L_P^2)+k_0+\mathcal O(L_P^2/A)\,\Big\}.
\label{eq:wesh-cone}
\end{equation}
States attempting $S>A/(4L_P^2)$ would overload the bilocal channel and stall $t$–fabrication, rendering them dynamically inadmissible for forward–oriented emergent time in the near–horizon wedge.

\paragraph{Corollaries.}
(i) \emph{BH saturation.} Stationary horizons saturate \eqref{eq:chrono-holo}: $\Gamma=0$ at HH; deformations into the exterior wedge satisfy the entropy–area flux identity $\dot S_{\rm rad}=-\dot A/(4L_P^2)\ge 0$.
(ii) \emph{Kerr generality.} The argument extends to rotating horizons by replacing $\partial_t$ with $\chi^\mu=\partial_t+\Omega_H\partial_\phi$ (Sec.~\ref{sec:Kerr-WESH}); the leading threshold remains $1/4$, while spin enters only through subleading coefficients (e.g., $\gamma(J)$) via horizon-local invariants.
(iii) \emph{EFT stability.} Buffer/coarse-graining corrections scale as $\mathcal O\!\big((\kappa\xi)^2\big)=\mathcal O\!\big((L_P/M)^2\big)$, while curvature-induced renormalizations of the kernel scale $\xi$ enter at $\mathcal O\!\big((L_P/M)^4\big)$ (cf.\ Sec.~\ref{sec:phase4-leading-robust}); for macroscopic holes both effects leave the leading threshold intact, with controlled deviations predicted only near the Planckian regime.

\paragraph{Pair–information functional (definition).}
Let $\mathcal A_{\rm out}$ and $\mathcal A_{\rm in}$ be the von Neumann algebras on the two sides of $\mathcal H$. For Planck–thickened surface elements $(x,y)$ with $x\in\mathrm{out}$, $y\in\mathrm{in}$, define
\begin{equation}
\mathcal I_\Psi(x:y)\;=\;\tfrac12\,S_{\mathfrak A_{xy}}\!\big(\rho_{xy}\Vert\rho_x\!\otimes\!\rho_y\big),\qquad
\overline{\mathcal I}_\Psi\;=\;\frac{\displaystyle \int_0^\infty\!\frac{d\omega}{2\pi}\,W_{\rm HH}(\omega)\,
\mathbb E_{(x,y)}[\mathcal I_\Psi(x:y)]}{\displaystyle \int_0^\infty\!\frac{d\omega}{2\pi}\,W_{\rm HH}(\omega)}.
\label{eq:pair-info}
\end{equation}

\vspace{0.3cm}
\noindent Here $W_{\rm HH}(\omega)$ is the near–horizon KMS weight (Sec.~\ref{sec:KMS-Rindler}), and $\mathbb E_{(x,y)}$ averages over trans–horizon pairs drawn with kernel density $K_\xi$. On the HH fixed point, $\overline{\mathcal I}_\Psi=\bar s$; near it, $\overline{\mathcal I}_\Psi\le \bar s+\mathcal O((\kappa\xi)^2)$ by null–plane relative–entropy inequalities.
\vspace{0.5cm}

\begin{remark}[Per-side information cost]
The factor $\tfrac12$ in the definition of $\mathcal I_\Psi$ reflects the bipartite 
structure of near-horizon entanglement. For thermofield-double-like states, the 
full mutual information satisfies $I(x:y)=2\,S(\rho_x)$: the correlation "bridge'' 
is supported equally by both sides. Since the Bekenstein--Hawking entropy 
$S_{\rm BH}=N_{\rm phys}\bar s$ counts the information cost \emph{per exterior mode}, 
the budget entering $\Gamma_{\rm cost}$ must likewise be the per-side share, 
$\mathcal I_\Psi=\tfrac12 I(x:y)$. This ensures $\overline{\mathcal I}_\Psi=\bar s$ 
at the HH fixed point in the regulated thermofield-double (two-sided) realization, 
without extraneous factors.
\end{remark}

\begin{remark}[Operational Bound (WESH Holography)]
For any near-horizon wedge where WESH dynamics admits a Rindler modular description 
and the mediator has range $\xi \simeq L_P$,
\vspace{0.3cm}
\begin{equation}
\boxed{\;
\begin{aligned}
S[\mathcal{H}] &\;\le\; \frac{A(\mathcal{H})}{4L_P^2} \;+\; \gamma\,\ln\!\frac{A}{L_P^2} 
\;+\; k_0 \;+\; \mathcal{O}\!\Big(\frac{L_P^2}{A}\Big),\\[0.3cm]
&\quad\text{with equality at the HH/BH fixed point.}
\end{aligned}
\;}
\label{eq:chrono-final}
\end{equation}
\vspace{0.3cm}

\noindent The coefficients $\gamma$ and $k_0$ are determined by the one--loop Hessian of the 
WESH effective action (Sec.~\ref{sec:subleading-GR}).
\end{remark}

\paragraph{Interpretation.}
Equation~\eqref{eq:chrono-final} reframes black--hole holography as a 
\emph{spacetime-emergence stability bound}. The area term sets the maximal sustainable 
cross-horizon information compatible with forward $t$--production; saturation pins 
$\Gamma$ to zero (stationarity), while sub--holographic configurations yield 
$\Gamma > 0$ and obey the entropy--area flux identity. Above the threshold the bilocal 
channel would dominate and stall chronogenesis. Thus an information--theoretic 
``holographic censorship'' follows directly from pre--geometric conservation and wedge 
modular KMS structure, completing the chain:
\begin{align*}
&\text{WESH--Noether} \;\Longrightarrow\; \text{KMS fixed point} 
 \;\Longrightarrow\; \text{physical projection (RAQ + swap--even)}\\[0.3cm]
&\qquad\qquad\qquad\qquad \Longrightarrow\; 
 S_{\rm BH} = \frac{A}{4L_P^2} + \gamma\,\ln\!\frac{A}{L_P^2} + \cdots,
\end{align*}
with the inequality \eqref{eq:chrono-final} as the dynamical stability envelope of the BH sector.

\vspace{0.8cm}

\section*{Epilogue}
\addcontentsline{toc}{section}{Epilogue}

\noindent This work has pursued a specific goal: to resolve the Wheeler--DeWitt 
frozen-time problem through its unique dissipative completion. The WESH master 
equation is not an alternative to Wheeler--DeWitt, but its dynamical extension: the 
minimal GKSL structure that unfreezes the constraint $\hat{H}_{\rm tot}|\Psi\rangle=0$ 
while preserving CPT symmetry, Noether charges, and complete positivity. From this 
completion, both the Einstein--Hilbert action and black hole thermodynamics emerge 
without assuming spacetime as a background structure.

\vspace{0.3cm}
\noindent The technical path connected several domains not usually placed in contact: GKSL dissipation in a timeless Wheeler--DeWitt sector, Noether-type constraints at the generator level, variational alignment as a dynamical fixed point, and near-horizon KMS structure as the bridge to thermodynamics. At each step, we required internal consistency and falsifiability rather than formal elegance alone. The result is a framework where spacetime emergence and thermodynamic equilibrium coincide: competing tendencies—loss of coherence, gradient flows, global conservation—self-organize into a stable configuration that satisfies Einstein's equations up to controlled corrections.

\vspace{0.3cm}
\noindent The framework offers concrete, testable predictions: collective coherence scaling as $\tau_{\rm coh}(N) \propto N^2$, angular modulation of decoherence rates following $1 + \varepsilon\cos^2\theta$, and holographic consistency across ${\sim}50$ orders of magnitude from the Planck scale to macroscopic horizons. These are not incidental features but direct consequences of the WESH structure, understood less as a derivation from Wheeler--DeWitt than as its dissipative Lindbladian completion. Preliminary tests on IBM and Rigetti quantum processors show consistency with the predicted scaling within current hardware limitations; the theory stands or falls on future experimental verdict.

\vspace{0.3cm}
\noindent We have deliberately confined this work to the gravitational sector and black hole thermodynamics. Natural extensions, compact astrophysical objects, matter couplings, scalar-tensor formulations, require dedicated studies with distinct observational protocols. At this stage, the priority was establishing the core derivation and its initial empirical contact.

\vspace{0.3cm}
\noindent The framework has known limitations. Near singularities, at trans-Planckian scales, or in regimes where the Markov condition $\tau_{\rm corr} \ll \tau_{\rm Eig}$ fails, the assumptions underlying WESH break down. We do not claim universality; we claim tractability within a well-defined domain.

\vspace{0.3cm}
\noindent Whether this particular formulation survives critical scrutiny is for the community to decide. If it does, the implications extend beyond the specific results derived here. If it does not, we hope it will have at least demonstrated that the functional relations between quantum dissipation, symmetry constraints, and geometric emergence admit richer structures than previously recognized, and that the attempt to derive spacetime rather than assume it remains a viable research direction.
\vspace{1cm}

\section*{Acknowledgments}

\noindent Mathematical content, physical arguments, and proofs have been verified by the author
and validated by the Lean~4 type-checker using the Mathlib library. The formalization comprises
over 6,100 lines of verified code. Axioms are limited to standard textbook results
and to QFTT--WESH-specific content derived elsewhere in the Lean corpus.

\noindent This work was developed with the assistance of a cross-inferencing multi-AI workflow, primarily involving ChatGPT-5.2 Pro and Claude Opus 4.5 for theoretical development, mathematical derivations, and iterative refinement, but also Gemini 3 Pro and other models. Aristotle v.~0.6.0 (Harmonic) was also used, in conjunction with ChatGPT-5.2 Pro and Claude Opus 4.5, as a translation tool in the final stages to convert established mathematical content into Lean~4 syntax. The author intends to further document this multi-AI architecture in future work, hoping it may contribute to AI-assisted theoretical physics research. Throughout the process, the research, conceptual framework and its details, workflow orchestration, and verification of every output remained with the author.

\noindent I am grateful and dedicate this work to my wife Margie and the little Elvia: thanks for the 
support, and for giving me the time to understand time. A heartfelt thank you also to 
Ka\"ira, the indomitable.

\vspace{0.8cm}

\section*{Appendix A: Kernel derivation and the scales \texorpdfstring{$\xi$}{ξ} and \texorpdfstring{$\gamma_0$}{γ0}}
\vspace{0.6cm}

\noindent\textbf{Setup.}
We motivate the causal, $L^1$-integrable weight used in the collapse kernel \(\gamma(x,y)\) of Eq.~\eqref{eq:gamma-kernel} from a minimal local mediator. Throughout we adopt the Minkowski convention \(\eta=\mathrm{diag}(-,+,+,+)\) and \(\Box=\eta^{\mu\nu}\partial_\mu\partial_\nu\).

\vspace{0.3cm}
\noindent\textbf{A.1 Minimal mediator and retarded propagator.}
We introduce a scalar mediator $\chi(x)$ with Klein--Gordon dynamics on the pre-chronogenesis sector:
\begin{equation}
S_\chi=\!\int d^4x\,\left[\tfrac12(\partial_\mu \chi)(\partial^\mu \chi)-\tfrac12 m_T^2 \chi^2\right].
\tag{A.1}
\end{equation}
The retarded Green function \(\Delta_R\) solves
\begin{equation}
(\Box+m_T^2)\,\Delta_R(z)=\delta^{(4)}(z),\qquad \mathrm{supp}\,\Delta_R\subseteq\{z^0\ge 0,\ z^2\le 0\}.
\tag{A.2}
\end{equation}
The correlation length associated with the mediator mass is the Compton scale
\begin{equation}
\xi=\frac{\hbar}{m_T c}.
\tag{A.3}\label{eq:scales-mediatormass}
\end{equation}

\vspace{0.3cm}
\noindent\textbf{A.2 Positive causal envelope and correlation four-volume.}
The retarded propagator $\Delta_R$ fixes the \emph{support} (future light cone) and the
range $\xi=\hbar/(m_Tc)$, but it is not used directly as a nonnegative $L^1$ weight.
Instead we adopt a coarse-grained, positive causal envelope $K_\xi$ of range $\xi$, e.g.
\begin{equation}
K_\xi(z)\;=\;\exp\!\Big(-\frac{z^0}{\xi}\Big)\,\Theta(z^0)\,\Theta(-z^2),
\qquad (\,\eta=\mathrm{diag}(-,+,+,+)\,),
\tag{A.4}
\end{equation}
whose $L^1$ norm defines the correlation four-volume
\begin{equation}
V_\xi \;:=\;\int d^4z\,K_\xi(z)\;=\;\mathcal O(\xi^4).
\tag{A.4$'$}
\end{equation}
With the normalized measures of Eq.~\eqref{eq:measure-conventions}, this $V_\xi$ is absorbed
at the level of the generator. (If desired one may also define the normalized kernel
$\bar K_\xi:=K_\xi/V_\xi$ so that $\int d^4z\,\bar K_\xi=1$.)

\vspace{0.3cm}
\noindent\textbf{A.3 Collapse kernel and dimensions.}
The collapse kernel of the main text reads \(\gamma(x,y)=\frac{\gamma_0}{N^2}\,K_\xi(x-y)\,\Theta[\mathrm{causal}(x,y)]\) (Eq.~\eqref{eq:gamma-kernel}). With $K_\xi$ chosen positive and $L^1$-integrable, $\gamma_0$ has the dimension of a rate.
(If one prefers an explicitly normalized kernel, define $\bar K_\xi:=K_\xi/V_\xi$ so that
$\int d^4z\,\bar K_\xi(z)=1$.) This choice yields a positive, integrable weight and, when composed with Hermitian jump operators, preserves complete positivity of the dissipative channel.

\vspace{0.3cm}
\noindent\textbf{A.4 WESH–Noether normalization.}
Coarse-grained consistency imposes a matching condition on the intensity scale:
\begin{equation}
\nu \;\simeq\; \gamma_0\,\bar C_x,\qquad 
\bar C_x:=\frac{1}{V_\xi}\!\int d^4y\,K_\xi(x-y)\,C(\Psi;x,y),\quad 
V_\xi=\!\int d^4z\,K_\xi(z)\sim \xi^4.
\label{eq:nu-matching} \tag{A.5}
\end{equation}
Here \(C(\Psi;x,y)\) is the state-dependent bilocal factor of the WESH channel and \(V_\xi\) is the effective correlation 4–volume. With the normalized measures of Eq.~\eqref{eq:measure-conventions}, $V_\xi$ is tracked consistently through the coarse-graining; $\gamma_0$ remains the fundamental rate scale, while $\nu$ is fixed by self-consistent matching (hence not independently tunable).

\vspace{0.3cm}
\noindent\textbf{A.5 Planck anchoring.}
With $m_T\sim M_P$ one has $\xi\sim L_P$ by (A.3). Combining this with the matching $\nu\simeq\gamma_0\,\bar C_x$ (Eq.~\eqref{eq:nu-matching}), with $\bar C_x=\mathcal O(1)$ on correlated domains, the natural anchoring is
\begin{equation}
\gamma_0 \;=\; \Theta_0\,t_P^{-1},\qquad \Theta_0=\mathcal O(1).
\tag{A.6}
\end{equation}
Here $\Theta_0$ collects normalized cumulants and geometric factors from the time sector,
ensuring that the anchoring introduces no additional free parameters or arbitrary scales.

\vspace{0.6cm}
\noindent\textbf{Remarks.}
(i) The limit \(\xi\to 0\) recovers a local channel; finite \(\xi\) implements coarse-grained causality with finite correlation volume.  
(ii) The static Yukawa form is a spatial illustration only; the operative kernel is the 4D $L^1$-integrable \(K_\xi\) with causal support.  
(iii) The construction is minimal and sufficient to support the collapse kernel of Eq.~\eqref{eq:gamma-kernel} and the no–signaling statement Eq.~\eqref{eq:no-signaling}, while fixing \(\gamma_0\) via Eq.~\eqref{eq:nu-matching}.

\vspace{1cm}

\section*{Appendix B: Parameter \(s\), Emergence of Physical Time, and the Conceptual Limits of a Language Within Time}
\addcontentsline{toc}{section}{Appendix~B: Parameter $s$, Physical Time, and Language Limits}

\vspace{0.8cm}

\subsection*{B.1 The Auxiliary Parameter $s$ and its Role in QFTT--WESH}

\begingroup
\setcounter{equation}{0}
\renewcommand{\theequation}{B.\arabic{equation}}

In QFTT--WESH, an intrinsic parameter $s$ is introduced to write the dynamics of a closed, timeless quantum universe. The universal state $\rho(s)$ is evolved in $s$ but $s$ is \emph{not} an observable: it is a non-physical ordering label used to formulate the GKSL evolution and to state constraints in the WDW sector. “Gauge-like’’ here means \emph{non-observable scaffolding}; No physical observable depends on the choice of the $s$-parametrization: one may view $s$ as a gauge-like ordering label (any monotone reparametrization is absorbed by the map $t(s)$). This addresses the \emph{problem of time} in a Wheeler--DeWitt universe.

\vspace{0.5cm}

\noindent The $s$-evolution uses the WESH generator referenced in Eq.~\eqref{eq:wesh-master}, with $\hat H_{\eff}$, $\mathcal D[\cdot]$, $L_{xy}$ and $\gamma$ given in Eqs.~\eqref{eq:superoperator}–\eqref{eq:gamma-kernel}.

\vspace{0.5cm}

\noindent\textit{Notation.} We adopt the operator and kernel conventions of Eqs.~\eqref{eq:superoperator}–\eqref{eq:gamma-kernel}; below we recall only the causal support needed for the rates.

\paragraph{Kernel and causality.}
We keep the \emph{exponential--causal} dissipative weight (see Eq.~\eqref{eq:gamma-kernel})

\vspace{0.5cm}

\[
\gamma(x,y)=\frac{\gamma_0}{N^2}\,e^{-d(x,y)/\xi}\,\Theta[\mathrm{causal}(x,y)].
\]

\noindent Here $d(x,y)$ denotes the invariant causal separation computed from the emergent scalar kernel argument
(e.g.\ via Synge's world function $\sigma(x,y)$ once $g^{(T)}_{\mu\nu}$ is defined; cf.\ Box~1).

\vspace{0.5cm}
\noindent We reserve 'Yukawa' for the \emph{mediator} kernel

\vspace{0.5cm}
\[
K(x,y)\ \propto\ \frac{e^{-|x-y|/\xi}}{4\pi |x-y|}\,\Theta[\mathrm{causal}(x,y)].
\]

\vspace{0.5cm}

\noindent This choice (Eq.~\eqref{eq:gamma-kernel}) avoids a short-distance $1/r$ singularity in the GKSL rates, keeps an $L^1$-integrable weight with finite correlation volume $\sim\xi^4$, and—via $\Theta[\mathrm{causal}]$—yields exact spacelike no‑signaling (Eq.~\eqref{eq:no-signaling}). 

\vspace{0.8cm}

\subsection*{B.2 Eigentimes and the Emergence of Physical Time $t$}

\vspace{0.5cm}

Eigentime events are the spontaneous collapses that make the arrow of time intrinsic. Physical time $t$ is the coarse‑grained monotone of $s$:

\vspace{0.5cm}

\[
\frac{dt}{ds} \;=\; \Gamma[\Psi(s)],
\qquad
t(s) \;=\; \int_0^{s} \Gamma[\Psi(s')]\,ds'
\]

\vspace{0.5cm}

\noindent The time-production functional is the dimensionless quantity defined in
Eq.~\eqref{eq:gamma-decomposition} using the normalized measures of Eq.~\eqref{eq:measure-conventions}:
\[
\Gamma[\Psi] = \tau_{\mathrm{Eig}} \left[
\nu \int_x \Tr\bigl[\tilde{T}^2(x)\,\rho\,\tilde{T}^2(x)\bigr]
+ \int_{xy} \gamma(x,y)\,C(\Psi;x,y)\,\Tr\bigl[L_{xy}^\dagger L_{xy}\,\rho\bigr]
\right],
\]
so that $dt/ds=\Gamma[\rho]\ge 0$, with $\Gamma>0$ under nondegeneracy (Lemma~\ref{lem:Gamma-positive}), and the local/bilocal matching is set by the coarse‑grained consistency condition (Eq.~\eqref{eq:nu-matching}). Strict positivity of $\Gamma$ (monotonicity $dt/ds>0$) follows under Lemma~\ref{lem:Gamma-positive}; eigentime activation on chronogenetic segments under Lemma~\ref{lem:eigentime-activation}.

\vspace{0.5cm}

\paragraph{Granularity $\to$ continuum (LLN).}

\vspace{0.5cm}
~\\ 

\noindent The law of large numbers links event counting to physical time:
\begin{equation}
t(s) \;\approx\; \tau_{\rm Eig}\,N_{\rm Eig}(s) \quad\text{(LLN)}
\label{eq:countingtotime-relation}
\end{equation}

\noindent More precisely:
\begin{align}
&N_{\mathrm{Eig}}(s)=\int_0^s \frac{\Gamma[\Psi(u)]}{\tau_{\rm Eig}}\,du + M(s), \notag\\
&t(s)=\int_0^s \Gamma[\Psi(u)]\,du, \qquad
\frac{t(s)}{\tau_{\rm Eig}\,N_{\mathrm{Eig}}(s)} \xrightarrow{\ \mathbb{P}\ } 1. \notag
\end{align}

\vspace{0.5cm}

\noindent with martingale $M(s)$ and $\tau_{\rm Eig}\equiv 1/\nu$. Hence, when $\mathbb{E}[N_{\mathrm{Eig}}]\gg1$, discrete eigentimes smooth to classical $t$ by the law of large numbers.

\vspace{0.5cm}

\paragraph{Operator‑ordering vs.\ moments (reader guardrail).}
The intensity functional $\Gamma[\Psi]$ contains the GKSL trace form 
$\Tr[\hat T^{2}\rho\,\hat T^{2}]$, \emph{not} the raw moment $\langle \hat T^{4}\rangle$; 
the two coincide only if $[\rho,\hat T^2]=0$. The trace form is required by complete 
positivity and appears in the definition of $\Gamma[\Psi]$ (Eq.~\eqref{eq:gamma-decomposition}).

\vspace{0.5cm}

\paragraph{Example (translation rule $s\to t$).}
A temporalized sentence like “collapse occurs \emph{before} the measurement’’ means: there exist $s_{\mathrm{col}}<s_{\mathrm{meas}}$; mapping by \eqref{eq:s-to-t-bootstrap} gives $t_{\mathrm{col}}<t_{\mathrm{meas}}$. “Before/after’’ acquires meaning only \emph{after} this mapping.

\vspace{0.8cm}

\subsection*{B.3 Comparison to Other Frameworks and Interpretations}
QFTT--WESH couples the flow of time with the state via $\Gtime$; $\nu$ is anchored by coarse‑grained self‑consistency Eq.~\eqref{eq:nu-matching} and the mass–length relation (Eq.~\eqref{eq:scales-mediatormass}); Planck anchoring (Eq.~\eqref{eq:fundamental-scales}). The arrow arises intrinsically at collapse, consistent with decoherence‑oriented discussions of temporal asymmetry (Zeh, 2007).

\vspace{0.5cm}

\begin{table}[H]
\centering
\small
\setlength{\tabcolsep}{6pt}
\renewcommand{\arraystretch}{1.15}
\begin{tabularx}{\linewidth}{>{\raggedright\arraybackslash}p{0.22\linewidth} >{\raggedright\arraybackslash}p{0.28\linewidth} >{\raggedright\arraybackslash}p{0.28\linewidth} >{\raggedright\arraybackslash}X}
\toprule
\textbf{Framework} & \textbf{Treatment of Time} & \textbf{Collapse Mechanism} & \textbf{Arrow of Time}\\
\midrule
Page--Wootters (Page \& Wootters, 1983) & Relational, static (timeless global state) & None (unitary) & Absent \\
GRW/CSL (Ghirardi et al., 1986) & External parameter $t$ (absolute) & Fixed rate per external time & Assumed \\
Penrose OR (Penrose, 1996) & External parameter $t$ (absolute) & Gravity-threshold reduction & Assumed \\
QFTT--WESH & \textbf{Emergent} via eigentimes & \textbf{State-dependent} $\Gtime$ & \textbf{Intrinsic}; each eigentime orients time\\
\bottomrule
\end{tabularx}
\end{table}

\vspace{0.8cm}

\subsection*{B.4 Concluding Remarks}
\vspace{0.4cm}

\noindent\textit{Auxiliary $s$ and emergence of $t$.} 
The auxiliary parameter $s$ is a scaffold to write the CP--TP WESH evolution (see the master equation). It disappears from observables once physical time is constructed by the bootstrap
$dt/ds=\Gamma[\Psi]$ (Eq.~\eqref{eq:s-to-t-bootstrap}): after this map, all operational statements are formulated in $t$.

\smallskip
\noindent\textit{Finite-$N$ imprint.}
Residual discreteness at finite $N$ should leave a measurable imprint in the intensity channel: the variance and cross–correlation structure of $\Gamma$ predicted in Sec. 1.5-1.6 (Eqs.~\eqref{eq:local-hazard}, \eqref{eq:gamma-decomposition}, 
\eqref{eq:Gamma-bilocal}) would provide a direct handle on the granularity of the eigentime mesh and on the causal range~$\xi$.

\smallskip
\noindent\textit{Bridge to GR.}
At the stationary fixed point selected by the WESH monotone (App.~D, Thm.~\ref{thm:alignment-PartII}), the gradient–alignment condition holds,
\(\partial_\mu \tau = k\,\partial_\mu \Phi\) (Eq.~\eqref{eq:D-alignment-IR-kconst});
with the matching \(\tfrac{k^2}{4\pi G}=\lambda_1+3\lambda_2\) (Eq.~\eqref{eq:parameter-relations}), this enforces the hidden–sector cancellation
\(T^{(T)}_{\mu\nu}+T^{(nl)}_{\mu\nu}=0\) (Eq.~\eqref{eq:hidden-cancel}) and yields the emergent Einstein equation with matter (Eq.~\eqref{eq:emergent-einstein}).

\smallskip
\noindent\textit{Pre–geometric consistency (App.~G).}
The correct foundational constraint at the $s$–level is WESH–Noether path independence, not thermodynamic detailed balance; the latter emerges only after the $s\!\to\!t$ bootstrap and in the near–horizon KMS regime (Sec.~\ref{sec:KMS-Rindler}, Eq.~\eqref{eq:UD-balance}). This completes the logical arc from pre-geometric consistency to emergent spacetime and thermodynamics, justifying the use of $s$ as a purely organizational device.


\vspace{0.5cm}

\[
T^{(T)}_{\mu\nu} + T^{(nl)}_{\mu\nu} = 0 \quad (N\to\infty)
\]

\vspace{0.5cm}

\noindent and the emergent Einstein equation

\vspace{0.5cm}

\[
G_{\mu\nu}+\Lambda g_{\mu\nu}=8\pi G\,T^{(m)}_{\mu\nu}+O(1/N)
\]

\vspace{0.5cm}

\noindent are independent of $s$; they are statements in the emergent $t$‑theory, confirming that $s$ leaves no footprint in observables (see Eqs.~\eqref{eq:hidden-cancel} and \eqref{eq:emergent-einstein}).

\vspace{0.8cm}

\subsection*{B.5 Conceptual Limits and the Paradox of Inherited Language}

\vspace{0.5cm}

Inevitably, a theory like QFTT--WESH, aiming to position itself upstream of the 4 dimensions, collides with the epistemological problem of using a language within the temporal line. Concepts such as 'before', 'after', 'cause', and 'effect' (but even more implicit assumptions and morphemes) are foundational to our linguistic and logical structures; yet the theory must describe a reality that exists at a more fundamental level than their emergence. At the same time, we identify a taxonomic challenge, implicitly calling for the development of a new mathematical language, perhaps more circular, topological, or inherently atemporal. In the present work, such tension forces a deliberate methodological strategy: to remain communicable, the theory must adopt a temporalized narrative for its exposition, even as its formalism in part describes an atemporal process. The auxiliary parameter $s$ is the controlled semantic scaffold, a \emph{necessary conceptual unit} that dissolves once the formalism is established, essential to navigate this paradox. We could see it in such a way that any temporal sentence admits an $s$‑label version (ordering only), then is translated to a $t$‑statement via Eq.~\eqref{eq:s-to-t-bootstrap}; Appendix~B's LLN line justifies the continuous limit for macroscopic descriptions. This keeps the exposition communicable while ensuring that predictions depend solely on the emergent, measurable $t$. Here lies the epistemological challenge, but at the same time the beauty of this theory.
\endgroup

\vspace{1cm}

\section*{Appendix C: WESH conservation law}

\vspace{0.5cm}
\subsection*{C.1 Preliminaries: T-neutrality}


\noindent\textbf{Lemma (T-neutrality \& commutators).} 
If the time field is T-neutral, $[\hat T(x), \hat Q_{\rm tot}] = 0$ 
for all $x$, then for any conserved charge $\hat Q_{\rm tot}$:
\[
[\hat T^{2}(x), \hat Q_{\rm tot}] = 0 \quad\text{and}\quad
[L_{xy}, \hat Q_{\rm tot}] = 0, \qquad L_{xy} \equiv \hat T^2(x) - \hat T^2(y).
\]

\vspace{0.5cm}
\subsection*{C.2 Generator-level WESH--Noether (atemporal)}

Using the standard GKSL identity
\[
\Tr\!\big(A\,\mathcal D[L]\,\rho\big)=\tfrac12\langle L^\dagger[A,L]+[L^\dagger,A]L\rangle,
\]
The Hamiltonian contribution is $-i\langle[\hat Q_{\rm tot},\hat H_{\rm eff}]\rangle$ and vanishes 
by the commutant condition $[\hat H_{\rm eff},\hat Q_{\rm tot}]=0$ (Eq.~\eqref{eq:commutant-conditions}).
Setting $A=\hat Q_{\rm tot}$ in the dissipative part, one obtains the decomposition
\[
\frac{d}{ds}\langle \hat{Q}_{\mathrm{tot}}\rangle
= \int d^4x\, S_{\mathrm{loc}}(x) \;+\; \iint d^4x\,d^4y\, S_{\mathrm{bi}}(x,y).
\]

\noindent\textbf{Local contribution (anticommutator form).}
\[
S_{\mathrm{loc}}(x)=\tfrac12\Big\langle \hat{T}^{2}(x)\,[\hat{Q}_{\mathrm{tot}},\hat{T}^{2}(x)]
+ [\hat{T}^{2}(x),\hat{Q}_{\mathrm{tot}}]\;\hat{T}^{2}(x)\Big\rangle.
\]

\noindent\textbf{Bilocal contribution.}
\[
S_{\mathrm{bi}}(x,y)=\tfrac12\Big\langle L_{xy}\,[\hat{Q}_{\mathrm{tot}},L_{xy}]
+ [L_{xy},\hat{Q}_{\mathrm{tot}}]\;L_{xy}\Big\rangle,\qquad
L_{xy}\equiv \hat T^2(x)-\hat T^2(y).
\]

\noindent By T-neutrality (C.1), $[\hat T^2(x),\hat Q_{\rm tot}]=0$ and $[L_{xy},\hat Q_{\rm tot}]=0$, hence
\[
S_{\mathrm{loc}}(x)=0,\qquad S_{\mathrm{bi}}(x,y)=0,
\]
and therefore
\[
\boxed{\ \frac{d}{ds}\,\langle \hat{Q}_{\mathrm{tot}}\rangle = 0\ }\quad\text{(WESH--Noether, generator form).}
\label{eq:wntheorem-formal}
\]

\vspace{0.3cm}
\noindent\textit{Remark (operator-level formulation).}
The statement above is equivalent to the generator identity
$\mathcal L^\dagger[\hat Q_{\rm tot}]=0$, with necessary and sufficient commutation
conditions and the double-commutator identity proved under explicit
domain/regularity assumptions. The complete operator-level derivation is in Appendix~G 
(Proposition~G.2 and surrounding discussion).

\vspace{0.5cm}
\subsection*{C.3 No $s$-footprint in observables (chain rule)}

On any chronogenetic interval where $\Gamma[\Psi]>0$ (Lemma~\ref{lem:Gamma-positive}), for any conserved charge
\[
\frac{d}{dt}\langle \hat{Q}_{\mathrm{tot}}\rangle
=\Big(\frac{ds}{dt}\Big)\,\frac{d}{ds}\langle \hat{Q}_{\mathrm{tot}}\rangle.
\]
If \eqref{eq:wntheorem-formal} holds exactly, then $\tfrac{d}{dt}\langle \hat Q_{\rm tot}\rangle=0$.
This confirms that the auxiliary parameter $s$ leaves no observable footprint in the conservation laws and matches the commutator check used in App.~D, Thm.~\ref{thm:alignment-PartII}.

\vspace{1cm}

\section*{Appendix D: Variational Alignment as the Unique Dynamical Fixed Point of WESH}
\vspace{0.8cm}

\begingroup
\setcounter{equation}{0}
\renewcommand{\theequation}{D.\arabic{equation}}
\setcounter{theorem}{0}
\renewcommand{\thetheorem}{D.\arabic{theorem}}
\setcounter{lemma}{0}
\renewcommand{\thelemma}{D.\arabic{lemma}}
\setcounter{remark}{0}
\renewcommand{\theremark}{D.\arabic{remark}}
\setcounter{corollary}{0}
\renewcommand{\thecorollary}{D.\arabic{corollary}}
\setcounter{proposition}{0}
\renewcommand{\theproposition}{D.\arabic{proposition}}

\begin{lemma}[Schauder--Tychonoff fixed point: existence of a self-consistent stationary state]
\label{lem:SchauderTychonoff}

\noindent\emph{Definition.} Fix a family of conserved charges $\{\hat Q_a\}$ (energy, momenta, $T$--charge, constraints) and
constants $c_a$ determined by the initial data. Define the physical state manifold as
\[
\mathcal S_{\rm phys}
:=\Big\{\rho\in\mathcal T_1(\mathcal H)\;:\;\rho\ge0,\ \Tr\rho=1,\ \langle \hat Q_a\rangle_\rho=c_a\ \forall a\Big\}.
\]
\noindent\emph{(Assumption)} The constraints are imposed using bounded charges (or bounded functions of the charges)
so that $\rho\mapsto\langle \hat Q_a\rangle_\rho$ is $\sigma$--continuous; hence $\mathcal S_{\rm phys}$ is $\sigma$--closed.

\medskip
\noindent\emph{Bootstrap one-step map.}
Fix a micro--step $\delta s>0$. For each $\rho\in\mathcal S_{\rm phys}$ define the (state--dependent) GKSL generator
$\mathcal L_{C[\rho]}$ by taking the WESH master equation \eqref{eq:wesh-master} and evaluating the scalar gate
$C[\rho;x,y]$ on the corresponding local/bilocal reductions (Appendix~H), i.e.\ the bilocal rate is
$\gamma(x,y)\,C[\rho;x,y]$ while the jump set is unchanged. Define the self-consistent bootstrap update
\begin{equation}
F_{\delta s}(\rho)\;:=\;\exp\!\big(\delta s\,\mathcal L_{C[\rho]}\big)(\rho)\,.
\label{eq:D-bootstrap-map}
\end{equation}
Assume the standing regularity hypotheses of Appendix~H: bounded rates $0\le C\le 1$, $\gamma\ge 0$,
and Lipschitz (hence continuous) dependence $\rho\mapsto C[\rho;x,y]$ through the reduced states, so that
for each frozen $\rho$ the map $\exp(\delta s\,\mathcal L_{C[\rho]})$ is a normal CPTP map.

\medskip
\noindent\textbf{Claim.}
$F_{\delta s}$ maps $\mathcal S_{\rm phys}$ into itself and is $\sigma(\mathfrak A(\mathcal W)^{*},\mathfrak A(\mathcal W))$--continuous on
$\mathcal S_{\rm phys}$. Hence, by Schauder--Tychonoff, $F_{\delta s}$ admits at least one fixed point
$\rho_{\delta s}^\star\in\mathcal S_{\rm phys}$:
\[
F_{\delta s}(\rho_{\delta s}^\star)=\rho_{\delta s}^\star.
\]
Moreover, there exists a cluster point $\rho^\star\in\mathcal S_{\rm phys}$ along $\delta s\downarrow 0$
satisfying the nonlinear stationarity condition
\begin{equation}
\mathcal L_{C[\rho^\star]}[\rho^\star]=0
\qquad\text{(in the weak/duality sense).}
\label{eq:D-nonlinear-stationarity}
\end{equation}
In particular, $\rho(s)\equiv\rho^\star$ is a stationary solution of the bootstrap WESH evolution in the
WDW sector.
\end{lemma}

\begin{proof}
\textit{(i) Compactness/topology.}
View states as normal functionals on the von Neumann algebra $\mathfrak A(\mathcal W)$.
The state space is weak-* compact in $\sigma(\mathfrak A(\mathcal W)^{*},\mathfrak A(\mathcal W))$ by Banach--Alaoglu.
Since each constraint $\rho\mapsto \langle \hat Q_a\rangle_\rho=\Tr(\rho\,\hat Q_a)$ is assumed $\sigma$--continuous,
$\mathcal S_{\rm phys}$ is $\sigma$--closed. Being an intersection of affine hyperplanes with the state space,
$\mathcal S_{\rm phys}$ is convex and weak-* compact.

\medskip
\textit{(ii) Invariance of $\mathcal S_{\rm phys}$ under $F_{\delta s}$.}
For each frozen $\rho$, the generator $\mathcal L_{C[\rho]}$ is GKSL with Hermitian jumps and nonnegative
scalar rates (Appendix~H), hence $\exp(\delta s\,\mathcal L_{C[\rho]})$ is CPTP and normal.
Therefore $F_{\delta s}(\rho)\ge 0$ and $\Tr\,F_{\delta s}(\rho)=1$.

For the conserved charges, WESH--Noether holds at the generator level for each frozen generator:
$\mathcal L_{C[\rho]}^\dagger[\hat Q_a]=0$ (Appendix~C, and the operator-level formulation in Appendix~G),
hence $\frac{d}{du}\langle \hat Q_a\rangle_{e^{u\mathcal L_{C[\rho]}}(\rho)}=0$ for all $u$.
In particular,
\[
\langle \hat Q_a\rangle_{F_{\delta s}(\rho)}
=\langle \hat Q_a\rangle_{\exp(\delta s\,\mathcal L_{C[\rho]})(\rho)}
=\langle \hat Q_a\rangle_{\rho}
=c_a,
\]
so $F_{\delta s}(\mathcal S_{\rm phys})\subseteq \mathcal S_{\rm phys}$.

\medskip
\textit{(iii) $\sigma$--continuity of $F_{\delta s}$.}
Let $\rho_n\to\rho$ in $\sigma(\mathfrak A(\mathcal W)^{*},\mathfrak A(\mathcal W))$ on $\mathcal S_{\rm phys}$.
By normality of partial traces, the reduced states entering $C[\rho;x,y]$ vary continuously in the induced
$\sigma$--topology. In the register/finite-block sectors relevant to the gate (Appendix~H), these reduced
state spaces are finite-dimensional, so the induced $\sigma$--topology coincides with trace-norm topology;
hence the Lipschitz assumption in Appendix~H implies
\[
C[\rho_n;x,y]\ \longrightarrow\ C[\rho;x,y]
\quad\text{(pointwise in $(x,y)$, and uniformly on bounded supports).}
\]
Consequently the coefficients of the frozen generators satisfy
$\mathcal L_{C[\rho_n]}\to \mathcal L_{C[\rho]}$ in the operator norm topology on the relevant bounded
observable core (finite-$N$ truncation / UV--IR cutoff picture, as already used elsewhere in this Appendix).

Fix any bounded observable $A\in\mathfrak A(\mathcal W)$ and write
\[
\Tr\!\big(A\,F_{\delta s}(\rho_n)\big)
=\Tr\!\big(\exp(\delta s\,\mathcal L_{C[\rho_n]}^\dagger)[A]\;\rho_n\big).
\]
By the Duhamel formula,
\[
\exp(\delta s\,\mathcal L_{C[\rho_n]}^\dagger)-\exp(\delta s\,\mathcal L_{C[\rho]}^\dagger)
=\int_0^{\delta s}\!\exp((\delta s-u)\mathcal L_{C[\rho_n]}^\dagger)\,
\big(\mathcal L_{C[\rho_n]}^\dagger-\mathcal L_{C[\rho]}^\dagger\big)\,
\exp(u\mathcal L_{C[\rho]}^\dagger)\,du,
\]
which yields (under the uniform boundedness of the frozen generators on the chosen core) the bound
\[
\big\|\exp(\delta s\,\mathcal L_{C[\rho_n]}^\dagger)[A]-\exp(\delta s\,\mathcal L_{C[\rho]}^\dagger)[A]\big\|
\;\le\;K_{\delta s}\,\|A\|\,\big\|\mathcal L_{C[\rho_n]}^\dagger-\mathcal L_{C[\rho]}^\dagger\big\|\ \xrightarrow[n\to\infty]{}\ 0.
\]
Therefore,
\[
\Tr\!\big(A\,F_{\delta s}(\rho_n)\big)\ \longrightarrow\ \Tr\!\big(A\,F_{\delta s}(\rho)\big)
\quad\text{for all }A\in\mathfrak A(\mathcal W),
\]
i.e.\ $F_{\delta s}(\rho_n)\to F_{\delta s}(\rho)$ in the $\sigma$--topology.

\medskip
\textit{(iv) Schauder--Tychonoff fixed point.}
$\mathcal S_{\rm phys}$ is compact and convex in the locally convex space
$(\mathfrak A(\mathcal W)^{*},\sigma(\mathfrak A(\mathcal W)^{*},\mathfrak A(\mathcal W)))$,
and $F_{\delta s}:\mathcal S_{\rm phys}\to\mathcal S_{\rm phys}$ is continuous.
By the Schauder--Tychonoff fixed point theorem, there exists
$\rho_{\delta s}^\star\in\mathcal S_{\rm phys}$ such that $F_{\delta s}(\rho_{\delta s}^\star)=\rho_{\delta s}^\star$.

\medskip
\textit{(v) Passage $\delta s\downarrow 0$ and nonlinear stationarity.}
Choose any sequence $\delta s_n\downarrow 0$, and let $\rho_n\in\mathcal S_{\rm phys}$ be fixed points:
$\rho_n=\exp(\delta s_n\,\mathcal L_{C[\rho_n]})(\rho_n)$.
By weak-* compactness of $\mathcal S_{\rm phys}$, extract a subsequence (not relabeled) with
$\rho_n\to\rho^\star$ in $\sigma$.

Fix any observable $A$ in a common invariant core $\mathcal D(\mathcal L^\dagger)$ for the adjoints
of the frozen generators (the standard choice in this Appendix; in finite-$N$ truncations this is simply
$\mathfrak A(\mathcal W)$). The fixed point identity implies
\[
0=\frac{1}{\delta s_n}\Tr\!\Big(\big[\exp(\delta s_n\,\mathcal L_{C[\rho_n]}^\dagger)[A]-A\big]\rho_n\Big).
\]
On the core $\mathcal D(\mathcal L^\dagger)$ one has the strong first-order expansion
\[
\frac{1}{\delta s_n}\big(\exp(\delta s_n\,\mathcal L_{C[\rho_n]}^\dagger)[A]-A\big)
=\mathcal L_{C[\rho_n]}^\dagger[A]\;+\;o(1)
\qquad(n\to\infty),
\]
and by the continuity already established in (iii) one has
$\mathcal L_{C[\rho_n]}^\dagger[A]\to \mathcal L_{C[\rho^\star]}^\dagger[A]$ in operator norm on the core.
Passing to the limit gives
\[
\Tr\!\big(\mathcal L_{C[\rho^\star]}^\dagger[A]\;\rho^\star\big)=0
\qquad\forall\,A\in\mathcal D(\mathcal L^\dagger).
\]
By duality, this is exactly \eqref{eq:D-nonlinear-stationarity}, i.e.\ $\mathcal L_{C[\rho^\star]}[\rho^\star]=0$
in the weak sense, hence $\rho(s)\equiv\rho^\star$ is a stationary solution of the bootstrap WESH master equation.
\end{proof}

\begin{remark}[Uniqueness and mixing; link to Lemma~\ref{lem:contraction}]
Lemma~\ref{lem:SchauderTychonoff} provides \emph{existence} of at least one stationary state
$\rho^\star\in\mathcal S_{\rm phys}$ for the self-consistent bootstrap map
$F_{\delta s}(\rho)=\exp(\delta s\,\mathcal L_{C[\rho]})(\rho)$ under the stated compactness/continuity assumptions.

\noindent
\textbf{Uniqueness and convergence.}
In the Markov window $\mu\ll1$, \emph{uniqueness} of the stationary state and \emph{trace-norm mixing}
follow already from the Dobrushin-type contraction mechanism of Lemma~\ref{lem:contraction}
(finite range $\xi$ and bounded per-site influence on blocks $L\gg\xi$), i.e. from strict
trace-norm contractivity of the update--mix map. This argument is entirely pre-thermal and does
\emph{not} require any KMS/detailed-balance input.

\noindent
\textbf{Optional independent route (specialization).}
In the near-horizon KMS detailed-balance setting one can alternatively deduce uniqueness/mixing
from primitivity together with KMS self-adjointness of $\mathcal L$, yielding a one-dimensional
kernel and a spectral gap $\lambda_{\rm gap}>0$ on the orthogonal complement, hence exponential
mixing
\[
\big\|\,e^{s\mathcal L}(\rho)-\rho^\star\,\big\|_1 \le C\,e^{-\lambda_{\rm gap}s}\,.
\]
This KMS-gap route is an \emph{independent, specialized} proof and provides a physical estimate of
mixing rates, but it is not the logical basis for fixed-point uniqueness in the general (pre-thermal)
WESH alignment analysis.
\end{remark}


\noindent\textbf{Theorem \ref{thm:alignment-PartII} (Variational alignment and metric consistency).}

\vspace{0.5cm}

\noindent Existence of $\rho^\star$ follows from Lemma D.1. 
We now establish uniqueness and mixing properties. Let $\rho(s)$ evolve under the WESH master equation (Eq.~\eqref{eq:wesh-master}) in the Markovian window $\tau_{\rm corr}\ll\tau_{\rm Eig}$, with exponential--causal weight
\vspace{0.3cm}
\[
\gamma(x,y)=\frac{\gamma_0}{N^2}e^{-d(x,y)/\xi}\,\Theta[{\rm causal}(x,y)]
\quad\text{and}\quad
L_{xy}=\tilde T^2(x)-\tilde T^2(y),\qquad \tilde T:=\hat T/\tau_s,
\]
\vspace{0.3cm}
\noindent and let the entanglement gate be the normalized Rényi–2 correlator $C[\rho;x,y]$ (see Eq.~\eqref{eq:renyi-gate}). 

\vspace{0.3cm}
\noindent Throughout this Appendix we use the normalized measures $\int_x$ and $\int_{xy}$
defined in Eq.~\eqref{eq:measure-conventions}.

\vspace{0.3cm}
\noindent Define the (regularized) WESH Lyapunov functional
\vspace{0.3cm}

\begin{align*}
\mathcal{M}_\epsilon[\rho] :=\;&
\int_x\Big(\langle \tilde T^4(x)\rangle_\rho - \langle \tilde T^2(x)\rangle_\rho^2\Big)\\
&+ \int_{xy}\gamma(x,y)\,C[\rho;x,y]\,\langle L_{xy}^2\rangle_\rho
\;+\; \epsilon\,\Tr\rho^2, \qquad \epsilon>0,
\end{align*}
\noindent where the term $\epsilon\,\Tr\rho^2$ is a Hilbert--Schmidt regularizer used to control
the commutator-norm dissipation estimates (note that $0<\Tr\rho^2\le1$ for states),
$\tilde T:=\hat T/\tau_s$ is dimensionless, and $L_{xy}:=\tilde T^2(x)-\tilde T^2(y)$.

\vspace{0.3cm}
\noindent Then:

\begin{enumerate}[label=(\roman*)]

\item \textbf{Monotonicity.} Along any non-stationary trajectory $\rho(s)$,
\[
\frac{d}{ds}\mathcal{M}_\epsilon[\rho(s)]
\;=\;-\mathcal{D}_\epsilon[\rho(s)]\;<\;0,
\]
with $\mathcal{D}_\epsilon[\rho]\ge 0$ a sum of weighted commutator norms for the Hermitian
jump set $\{\tilde T^2(x),\,L_{xy}\}$ (by the standard GKSL identity for Hermitian jumps under the standing assumptions).
In particular, the Hilbert--Schmidt regularizer contributes the exact term
\[
\frac{d}{ds}\big(\epsilon\,\Tr\rho^2\big)
=-2\epsilon\left(
\nu\int_x\|[\tilde T^2(x),\rho]\|_2^2
+\int_{xy}\gamma(x,y)\,C[\rho;x,y]\;\|[L_{xy},\rho]\|_2^2
\right)\le 0,
\]
so equality in $d\mathcal M_\epsilon/ds\le 0$ forces $[\tilde T^2(x),\rho]=[L_{xy},\rho]=0$ for all $x,y$ (cf.\ Appendix~G).

\vspace{0.3cm}
Across concatenated micro-steps, coefficient-update errors are controlled by the Markov parameter
$\mu=\tau_{\rm corr}/\tau_{\rm Eig}\ll1$ (Remark~D.3).

\vspace{0.3cm}
\item \textbf{Unique stationary point (variational alignment).} There is a unique fixed point $\rho^\star$ for the $s$‑flow. At $\rho^\star$ the time field satisfies the \emph{gradient‑alignment} condition
\[
\partial_\mu \tau(x)=k\,\partial_\mu\Phi(x),\qquad
\Phi(x):=\int d^4y\ K(x-y)\,C[\rho^\star;x,y],
\]
on chronogenetic IR domains $L\gg\xi$, up to the controlled Markov error $\mathcal O(\mu)$ (with $\mu\to0$ as $N\to\infty$, Remark~D.3)
and the controlled IR gradient-expansion remainder $\mathcal O(\xi^2\partial^3\tilde\tau)$ (Step~3, $L\gg\xi$).
Here $K$ is the (Yukawa-type) mediator kernel defining $\Phi$ in the main text
(cf.\ the Terminological note distinguishing $K$ from the causal rate envelope $K_\xi$;
Appendix~A fixes the common range $\xi$; $\Phi$ is defined with the unnormalized measure as in main-text conventions).
The normalization is fixed by the GR matching
\[
\frac{k^2}{4\pi G}=\lambda_1+3\lambda_2\qquad\text{(Eq.~\eqref{eq:parameter-relations}).}
\]
The full derivation is given in Proposition~D.2 below.

\vspace{0.3cm}
\item \textbf{Global attractivity.} If $\rho(0)$ satisfies the nondegeneracy hypothesis of Lemma~\ref{lem:Gamma-positive}, then $\rho(s)\to\rho^\star$ in trace norm as $s\to\infty$. 
Uniqueness and global convergence follow from mixing/primitivity (Remark~D.1) or from a
finite-range Dobrushin contraction on blocks $L\gg\xi$ (Lemma~\ref{lem:contraction}).
The $\epsilon\,\Tr\rho^2$ term is a technical regularizer for the Lyapunov estimates.

\vspace{0.3cm}
\item \textbf{Collective–stability scaling.}
At the fixed point, the causal bilocal channel carries an $N^{-2}$ prefactor, so the per–site collapse rate scales as $\Gamma_i\!\sim\!\gamma_0/N$.
The fixed‑point balance $\mu\,\sigma_\alpha=\mathcal O(1)$ (with $\mu=\tau_{\rm corr}/\tau_{\rm Eig}$) then selects $\alpha=2$, hence
\[
\tau_{\rm coh}(N)\ \propto\ N^2 .
\]

\vspace{0.3cm}
\item \textbf{Metric emergence and hidden–sector cancellation.}
At $\rho^\star$, gradient alignment cancels the quadratic pieces of the time–sector stress and the nonlocal backreaction,
\[
T^{(T)}_{\mu\nu}+T^{(\mathrm{nl})}_{\mu\nu}= \mathcal O(1/N),
\]
so that the full metric $g_{\mu\nu}=g^{(T)}_{\mu\nu}+g^{(C)}_{\mu\nu}+g^{(E)}_{\mu\nu}$ satisfies, up to $1/N$ corrections,
\[
G_{\mu\nu}+\Lambda g_{\mu\nu}=8\pi G\,T^{(m)}_{\mu\nu}.
\]
(For the precise statements see Eqs.~\eqref{eq:hidden-cancel} and \eqref{eq:emergent-einstein}.)

\vspace{0.3cm}
\item \textbf{Continuum and regularization limit.} As $\epsilon\downarrow 0$, the unique minimizer of $\mathcal{M}_\epsilon$ converges to a minimizer of the unregularized functional $\mathcal{M}$; fluctuations are controlled at $\mathcal O(\epsilon)$ (standard $\Gamma$–convergence under the compactness of sublevel sets in trace norm).
\end{enumerate}

\vspace{0.5cm}
\begin{proposition}[Stationarity $\Rightarrow$ alignment derivation]
\label{prop:alignment-derivation}

\vspace{0.3cm}
\noindent We make explicit the only logically admissible route compatible with the QFTT--WESH
formalism: the monotone is a functional of the state $\rho$ (not of the operator $\tilde T$),
hence the relevant first variation is $\delta \mathcal M_\epsilon/\delta \rho$.
Gradient alignment emerges as the Euler--Lagrange condition of the induced IR
(coarse-grained) functional on the effective time-field configuration.

\medskip
\noindent\textbf{Step 0 (what is varied).}
Fix the conserved charges $\{\hat Q_a\}$ and values $c_a$ (WESH--Noether), and consider
variations $\delta\rho$ tangent to $\mathcal S_{\rm phys}$, i.e.
\[
\Tr\,\delta\rho=0,\qquad \Tr(\delta\rho\,\hat Q_a)=0\ \ \forall a,
\qquad \rho+\delta\rho\ge 0\ \text{(to first order)}.
\]
We write the monotone (same as in the statement) as
\[
\mathcal M_\epsilon[\rho]=\mathcal M_{\rm loc}[\rho]+\mathcal M_{\rm bi}[\rho]+\epsilon\,\Tr\rho^2,
\]
with
\[
\mathcal M_{\rm loc}[\rho]=\int_x \Var_\rho\!\big(\tilde T^2(x)\big)
=\int_x\Big(\langle \tilde T^4(x)\rangle_\rho-\langle \tilde T^2(x)\rangle_\rho^2\Big),
\]
\[
\mathcal M_{\rm bi}[\rho]=\int_{xy}\gamma(x,y)\,C[\rho;x,y]\,
\langle (\tilde T^2(x)-\tilde T^2(y))^2\rangle_\rho .
\]
Here $\int_x,\int_{xy}$ are the normalized measures of Eq.~\eqref{eq:measure-conventions}.
All terms are finite under the standing moment assumptions (finite fourth moments of $\tilde T$)
and boundedness of the gate $0\le C\le 1$.

\medskip
\noindent\textbf{Step 1 (first variation in $\rho$; gate contribution included).}
For any admissible operator $A$ (possibly unbounded) under the standing domain assumptions
(i.e.\ all traces below are finite), one has $\delta\langle A\rangle_\rho=\Tr(A\,\delta\rho)$.
Hence, for each spacetime point $x$,
\[
\delta\,\Var_\rho(\tilde T^2(x))
=\Tr\!\Big(\big[\tilde T^4(x)-2\langle \tilde T^2(x)\rangle_\rho\,\tilde T^2(x)\big]\delta\rho\Big).
\]
Therefore
\begin{equation}
\delta\mathcal M_{\rm loc}[\rho]
=\Tr\!\Big(\mathcal A_{\rm loc}[\rho]\,\delta\rho\Big),\qquad
\mathcal A_{\rm loc}[\rho]:=\int_x\Big(\tilde T^4(x)-2\langle \tilde T^2(x)\rangle_\rho\,\tilde T^2(x)\Big).
\label{eq:D-variation-local}
\end{equation}
For the bilocal part, define $L_{xy}:=\tilde T^2(x)-\tilde T^2(y)$.
Then $\langle L_{xy}^2\rangle_\rho=\Tr(L_{xy}^2\,\rho)$ and
\[
\delta\langle L_{xy}^2\rangle_\rho=\Tr(L_{xy}^2\,\delta\rho).
\]
Thus
\[
\delta\mathcal M_{\rm bi}[\rho]
= \int_{xy}\gamma(x,y)\Big(
\delta C[\rho;x,y]\;\langle L_{xy}^2\rangle_\rho
+ C[\rho;x,y]\;\Tr(L_{xy}^2\,\delta\rho)\Big).
\]
The second term is linear in $\delta\rho$:
\begin{equation}
\begin{split}
\int_{xy}\gamma(x,y)\,C[\rho;x,y]\;\Tr(L_{xy}^2\,\delta\rho)
&=\Tr\!\Big(\mathcal A_{\rm bi}^{(1)}[\rho]\;\delta\rho\Big),\\[4pt]
\mathcal A_{\rm bi}^{(1)}[\rho]&:=\int_{xy}\gamma(x,y)\,C[\rho;x,y]\;L_{xy}^2.
\end{split}
\label{eq:D-variation-biloc1}
\end{equation}
At the fixed point $\rho^\star$, the bootstrap implies that $C[\rho;x,y]$ depends on the state
(via local/bilocal reductions). We denote its first-order response along admissible variations by
a bounded operator $\mathcal G_{xy}$ such that
\[
\delta C[\rho;x,y]\big|_{\rho=\rho^\star}
=\Tr\!\big(\mathcal G_{xy}\,\delta\rho\big).
\]
Accordingly, the first term yields an additional contribution from the gate variation:
\begin{equation}
\int_{xy}\gamma(x,y)\,\delta C[\rho;x,y]\;\langle L_{xy}^2\rangle_\rho\Big|_{\rho=\rho^\star}
=\Tr\!\Big(\mathcal A_{\rm bi}^{(2)}[\rho^\star]\;\delta\rho\Big),
\label{eq:D-variation-biloc2}
\end{equation}
with
\[
\mathcal A_{\rm bi}^{(2)}[\rho^\star]
:=\int_{xy}\gamma(x,y)\,\langle L_{xy}^2\rangle_{\rho^\star}\;\mathcal G_{xy}.
\]
This is the term that generates the $\Phi$-dependent forcing in the coarse-grained EL equation (Step~5).

Finally, the regularizer gives
\begin{equation}
\delta(\epsilon\,\Tr\rho^2)=2\epsilon\,\Tr(\rho\,\delta\rho).
\label{eq:D-variation-eps}
\end{equation}
Collecting \eqref{eq:D-variation-local}, \eqref{eq:D-variation-biloc1}, \eqref{eq:D-variation-biloc2}, \eqref{eq:D-variation-eps}:
\begin{equation}
\delta\mathcal M_\epsilon[\rho]\big|_{\rho=\rho^\star}
=\Tr\!\Big(\big[\mathcal A_{\rm loc}[\rho^\star]+\mathcal A_{\rm bi}^{(1)}[\rho^\star]+\mathcal A_{\rm bi}^{(2)}[\rho^\star]+2\epsilon\,\rho^\star\big]\delta\rho\Big)
\quad\text{for all admissible }\delta\rho\text{ in }\mathcal S_{\rm phys}.
\label{eq:D-variation-total}
\end{equation}

\medskip
\noindent\textbf{Step 2 (Euler--Lagrange condition on $\mathcal S_{\rm phys}$).}
Introduce Lagrange multipliers for the constraints in $\mathcal S_{\rm phys}$:
a scalar $\alpha$ for $\Tr\rho=1$ and coefficients $\{\beta_a\}$ for $\Tr(\rho\hat Q_a)=c_a$.
Stationarity of $\rho^\star$ as a minimizer of $\mathcal M_\epsilon$ on $\mathcal S_{\rm phys}$ means:
\[
\delta\Big(\mathcal M_\epsilon[\rho]-\alpha\,\Tr\rho-\sum_a \beta_a \Tr(\rho\hat Q_a)\Big)\Big|_{\rho=\rho^\star}=0
\quad\text{for all admissible }\delta\rho.
\]
Using \eqref{eq:D-variation-total}, this is equivalent to the operator identity
\begin{equation}
\mathcal A_{\rm loc}[\rho^\star]+\mathcal A_{\rm bi}^{(1)}[\rho^\star]+\mathcal A_{\rm bi}^{(2)}[\rho^\star]+2\epsilon\,\rho^\star
=\alpha\,\mathbb 1+\sum_a \beta_a \hat Q_a\,,
\label{eq:D-EL-operator}
\end{equation}
\noindent understood as an equality of densely-defined quadratic forms on a common invariant core.
Equation \eqref{eq:D-EL-operator} is the \emph{correct} first-order optimality condition:
it varies the state, not the operator, and includes the gate variation.

\medskip
\noindent\textbf{Step 3 (IR/coarse-grained reduction: from \eqref{eq:D-EL-operator} to a field equation).}
We now pass to the IR regime $L\gg\xi$ (Box~1) where the bilocal kernel has finite range $\xi$ and
the dynamics mixes on blocks $L\gg\xi$ (Lemma~\ref{lem:contraction}).
Define the coarse-grained classical field
\[
\tau(x):=\langle \hat T(x)\rangle_{\rho^\star},\qquad
u(x):=\langle \tilde T^2(x)\rangle_{\rho^\star}=\frac{1}{\tau_s^2}\langle \hat T^2(x)\rangle_{\rho^\star}.
\]
In this regime, spectral regularity and mixing imply that fluctuations of $\tilde T^2(x)$ around its mean
are suppressed on coarse blocks (the local variance term in $\mathcal M_{\rm loc}$ is minimized), so that
\[
\Var_{\rho^\star}(\tilde T^2(x))=\mathcal O(\mu)\quad\text{with }\mu:=\tau_{\rm corr}/\tau_{\rm Eig}\ll 1,
\]
and consequently $u(x)$ is a smooth field at scale $L$.
Moreover, the bilocal term becomes a quadratic form in field differences:
\[
\mathcal M_{\rm bi}[\rho^\star]\ \approx\ \int_{xy}\gamma(x,y)\,C[\rho^\star;x,y]\,(u(x)-u(y))^2,
\]
with an error of order $\mathcal O(\mu)$ controlled by the Markov window.

Now expand the finite-range quadratic form for smooth $u$.
Since $L_{xy}^2=L_{yx}^2$ and the gate is symmetric $C[\rho^\star;x,y]=C[\rho^\star;y,x]$,
all bilinear forms in $(x,y)$ depend on $\gamma$ only through its symmetrization
$\gamma_{\rm sym}(x,y):=\tfrac12(\gamma(x,y)+\gamma(y,x))$.
Hence, without loss of generality we may assume $\gamma=\gamma_{\rm sym}$ in the IR moment expansion,
so odd kernel moments vanish.
Using symmetry $\gamma_{\rm sym}(x,y)=\gamma_{\rm sym}(y,x)$ together with finite range $\xi$,
$L^1$-integrability, and smoothness of $u$ on $L\gg\xi$, one obtains the standard Dirichlet-to-gradient reduction:
\begin{equation}
\int_{xy}\gamma(x,y)\,C[\rho^\star;x,y]\,(u(x)-u(y))^2
= \int_x \Big(\lambda_2\, g^{\mu\nu}_{(C)}(x)\,\partial_\mu u(x)\,\partial_\nu u(x)\Big)
+\mathcal O(\xi^2\partial^3 u),
\label{eq:D-Dirichlet-expansion}
\end{equation}
where $\lambda_2>0$ is the (fixed) IR coefficient induced by the kernel moments
and $g^{\mu\nu}_{(C)}$ is the emergent inverse metric weight built from the second moments
of the causal kernel (as in the main-text covariantization prescription; Box~1).

Similarly, the local term enforces small variance and fixes the relation between $u$ and $\tau$
on the stationary manifold: since $\rho^\star$ minimizes $\Var(\tilde T^2)$,
to leading IR order one has $u(x)\approx \tilde\tau(x)^2$ with $\tilde\tau:=\tau/\tau_s$.
Thus we may write $u=\tilde\tau^2$ at the level of the Euler--Lagrange equation, with controlled
$\mathcal O(\mu)$ corrections.

\medskip
\noindent\textbf{Step 4 (introducing the entanglement potential $\Phi$).}
Define the entanglement potential exactly as in the main text:
\[
\Phi(x):=\int d^4y\;K(x-y)\,C[\rho^\star;x,y],
\]
where $K$ is the Yukawa mediator kernel (distinct from the rate kernel $K_\xi$; see Appendix~A and the
Terminological note in Sec.~1).
In an IR/slowly-varying regime where $C[\rho^\star;x,y]\approx \bar C(x)\delta_\xi(x-y)$ with $\bar C(x)$ the locally averaged gate, 
one may write
\[
(\Box+m_T^2)\Phi(x)\approx\mathcal N\,\bar C(x)
\quad\text{(in the distributional sense)},
\]
for an $\mathcal O(1)$ normalization $\mathcal N$.
This relation is the standard Yukawa/Helmholtz inversion.

\medskip
\noindent\textbf{Step 5 (field-level Euler--Lagrange equation).}

\noindent\textit{Index convention (IR).} In Steps 5–6 we raise indices with the emergent IR weight
$g^{\mu\nu}_{(C)}$ from Eq.~\eqref{eq:D-Dirichlet-expansion}, i.e.\ $\partial^\mu := g^{\mu\nu}_{(C)}\partial_\nu$
(equivalently $\nabla^\mu$ once the Levi--Civita connection is defined).

\medskip
Project \eqref{eq:D-EL-operator} onto the coarse-grained sector by taking expectations
against local variations supported in a block $B_L(x)$ and using the reduction
\eqref{eq:D-Dirichlet-expansion} together with $u\approx\tilde\tau^2$.
The result is a deterministic IR Euler--Lagrange equation of the form
\begin{equation}
\partial_\mu\Big(\lambda_2\,\partial^\mu(\tilde\tau^2)\Big)=\lambda_1\,(\tilde\tau^2-\Phi)+\mathcal O(\mu)+\mathcal O(\xi^2\partial^3\tilde\tau),
\label{eq:D-EL-IR}
\end{equation}
where $\lambda_1,\lambda_2>0$ are the same IR coefficients appearing in the main-text matching
(and are fixed by the WESH generator; no new free parameters are introduced).
Equation \eqref{eq:D-EL-IR} is the precise IR stationarity condition:
it equates the Laplacian flow induced by the bilocal channel to the Yukawa-sourced potential term
(encoded by the gate-response term $\mathcal A_{\rm bi}^{(2)}$ from Step~1).

\medskip
\noindent\textbf{Step 6 (alignment as the unique smooth stationary branch).}
In the stationary IR regime, the admissible solutions are those with vanishing dissipative current.
Define the (coarse-grained) WESH mismatch current
\[
J_\mu(x):=\partial_\mu(\tilde\tau^2)-\frac{\lambda_1}{\lambda_2}\,\partial_\mu\Phi(x).
\]
Equation \eqref{eq:D-EL-IR} defines the IR stationary manifold. The \emph{aligned} branch is characterized by
vanishing mismatch current $J_\mu\equiv 0$.
By mixing (Lemma~\ref{lem:contraction}) and the existence of a spectral gap in the primitive sector
(when applicable), any smooth stationary profile with $J_\mu\not\equiv 0$ sustains strictly positive
coarse-grained dissipation on some block (item (i)), contradicting stationarity. Hence the only globally
attractive smooth stationary branch satisfies $J_\mu\equiv 0$.

Therefore, on the stationary manifold and in the IR window,
\begin{equation}
\partial_\mu(\tilde\tau^2)=\frac{\lambda_1}{\lambda_2}\,\partial_\mu\Phi
\qquad\Longleftrightarrow\qquad
\partial_\mu\Big(\tilde\tau^2-\frac{\lambda_1}{\lambda_2}\Phi\Big)=0,
\label{eq:D-IR-branch}
\end{equation}
up to the controlled IR errors already present in \eqref{eq:D-EL-IR}
(i.e.\ $\mathcal O(\mu)+\mathcal O(\xi^2\partial^3\tilde\tau)$).
On the globally attractive stationary branch selected by mixing/primitivity
(Remark~D.1 and Lemma~\ref{lem:contraction}), the fixed-point profile is
block-homogeneous on scales $L\gg\xi$:
\begin{equation}
\tilde\tau(x)=\tilde\tau_\star+\mathcal O(\mu)\qquad (L\gg\xi),
\label{eq:D-IR-sync}
\end{equation}
so that the conversion from $\partial_\mu(\tilde\tau^2)$ to $\partial_\mu\tau$ does not
introduce a space-dependent proportionality factor. Indeed,
\begin{equation}
\partial_\mu\tau(x)
=\frac{\tau_s}{2\tilde\tau_\star}\,\partial_\mu(\tilde\tau^2)(x)
=\underbrace{\frac{\tau_s}{2\tilde\tau_\star}\frac{\lambda_1}{\lambda_2}}_{=:k}\,
\partial_\mu\Phi(x)\;+\;\mathcal O(\mu)\;+\;\mathcal O(\xi^2\partial^3\tilde\tau).
\label{eq:D-alignment-IR-kconst}
\end{equation}
In the thermodynamic/continuum limit where $\mu\to0$ (Remark~D.3), this yields the
alignment condition of the main text with a constant coefficient:
\[
\partial_\mu \tau = k\,\partial_\mu\Phi
\quad\text{on chronogenetic IR domains.}
\]

\medskip
\noindent\textbf{Step 7 (matching fixes $k$; no tuning).}
The normalization $k$ is not free. The IR matching condition of the main text,
\[
\frac{k^2}{4\pi G}=\lambda_1+3\lambda_2,
\]
fixes $k$ uniquely in terms of the WESH couplings $\lambda_{1,2}$ and Newton's constant,
as stated in Eq.~\eqref{eq:parameter-relations}. No additional dimensionless parameters enter.
\end{proposition}

\medskip
\noindent\textbf{Conclusion.}
The alignment condition is not imposed and not obtained by an illegitimate operator-variation.
It follows from:
(i) correct stationarity of $\mathcal M_\epsilon[\rho]$ under state variations, including the gate variation $\delta C$;
(ii) finite-range IR reduction of the bilocal quadratic form (with symmetrized kernel);
(iii) the Yukawa definition of $\Phi$, whose source term arises from $\delta C/\delta\rho$;
(iv) vanishing of the stationary mismatch current and block-homogeneity at the fixed point, which together select the unique globally attractive branch with constant alignment coefficient $k$.

\vspace{0.5cm}
\begin{proof}[Proof sketch of Theorem \ref{thm:alignment-PartII}]

\vspace{0.3cm}
~\\

\noindent\emph{(i) Lyapunov decay.}
Freeze $C[\rho]$ and $\gamma$ on a micro–interval $[s,s+\delta s]$.
With Hermitian jumps and unital evolution, the GKSL identity yields $\tfrac{d}{ds}\mathcal M_\epsilon\le 0$;
the term $\epsilon\,\Tr\rho^2$ provides the regularization needed for the commutator-norm estimates.
Concatenating micro–intervals with piecewise–constant coefficient freezing via the Kato–Trotter product formula
gives monotone decay along $s$.

\vspace{0.3cm}
\noindent\emph{(ii) Stationarity $\Rightarrow$ alignment.}
See Proposition~D.2 above for the complete derivation.

\vspace{0.3cm}
\noindent\emph{(iii) Uniqueness and mixing.}
Global convergence follows either from primitivity in the KMS geometry (spectral gap),
or from a Dobrushin–type contraction for finite–range bilocal mixing on blocks $L\gg\xi$
with rate $\varepsilon=\Theta(\tau_{\rm corr}/\tau_{\rm Eig})$ (Lemma~\ref{lem:contraction}, if available).

\vspace{0.3cm}
\noindent\emph{(iv) Scaling in item (iv).}
Because the bilocal channel carries an $N^{-2}$ prefactor, the per–site hazard scales as $\Gamma_i\sim \gamma_0/N$; at the fixed point the balance $\mu\,\sigma_\alpha=\mathcal O(1)$ (with $\mu=\tau_{\rm corr}/\tau_{\rm Eig}$) enforces the $N^2$ coherence‑time scaling stated in the collective–stability item above.

\vspace{0.3cm}
\noindent\emph{(v) GR bridge.}
With $\partial \hat T = k\,\partial \Phi$ and the matching (Eq.~\eqref{eq:parameter-relations}),
the quadratic terms cancel between $T^{(T)}$ and $T^{(\mathrm{nl})}$ up to $\mathcal O(1/N)$;
discrete–to–continuum control bounds supply the remainder, yielding Einstein's equation with matter
(Eqs.~\eqref{eq:hidden-cancel}, \eqref{eq:emergent-einstein}).

\vspace{0.3cm}
\noindent\emph{(vi) $\Gamma$–convergence.}
Lower semicontinuity of $\mathcal M$ and compactness of sublevel sets (trace norm)
imply $\Gamma$–convergence $\mathcal M_\epsilon\to\mathcal M$ and convergence of minimizers as $\epsilon\downarrow 0$.

\end{proof}

\paragraph{Notation \& guardrails (consistency with the main text).}
The dissipative weight $\gamma$ is \emph{exponential–causal} (Eq.~\eqref{eq:gamma-kernel}). 
The mediator kernel $K$ entering $\Phi$ is Yukawa-type (cf.\ the Terminological note distinguishing $K$ from the causal rate envelope $K_\xi$; Appendix~A fixes the common range $\xi$).
The local rate $\nu$ is fixed by coarse–grained consistency (Eq.~\eqref{eq:nu-matching}). 
The $s\!\to\!t$ bootstrap is $dt/ds=\Gamma[\rho]\ge0$, with $\Gamma>0$ under the nondegeneracy
hypothesis (Lemma~\ref{lem:Gamma-positive}) and on chronogenetic intervals. 
$\Gamma$ is defined in Eq.~\eqref{eq:gamma-decomposition}.

\vspace{0.3cm}
\begin{remark}[Endogenous mechanism — no external metric inputs]
Alignment is the unique attractor of the endogenous WESH \emph{bootstrap dynamics} 
$\rho\mapsto C[\rho]\mapsto\text{dynamics}\mapsto\rho$, a continuous–stochastic 
process (no discrete loop): the GKSL generator (Eq.~\eqref{eq:wesh-master}), 
the Rényi‑2 gate $C$, the causal kernel (Eq.~\eqref{eq:gamma-kernel}, mediator in Appendix~A), 
and the normalization/matching (Eq.~\eqref{eq:nu-matching}) together with the parameter relations (Eq.~\eqref{eq:parameter-relations}) 
are specified within the formalism. A differentiable manifold structure for the label
space $(x,y)$ is presupposed to define fields and integrals, but no background
\emph{metric geometry or causal structure} is assumed; these emerge dynamically
from the causal support of the kernel and the eigentime density.
In Steps 5–6 of Proposition~D.2, indices are raised with the emergent weight $g^{\mu\nu}_{(C)}$
(Eq.~\eqref{eq:D-Dirichlet-expansion}), consistent with this endogenous construction.
\end{remark}

\begin{remark}[Markovian error control]
Coefficient–freezing errors are controlled by the Markov parameter
$\mu:=\tau_{\rm corr}/\tau_{\rm Eig}\ll1$.
In the collective regime one may equivalently express the suppression as
$\tau_{\rm corr}/\tau_{\rm Eig}^{(\rm eff)}\sim N^{-2}$ (cf.\ Eq.~\eqref{eq:ratio-scaling}),
so these errors vanish in the large-$N$ limit on coarse-grained windows $\Delta t\gg\tau_{\rm corr}$.
\end{remark}

\begin{remark}[Falsifiable signatures]
The fixed point entails: (a) $\tau_{\rm coh}(N)\propto N^2$; (b) an angular law $\Gamma(\theta)\propto 1+\varepsilon\cos^2\theta$;
(c) $\mathcal O(1/N)$ corrections to Einstein's equation (see Eq.~\eqref{eq:emergent-einstein}).
The first two predictions are accessible to current NISQ platforms; the third awaits future precision tests.
\end{remark}
\vspace{0.3cm}
\endgroup

\section*{Appendix E: Heuristic derivations for $N$–scaling}
\label{app:heuristics}

\noindent\textbf{Setup.} The WESH bilocal channel uses a causal, normalized kernel with prefactor $N^{-2}$ (Eq.~\eqref{eq:gamma-kernel}); the intensity channel is quadratic in $\hat T^2$ (Eq.~\eqref{eq:gamma-decomposition}). The mediator kernel entering $\Phi$ is Yukawa (Appendix~A), with range $\xi$ and correlation time $\tau_{\corr}\sim \xi/c$ (independent of $N$).

\vspace{0.6em}
\noindent\textbf{H1 — Pairwise normalization.} Let the effective pair couplings be $\gamma_{ij}=\gamma_0 N^{-2} w_{ij}$ with $w_{ij}=\mathcal O(1)$ and $\sum_{i<j} w_{ij}=\Theta(N^2)$ (bounded, causal weights; equivalently $\langle w\rangle=\mathcal O(1)$ over $\Theta(N^2)$ pairs). Then
\[
\sum_{j\ne i}\gamma_{ij}=\mathcal O(\gamma_0/N),
\qquad
\sum_{i<j}\gamma_{ij}=\mathcal O(\gamma_0).
\]
Interpretation: the per–site collapse power scales as $1/N$ while the total power remains $\mathcal O(1)$, consistent with the $N^{-2}$ prefactor in Eq.~\eqref{eq:gamma-kernel}.

\vspace{0.6em}
\noindent\textbf{H2 — Local hazards from block averaging.} For block averages $X_B=\frac{1}{|B|}\sum_{i\in B}X_i$ (with $X_i$ a bounded local contribution to the $\hat T^2$-intensity) and short-range correlations (range $\xi$),
\[
\mathrm{Var}(X_B)=\mathcal O(|B|^{-1}).
\]
Under WESH weighting, the relevant block tracks the entangled set size, $|B|\sim N$. Because the intensity channel is quadratic in $\hat T^2$ (Eq.~\eqref{eq:gamma-decomposition}), the coarse–grained hazard inherits a square on the variance,
\[
\lambda \;\propto\; \mathrm{Var}(X_B)^2 \;=\; \mathcal O(N^{-2}).
\]
Thus $\lambda=\mathcal O(N^{-2})$ implies $\tau_{\mathrm{coh}}\sim\lambda^{-1}\propto N^2$, while the per-site rate $\Gamma_i\sim \gamma_0/N$ aligns with H1.

\vspace{0.6em}
\noindent\textbf{H3 — Correlation time vs.\ eigentime spacing.} The kernel range fixes $\tau_{\mathrm{corr}}\sim \xi/c$ (independent of $N$), while the effective macroscopic inter-event spacing grows under H1–H2 as $\tau_{\mathrm{Eig}}^{(\mathrm{eff})}\propto N^{2}$ (see Appendix~D). Hence
\[
\mu \;:=\; \frac{\tau_{\corr}}{\tau_{\rm Eig}^{(\mathrm{eff})}} \;=\; \mathcal O(N^{-2}),
\]
placing the dynamics deep in the Markov window for large $N$.

\vspace{0.6em}
\noindent\textbf{Consequence.} H1–H3 jointly underwrite the collective–stability law $\tau_{\coh}(N)\propto N^{2}$ and the small–parameter regime $\mu=\mathcal O(N^{-2})$ used in the variational analysis (Appendix~D) and in the GR bridge.

\vspace{1cm}

\section*{Appendix F: Derivation of the angular dependence law}
\addcontentsline{toc}{section}{Appendix F — Derivation of the angular decoherence law}
\label{app:ang}

\begingroup
\setcounter{equation}{0}
\renewcommand\theequation{F.\arabic{equation}}

\paragraph{Scope.}
We derive the observed angular law
\[
\Gdec(\theta)=\overline\Gamma\big(1+\varepsilon\cos^2\theta\big),
\]
starting from the bilocal sector of the intensity functional (see the bilocal term in Eq.~\eqref{eq:gamma-decomposition}) under the causal–exponential kernel and within the Markov window \(\Delta t\gg\tau_{\rm corr}\) (equal‑\(s\) evaluation). Assumptions: short–range dominance \(r\sim\xi\); weak, even gate anisotropy; rotated‑parity readout.
\vspace{0.3cm}

\begin{figure}[H]
  \centering
  \includegraphics[width=\linewidth]{Picture13.png}
  \label{fig:ang-geometry}
\end{figure}

\paragraph{Notation.}
We use \(\Ttil=\hat T/\tau_*\), \(\Lxy=\Ttil^2(x)-\Ttil^2(y)\); \(\Iloc,\Ibi\) denote the pre‑normalized local/bilocal parts with \(\Gamma_{\rm dec}=\tau_*^{-1}(\Iloc+\Ibi)\). The kernel weight follows Eq.~\eqref{eq:gamma-kernel} (causal exponential with range \(\xi\)).

\paragraph{Gate anisotropy parametrization.}
For GHZ states prepared with directional bias \(\hat{\mathbf n}\), the entanglement gate admits the even expansion
\begin{equation}
C(\Psi;\hat{\mathbf r}) \;=\; a_0 \;+\; a_2\,(\hat{\mathbf r}\!\cdot\!\hat{\mathbf n})^2 \;+\; \mathcal O(a_4),
\qquad a_0>0,\; |a_2|\ll a_0,
\label{eq:G-Cexp}
\end{equation}
where \(a_0\) is the isotropic term and \(a_2\) quantifies the intrinsic quadrupolar anisotropy of the prepared state (sign‐definite by state class; cf.\ Sec.~\ref{sec:F7}).

\subsection*{F.0 Dynamical origin: from the WESH master equation to parity decay}

The $\cos^2\theta$ law is not a generic consequence of bilocal geometry—it emerges specifically from the WESH dissipator structure. We establish this connection before proceeding to the geometric derivation.

\paragraph{Master equation in the toggling frame.}
Under CPMG dynamical decoupling, the reduced dynamics of the $N$-qubit register takes the GKSL form (cf.\ Eq.~\eqref{eq:wesh-master}):
\begin{equation}
\frac{d\rho}{ds} \;=\; \mathcal{L}_{\rm hw}[\rho] 
\;+\; \nu \int d^3x\, \mathcal{D}[\hat{T}^2(x)]\rho 
\;+\; \iint d^3x\, d^3y\, \gamma(x,y)\, C(\Psi;x,y)\, \mathcal{D}[L_{xy}]\rho,
\label{eq:F-master}
\end{equation}
where $\mathcal{L}_{\rm hw}$ captures hardware noise filtered by the echo sequence, and the WESH bilocal channel is governed by the jump operator $L_{xy} = \tilde{T}^2(x) - \tilde{T}^2(y)$ with kernel $\gamma(x,y)$ and entanglement gate $C(\Psi;x,y)$.

\paragraph{Parity as the observable.}
We consider the rotated-parity observable $\hat{\Pi}(\theta)$ used throughout this Appendix (see Sec.~F.2); the $X$-parity $\sigma_x^{\otimes N}$ is recovered as the special case $\hat{\Pi}(\pi/2)$. In the Heisenberg picture, the WESH contribution to parity evolution is:
\begin{equation}
\frac{d}{ds}\langle \hat{\Pi}(\theta) \rangle_{\rm WESH} 
\;=\; \iint d^3x\, d^3y\, \gamma(x,y)\, C(\Psi;x,y)\, 
\big\langle L_{xy}^\dagger \hat{\Pi}(\theta) L_{xy} - \tfrac{1}{2}\{L_{xy}^\dagger L_{xy}, \hat{\Pi}(\theta)\} \big\rangle.
\label{eq:F-parity-evol}
\end{equation}

\paragraph{Echo-frame reduction (effective scalar modulation).}
In the echo-narrowband regime at fixed $(T,n)$, the bilocal WESH sector acts as a state-dependent scalar rescaling of an effective parity-decay channel. Equivalently, we can write
\begin{equation}
\frac{d}{ds}\langle \hat{\Pi}(\theta)\rangle_{\rm WESH}
\;=\; \sum_j \gamma_j\, M(\Psi)\,
\Big\langle \hat{A}_j^\dagger \hat{\Pi}(\theta)\hat{A}_j
-\tfrac12\{\hat{A}_j^\dagger\hat{A}_j,\hat{\Pi}(\theta)\}\Big\rangle
\;=\; -R_{\Pi}\,M(\Psi)\,\langle \hat{\Pi}(\theta)\rangle,
\label{eq:F-parity-modulated}
\end{equation}
where $R_{\Pi}$ depends on the hardware channel and the CPMG filter function but is independent of $\theta$ and entanglement class for matched circuits at fixed $(T,n)$.

\paragraph{State-structural modulator.}
For GHZ-family states $|\Psi(\vartheta)\rangle = \cos\vartheta|0\cdots 0\rangle + \sin\vartheta|1\cdots 1\rangle$, the effective modulator $M(\Psi)$ factorizes into two contributions:
\begin{enumerate}
\item The entanglement correlator $\bar{C}(\Psi)$, which is $\mathcal{O}(1)$ for genuinely entangled states and $\approx 0$ for product states;
\item The structural overlap $|\langle 0\cdots 0|\Psi\rangle|^2 = \cos^2\vartheta$, which controls how the CPMG-filtered channel projects onto the decoherence-resistant subspace.
\end{enumerate}
Hence:
\begin{equation}
M(\Psi(\vartheta)) \;\propto\; \bar{C}(\Psi)\,\cos^2\vartheta.
\label{eq:F-modulator}
\end{equation}

\paragraph{Parity decay law.}
Integrating Eq.~\eqref{eq:F-parity-modulated} over the fixed evolution time $T$, the parity contrast between GHZ and product states becomes:
\begin{equation}
\Delta(\vartheta) \;\equiv\; |\langle\hat{\Pi}(\theta)\rangle|_{\rm GHZ(\vartheta)} - |\langle\hat{\Pi}(\theta)\rangle|_{\rm PROD(\vartheta)}
\;=\; \beta_0 + \beta_1\,\cos^2\vartheta + \mathcal{O}(T^2),
\label{eq:F-delta}
\end{equation}
with $\beta_1 \propto \bar{C}_{\rm GHZ} > 0$. This is the dynamical origin of the $\cos^2\vartheta$ law: it emerges from the WESH master equation through the state-structural modulator $M(\Psi)$, not from geometry alone.

\paragraph{W-state control (sign).}
For $W$-type states, the angular signature is controlled by the sign of the quadrupolar coefficient in the gate expansion: $a_2^{(W)}<0$, hence $\varepsilon_W<0$ (anti-modulation), consistent with Sec.~F.7.

\paragraph{Scope of the geometric derivation.}
The sections F.1–F.4 that follow derive the geometric factors ($G_0, G_2$) and the gate anisotropy ($a_0, a_2$) that determine the amplitude $\varepsilon$ of the modulation. These geometric elements explain \emph{how} the $\cos^2\theta$ dependence is shaped by the device layout and state preparation, but the \emph{existence} of this angular form—rather than $\cos^4\theta$ or another function—is fixed by the WESH dissipator structure established here.

\subsection*{F.1 Equal‑\(s\) reduction and gradient expansion}

Within the Markov window, the causal bilocal integral reduces to a 3D spatial average at fixed \(s\) (the effective kernel results from integrating the 4D causal weight over the eigentime coordinate).
Let \(y=x+r\,\hat{\mathbf r}\) with \(r\sim\xi\). A second–order Taylor expansion gives
\[
\Ttil^2(y)=\Ttil^2(x)+r\,\hat{\mathbf r}\!\cdot\!\nabla\Ttil^2(x)
+\tfrac12 r^2 \hat r_i\hat r_j\,\partial_i\partial_j\Ttil^2(x)+\mathcal O(r^3),
\]
hence
\begin{equation}
\big(\Ttil^2(x)-\Ttil^2(y)\big)^2
= r^2\,(\hat{\mathbf r}\!\cdot\!\nabla\Ttil^2)^2
+\mathcal O\!\big(r^3\,\|\nabla\Ttil^2\|\,\|\nabla^2\Ttil^2\|\big).
\label{eq:G-grad}
\end{equation}
The leading \(r^2\) term links scalar differences to a direction‑selective form.

\subsection*{F.2 Measurement projection: rotated parity selects $\cos^2\theta$}

With \(R_y(\theta)=e^{-i\theta\sigma_y/2}\),
\[
\hat\Pi(\theta)=R_y(\theta)^{\otimes N}\!\Big(\!\bigotimes_{i}\sigma_z^{(i)}\!\Big)R_y(-\theta)^{\otimes N}
=\big(\cos\theta\,\sigma_z+\sin\theta\,\sigma_x\big)^{\otimes N}.
\]
Choose the \(z\)–axis as angular origin in the \(x\)–\(z\) plane. Then
\[
\hat{\mathbf m}(\theta)=\sin\theta\,\hat{\mathbf x}+\cos\theta\,\hat{\mathbf z},\qquad
\hat{\mathbf r}_{ij}=\sin\alpha_{ij}\,\hat{\mathbf x}+\cos\alpha_{ij}\,\hat{\mathbf z},
\]
so that
\[
\big(\hat{\mathbf m}(\theta)\!\cdot\!\hat{\mathbf r}_{ij}\big)^2
=\big(\sin\theta\sin\alpha_{ij}+\cos\theta\cos\alpha_{ij}\big)^2
=\cos^2\big(\theta-\alpha_{ij}\big).
\]
Rotated parity thus acts as a geometric projector onto the pair orientation \(\alpha_{ij}\), selecting the \(\cos^2\theta\) harmonic at pair level.

\subsection*{F.3 From microscopic to mesoscopic: lattice–weighted averaging}

\noindent\textit{Bridge.} The microscopic average \(\langle(\hat{\mathbf r}\!\cdot\!\nabla\Ttil^2)^2\rangle_{S^2}\) becomes a discrete sum over physical pairs once the causal–exponential kernel collapses the support to \(r\!\sim\!\xi\) along device edges. Label pairs by \(k\), with orientation \(\alpha_k\) and radial weight
\[
w_k \;\equiv\; r_k^2\,\gamma(r_k)\;\propto\; r_k^2\,e^{-r_k/\xi}.
\]
Collecting the gradient factor and the measurement projector from F.1–F.2,
\begin{equation}
\Gamma_{\rm bi}(\theta)\;\propto\;\sum_{k} w_k\,C(\Psi;\hat{\mathbf r}_k)\,\cos^2\!\big(\theta-\alpha_k\big).
\label{eq:G-bridge}
\end{equation}
Use \(\cos^2(\theta-\alpha)=\tfrac12[\,1+\cos 2\theta\,\cos 2\alpha + \sin 2\theta\,\sin 2\alpha\,]\) and define
\begin{equation}
G_0:=\sum_k w_k,\qquad
G_2:=\sum_k w_k\cos 2\alpha_k,\qquad
G_2^\perp:=\sum_k w_k\sin 2\alpha_k.
\label{eq:G-Gdefs}
\end{equation}
For symmetric heavy–hex subsets (Fig.~\ref{fig:ang-geometry}), \(G_2^\perp=0\) by reflections, and
\[
\sum_k w_k \cos^2(\theta-\alpha_k)=\tfrac12\,G_0+\tfrac12\,\cos 2\theta\,G_2.
\]
\footnotesize\emph{Micro‑foundation (for completeness).} Standard spherical identities
\(\int_{S^2}(\hat{\mathbf r}\!\cdot\!\mathbf A)^2 d\Omega=\tfrac{4\pi}{3}\|\mathbf A\|^2\),
\(\int_{S^2}(\hat{\mathbf r}\!\cdot\!\mathbf A)^2(\hat{\mathbf r}\!\cdot\!\mathbf B)^2 d\Omega=\tfrac{4\pi}{15}(\|\mathbf A\|^2\|\mathbf B\|^2+2(\mathbf A\!\cdot\!\mathbf B)^2)\)
justify the elimination of the microscopic direction \(\hat{\mathbf r}\) under angular averaging.\normalsize

\subsection*{F.4 Radial integral and scaling}

With an exponential kernel \(K_\xi(r)\propto e^{-r/\xi}\) in 3D,
\begin{equation}
I_\xi=\int_0^\infty r^{4} e^{-r/\xi}\,dr = 24\,\xi^{5}.
\label{eq:G-Ixi}
\end{equation}
Thus the bilocal contribution scales as \(\xi^{5}\) in \(d=3\) (in general \(I_\xi\propto\xi^{d+2}\)).

\subsection*{F.5 Final law and parameterization}

Collecting gradient, projection, lattice weights, and radial factor:
\begin{equation}
\Gdec(\theta)=\overline\Gamma\,\big(1+\varepsilon\cos^2\theta\big),
\qquad
\overline\Gamma \propto \gamma_0\,\xi^5\,a_0\,\big\langle \|\nabla\Ttil^2\|^2\big\rangle\,G_0,
\label{eq:G-law}
\end{equation}
with modulation
\begin{equation}
\varepsilon = \frac{a_2}{a_0}\,\frac{G_2}{G_0}.
\label{eq:G-epsilon}
\end{equation}
Equation~\eqref{eq:G-epsilon} follows by inserting the gate expansion \eqref{eq:G-Cexp} into \eqref{eq:G-bridge}: $a_0$ multiplies the isotropic part ($G_0$), while $a_2$ multiplies the quadrupolar harmonic ($G_2$). This factorization completes the bridge from the WESH master equation (F.0) to the observable angular law: Eq.~\eqref{eq:F-parity-evol} and the echo-frame reduction \eqref{eq:F-parity-modulated} show how the bilocal WESH dissipator governs parity decay, while Eqs.~\eqref{eq:G-bridge}--\eqref{eq:G-epsilon} evaluate its angular projection and yield the robust $\cos^2\theta$ harmonic with amplitude $\varepsilon=(a_2/a_0)(G_2/G_0)$.

\subsection*{F.6 Device factor and inversion}

For any layout,
\[
\frac{G_2}{G_0} \;=\; \frac{\sum_k w_k \cos 2\alpha_k}{\sum_k w_k}, \qquad
w_k \propto r_k^2 e^{-r_k/\xi}.
\]
\textit{Eagle subset (Fig.~\ref{fig:ang-geometry}).} With three representative orientations \((\alpha_h,\alpha_v,\alpha_d)=(90^\circ,0^\circ,\alpha_3)\),
\[
\frac{G_2}{G_0} \;=\; \frac{w_v - w_h + w_d\cos 2\alpha_3}{w_h+w_v+w_d}.
\]
This exposes the threshold condition for the sign and magnitude of \(G_2/G_0\) in terms of geometric weights \((w_h,w_v,w_d)\) extracted from the device map (Fig.~\ref{fig:ang-geometry}).

\noindent\textit{Inversion to state anisotropy.} From \(\varepsilon=(a_2/a_0)\,(G_2/G_0)\),
\[
\frac{a_2}{a_0} \;=\; \frac{\varepsilon}{G_2/G_0}.
\]
Thus a measurement of \(\varepsilon\) combined with a device–level \(G_2/G_0\) (computed from the chip layout) yields the intrinsic anisotropy of the prepared state, independently of kernel strength.

\subsection*{F.7 W-state anti-modulation (sign sketch)}
\label{sec:F7}

For the symmetric single-excitation state \(|W_N\rangle\), the rotated-parity correlator is locally concave at \(\theta=0\) in the preparation plane, giving a negative quadrupolar coefficient in the gate expansion:
\[
C_W(\Psi;\hat{\mathbf r}) \;=\; a_0^{(W)} \;-\; |a_2^{(W)}|\,(\hat{\mathbf r}\!\cdot\!\hat{\mathbf n})^2 + \cdots,
\]
hence \(a_2^{(W)}<0\) and \(\varepsilon_W = (a_2^{(W)}/a_0^{(W)})(G_2/G_0)<0\), in agreement with the observed anti–modulation in the \(W\)–state control runs (main text).

\subsection*{F.8 Validity and corrections}

\emph{Regime.} Short‑range dominance \(r\sim\xi\), weak gate anisotropy \(a_2\ll a_0\), rotated‑parity readout.\\
\emph{Corrections.} Higher even harmonics in \(C(\Psi)\) (e.g.\ \(a_4\mu^4\)) and next order in~\eqref{eq:G-grad} add small \(\cos(4\theta)\) admixtures at high precision; environmental channels add an angle‑independent baseline.\\
\emph{Scaling.} Device dependence enters only via \(G_2/G_0\); \(\overline\Gamma\) inherits the \(\xi^5\) scaling.
\endgroup
\vspace{1cm}

\section*{Appendix G — WESH--Noether as Pre-Geometric Path Independence}
\addcontentsline{toc}{section}{Appendix G — WESH--Noether as Pre-Geometric Path Independence}
\label{app:wesh-noether-pregeom}

\begingroup
\setcounter{equation}{0}
\renewcommand\theequation{G.\arabic{equation}}
\setcounter{theorem}{0}
\renewcommand\thetheorem{G.\arabic{theorem}}
\setcounter{corollary}{0}
\renewcommand\thecorollary{G.\arabic{corollary}}
\setcounter{remark}{0}
\renewcommand\theremark{G.\arabic{remark}}
\setcounter{lemma}{0}
\renewcommand\thelemma{G.\arabic{lemma}}

\paragraph{Context.}
In the timeless Wheeler--DeWitt sector, the auxiliary parameter $s$ is an \emph{ordinal label} of configurations; no physical time, metric, or thermodynamics exist yet. The foundational constraint is a \emph{pre-geometric} conservation law compatible with the WESH generator (see the master equation, \eqref{eq:wesh-master}).
\emph{Terminology.} “Flow” is the shorthand of an \emph{ordinal vector field} in state space; $s$ is not physical time and no temporal structure is assumed prior to the $s\!\to\!t$ bootstrap.

\begin{center}
\fbox{\begin{minipage}{0.95\linewidth}
\textbf{Box G.1: Static vs Dynamic in the Pre-Geometric Regime.}
\vspace{0.3cm}
~\\
In the timeless WDW sector, there is no external time parameter $t$ to define “flow” in the standard sense. Instead, the auxiliary parameter $s$ labels configurations in a pre-causal ordering structure $(\mathcal{S}, \prec, \mathcal{M})$. This is a \emph{static dynamism}: a frozen landscape of potential transitions, characterized by:
\begin{itemize}[leftmargin=1.2em,itemsep=0.2em]
\item \textbf{Ordinal structure:} States are ordered by the Lyapunov functional $\mathcal{M}[\rho]$;
\item \textbf{Path independence:} Global charges $\langle\hat{Q}_{\rm tot}\rangle$ are constant along any WESH flowline in $\mathcal{S}$;
\item \textbf{Gradient architecture:} The vector field $\nabla_\rho\mathcal{M}$ defines potential streamlines.
\end{itemize}
Physical time emerges when this static configuration is \emph{actualized} via the bootstrap $dt/ds=\Gamma[\Psi]>0$, converting potential order into temporal flow.
\end{minipage}}
\end{center}

\vspace{0.3cm}
\subsection*{G.1 Operator formulation of pre-geometric Noether}
\vspace{0.3cm}

\noindent Let $\mathcal L$ be the WESH GKSL generator (see \eqref{eq:wesh-master}) and $\mathcal L^\dagger$ its adjoint on observables:
\[
\mathcal L^\dagger[A] \;=\; -\,i[H_{\mathrm{eff}},A] \;+\; \nu\!\int_x\!\mathcal D^\dagger_{\hat T^2(x)}[A] \;+\; \int_{xy}\gamma(x,y)\,C(\Psi;x,y)\,\mathcal D^\dagger_{L_{xy}}[A],
\]
with $\int_x,\int_{xy}$ as in Eq.~\eqref{eq:measure-conventions}, $\mathcal D^\dagger_{L}[A]= L^\dagger A L - \tfrac12\{L^\dagger L, A\}$, and $L_{xy}=\hat T^2(x)-\hat T^2(y)$ (Hermitian).

\vspace{0.5cm}
\noindent\textbf{Definition G.1 (Pre-geometric WESH--Noether).}
For every global conserved charge $\hat Q_{\rm tot}$,
\begin{equation}
\mathcal L^\dagger[\hat Q_{\rm tot}] \;=\; 0 \,.
\label{eq:H-noether}
\end{equation}
This is equivalent to $\tfrac{d}{ds}\langle \hat Q_{\rm tot}\rangle=0$ for all $\rho(s)$. The kernel weight is causal–exponential with range $\xi$ (see \eqref{eq:gamma-kernel}).

\begin{remark}[Self-consistent physical subset]
The “physical subset” is \emph{not} pre-imposed. Operationally, it is the basin of attraction of the bootstrap fixed point(s) $\rho^\star$ that (i) minimize the Lyapunov functional (Appendix~D), (ii) satisfy WESH--Noether at the fixed point, and (iii) are reached by iterating the bootstrap from RAQ-physical initial conditions (i.e., they belong to the basin of attraction).
WESH--Noether thus acts as a \emph{local structural constraint} of the (state-dependent) generator throughout the evolution, without a pre/post-bootstrap dichotomy.
\end{remark}

\vspace{0.3cm}

\begin{theorem}[WESH--Noether]\label{thm:wesh-noether}
Let $\mathcal L$ be a GKSL generator governing evolution in the atemporal ordering parameter $s$ within the Wheeler--DeWitt sector, with Heisenberg adjoint $\mathcal L^\dagger$. Assume Hermitian jump operators $L_\alpha \in \{\hat T^2(x),\,L_{xy}=\hat T^2(x)-\hat T^2(y)\}$ and the mild spectral regularity of Proposition~G.2. Then, for any global charge operator $\hat Q_a$, the following are equivalent:
\begin{equation*}
\mathcal L^\dagger[\hat Q_a]=0 
\quad\Longleftrightarrow\quad
[H_{\mathrm{eff}},\hat Q_a]=0\ \ \text{and}\ \ [L_\alpha,\hat Q_a]=0\ \ \text{for all }\alpha\text{ with nonzero rate}.
\end{equation*}
No geometric or thermal structure is presupposed.
\end{theorem}

\begin{corollary}[Path independence / $s$-conservation]\label{cor:wesh-path-indep}
For all states $\rho(s)$ and all $a$, $\tfrac{d}{ds}\langle \hat Q_a\rangle=0$. Equivalently, the charge one-form $\omega_a[\rho]:=\Tr(\hat Q_a\,\mathcal L[\rho])\,ds$ is exact (zero holonomy).
\end{corollary}

\begin{remark}[Specialization to the KMS wedge]
In the near-horizon Rindler wedge, the Modular--KMS spectral regularity (Definition~\ref{def:modular-KMS-regularity}) implies the hypothesis of Prop.~G.2 (Lemma~\ref{lem:KMS-regularity}); hence the equivalence holds there verbatim.
\end{remark}

\vspace{0.3cm}

\subsection*{G.2 Necessary \& sufficient algebraic conditions (Hermitian Lindblad structure)}
\vspace{0.3cm}

\noindent In the present framework the Lindblad operators are Hermitian, $L\in\{\hat T^2(x),\,L_{xy}\}$, and the standard GKSL identity
\[
\Tr\!\big(A\,\mathcal D[L]\rho\big)=\tfrac12\,\big\langle\,L^\dagger[A,L]+[L^\dagger,A]L\,\big\rangle
\]
yields, for Hermitian $L$,
\begin{equation}
\mathcal D_{L}^\dagger[\hat Q]\;=\;-\tfrac12\,[L,[L,\hat Q]].
\label{eq:H-double-comm}
\end{equation}

\noindent\textbf{Proposition G.2 (Equivalence under mild spectral regularity).}\label{prop:G2}
\vspace{0.3cm}
~\\

\noindent Assume:
\begin{enumerate}[label=(\roman*),leftmargin=1.9cm,itemsep=0.2cm]
\item $\hat{Q}_{\rm tot}$ is bounded (or the identities are understood as quadratic forms on a common invariant core);
\item Hermitian jump set $L\in\{\hat T^2(x),\,L_{xy}\}$ in the generator \eqref{eq:wesh-master};
\item (\emph{Mild spectral regularity}) For each such $L$, the global charge $\hat Q_{\rm tot}$ is block–diagonal in the spectral decomposition of $L$ (equivalently: $\hat Q_{\rm tot}$ commutes with the spectral projectors of $L$ on the relevant support).
\end{enumerate}
Then the pre–geometric WESH–Noether condition \eqref{eq:H-noether} holds \emph{iff}
\begin{equation}
[H_{\mathrm{eff}},\hat Q_{\rm tot}] = 0,\qquad
[\hat T^2(x),\hat Q_{\rm tot}] = 0\ \ \forall x,\qquad
[L_{xy},\hat Q_{\rm tot}] = 0\ \ \forall x,y.
\label{eq:H-comm-all}
\end{equation}

\vspace{0.3cm}
\begin{proof}
\emph{Sufficiency.} If \eqref{eq:H-comm-all} holds, each dissipative term in $\mathcal L^\dagger[\hat Q_{\rm tot}]$ vanishes by \eqref{eq:H-double-comm}, and the Hamiltonian part yields $[H_{\mathrm{eff}},\hat Q_{\rm tot}]=0$, hence \eqref{eq:H-noether}.

\vspace{0.3cm}
\noindent\emph{Necessity.}
Assume $\mathcal L^\dagger[\hat Q_{\rm tot}]=0$ with $\hat Q_{\rm tot}=\hat Q_{\rm tot}^\dagger$.
Take the Hilbert--Schmidt pairing with $\hat Q_{\rm tot}$:
\[
0=\Tr\!\big(\hat Q_{\rm tot}\,\mathcal L^\dagger[\hat Q_{\rm tot}]\big).
\]
The Hamiltonian contribution vanishes by cyclicity,
$\Tr\!\big(\hat Q_{\rm tot}[-iH_{\rm eff},\hat Q_{\rm tot}]\big)=0$.
For Hermitian Lindblad operators, one has
\[
\Tr\!\big(\hat Q_{\rm tot}\,\mathcal D_L^\dagger[\hat Q_{\rm tot}]\big)
=-\tfrac12\,\Tr\!\big([L,\hat Q_{\rm tot}]^\dagger [L,\hat Q_{\rm tot}]\big)\le 0.
\]
Using $\nu>0$ and $\gamma(x,y)\,C(\Psi;x,y)\ge 0$, this yields
\begin{align*}
\int_x \Tr\!\big([\hat T^2(x),\hat Q_{\rm tot}]^\dagger[\hat T^2(x),\hat Q_{\rm tot}]\big) &= 0, \\[6pt]
\int_{xy}\gamma(x,y)\,C(\Psi;x,y)\,
\Tr\!\big([L_{xy},\hat Q_{\rm tot}]^\dagger[L_{xy},\hat Q_{\rm tot}]\big) &= 0,
\end{align*}
hence $[\hat T^2(x),\hat Q_{\rm tot}]=0$ (on the support of $\int_x$) and
$[L_{xy},\hat Q_{\rm tot}]=0$ wherever $\gamma(x,y)C(\Psi;x,y)\neq 0$.
Plugging back into $\mathcal L^\dagger[\hat Q_{\rm tot}]=0$ leaves
$-i[H_{\rm eff},\hat Q_{\rm tot}]=0$, hence $[H_{\rm eff},\hat Q_{\rm tot}]=0$.
\end{proof}

\begin{remark}[Athermal generality and relation to Section~6]
Proposition~G.2 is pre–geometric and athermal: no KMS structure, thermal equilibrium, or wedge geometry is required. The proposition relies solely on mild spectral regularity of the Lindblad operators in the kinematical Hilbert space. Section~6 shows that the near–horizon Rindler wedge, equipped with Modular--KMS spectral regularity (Definition~\ref{def:modular-KMS-regularity}), provides a concrete regime where the hypothesis of Proposition~G.2 is satisfied (Lemma~\ref{lem:KMS-regularity}); the wedge analysis is thus a specialization of the general algebraic framework established here, not an alternative foundation.
\end{remark}

\begin{lemma}[Spectral genericity]\label{lem:spectral-generic}
For the time–field operator $\hat T(x)$ with continuous spectrum in the WDW sector, the quadratic operator $\hat T^2(x)$ has a non–degenerate spectral measure on any bounded interval $[E_{\min},E_{\max}]$ with probability $1$ under generic local perturbations.
\end{lemma}

\begin{proof}[Proof sketch]
The spectrum of $\hat T^2(x)$ is the image of $\sigma(\hat T)$ under $\lambda\mapsto\lambda^2$. For continuous $\sigma(\hat T)\subset\mathbb{R}$, generic perturbations preserve measure–theoretic nondegeneracy (Kato perturbation theory), which supports the projector–commutation regularity used in Prop.~G.2. \qedhere
\end{proof}

\begin{remark}[Regularity note]
Commutation with spectral projectors is the minimal operational hypothesis required by the double–commutator identity; it is generic for local polynomials of $\hat T$ and stable under local perturbations. Since $L_{xy}=\hat T^2(x)-\hat T^2(y)$, $[L_{xy},\hat Q_{\rm tot}]=0$ follows by linearity once $[\hat T^2(x),\hat Q_{\rm tot}]=[\hat T^2(y),\hat Q_{\rm tot}]=0$.

\noindent The mild spectral regularity is generic for local polynomials of 
$\hat T$ and their differences. It is strictly weaker than assuming 
any thermal structure and does \emph{not} presuppose the emergence of $t$.
\end{remark}

\begin{remark}[T-neutrality as a consistency requirement]
If $[\hat T,\hat Q_{\rm tot}]\neq0$, the state–dependent coefficients of the generator induce a charge–dependent intensity; the ordinal speed $dt/ds$ becomes background–dependent, spoiling the universality of physical time. Moreover, along different state–space paths one generically sources $\hat Q_{\rm tot}$, violating path independence. Hence T–neutrality is not an external postulate but the unique choice compatible with path independence, universality of emergent time, and no–signaling.
\end{remark}

\begin{remark}[Operator–level vs expectation–level conservation]
The WESH–Noether condition \eqref{eq:H-noether} is an operator–level statement, not merely conservation of expectations. Operator–level conservation guarantees zero holonomy of the charge one–form on state space; conserving only $\langle\hat Q_{\rm tot}\rangle$ allows hidden anomalies in the commutant that re–emerge under block decompositions or sub–sector measurements, obstructing the emergence of a unique geometry.
\end{remark}

\vspace{0.3cm}
\subsection*{G.3 From potential order to geometry (bridge to alignment)}
\vspace{0.3cm}
Let $\mathcal M[\rho]$ be the WESH Lyapunov functional (strictly convex in the regime of interest; see Appendix~D). Consider the constrained minimization “minimize $\mathcal M$ subject to WESH--Noether”. At a stationary point, the Karush–Kuhn–Tucker optimality conditions require that the state-space gradient $\nabla_\rho\mathcal M$ aligns with the gradient of the constraint manifold. For the quadratic structure of $\mathcal M$ (in $\|\partial_\mu\tau\|^2$) and the Yukawa-weighted entanglement potential, this forces the field-level relation $\partial_\mu\tau \parallel \partial_\mu\Phi$, i.e.:

\begin{equation}
\partial_\mu \tau \;=\; k\,\partial_\mu \Phi,
\qquad
\Phi(x)=\!\int\! K(x-y)\,C(\rho^\star;x,y)\,d^4y,
\label{eq:alignmentG}
\end{equation}
i.e.\ the pre-geometric consistency crystallizes into an emergent metric relation.

\begin{remark}[Existence vs uniqueness]\label{rem:H-unique}
WESH--Noether ensures \emph{existence} of a stationary state (Appendix~D, fixed-point lemma via Schauder–Tychonoff) but not uniqueness. Uniqueness follows from additional physical constraints:
\begin{enumerate}[label=(\roman*),leftmargin=1.5cm,itemsep=0.2em]
\item Causal support of $\gamma(x,y)$ (no superluminal signaling);
\item Finite correlation time $\tau_{\rm corr}\sim\xi/c$ (Markov window);
\item Cross-horizon connectivity (null-plane Markov mixing).
\end{enumerate}
These are not fine-tunings but \emph{consistency requirements} for the scaling hypothesis $\xi\simeq L_P$. Together with WESH--Noether, they select the unique fixed point where gradient alignment \eqref{eq:alignmentG} holds.
\end{remark}

\subsection*{G.4 Empirical signatures}
\vspace{0.3cm}
The pre-geometric constraint has distinct falsifiable consequences:
\begin{enumerate}[leftmargin=1.2em,itemsep=0.25em]
\item \textbf{Collective stability.} The coherence time scales as $\tau_{\mathrm{coh}}(N)\propto N^2$, reflecting causal mixing under the WESH constraint.
\item \textbf{Angular law.} The decoherence rate obeys $\overline\Gamma\,[1+\varepsilon\cos^2\theta]$ with $\varepsilon=(a_2/a_0)\,(G_2/G_0)$, factorizing state anisotropy and device geometry.
\item \textbf{BH thermodynamics/holography.} Near-horizon KMS structure yields the thermodynamic law (area term and logarithmic corrections) without external parameters; the framework establishes WESH validity through a holographic criterion spanning ${\sim}50$ orders of magnitude in radius, from Planck-scale correlations to macroscopic event horizons.
\end{enumerate}
Each signature tests a different aspect of the structure: collective stability probes mixing, angular dependence tests locality and causal connectivity, and black hole thermodynamics verifies the bridge to general relativity. 
\vspace{0.3cm}
\subsection*{G.5 Physical meaning and scope}
\vspace{0.3cm}
\emph{Meaning.} Eqs.~\eqref{eq:H-noether}–\eqref{eq:H-comm-all} formalize the “no-vorticity” intuition without thermodynamic inputs: in the minimal GKSL representation each Hermitian Lindblad channel commutes with the global charges, hence the WESH flow cannot generate spurious global sources.  
\emph{Scope.} Detailed balance is \emph{derived} at the emergent level (near-horizon KMS), not assumed: see Sec.~\ref{sec:KMS-Rindler}, Eq.~\eqref{eq:UD-balance}.

\paragraph{Laminar-flow analogy.}
Imagine a stationary laminar fluid where the velocity field is globally time-independent yet locally directed along well-defined streamlines: this captures the pre-geometric WESH structure in state space. "No‑vorticity'' corresponds to path independence of global charges (zero holonomy of $\omega_Q$); stagnation points correspond to eigentimes where geometric structure crystallizes. Physical time $t$ then emerges as the local flow speed via $dt/ds=\Gamma[\Psi]$, actualizing the potential streamlines into worldlines. This clarifies why classical detailed balance cannot be imposed at this stage: temperature and KMS structure arise only after the flow has crystallized into a geometry.

\paragraph{Operational note n.1 (compatibility with BH sector).}
The horizon-symmetric (gauge-even) conditional expectation $\mathbb E_{\rm even}(X)=\tfrac12(X+GXG)$ used in the BH discussion is the minimal CP idempotent onto the even subalgebra. It is \emph{compatible} with \eqref{eq:H-noether}–\eqref{eq:H-comm-all} but not required to derive them.

\paragraph{Operational note n.2 (RAQ compatibility).}
The WESH dynamics preserves the RAQ-physical subspace: T-neutrality and Hermitian jumps ensure that the generator neither sources nor leaks gauge constraints. The global WESH--Noether condition is the observable face of this preservation in the emergent sector.

\vspace{1cm}

\section*{Appendix H — CP/TP preservation and no–signaling in WESH}
\addcontentsline{toc}{section}{Appendix H — CP/TP preservation and no–signaling in WESH}
\label{app:CPTP}

\begingroup
\setcounter{equation}{0}
\renewcommand\theequation{H.\arabic{equation}}
\setcounter{theorem}{0}
\renewcommand\thetheorem{H.\arabic{theorem}}
\setcounter{lemma}{0}
\renewcommand\thelemma{H.\arabic{lemma}}
\setcounter{remark}{0}
\renewcommand\theremark{H.\arabic{remark}}
\setcounter{corollary}{0}
\renewcommand\thecorollary{H.\arabic{corollary}}
\setcounter{proposition}{0}
\renewcommand\theproposition{H.\arabic{proposition}}

\vspace{0.5cm}
\noindent\textbf{Setting.}
Consider the pre–geometric master equation \eqref{eq:wesh-master} with Hermitian channels
\[
L_\alpha\in\big\{\hat T^2(x),\ L_{xy}=\hat T^2(x)-\hat T^2(y)\big\},
\]
and scalar rates
\[
c_\alpha(s)\in\{\ \nu,\ \gamma(x,y)\,C(\Psi;x,y)\ \},\qquad 0\le C\le1,\ \gamma\ge0,
\]
assumed bounded and piecewise continuous in $s$ (hence strongly measurable and uniformly bounded on compact intervals). Here $C(\Psi;x,y)$ depends only on \emph{local/bilocal reductions}
\[
\rho_x=\Tr_{x^c}\rho,\qquad \rho_y=\Tr_{y^c}\rho,\qquad \rho_{xy}=\Tr_{(xy)^c}\rho,
\]
through bounded, Borel–measurable functionals that are Lipschitz in the trace norm. The effective Hamiltonian $H_{\mathrm{eff}}$ is self–adjoint so that $-i[H_{\mathrm{eff}},\cdot]$ generates a strongly continuous one–parameter group of trace–norm isometries on the trace class.

\begin{theorem}[CP/TP preservation for the nonlinear WESH evolution]\label{thm:H-CPTP}
Under the above assumptions, for any $s\ge0$ the non–autonomous evolution $E_s:\rho_0\mapsto\rho(s)$ generated by \eqref{eq:wesh-master} preserves positivity and unit trace; each frozen micro-step is CPTP, and the product-integral limit preserves these properties.
\end{theorem}

\begin{proof}[Proof sketch]
Partition $[0,s]$ into $0=s_0<\dots<s_N=s$ and \emph{freeze} the rates at the left endpoints:
\[
\mathcal L_k(\cdot)=-\,i[H_{\mathrm{eff}},\cdot]+\nu\!\int d^4x\,\mathcal D[\hat T^2(x)](\cdot)
+\!\iint d^4x\,d^4y\,\gamma(x,y)\,C(\Psi_{s_k};x,y)\,\mathcal D[L_{xy}](\cdot).
\]
Each $\mathcal L_k$ is GKSL (Hermitian jumps, nonnegative rates), thus $\Phi_k:=\exp(\Delta s_k\,\mathcal L_k)$ is CPTP and $E_s^{(N)}=\Phi_{N-1}\circ\cdots\circ\Phi_0$ is CPTP by composition. 

\noindent\textit{Non–autonomous convergence.} 
Assume $s\mapsto c_\alpha(s)$ are strongly measurable and uniformly bounded, and that the map $\rho\mapsto C(\Psi;x,y)$ is Lipschitz in the trace norm via the reduced states. Then the product–integral scheme with frozen coefficients converges strongly to the unique non–autonomous evolution solving \eqref{eq:wesh-master}. Strong limits of CPTP maps remain CP and TP, hence $E_s$ is positivity- and trace-preserving (stepwise CPTP).
\end{proof}

\begin{corollary}[System–ancilla consistency]\label{cor:H-ancilla}
For any ancilla $A$ and initial $\rho_{SA}$, $(E_s\!\otimes\!\mathbb I_A)(\rho_{SA})$ preserves positivity and unit trace.
\end{corollary}

\begin{proposition}[Spacelike no–signaling]\label{prop:H-nosig}
Let $\Sigma$ be a Cauchy slice and $A\subset\Sigma$ with complement $A^c$. If (i) $\gamma(x,y)=0$ for spacelike $(x,y)$ and (ii) $[\hat T(x),\hat T(y)]=0$ for spacelike $(x,y)$, then
\[
\frac{d}{ds}\rho_A(s)\;=\;\Tr_{A^c}\big(\mathcal L_{J(A)}[\rho(s)]\big),
\]
i.e.\ $\rho_A$ depends only on operators supported in the causal domain $J(A)$; spacelike operations on $A^c$ cannot instantaneously affect $\rho_A$.
\end{proposition}

\begin{remark}[Nonlinearity vs CP]
Nonlinearity enters only through the \emph{scalar} rates $c_\alpha(s)$ (via $C(\Psi;x,y)$), not in the operator set $\{L_\alpha\}$. Freezing $c_\alpha$ yields GKSL steps; product–integral concatenation preserves CP/TP in the limit, avoiding pathologies such as loss of positivity or violation  of complete positivity that can occur in arbitrary nonlinear  master equations.
\end{remark}

\section*{Appendix I — Consistency selection of the quadratic dissipator \texorpdfstring{$\mathcal D[\hat T^2]$}{D[T\textasciicircum 2]}}
\addcontentsline{toc}{section}{Appendix I — Consistency selection of the quadratic dissipator}
\label{app:QuadSelection}

\begingroup
\setcounter{equation}{0}
\renewcommand\theequation{I.\arabic{equation}}
\setcounter{theorem}{0}
\renewcommand\thetheorem{I.\arabic{theorem}}
\setcounter{lemma}{0}
\renewcommand\thelemma{I.\arabic{lemma}}
\setcounter{remark}{0}
\renewcommand\theremark{I.\arabic{remark}}
\setcounter{proposition}{0}
\renewcommand\theproposition{I.\arabic{proposition}}

\vspace{0.5cm}
\noindent\textbf{Setting.}
Local channels are even, real–analytic functionals of the time field,
\[
L_x=F(\hat T(x))=\sum_{m\ge1}a_{2m}\,\hat T^{2m}(x),\qquad
L_{xy}=F(\hat T(x))-F(\hat T(y)),
\]
consistent with CPT–evenness. The operator–level WESH–Noether constraint (App.~G, Eq.~\eqref{eq:H-noether}) fixes the commutant structure; here we address the admissible \emph{local} form in the infrared.

\subsection*{I.1 Constraints and selection principle}
\vspace{0.2cm}

\noindent\textbf{Constraints.}
\begin{enumerate}[label=(\roman*),leftmargin=1.8cm,itemsep=0.2em]
\item \textbf{CPT–evenness:} time–reversal on $\hat T$ excludes odd powers.
\item \textbf{WESH–Noether:} compatibility with \eqref{eq:H-noether} throughout the flow.
\item \textbf{Collective stability (structural requirement):}
the bootstrap closure condition $\mu \cdot \sigma_\alpha = \mathcal{O}(1)$ (Sec.~2.6) uniquely fixes $\alpha = 2$, yielding $\tau_{\mathrm{coh}}(N) \propto N^{2}$ (Sec.~\ref{sec:theoretical-predictions}; cf.\ Eq.~\eqref{eq:scaling-condition} in Sec.~1). This is a structural consistency requirement of the fixed-regime, not an empirical postulate.
\end{enumerate}

\begin{proposition}[Quadratic local normal form]\label{prop:I-quadratic}
Under (i)–(iii), the admissible local normal form is uniquely quadratic:
\[
F(z)=a_2 z^2,\qquad \text{i.e.}\ \ \mathcal D[\hat T^2(x)].
\]
\end{proposition}

\begin{proof}
CPT symmetry implies that $F$ is even, hence the leading monomial can be written as $F(z) \sim z^{2n}$ with $n \ge 1$. As shown in Sec.~1 (Eq.~\eqref{eq:scaling-condition}), under the WESH normalization $\gamma \propto N^{-2}$ one has
\[
F(\hat T) \sim \hat T^{2n} \quad \Rightarrow \quad \tau_{\mathrm{coh}}(N) \propto N^{2n}.
\]
Imposing collective stability $\tau_{\mathrm{coh}}(N) \propto N^{2}$ fixes $n = 1$ uniquely, hence $F(z) = a_2 z^2$ and the local channel is $\mathcal{D}[\hat T^2(x)]$.
\end{proof}

\begin{remark}[Empirical validation]
The $N^2$ scaling is used in the theory as a fixed-regime consistency condition (collective stability) selecting the quadratic channel. Experimentally, observing $\tau_{\mathrm{coh}} \propto N^2$ acts as a direct validation of this structural requirement and excludes higher even monomials in the collective regime.
\end{remark}

\begin{remark}[Convergence with effective field theory]
The quadratic selection also follows from IR minimality: among CPT-even operators, the lowest-dimension monomial dominates at long wavelengths in the Wilsonian sense. This convergence of two independent arguments—structural (collective stability from bootstrap closure) and effective (Wilsonian IR dominance)—strengthens confidence in the uniqueness of $\mathcal{D}[\hat T^2]$.
\end{remark}

\endgroup

\section*{Appendix J — $\Lambda$ as intrinsic shot noise of eigentime production}
\addcontentsline{toc}{section}{Appendix J — $\Lambda$ as intrinsic shot noise of eigentime production}
\label{app:LambdaShotNoise}

\begingroup
\setcounter{equation}{0}
\renewcommand\theequation{J.\arabic{equation}}

\vspace{0.5cm}
\paragraph{Scope and claim.}
This appendix establishes the \emph{natural magnitude} of the infrared cosmological constant in QFTT--WESH as a consequence of the intrinsic fluctuations in eigentime production.
The mechanism is dynamical: $\Lambda$ emerges as the irreducible scalar shot-noise residue of the Lindblad channel generating time.
We derive the scaling of the typical fluctuation width $\delta\Lambda_{\rm typ}$ on a four-volume $V_4$, demonstrating that it follows a parameter-free inverse-volume law.

% ----------------------------
\paragraph{J.0 The Martingale structure of eigentime generation.}
In QFTT--WESH, eigentime production is a stochastic process governed by the local hazard $\lambda(x)$ (Eq.~\eqref{eq:local-hazard}) and the global GKSL intensity functional $\Gamma[\rho]$.
The emergent physical time $t(s)$ is bootstrapped via
\begin{equation}
\frac{dt}{ds}=\Gamma[\Psi(s)],\qquad t(s)=\int_0^s \Gamma[\Psi(s')]\,ds',
\label{eq:J_t_of_s}
\end{equation}
generating a counting process $N_{\rm Eig}(s)$.
This process admits the Doob--Meyer decomposition:
\begin{equation}
N_{\rm Eig}(s)\ =\ \underbrace{\int_0^s \frac{\Gamma[\Psi(u)]}{\tau_{\rm Eig}}\,du}_{\text{Compensator}}\ +\ \underbrace{M(s)}_{\text{Martingale}}.
\label{eq:J_DoobMeyer}
\end{equation}
The martingale term $M(s)$ represents the irreducible quantum fluctuations of the event count around its compensator.

\paragraph{J.1 Regional counting on the physical manifold.}
Consider a spacetime region $\mathcal R$ of the emergent IR manifold with physical four-volume
\begin{equation}
V_4(\mathcal R)\ :=\ \int_{\mathcal R} d^4x\sqrt{-g}\,.
\label{eq:J_V4}
\end{equation}
We define the \emph{regional eigentime count} $N(\mathcal R)$ as the cardinality of the set of eigentime events $\{x_i\} \subset \mathcal R$.
Since eigentime events seed the geometric estimators on scales $L\gg\xi$, they form a quasi-uniform mesh with filling density governed by the hazard.
For regions much larger than the correlation length, the expected count is extensive:
\begin{equation}
\mathbb E\big[N(\mathcal R)\big]\ \simeq\ \bar\lambda\,V_4(\mathcal R)\ \equiv\ \frac{V_4(\mathcal R)}{V_\xi}\,,
\label{eq:J_N_of_R}
\end{equation}
where $V_\xi \sim \xi^4$ is the correlation 4-volume.

\paragraph{J.2 Central Limit scaling from finite-range mixing.}
The WESH bilocal channel is constructed with an exponential-causal kernel of finite range $\xi$. This implies a finite correlation time $\tau_{\rm corr}$ and a Markov window $\mu \ll 1$.
Under the block-mixing hypothesis (Dobrushin-type contractivity for $L \gg \xi$), the event count on large regions obeys the Central Limit Theorem despite local correlations:
\begin{equation}
\mathrm{Var}\big[N(\mathcal R)\big]\ =\ \alpha_N\,\mathbb E\big[N(\mathcal R)\big],\qquad
\delta N(\mathcal R)\ \sim\ \sqrt{N(\mathcal R)},
\label{eq:J_CLT_count}
\end{equation}
where $\alpha_N=\mathcal O(1)$ accounts for short-range correlations.

\paragraph{J.3 Derivation of the $\Lambda$ fluctuation scaling.}
By IR uniqueness, the cosmological constant couples to the volume element $V_4$.
We identify $\Lambda_{\rm eff}$ via the zero-derivative action term:
\begin{equation}
S_{\Lambda}[g]\ =\ -\frac{\Lambda_{\rm eff}}{8\pi G}\,V_4(\mathcal R).
\label{eq:J_SLambda}
\end{equation}
Discretizing $\mathcal R$ into $N(\mathcal R)$ correlation cells of volume $V_\xi$, the total action fluctuates due to the stochastic residue of the GKSL channel in each cell.
The action residue obeys random-walk scaling:
\begin{equation}
\delta S_{\Lambda}(\mathcal R)\ \sim\ \sigma_\Lambda\,\sqrt{N(\mathcal R)}, \qquad \sigma_\Lambda = \mathcal O(1).
\label{eq:J_dS}
\end{equation}
Solving \eqref{eq:J_SLambda} for $\Lambda_{\rm eff}$ and propagating the uncertainty:
\begin{align}
\delta\Lambda_{\rm typ}(\mathcal R)
&\sim\ \frac{8\pi G}{V_4(\mathcal R)}\,\delta S_{\Lambda}(\mathcal R)
\sim \frac{8\pi G}{N(\mathcal R)\,V_\xi}\,\sigma_\Lambda\sqrt{N(\mathcal R)}
\nonumber\\
&\sim\ \frac{\alpha_\Lambda G}{V_\xi}\,\frac{1}{\sqrt{N(\mathcal R)}}.
\label{eq:J_dLambda_intermediate}
\end{align}
Imposing the Planck anchoring condition $G \sim \xi^2$ (in natural units) and using $V_\xi \sim \xi^4$, the microscopic scales cancel:
\begin{equation}
\delta\Lambda_{\rm typ}(\mathcal R)\ \sim\ \frac{1}{\xi^2 \sqrt{N(\mathcal R)}}.
\end{equation}
Substituting $N(\mathcal R) \simeq V_4(\mathcal R)/\xi^4$ yields the fundamental scaling:
\begin{equation}
\boxed{\quad
\delta\Lambda_{\rm typ}(\mathcal R)\ \sim\ \frac{\alpha_\Lambda}{\sqrt{V_4(\mathcal R)}},\qquad
\alpha_\Lambda=\mathcal O(1).\quad}
\label{eq:J_dLambda_V4}
\end{equation}
This result depends solely on the counting statistics of the underlying process and contains no free parameters beyond the $\mathcal O(1)$ coefficient $\alpha_\Lambda$.

\paragraph{J.4 Cosmological magnitude: $\delta\Lambda \sim H^2$.}
Taking the coarse-graining window to be the observable universe at an epoch with Hubble rate $H$, we have $V_4 \sim H^{-4}$.
Equation \eqref{eq:J_dLambda_V4} then yields:
\begin{equation}
\delta\Lambda_{\rm typ}\ \sim\ \alpha_\Lambda\,H^2.
\label{eq:J_dLambda_H2}
\end{equation}
At the present epoch, this predicts a magnitude $\sim 10^{-122}\, L_P^{-2}$, consistent with observation.
This value is not obtained by fine-tuning the microscopic scale $\xi$ (which is fixed at $L_P$), but arises dynamically from the large-$N$ suppression factor $1/\sqrt{V_4}$.

\medskip
\noindent\textit{Note on interpretation.}
This derivation predicts the \emph{typical fluctuation amplitude} on a finite volume. The sign and instantaneous value remain stochastic.

\paragraph{J.5 Unified origin of Lorentz emergence and $\Lambda$.}
The tensor and scalar sectors emerge from the same stochastic substrate but scale differently in the continuum limit:
\begin{itemize}
\item \textbf{Tensor sector (geometry):} Tensorial anisotropies are averaged out by kernel smoothing. Convergence to smooth GR is controlled by the sampling density, with error scaling as $\delta g \sim N^{-1/4}$.
\item \textbf{Scalar sector ($\Lambda$):} The vacuum residue is a global intensive quantity. Local averaging does not cancel it; it persists as a global fluctuation scaling as $\delta \Lambda \sim N^{-1/2}$.
\end{itemize}
Lorentz symmetry emergence and the cosmological constant are thus dual manifestations of the same eigentime statistics, distinguished only by their symmetry structure under local averaging.

\paragraph{J.6 Relation to Causal Set Theory.}
The scaling $\delta\Lambda \sim H^2$ is predicted by Causal Set Theory as ``everpresent $\Lambda$'' (Ahmed et al., 2004; Sorkin, 2007).
In QFTT--WESH, discreteness is not postulated axiomatically but derived from the GKSL eigentime production. The agreement in scaling serves as a universality check across discrete-measure quantum gravity approaches.

\vspace{0.5cm}

\begin{tcolorbox}[
  colback=gray!5, colframe=black, boxrule=0.4pt, arc=2pt,
  title={\centering\textbf{Scaling hierarchy of the stochastic substrate}}
]
\begin{align*}
\textbf{Metric reconstruction (tensor):} \quad & \delta g_{\mu\nu} \sim N^{-1/4} \to 0 \\[6pt]
\textbf{Vacuum residue (scalar):} \quad & \delta\Lambda \sim \frac{1}{\sqrt{V_4}} \sim H^2
\end{align*}
\end{tcolorbox}

\endgroup


\section*{\centering References}
\vspace{0.3cm}
\begin{itemize}[leftmargin=0.5in, itemindent=-0.5in, label={}]

\item Adesso, G., Girolami, D., \& Serafini, A. (2012). Measuring Gaussian quantum information and correlations using the Rényi entropy of order 2. \textit{Physical Review Letters}, 109(19), 190502. \url{https://doi.org/10.1103/PhysRevLett.109.190502}

\item Ahmed, M., Dodelson, S., Greene, P. B., \& Sorkin, R. D. (2004). Everpresent $\Lambda$. \textit{Physical Review D}, 69, 103523. \url{https://doi.org/10.1103/PhysRevD.69.103523}

\item Araki, H. (1976). Relative entropy of states of von Neumann algebras. \textit{Publications of the Research Institute for Mathematical Sciences}, 11(3), 809--833. \url{https://doi.org/10.2977/PRIMS/1195191148}

\item Ashtekar, A., Lewandowski, J., Marolf, D., Mourão, J., \& Thiemann, T. (1995). Quantization of diffeomorphism invariant theories of connections with local degrees of freedom. \textit{Journal of Mathematical Physics}, 36(11), 6456--6493. \url{https://doi.org/10.1063/1.531252}

\item Blanco, D. D., Casini, H., Hung, L.-Y., \& Myers, R. C. (2013). Relative entropy and holography. \textit{Journal of High Energy Physics}, 2013(8), 060. \url{https://doi.org/10.1007/JHEP08(2013)060}

\item Bousso, R. (1999). A covariant entropy conjecture. \textit{Journal of High Energy Physics}, 1999(07), 004. \url{https://doi.org/10.1088/1126-6708/1999/07/004}

\item Carlen, E. A., \& Maas, J. (2017). Gradient flow and entropy inequalities for quantum Markov semigroups with detailed balance. \textit{Journal of Functional Analysis}, 273(5), 1810--1869. \url{https://doi.org/10.1016/j.jfa.2017.05.003}

\item Casini, H., Testé, E., \& Torroba, G. (2017). Modular Hamiltonians on the null plane and the Markov property of the vacuum state. \textit{Journal of Physics A: Mathematical and Theoretical}, 50(36), 364001. \url{https://doi.org/10.1088/1751-8121/aa7eaa}

\item Chew, G. F. (1961). \textit{S-matrix theory of strong interactions}. W. A. Benjamin.

\item Facchinei, F., \& Pang, J. S. (2003). \textit{Finite-dimensional variational inequalities and complementarity problems} (Vols. 1--2). Springer. \url{https://doi.org/10.1007/b97543}

\item Ghirardi, G. C., Rimini, A., \& Weber, T. (1986). Unified dynamics for microscopic and macroscopic systems. \textit{Physical Review D}, 34(2), 470--491. \url{https://doi.org/10.1103/PhysRevD.34.470}

\item Gough, J. E., Ratiu, T. S., \& Smolyanov, O. G. (2015). Noether's theorem for dissipative quantum dynamical semi-groups. \textit{Journal of Mathematical Physics}, 56(2), 022108. \url{https://doi.org/10.1063/1.4907985}

\item Guerreiro, T. (2025). Entanglement and squeezing of gravitational waves. \textit{Physical Review D}, 112, L101904. \url{https://doi.org/10.1103/fn5d-mrsj}

\item Haag, R. (1996). \textit{Local quantum physics: Fields, particles, algebras} (2nd ed.). Springer. \url{https://doi.org/10.1007/978-3-642-61458-3}

\item Haag, R., Hugenholtz, N. M., \& Winnink, M. (1967). On the equilibrium states in quantum statistical mechanics. \textit{Communications in Mathematical Physics}, 5(3), 215--236. \url{https://doi.org/10.1007/BF01646342}

\item Lions, J. L., \& Stampacchia, G. (1967). Variational inequalities. \textit{Communications on Pure and Applied Mathematics}, 20(3), 493--519. \url{https://doi.org/10.1002/cpa.3160200302}

\item McCauley, G., Cruikshank, B., Bondar, D. I., \& Jacobs, K. (2020). Accurate Lindblad-form master equation for weakly damped quantum systems across all regimes. \textit{npj Quantum Information}, 6(1), 74. \url{https://doi.org/10.1038/s41534-020-00299-6}

\item Page, D. N., \& Wootters, W. K. (1983). Evolution without evolution: Dynamics described by stationary observables. \textit{Physical Review D}, 27(12), 2885--2892. \url{https://doi.org/10.1103/PhysRevD.27.2885}

\item Penrose, R. (1996). On gravity's role in quantum state reduction. \textit{General Relativity and Gravitation}, 28(5), 581--600. \url{https://doi.org/10.1007/BF02105068}

\item Petz, D. (1986). Sufficient subalgebras and the relative entropy of states of a von Neumann algebra. \textit{Communications in Mathematical Physics}, 105(1), 123--131. \url{https://doi.org/10.1007/BF01212345}

\item Rattazzi, R., Rychkov, V. S., Tonni, E., \& Vichi, A. (2008). Bounding scalar operator dimensions in 4D CFT. \textit{Journal of High Energy Physics}, 2008(12), 031. \url{https://doi.org/10.1088/1126-6708/2008/12/031}

\item Rivas, Á., \& Huelga, S. F. (2012). \textit{Open quantum systems: An introduction}. Springer. \url{https://doi.org/10.1007/978-3-642-23354-8}

\item Sakharov, A. D. (1967). Vacuum quantum fluctuations in curved space and the theory of gravitation. \textit{Soviet Physics Doklady}, 12, 1040--1041.

\item Sorkin, R. D. (2007). Is the cosmological "constant" a nonlocal quantum residue of discreteness of the causal set type? \textit{AIP Conference Proceedings}, 957, 142--153. \url{https://doi.org/10.1063/1.2823750}

\item Spohn, H. (1991). \textit{Large scale dynamics of interacting particles}. Springer. \url{https://doi.org/10.1007/978-3-642-84371-6}

\item Takesaki, M. (1970). \textit{Tomita's theory of modular Hilbert algebras and its applications}. Lecture Notes in Mathematics, Vol. 128. Springer. \url{https://doi.org/10.1007/BFb0065832}

\item Zeh, H. D. (2007). \textit{The physical basis of the direction of time} (5th ed.). Springer. \url{https://doi.org/10.1007/978-3-540-68001-7}
\end{itemize}

\endgroup
\end{document}